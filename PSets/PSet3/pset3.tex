\documentclass[../psets.tex]{subfiles}

\pagestyle{main}
\renewcommand{\leftmark}{Problem Set \thesection}
\setcounter{section}{2}

\begin{document}




\section{Dilute Solutions}
\noindent Name: Steven Labalme\\
\noindent Email: \href{mailto:labalme@mit.edu}{labalme@mit.edu}\\
\noindent Collaborator(s): \emph{Isa Pan, Jordan Gray}
\begin{enumerate}
    \item \marginnote{10/21:}(35 pts) Estimate, as a function of molecular weight, the characteristic diffusion time $\tau$ for a polymer in a solution of viscosity $\eta$, where $\tau$ is defined as
    \begin{equation*}
        \tau = \frac{R^2}{D}
    \end{equation*}
    where $R$ is the size of the polymer (function of $N$ and $l_k$) and $D$ is the diffusion coefficient (remember $D$ is different for different coil models). Determine $\tau$ for polymer coils in a theta-solvent for\dots
    \begin{enumerate}
        \item Freely-draining (Rouse-like) model;
        \begin{proof}
            Recall from class that in general,
            \begin{align*}
                R &= N^\nu l_k&
                D_t &= \frac{\kB T}{f}
            \end{align*}
            Additionally, in the specific case of a theta solvent and free-draining model, we have respectively that
            \begin{align*}
                \nu &= \frac{1}{2}&
                f &= N\xi^*
            \end{align*}
            Therefore,
            \begin{align*}
                \tau &= \frac{(N^{1/2}l_k)^2}{\kB T/N\xi^*}\\
                \Aboxed{\tau &= \frac{N^2l_k^2\xi^*}{\kB T}}
            \end{align*}
        \end{proof}
        \item Non-draining (Zimm-like) model.
        \begin{proof}
            Everything from Q1a holds except that we are now in a non-draining regime, so $f\neq N\xi^*$ but rather
            \begin{equation*}
                f = 6\pi\eta R_H
            \end{equation*}
            Additionally, we have in general that
            \begin{equation*}
                R_H = \gamma R_g = \frac{\gamma R}{\sqrt{6}}
            \end{equation*}
            Therefore,
            \begin{align*}
                \tau &= \frac{R^2}{\kB T/(6\pi\eta\gamma R/\sqrt{6})}\\
                \Aboxed{\tau &= \frac{\sqrt{6}\pi N^{3/2}l_k^3\eta\gamma}{\kB T}}
            \end{align*}
        \end{proof}
        \pagebreak
        \item You are performing a GPC packed with beads with diameter \SI{5}{\micro\meter}. Estimate the time for a DNA molecule with $D=\SI[per-mode=symbol]{0.01}{\micro\meter\squared\per\second}$ to diffuse a distance equal to a bead diameter.
        \begin{proof}
            We want to find the time $\tau$ it takes for diffusion to displace a polymer with $D=\SI[per-mode=symbol]{0.01}{\micro\meter\squared\per\second}$ by $\Delta r=\SI{5}{\micro\meter}$. Per class,
            \begin{align*}
                \prb{\Delta r^2} &= 6D\tau\\
                \tau &= \frac{(\SI{5}{\micro\meter})^2}{6(\SI{0.01}{\micro\meter\squared\per\second})}\\
                \Aboxed{\tau &\approx \SI{417}{\second}}
            \end{align*}
        \end{proof}
    \end{enumerate}
    \item (30 pts) Ultracentrifugation is a means to separate polymers. The same is placed in a tube and spun quickly to exert a centrifugual force on the solution. Three main forces on a polymer molecule are
    \begin{align*}
        F_s &= m\omega^2r\tag{Sedimentation}\\
        F_b &= -m_0\omega^2r\tag{Buoyancy / Archimedes' Principle}\\
        F_d &= -\zeta U\tag{Drag}
    \end{align*}
    where $m$ is the mass of a single polymer, $\omega$ is the angular speed of the centrifuge, $r$ is the stator arm of the centrifuge, $m_0$ is the mass of fluid displaced by the polymer (given by $m_0=m\rho v$), $v$ is the partial specific volume of a polymer, $\rho$ is the solvent density, $\zeta$ is the drag on the polymer, and $U$ is the speed at which the polymer moves.
    \begin{enumerate}
        \item Assuming the polymer has reached its terminal velocity (constant velocity, sum of forces equal zero), calculate the velocity $U$ as a function of $\omega,r,v,\zeta,\rho,M,N_\text{A}$, where $M$ is the molar mass of the polymer and $N_\text{A}$ is Avogadro's number.
        \begin{proof}
            By the definition of molar mass,
            \begin{equation*}
                M = mN_\text{A}
            \end{equation*}
            Thus, since the polymer has no net force acting on it, we have by Newton's second law that
            \begin{align*}
                0 &= F_s+F_b+F_d\\
                &= m\omega^2r-m_0\omega^2r-\zeta U\\
                &= \frac{M}{N_\text{A}}\cdot\omega^2r-\frac{M}{N_\text{A}}\cdot\rho v\omega^2r-\zeta U\\
                \Aboxed{U &= \frac{M\omega^2r}{N_\text{A}\zeta}(1-\rho v)}
            \end{align*}
        \end{proof}
        \item Calculate how $U$ scales with $M$ for a flexible polymer in a theta solvent. \emph{Hint}: Don't forget that $\zeta(M)$! Also, $v$ is not a function of $M$.
        \begin{proof}
            For an impenetrable sphere (which is how a flexible polymer would behave in a theta solvent), we have from class that $\zeta\propto R_H\propto R_g\propto R\propto N^\nu$. Since $N\propto M$ and $\nu=1/2$ in a theta solvent, it follows that $\zeta\propto M^{1/2}$. Thus, in the context of the above equation, $U\propto M/M^{1/2}$, or
            \begin{equation*}
                \boxed{U \propto M^{1/2}}
            \end{equation*}
        \end{proof}
    \end{enumerate}
    \item (35 pts) This problem will explore viscometry of polymer solutions and data obtained from \textcite{bib:viscometerInt} in their microfluidic rheometer, discussed in Lecture 11.
    \begin{enumerate}
        \item The authors measure an intrinsic viscosity of a polyethylene glycol solution to be \SI[per-mode=symbol]{44.6}{\milli\liter\per\gram}. The MW of the polymer is \SI[per-mode=symbol]{35000}{\gram\per\mole}. Estimate the radius of gyration of the molecule assuming a good solvent. Please use the experimental value of $\gamma$ from the Lecture 10 slides.
        \begin{proof}
            From the Lecture 10 slides, $\gamma=1.5$ for a linear, monodisperse polymer in a good solvent. PEG tends to be linear, hence that selection, and there is no other provided experimental $\gamma$ value for a linear polymer, so I believe I have to assume it's monodisperse?? Thus, based on our theory from class,
            \begin{align*}
                \prb{R_g^2}^{1/2} &= \sqrt[3]{\frac{3M[\eta]}{10\pi N_\text{A}\gamma^3}}\\
                &= \sqrt[3]{\frac{3(\SI{35000}{\gram\per\mole})(\SI{44.6}{\milli\liter\per\gram})}{10\pi(\SI{6.02e23}{\per\mole})(1/1.5)^3}}\\
                \Aboxed{\prb{R_g^2}^{1/2} &\approx \SI{9.4}{\nano\meter}}
            \end{align*}
        \end{proof}
        \item What is your estimate for the overlap concentration $c^*$ for this polymer?
        \begin{proof}
            From class, we have that
            \begin{align*}
                c^* &= \frac{2.5}{[\eta]}\\
                &= \frac{2.5}{\SI{44.6}{\milli\liter\per\gram}}\\
                \Aboxed{c^* &\approx \SI[per-mode=symbol]{5.6e-2}{\gram\per\milli\liter}}
            \end{align*}
        \end{proof}
        \item Assuming the solvent has a viscosity of \SI{1.1}{\pascal\second}, calculate the viscosity of the polymer solution at $c=c^*/2$.
        \begin{proof}
            From class, the approximation of $\eta$ in terms of $[\eta]$ to one term is
            \begin{align*}
                \eta &= \eta_s\left( 1+c\cdot[\eta] \right)\\
                &= (\SI{1.1}{\pascal\second})\left[ 1+\frac{1}{2}(\SI{0.056}{\gram\per\milli\liter})(\SI{44.6}{\milli\liter\per\gram}) \right]\\
                \Aboxed{\eta &= \SI{2.5}{\pascal\second}}
            \end{align*}
        \end{proof}
        \item The authors measured the intrinsic viscosity of the $\text{MW}=\SI[per-mode=symbol]{10000}{\gram\per\mole}$ sample in water to be \SI[per-mode=symbol]{11.9}{\milli\gram\per\milli\liter} and in a water/methanol mixture to be \SI[per-mode=symbol]{22.3}{\milli\gram\per\milli\liter}. What can you conclude about the relative solvent quality of these solvents?
        \begin{proof}
            From class, intrinsic viscosity obeys the following proportionality.
            \begin{equation*}
                [\eta] \propto M^{3\nu-1}
            \end{equation*}
            Thus, if the intrinsic viscosity is higher in the water/methanol mixture, $\nu$ must be higher for this solvent as well. Therefore, \fbox{water/methanol is a better solvent than water.}
        \end{proof}
    \end{enumerate}
\end{enumerate}




\end{document}