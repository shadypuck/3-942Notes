\documentclass[../psets.tex]{subfiles}

\pagestyle{main}
\renewcommand{\leftmark}{Problem Set \thesection}
\setcounter{section}{3}

\begin{document}




\section{Networks}
\noindent Name: Steven Labalme\\
\noindent Email: \href{mailto:labalme@mit.edu}{labalme@mit.edu}\\
\noindent Collaborator(s): Joanne, Isa Pan, Jordan Gray
\begin{enumerate}
    \item \marginnote{10/30:}Rayleigh ratios ($R_\theta$) were obtained at \SI{25}{\celsius} for a series of solutions of a polystyrene sample in benzene, with a detector situated at various angles $\theta$ to the incident beam of unpolarized, monochromatic light with wavelength $\lambda=\SI{546.1}{\nano\meter}$. The results of these measurements appear below.
    \begin{center}
        \small
        \renewcommand{\arraystretch}{1.5}
        \begin{tabular}{l@{\hspace{1.5cm}}S@{\hspace{1cm}}S@{\hspace{1cm}}S@{\hspace{1cm}}S}
            \toprule
            \multirow{2}{*}{\begin{tabular}{@{}l}\\[-7mm]Polystyrene\\[-2mm]concentration\\[-2mm](\si{\gram\per\cubic\deci\meter})\end{tabular}} & \multicolumn{4}{c}{$10^4\times R_\theta/\si{\per\meter}$ measured at $\theta=$}\\ \cline{2-5}
             & \ang{30} & \ang{60} & \ang{90} & \ang{120}\\
            \midrule
            0.50 & 72.3  & 69.4  & 66.2 & 64.3\\[-1.5mm]
            1.00 & 89.8  & 85.7  & 81.3 & 78.2\\[-1.5mm]
            1.50 & 100.8 & 97.1  & 92.0 & 88.1\\[-1.5mm]
            2.00 & 108.7 & 103.8 & 99.7 & 95.9\\
            \bottomrule
        \end{tabular}
    \end{center}
    Under the conditions of the measurements, the Rayleigh ratio and the refractive index of benzene are \SI{46.5e-4}{\per\meter} and 1.502 respectively, and the refractive index increment for the polystyrene solutions is \SI[per-mode=symbol]{1.08e-4}{\cubic\meter\per\kilo\gram}. The density of polystyrene is \SI[per-mode=symbol]{1.05}{\gram\per\cubic\centi\meter}, the density of benzene is \SI[per-mode=symbol]{0.8787}{\gram\per\centi\meter\cubed}, the molecular weight of benzene is \SI[per-mode=symbol]{78.11}{\gram\per\mole}, and the molar volume of benzene can be approximated as its molecular weight divided by its density.\par
    Using a Zimm plot, determine\dots
    \begin{proof}
        I will first construct a Zimm plot for the given data, and then give the requested information based on it.\par
        For the $x$-axis, let's set $k=\SI{0.1}{\cubic\deci\meter\per\gram}$. This will "normalize" the $x$-axis of our Zimm plot to the same roughly 0-1 scale from the in-class example. Thus, to calculate a unitless $x$-value for our Zimm plot from a concentration $c$ and angle $\theta$ in the table, we will use
        \begin{equation*}
            x = \sin^2\left( \frac{\theta}{2} \right)+0.1c
        \end{equation*}
        For example, the $x$-value we should use to plot the upper-left value in the table is
        \begin{equation*}
            x = \sin^2\left( \frac{\ang{30}}{2} \right)+(\SI{0.1}{\deci\meter\cubed\per\gram})(\SI{0.5}{\gram\per\deci\meter\cubed})
            \approx 0.117
        \end{equation*}
        For the $y$-axis, we will need to convert each $R_\theta$ value to a $Kc/\Delta R$ value. Thus, to calculate a $y$-value (in \si{\mole\per\kilo\gram}) for our Zimm plot from a concentration $c$ and Rayleigh ratio $R_\theta$ in the table, we will use
        \begin{align*}
            y &= \frac{Kc}{\Delta R}\\
            &= \frac{2\pi^2n_0^2(\dv*{n}{c})^2c}{N_\text{A}\lambda^4(R_\theta-R_\text{benzene})}\\
            &= \frac{2\pi^2(1.502)^2(\SI{1.08e-4}{\meter\cubed\per\kilo\gram})^2c}{(\SI{6.02e23}{\per\mole})(\SI{5.461e-7}{\meter})^4[(R_\theta-46.5)\cdot\SI{e-4}{\per\meter}]}\\
            &= 0.0970\cdot\frac{c}{R_\theta-46.5}
        \end{align*}
        For example, the $y$-value we should use to plot the upper-left value in the table is
        \begin{equation*}
            y = (\SI{0.0970}{\mole\cubic\meter\per\kilo\gram\squared})\cdot\frac{\SI{0.50}{\kilo\gram\per\cubic\meter}}{72.3-46.5}
            \approx \SI{0.00188}{\mole\per\kilo\gram}
        \end{equation*}
        \par\pagebreak
        Using these equations, we can now calculate ordered pairs $(x,y)$ for each of the 16 data points in the table. Note that for ease of display, the $y$-coordinates will be expressed in units of \si{\mole\per\mega\gram}
        \begin{center}
            \small
            \renewcommand{\arraystretch}{1.2}
            \begin{tabular}{|c|c|c|c|}
                \hline
                $(0.117,1.88)$ & $(0.3,2.12)$  & $(0.55,2.46)$ & $(0.8,2.72)$ \\ \hline
                $(0.167,2.24)$ & $(0.35,2.47)$ & $(0.6,2.79)$  & $(0.85,3.06)$\\ \hline
                $(0.217,2.68)$ & $(0.4,2.88)$  & $(0.65,3.20)$ & $(0.9,3.50)$ \\ \hline
                $(0.267,3.12)$ & $(0.45,3.39)$ & $(0.7,3.65)$  & $(0.95,3.93)$\\
                \hline
            \end{tabular}
        \end{center}
        We may now create our Zimm plot. To do so, we will\dots
        \begin{itemize}
            \item Plot these points (black circles);
            \item Calculate linear regression lines\dots
            \begin{itemize}
                \item For each of the four constant-$\theta$ series (red);
                \item For each of the four constant-$c$ series (blue);
            \end{itemize}
            \item Extrapolate these regression lines\dots
            \begin{itemize}
                \item To $c=\SI{0}{\gram\per\cubic\deci\meter}$ (red squares);
                \item To $\theta=\ang{0}$ (blue squares);
            \end{itemize}
            \item Calculate regression lines for each series of squares (green);
            \item Find these last regression lines' slopes, and their point of intersection (green star).
        \end{itemize}
        The final result appears as follows.
        \begin{center}
            \begin{tikzpicture}[xscale=5,yscale=2]
                \small
                \draw [-latex] (0,1) -- node[below=4.7mm]{$\sin^2(\theta/2)+kc$} ++(1.06,0);
                \draw [-latex] (0,1) -- node[left=6mm]{$\dfrac{Kc}{\Delta R}/10^{-6}\si[per-mode=fraction]{\mole\per\gram}$} ++(0,3.65);

                \footnotesize
                \foreach \x in {0.2,0.4,0.6,0.8,1.0} {
                    \draw (\x,1.05) -- ++(0,-0.1) node[below]{$\x$};
                }
                \foreach \y in {1.5,2.0,2.5,3.0,3.5,4.0,4.5} {
                    \draw (0.02,\y) -- ++(-0.04,0) node[left]{$\y$};
                }

                \node [fill=white,inner sep=5pt] at (-0.001,1.38) {};
                \draw [grx,thick] (-0.001,1.36) node[star,star point ratio=2.5,fill,inner sep=1.2pt]{} -- (1,2.63);
                \draw [grx,thick] (-0.001,1.36) -- (0.227,3.26);

                \draw [blx,thick] (0.05,1.81) node[fill,inner sep=1.5pt]{} -- (1,2.99) node[black,right,yshift=2pt]{0.50};
                \draw [blx,thick] (0.10,2.17) node[fill,inner sep=1.5pt]{} -- (1,3.26) node[black,right,yshift=2pt]{1.00};
                \draw [blx,thick] (0.15,2.59) node[fill,inner sep=1.5pt]{} -- (1,3.62) node[black,right,yshift=2pt]{1.50};
                \draw [blx,thick] (0.20,3.06) node[fill,inner sep=1.5pt]{} -- (1,3.99) node[black,right,yshift=2pt]{2.00};

                \draw [rex,thick] (0.067,1.44) node[fill,inner sep=1.5pt]{} -- (0.296,3.34) node[black,rotate=73,right]{\ang{30}};
                \draw [rex,thick] (0.25,1.66) node[fill,inner sep=1.5pt]{} -- (0.474,3.55) node[black,rotate=73,right]{\ang{60}};
                \draw [rex,thick] (0.5,2.03) node[fill,inner sep=1.5pt]{} -- (0.728,3.84) node[black,rotate=73,right]{\ang{90}};
                \draw [rex,thick] (0.75,2.29) node[fill,inner sep=1.5pt]{} -- (0.977,4.13) node[black,rotate=73,right]{\ang{120}};

                \fill (0.117,1.88) ellipse (0.24pt and 0.6pt);
                \fill (0.167,2.24) ellipse (0.24pt and 0.6pt);
                \fill (0.217,2.68) ellipse (0.24pt and 0.6pt);
                \fill (0.267,3.12) ellipse (0.24pt and 0.6pt);

                \fill (0.3, 2.12) ellipse (0.24pt and 0.6pt);
                \fill (0.35,2.47) ellipse (0.24pt and 0.6pt);
                \fill (0.4, 2.88) ellipse (0.24pt and 0.6pt);
                \fill (0.45,3.39) ellipse (0.24pt and 0.6pt);

                \fill (0.55,2.46) ellipse (0.24pt and 0.6pt);
                \fill (0.6, 2.79) ellipse (0.24pt and 0.6pt);
                \fill (0.65,3.20) ellipse (0.24pt and 0.6pt);
                \fill (0.7, 3.65) ellipse (0.24pt and 0.6pt);

                \fill (0.8, 2.72) ellipse (0.24pt and 0.6pt);
                \fill (0.85,3.06) ellipse (0.24pt and 0.6pt);
                \fill (0.9, 3.50) ellipse (0.24pt and 0.6pt);
                \fill (0.95,3.93) ellipse (0.24pt and 0.6pt);
            \end{tikzpicture}
        \end{center}
    \end{proof}
    \begin{enumerate}
        \item The weight-average molar mass of the polystyrene sample.
        \begin{proof}
            From our theory in class, the green star lies at $(0,1/\Mw)$. The actual green star lies at $(-0.001,\SI{1.36e-6}{\mole\per\gram})$. Thus, setting the two $y$-coordinates equal and solving, we obtain
            \begin{align*}
                \frac{1}{\Mw} &= \SI{1.36e-6}{\mole\per\gram}\\
                \Aboxed{\Mw &= \SI{7.35e5}{\gram\per\mole}}
            \end{align*}
        \end{proof}
        \item The average radius of gyration, $\prb{R_g^2}^{1/2}$ of the polystyrene molecules in benzene at \SI{25}{\celsius}.
        \begin{proof}
            From our theory in class, the slope of the green line at the bottom of the Zimm plot is $16\pi^2n_0^2\prb{R_g^2}/3\lambda^2\Mw$. The actual green line has slope \SI{1.27e-6}{\mole\per\gram}. Thus, setting equal and solving, we obtain
            \begin{align*}
                \frac{16\pi^2(1.502)^2\prb{R_g^2}}{3(\SI{546.1}{\nano\meter})^2(\SI{7.35e5}{\gram\per\mole})} &= \SI{1.27e-6}{\mole\per\gram}\\
                \Aboxed{\prb{R_g^2}^{1/2} &= \SI{48.4}{\nano\meter}}
            \end{align*}
        \end{proof}
        \item The second virial coefficient $A_2$ and $\chi$ for the polystyrene-benzene interaction at \SI{25}{\celsius}.
        \begin{proof}
            From our theory in class, the slope of the green line at the left of the Zimm plot is $2A_2/k$. The actual green line has slope \SI{8.34e-6}{\mole\per\gram}. Thus, setting equal and solving, we obtain
            \begin{align*}
                \frac{2A_2}{\SI{100}{\centi\meter\cubed\per\gram}} &= \SI{8.34e-6}{\mole\per\gram}\\
                \Aboxed{A_2 &= \SI{4.17e-4}{\mole\centi\meter\cubed\per\gram\squared}}
            \end{align*}
            As to $\chi$, we have from the problem statement that $\rho_2=\SI{1.05}{\gram\per\cubic\centi\meter}$, $\rho_1=\SI{0.8787}{\gram\per\cubic\centi\meter}$, and $M_1=\SI{78.11}{\gram\per\mole}$. It follows from the latter two that
            \begin{equation*}
                \bar{V}_1 \approx \hat{V}_1
                = \frac{M_1}{\rho_1}
                = \frac{\SI{78.11}{\gram\per\mole}}{\SI{0.8787}{\gram\per\cubic\centi\meter}}
                = \SI{88.89}{\cubic\centi\meter\per\mole}
            \end{equation*}
            Therefore, by the definition of the second virial coefficient,
            \begin{align*}
                A_2 &= \frac{\frac{1}{2}-\chi}{\bar{V}_1\rho_2^2}\\
                \chi &= \frac{1}{2}-A_2\bar{V}_1\rho_2^2\\
                &= \frac{1}{2}-(\SI{4.17e-4}{\mole\centi\meter\cubed\per\gram\squared})(\SI{88.89}{\cubic\centi\meter\per\mole})(\SI{1.05}{\gram\per\cubic\centi\meter})^2\\
                \Aboxed{\chi &= 0.459}
            \end{align*}
        \end{proof}
    \end{enumerate}
    \item 
    \begin{enumerate}
        \item Find the relation between the stress $\sigma$ and the strain $\lambda$ for a piece of ideal rubber undergoing biaxial extension. Assume the rubber has initial area $A_0$ and thickness $d_0$, and let the final area be $A=\lambda^2A_0$.
        \begin{proof}
            % From a mathematical perspective, biaxial stretching is equivalent to uniaxial compression. Thus, we need only reexpress $\sigma_{xx}(\alpha_x)$ in terms of $\lambda$! To make this easier, view the rubber as being stretched in the $yz$-plane, and therefore compressed in the $x$-direction. It follows from Assumption 8 that
            % \begin{align*}
            %     \alpha_y &= \alpha_z = \lambda&
            %     \alpha_x &= \frac{1}{\alpha_y\alpha_z} = \frac{1}{\lambda^2}
            % \end{align*}
            % Making this substitution, we obtain
            % \begin{align*}
            %     \sigma &= \frac{\rho RT}{M_x}\left( \alpha_x-\frac{1}{\alpha_x^2} \right)\\
            %     \Aboxed{\sigma &= \frac{\rho RT}{M_x}\left( \frac{1}{\lambda^2}-\lambda^4 \right)}
            % \end{align*}

            Let the $xy$-plane be the one in which the rubber is being stretched. Then per Assumption 8, we have
            \begin{align*}
                \alpha_x &= \alpha_y = \lambda&
                \alpha_z &= \frac{1}{\alpha_x\alpha_y} = \frac{1}{\lambda^2}
            \end{align*}
            Thus, the change in entropy upon deforming a single subchain in this manner is
            \begin{equation*}
                \Delta S = -\frac{\kB}{2}\left( \alpha_x^2+\alpha_y^2+\alpha_z^2-3 \right)
                = -\frac{\kB}{2}\left( 2\lambda^2+\frac{1}{\lambda^4}-3 \right)
            \end{equation*}
            Let the full rubber contain $z$ subchains, each identical to the one represented by the above equation and each of which contributes equally to the total entropic spring force. Then the total change in entropy upon stretching the rubber is
            \begin{equation*}
                \Delta S_\text{tot} = -\frac{z\kB}{2}\left( 2\lambda^2+\frac{1}{\lambda^4}-3 \right)
            \end{equation*}
            % Now from a mathematical perspective, the change in energy required to stretch the rubber a certain amount in the $xy$-plane is equivalent to that required to compress it by a proportional amount in the $z$-direction. Since uniaxial derivatives and planar cross sections are easier to conceptualize than the other way around, we will evaluate $F_{z,\text{tot}}$ here. Essentially, the force $F_{z,\text{tot}}$ required to compress the rubber a certain amount is
            Now since we're stretching the rubber the same amount in the $x$- and $y$-directions, the force needed to stretch it should be the same in both directions. Thus, we can evaluate either. WLOG, let's evaluate it in the $x$-direction. Defining $l_x=\sqrt{A}$, we obtain
            \begin{align*}
                F_{x,\text{tot}} &= -T\pdv{\Delta S_\text{tot}}{l_x}\\
                &= -T\pdv{\Delta S_\text{tot}}{\lambda}\cdot\pdv{\lambda}{l_x}\\
                &= \frac{z\kB T}{2}\pdv{\lambda}(2\lambda^2+\frac{1}{\lambda^4}-3)\cdot\pdv{l_x}(\frac{l_x}{l_{x0}})\\
                &= \frac{z\kB T}{2}\left( 4\lambda-\frac{4}{\lambda^5} \right)\cdot\frac{1}{l_{x0}}\\
                &= \frac{2z\kB T}{A_0^{1/2}}\left( \lambda-\frac{1}{\lambda^5} \right)
            \end{align*}
            The (initial) cross-sectional area of this gel in the $yz$-plane will be $A_0^{1/2}d_0$. Additionally, we still have $z/V=\rho N_\text{A}/M_x$, as in class. Therefore, the (engineering) stress is
            \begin{equation*}
                \boxed{\sigma = \frac{2\rho N_\text{A}\kB T}{M_x}\left( \lambda-\frac{1}{\lambda^5} \right)}
            \end{equation*}
        \end{proof}
        \item Use the result from part (a) to calculate the relation between the pressure $p$ of an ideal gas inside a balloon made from an ideal elastomer, expanded to a radius $R=\lambda R_0$, where $R_0$ is the initial radius. Use the following version of the \textbf{Young-Laplace equation} to relate the excess pressure $\Delta p$ (inside the balloon minus outside) to the stress in the rubber, where $d$ is the thickness of the balloon skin.
        \begin{equation*}
            \Delta p = \frac{2d\sigma}{R}
        \end{equation*}
        Empirically, it often seems harder to "get started" blowing up a balloon, than to blow it up further beyond a certain point; explain this observation based on your result for $p$ versus $\lambda$.
        \begin{proof}
            In addition to $R=\lambda R_0$, we can express $d$ in terms of $d_0$ and $\lambda$ using Assumption 8. Running the math, this turns out to be $d=d_0/\lambda^2$. Thus, we have
            \begin{align*}
                p &= \frac{2d}{R}\sigma\\
                &= \frac{2(d_0/\lambda^2)}{\lambda R_0}\cdot\frac{2\rho N_\text{A}\kB T}{M_x}\left( \lambda-\frac{1}{\lambda^5} \right)\\
                \Aboxed{p &= \frac{4\rho N_\text{A}\kB Td_0}{M_xR_0}\left( \frac{1}{\lambda^2}-\frac{1}{\lambda^8} \right)}
            \end{align*}
            Plotting this equation yields a graph that peaks shortly after $\lambda=1$, and then decreases. This peak is the hard point you have to get past in order to get started; past it, you're just adding more volume without working against a more difficult pressure.
        \end{proof}
        \item Finally, suppose you have a balloon made of an ideal elastomer that is inflated to a reasonable size with an ideal gas at room temperature. If the temperature of the balloon plus gas system is then increased to \SI{100}{\celsius}, will the balloon expand, contract, or stay the same size? Justify your answer.
        \begin{proof}
            By the ideal gas law, the pressure $p$ of the ideal gas inside the balloon is equal to $nRT/V$. Part (b) gives an additional expression for the pressure of this gas. Setting these equal to each other, we obtain
            \begin{align*}
                \frac{nRT}{V} &= \frac{4\rho RTd_0}{M_xR_0}\left( \frac{1}{\lambda^2}-\frac{1}{\lambda^8} \right)\\
                \frac{n}{V} &= \frac{4\rho d_0}{M_xR_0}\left( \frac{1}{\lambda^2}-\frac{1}{\lambda^8} \right)
            \end{align*}
            Importantly, the above equation shows that the volume of the balloon is temperature independent! Thus, the balloon will \fbox{stay the same size.}
        \end{proof}
    \end{enumerate}
    \item Consider a cross-linked polyelectrolyte gel in a reservoir of water. Our goal is to understand the swelling behavior of this gel. A polyelectrolyte gel consists of a polymer where some (or all) monomers have ionizable functional groups that may or may not be charged. For each group that is charged, a corresponding counterion of opposite charge must exist to maintain overall charge neutrality. Calculations with charges are complicated, so for this problem we will consider a system composed of solvent, free particles (counterions), and polymers. The number of free particles is regulated by the degree of ionization $f$. Again, for the sake of the problem, ignore any actual charges and only consider the effect of the polyelectrolyte as giving rise to (uncharged) counterions in the system.
    \begin{enumerate}
        \item In the absence of charges ($f=0$), what is the thermodynamic condition that determines the equilibrium swelling ratio of the gel? Describe the qualitative competition between free energy terms in the context of Flory-Rehner theory.
        \begin{proof}
            The chemical potential $\mu_1$ of the water inside the gel needs to be equal to its chemical potential $\mu_1^\circ$ outside the gel in the reservoir of water.\par
            Qualitatively, osmotic pressures try to push water into the gel ($\Delta G_\text{mix}$), while the entropic stretching force of the chains within the gel opposes this mixing ($\Delta G_\text{elastic}$).
        \end{proof}
        \item Now allowing for the possibility of charges ($f>0$), write the Flory-Huggins free energy for this system. Remember that the polymer is a big cross-linked network with an enormous degree of polymerization $N$. Again, we are treating counterions like uncharged particles, so do not explicitly include Coulombic energies or other electrostatic interactions. \emph{Hint}: You can imagine that the counterions act as a gas which exerts a hydrostatic pressure on the "walls" of the gel related to the number of counterions and the volume of the system by the ideal gas law.
        \begin{proof}
            The total free energy for this is
            \begin{align*}
                \Delta G_\text{tot} &= \Delta G_\text{mix}+\Delta G_\text{elastic}+\Delta G_\text{counterions}\\
                \Aboxed{\Delta G_\text{tot} &= \left[ \kB TX_0\left( \chi\phi_1\phi_2+\phi_1\ln\phi_1 \right) \right]+\left[ \frac{z\kB T}{2}\left( 3\alpha_s^2-3-\ln\frac{V}{V_0} \right) \right]-N_\text{counterions}\kB T\ln\frac{V}{V_0}}
            \end{align*}
        \end{proof}
        \item What is the chemical potential of the solvent in part (b) in terms of $\chi$ and $f$?
        \begin{proof}
            Analogously to in class, we have that
            \begin{align*}
                \left( \pdv{\Delta G_\text{counterions}}{n_1} \right)_{T,P,n_j} &= -N_\text{counterions}\kB T\cdot\pdv{\alpha_s}(\ln\frac{\alpha_s^3V_0}{V_0})\cdot\pdv{\alpha_s}{n_1}\\
                &= -fzN\kB T\cdot\frac{3\alpha_s^2}{\alpha_s^3}\cdot\frac{\hat{V}_1}{3\alpha_s^2V_0}\\
                &= -fN\kB T\hat{V}_1\cdot\frac{z}{V_0}\cdot\frac{1}{\alpha_s^3}\\
                &= -fN\kB T\hat{V}_1\cdot\frac{\rho N_\text{A}}{M_x}\cdot\phi_2\\
                &= -\frac{fN\rho RT\hat{V}_1\phi_2}{M_x}
            \end{align*}
            Combining this with our other results from class, we have that
            \begin{equation*}
                \boxed{\frac{\mu_1-\mu_1^\circ}{RT} = \left( \chi-\frac{1}{2} \right)\phi_2^2+\frac{\rho\hat{V}}{M_x}\left( \phi_2^{1/3}-\frac{\phi_2}{2}-fN\phi_2 \right)}
            \end{equation*}
        \end{proof}
        \item Construct a Flory-Rehner type theory to describe the swelling of this system, based on our description of Flory-Rehner theory in class.
        \begin{proof}
            Setting the chemical potential equal to zero at equilibrium swelling --- as suggested in part (a) --- we obtain
            \begin{align*}
                \left( \chi-\frac{1}{2} \right)\phi_2^2 &= -\frac{\rho\hat{V}}{M_x}\left( \phi_2^{1/3}-\frac{\phi_2}{2}-fN\phi_2 \right)\\
                \Aboxed{\left( \chi-\frac{1}{2} \right)\phi_2^{5/3} &= -\frac{\rho\hat{V}}{M_x}+\frac{\rho\hat{V}}{M_x}\left( \frac{1}{2}+fN \right)\phi_2^{2/3}}
            \end{align*}
        \end{proof}
        \item From your expression in part (d), identify which terms can be ignored and why. Show that the final perturbed dimension of the gel scales as
        \begin{equation*}
            L \propto f^{1/2}N
        \end{equation*}
        where $L$ is the swollen size of one side of the (isotropic) gel, and $N$ is the degree of polymerization.
        \begin{proof}
            $|fN|\gg|1/2|$, so I can ignore the $1/2$ term.
            Additionally, we can assume that $\chi\approx 1/2$; in the context of ionic gels, $\chi$ can be a bit bigger or smaller than 1/2, not strictly smaller. Thus, we get
            \begin{align*}
                0 &= -\frac{\rho\hat{V}}{M_x}+\frac{\rho\hat{V}}{M_x}\left( 0+fN \right)\phi_2^{2/3}\\
                1 &= fN\phi_2^{2/3}\\
                &= fN\left( \frac{V_0}{V} \right)^{2/3}\\
                &= fN\left( \frac{zN^{3/2}b^3}{L^3} \right)^{2/3}\\
                % &= fN(z^{1/3}N^{1/2}b/L)^2\\
                L^2 &= z^{2/3}b^2fN^2\\
                L^2 &\propto fN^2\\
                \Aboxed{L &\propto f^{1/2}N}
            \end{align*}
        \end{proof}
        \item We now are interested in using our polyelectrolyte gel as an actuator that can expand and contract in response to an electrical signal, recognizing that applying a voltage to the system will change the degree of dissociation $f$ (Figure \ref{fig:PSet4Q3fa}).
        \begin{figure}[H]
            \centering
            \begin{subfigure}[b]{0.35\linewidth}
                \centering
                \includegraphics[width=0.8\linewidth]{PSet4Q3fa.png}
                \caption{Experimental setup.}
                \label{fig:PSet4Q3fa}
            \end{subfigure}
            \begin{subfigure}[b]{0.35\linewidth}
                \centering
                \includegraphics[width=0.8\linewidth]{PSet4Q3fb.png}
                \caption{Degree of dissociation vs. voltage.}
                \label{fig:PSet4Q3fb}
            \end{subfigure}
            \caption{Polyelectrolyte gel as a piston.}
        \end{figure}
        The change in the number of dissociated ions is shown in Figure \ref{fig:PSet4Q3fb}. On the same graph, sketch the relative height $H/H_0$ of the piston as a function of the applied voltage, and derive a scaling law that relates the number of dissociated ions to the height $H$ of the gel divided by the height $H_0$ of the gel in the absence of voltage. Assume the cross-sectional area of the gel is fixed at a value $A$ by the walls of the apparatus.
        \begin{proof}
            Let the cylinder have base area $A$ and height $H$. The pressure from the gas of counterions is
            \begin{equation*}
                P_\text{gas} = \frac{N_\text{counterions}\kB T}{V}
                = \frac{fnN\kB T}{V}
            \end{equation*}
            where $f$ is the degree of ionization, $n$ is the number of chains, and $N$ is the degree of polymerization of each chain. On the other hand, recall from Lecture 4 that the elastic force $F$ generated by a single polymer chain is
            \begin{equation*}
                F \propto\frac{\kB TH}{Nb^2}
            \end{equation*}
            where $b$ is a Kuhn length. Thus since $V=AH$, the pressure (force/area) for all $n$ chains is
            \begin{equation*}
                P_\text{elastic} = \frac{nF}{A}
                = \frac{n\kB TH}{Nb^2A}
                = \frac{n\kB TH^2}{Nb^2V}
            \end{equation*}
            Therefore,
            \begin{align*}
                P_\text{gas} &= P_\text{elastic}\\
                \frac{fnN\kB T}{V} &= \frac{n\kB TH^2}{Nb^2V}\\
                fN^2 &= \frac{H^2}{b^2}\\
                \Aboxed{H &\propto Nbf^{1/2}}
            \end{align*}
            It follows that --- like in part (e) --- the scaling is approximately the square root function:
            \begin{center}
                \begin{tikzpicture}
                    \small
                    \draw [-latex] (0,0) -- node[below]{Voltage} (3.3,0);
                    \draw [-latex] (0,0) -- ++(0,2.3) node[above]{$f$};
                    \draw [-latex,dashed] (3,0) -- ++(0,2.3) node[above]{$H/H_0$};

                    \footnotesize
                    \draw [dashed] (0,1) node[left]{1} -- (3,1);
                    \node [left] {0};

                    \draw [rex,thick] (0,0) -- (1.5,0) -- (3,1);
                    \draw [blx,thick] (0,1) -- (1.5,1) plot[domain=1.5:3,smooth,samples=100] (\x,{sqrt(\x-1.5)/sqrt(1.5)+1});
                \end{tikzpicture}
            \end{center}
        \end{proof}
    \end{enumerate}
\end{enumerate}




\end{document}