\documentclass[../psets.tex]{subfiles}

\pagestyle{main}
\renewcommand{\leftmark}{Problem Set \thesection}
\setcounter{section}{3}

\begin{document}




\section{Networks}
\noindent Name: Steven Labalme\\
\noindent Email: \href{mailto:labalme@mit.edu}{labalme@mit.edu}\\
\noindent Collaborator(s): \emph{None}
\begin{enumerate}
    \item \marginnote{10/30:}Rayleigh ratios ($R_\theta$) were obtained at \SI{25}{\celsius} for a series of solutions of a polystyrene sample in benzene, with a detector situated at various angles $\theta$ to the incident beam of unpolarized, monochromatic light with wavelength $\lambda=\SI{546.1}{\nano\meter}$. The results of these measurements appear below.
    \begin{center}
        \small
        \renewcommand{\arraystretch}{1.5}
        \begin{tabular}{l@{\hspace{1.5cm}}S@{\hspace{1cm}}S@{\hspace{1cm}}S@{\hspace{1cm}}S}
            \toprule
            \multirow{2}{*}{\begin{tabular}{@{}l}\\[-7mm]Polystyrene\\[-2mm]concentration\\[-2mm](\si{\gram\per\cubic\deci\meter})\end{tabular}} & \multicolumn{4}{c}{$10^4\times R_\theta/\si{\per\meter}$ measured at $\theta=$}\\ \cline{2-5}
             & \ang{30} & \ang{60} & \ang{90} & \ang{120}\\
            \midrule
            0.50 & 72.3  & 69.4  & 66.2 & 64.3\\[-1.5mm]
            1.00 & 89.8  & 85.7  & 81.3 & 78.2\\[-1.5mm]
            1.50 & 100.8 & 97.1  & 92.0 & 88.1\\[-1.5mm]
            2.00 & 108.7 & 103.8 & 99.7 & 95.9\\
            \bottomrule
        \end{tabular}
    \end{center}
    Under the conditions of the measurements, the Rayleigh ratio and the refractive index of benzene are \SI{46.5e-4}{\per\meter} and 1.502 respectively, and the refractive index increment for the polystyrene solutions is \SI[per-mode=symbol]{1.08e-4}{\cubic\meter\per\kilo\gram}. The density of polystyrene is \SI[per-mode=symbol]{1.05}{\gram\per\cubic\centi\meter}, the density of benzene is \SI[per-mode=symbol]{0.8787}{\gram\per\centi\meter\cubed}, the molecular weight of benzene is \SI[per-mode=symbol]{78.11}{\gram\per\mole}, and the molar volume of benzene can be approximated as its molecular weight divided by its density.\par
    Using a Zimm plot, determine\dots
    \begin{enumerate}
        \item The weight-average molar mass of the polystyrene sample.
        \item The average radius of gyration, $\prb{R_g^2}^{1/2}$ of the polystyrene molecules in benzene at \SI{25}{\celsius}.
        \item The second virial coefficient $A_2$ and $\chi$ for the polystyrene-benzene interaction at \SI{25}{\celsius}.
    \end{enumerate}
    \item 
    \begin{enumerate}
        \item Find the relation between the stress $\sigma$ and the strain $\lambda$ for a piece of ideal rubber undergoing biaxial extension. Assume the rubber has initial area $A_0$ and thickness $d_0$, and let the final area be $A=\lambda^2A_0$.
        \item Use the result from part (a) to calculate the relation between the pressure $p$ of an ideal gas inside a balloon made from an ideal elastomer, expanded to a radius $R=\lambda R_0$, where $R_0$ is the initial radius. Use the following version of the \textbf{Young-Laplace equation} to relate the excess pressure $\Delta p$ (inside the balloon minus outside) to the stress in the rubber, where $d$ is the thickness of the balloon skin.
        \begin{equation*}
            \Delta p = \frac{2d\sigma}{R}
        \end{equation*}
        Empirically, it often seems harder to "get started" blowing up a balloon, than to blow it up further beyond a certain point; explain this observation based on your result for $p$ versus $\lambda$.
        \item Finally, suppose you have a balloon made of an ideal elastomer that is inflated to a reasonable size with an ideal gas at room temperature. If the temperature of the balloon plus gas system is then increased to \SI{100}{\celsius}, will the balloon expand, contract, or stay the same size? Justify your answer.
    \end{enumerate}
    \item Consider a cross-linked polyelectrolyte gel in a reservoir of water. Our goal is to understand the swelling behavior of this gel. A polyelectrolyte gel consists of a polymer where some (or all) monomers have ionizable functional groups that may or may not be charged. For each group that is charged, a corresponding counterion of opposite charge must exist to maintain overall charge neutrality. Calculations with charges are complicated, so for this problem we will consider a system composed of solvent, free particles (counterions), and polymers. The number of free particles is regulated by the degree of ionization $f$. Again, for the sake of the problem, ignore any actual charges and only consider the effect of the polyelectrolyte as giving rise to (uncharged) counterions in the system.
    \begin{enumerate}
        \item In the absence of charges ($f=0$), what is the thermodynamic condition that determines the equilibrium swelling ratio of the gel? Describe the qualitative competition between free energy terms in the context of Flory-Rehner theory.
        \item Now allowing for the possibility of charges ($f>0$), write the Flory-Huggins free energy for this system. Remember that the polymer is a big cross-linked network with an enormous degree of polymerization $N$. Again, we are treating counterions like uncharged particles, so do not explicitly include Coulombic energies or other electrostatic interactions. \emph{Hint}: You can imagine that the counterions act as a gas which exerts a hydrostatic pressure on the "walls" of the gel related to the number of counterions and the volume of the system by the ideal gas law.
        \item What is the chemical potential of the solvent in part (b) in terms of $\chi$ and $f$?
        \item Construct a Flory-Rehner type theory to describe the swelling of this system, based on our description of Flory-Rehner theory in class.
        \item From your expression in part (d), identify which terms can be ignored and why. Show that the final perturbed dimension of the gel scales as
        \begin{equation*}
            L \propto f^{1/2}N
        \end{equation*}
        where $L$ is the swollen size of one side of the (isotropic) gel, and $N$ is the degree of polymerization.
        \item We now are interested in using our polyelectrolyte gel as an actuator that can expand and contract in response to an electrical signal, recognizing that applying a voltage to the system will change the degree of dissociation $f$ (Figure \ref{fig:PSet4Q3fa}).
        \begin{figure}[H]
            \centering
            \begin{subfigure}[b]{0.35\linewidth}
                \centering
                \includegraphics[width=0.8\linewidth]{PSet4Q3fa.png}
                \caption{Experimental setup.}
                \label{fig:PSet4Q3fa}
            \end{subfigure}
            \begin{subfigure}[b]{0.35\linewidth}
                \centering
                \includegraphics[width=0.8\linewidth]{PSet4Q3fb.png}
                \caption{Degree of dissociation vs. voltage.}
                \label{fig:PSet4Q3fb}
            \end{subfigure}
            \caption{Polyelectrolyte gel as a piston.}
        \end{figure}
        The change in the number of dissociated ions is shown in Figure \ref{fig:PSet4Q3fb}. On the same graph, sketch the relative height $H/H_0$ of the piston as a function of the applied voltage, and derive a scaling law that relates the number of dissociated ions to the height $H$ of the gel divided by the height $H_0$ of the gel in the absence of voltage. Assume the cross-sectional area of the gel is fixed at a value $A$ by the walls of the apparatus.
    \end{enumerate}
\end{enumerate}




\end{document}