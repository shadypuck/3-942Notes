\documentclass[../psets.tex]{subfiles}

\pagestyle{main}
\renewcommand{\leftmark}{Problem Set \thesection}
\stepcounter{section}

\begin{document}




\section{Solution Thermodynamics}
\noindent Name: Steven Labalme\\
\noindent Email: \href{mailto:labalme@mit.edu}{labalme@mit.edu}\\
\noindent Collaborator(s): \emph{None}\\
\noindent Total grade: {\color{rex}Corrected}
\begin{enumerate}
    \item 
    \begin{enumerate}
        \item \marginnote{9/30:}Calculate a general expression for chemical potential $\mu-\mu_0$ for species 1 from the Flory-Huggins free energy of mixing that we have seen in class.
        \begin{proof}
            The Flory-Huggins expression for the free energy of mixing is
            \begin{equation*}
                \frac{\Delta G_M}{X_0} = \kB T\left( \chi\phi_1\phi_2+\frac{\phi_1}{N_1}\ln\phi_1+\frac{\phi_2}{N_2}\ln\phi_2 \right)
            \end{equation*}
            Using the identities
            \begin{align*}
                \phi_i &= \frac{X_i}{X_0}&
                X_i &= n_iN_\text{A}N_i\tag{$i=1,2$}
            \end{align*}
            we can rewrite the original expression as
            \begin{equation*}
                \Delta G_M = RT(N_1\chi\phi_2n_1+n_1\ln\phi_1+n_2\ln\phi_2)
            \end{equation*}
            Since the chemical potential requires us to take a derivative with respect to the number $n_1$ of moles of species 1, let's also investigate two helpful derivatives.
            \begin{align*}
                \pdv{\phi_1}{n_1} &= \pdv{n_1}(\frac{n_1N_1}{n_1N_1+n_2N_2})&
                    \pdv{\phi_2}{n_1} &= \pdv{n_1}(\frac{n_2N_2}{n_1N_1+n_2N_2})\\
                &= \frac{(n_1N_1+n_2N_2)\cdot N_1-n_1N_1\cdot N_1}{(n_1N_1+n_2N_2)^2}&
                    &= \frac{(n_1N_1+n_2N_2)\cdot 0-n_2N_2\cdot N_1}{(n_1N_1+n_2N_2)^2}\\
                &= \frac{n_2N_2N_1}{(n_1N_1+n_2N_2)^2}&
                    &= -\frac{n_2N_2N_1}{(n_1N_1+n_2N_2)^2}\\
                &= \frac{\phi_2N_1}{n_1N_1+n_2N_2}&
                    &= -\frac{\phi_2N_1}{n_1N_1+n_2N_2}\\
                &= \frac{\phi_2\phi_1}{n_1}&
                    &= -\frac{\phi_2\phi_1}{n_1}
            \end{align*}
            It follows that
            \begin{align*}
                \mu_1-\mu_1^\circ &= \left( \pdv{\Delta G_M}{n_1} \right)_{T,P,n_2}\\
                &= RT\left[ N_1\chi(\phi_2'\cdot n_1+\phi_2\cdot 1)+\left( 1\cdot\ln\phi_1+n_1\cdot\frac{\phi_1'}{\phi_1} \right)+n_2\cdot\frac{\phi_2'}{\phi_2} \right]\\
                &= RT\left[ N_1\chi(\phi_2-\phi_1\phi_2)+(\ln\phi_1+\phi_2)-\frac{\phi_1n_2}{n_1} \right]\\
                &= RT\left[ N_1\chi\phi_2(1-\phi_1)+(\ln\phi_1+\phi_2)-\frac{N_1\phi_2}{N_2} \right]\\
                &= RT\left[ N_1\chi\phi_2^2+\ln\phi_1+\left( 1-\frac{N_1}{N_2} \right)\phi_2 \right]\\
                \Aboxed{\mu_1-\mu_1^\circ &= RT\left[ \ln\phi_1+\left( 1-\frac{N_1}{N_2} \right)\phi_2+N_1\chi\phi_2^2 \right]}
            \end{align*}
            Note that we need $\mu_1^\circ$ as a reference because we are taking the partial derivative of the \emph{change} in free energy, as opposed to the partial derivative of free energy \emph{alone} (as in the definition of the chemical potential).
        \end{proof}
        \item Calculate a general expression for $\chi$ that satisfies the spinodal criterion of a polymer blend from Flory-Huggins theory. Remember that the spinodal satisfies the expression
        \begin{equation*}
            \pdv[2]{\Delta G_M}{\phi_1} = 0
        \end{equation*}
        where $\phi_1$ is the volume fraction of one of the species.
        \begin{proof}
            The equation
            \begin{align*}
                \frac{\Delta G_M}{X_0} &= \kB T\left( \chi\phi_1\phi_2+\frac{\phi_1}{N_1}\ln\phi_1+\frac{\phi_2}{N_2}\ln\phi_2 \right)\\
                \Delta G_M &= X_0\kB T\left[ \chi\phi_1-\chi\phi_1^2+\frac{\phi_1}{N_1}\ln\phi_1+\frac{1-\phi_1}{N_2}\ln(1-\phi_1) \right]
            \end{align*}
            can be viewed as generating a family of $\Delta G_M$ vs. $\phi_1$ curves in 2D space, indexed by the variable $\chi$. To find the set of points in each curve that satisfy the spinodal criterion, we may directly evaluate said criterion and then solve for $\chi$ as a function of $\phi_1$. Let's begin.\par
            We have that
            \begin{align*}
                \pdv{\Delta G_M}{\phi_1} &= X_0\kB T\left\{ \pdv{\phi_1}\left[ \chi\phi_1 \right]-\pdv{\phi_1}\left[ \chi\phi_1^2 \right]+\frac{1}{N_1}\pdv{\phi_1}\left[ \phi_1\ln\phi_1 \right]+\frac{1}{N_2}\pdv{\phi_1}\left[ (1-\phi_1)\ln(1-\phi_1) \right] \right\}\\
                &= X_0\kB T\left\{ \chi-2\chi\phi_1+\frac{1}{N_1}\left[ 1\cdot\ln\phi_1+\phi_1\cdot\frac{1}{\phi_1} \right]+\frac{1}{N_2}\left[ -1\cdot\ln(1-\phi_1)+(1-\phi_1)\cdot\frac{-1}{1-\phi_1} \right] \right\}\\
                &= X_0\kB T\left[ \chi-2\chi\phi_1+\frac{1+\ln\phi_1}{N_1}-\frac{1+\ln(1-\phi_1)}{N_2} \right]\\
                &= \frac{X_0\kB T}{N_1N_2}\left[ \left( N_1N_2\chi+N_2-N_1 \right)-2N_1N_2\chi\phi_1+N_2\ln\phi_1-N_1\ln(1-\phi_1) \right]
            \end{align*}
            It follows that
            \begin{equation*}
                \pdv[2]{\Delta G_M}{\phi_1} = \frac{X_0\kB T}{N_1N_2}\left[ -2N_1N_2\chi+\frac{N_2}{\phi_1}+\frac{N_1}{1-\phi_1} \right]
            \end{equation*}
            Therefore, the desired general expression is
            \begin{align*}
                0 &= -2N_1N_2\chi+\frac{N_2}{\phi_1}+\frac{N_1}{1-\phi_1}\\
                \chi &= \frac{1}{2N_1\phi_1}+\frac{1}{2N_2(1-\phi_1)}\\
                \Aboxed{\chi &= \frac{1}{2}\left( \frac{1}{N_1\phi_1}+\frac{1}{N_2\phi_2} \right)}
            \end{align*}
        \end{proof}
        \pagebreak
        \item From your expression for $\chi$ above, plot the spinodal line as a function of $\phi_1$ for the following cases. Please plot all 4 cases separately, and label which region corresponds to 2 phases (only label the 2 phase/phase-separated region; note that from the spinodal alone you cannot with certainty label a homogeneous/1 phase region). MIT has free student site licenses for Excel, Matlab, and Mathematica that you can use for plotting.
        \begin{enumerate}
            \item $N_1=N_2=1$.
            \begin{proof}
                {\color{white}hi}\\[-0.8em]
                \begin{center}
                    \begin{tikzpicture}[xscale=3,yscale=0.25]
                        \small
                        \draw [-latex] (-0.1,0) -- (1.1,0) node[right]{$\phi_1$};
                        \draw [-latex] (0,-1.2) -- (0,11.2) node[above]{$\chi$};

                        \footnotesize
                        \node [below left] {0};
                        \foreach \x in {0.25,0.5,0.75,1} {
                            \draw (\x,0.4) -- ++(0,-0.8) node[below]{$\x$};
                        }
                        \foreach \y in {2,4,...,10} {
                            \draw (0.033,\y) -- ++(-0.066,0) node[left]{$\y$};
                        }

                        \def\None{1}
                        \def\Ntwo{1}
                        \draw [rex,thick] plot [domain=0.052:0.948,smooth] (\x,{1/(2*\None*\x)+1/(2*\Ntwo*(1-\x))});

                        \normalsize
                        \node at (0.5,4) {2 phases};
                    \end{tikzpicture}
                \end{center}\vspace{-1em}
            \end{proof}
            \item $N_1=1$ and $N_2=1000$.
            \begin{proof}
                {\color{white}hi}\\[-0.8em]
                \begin{center}
                    \begin{tikzpicture}[xscale=3,yscale=0.25]
                        \small
                        \draw [-latex] (-0.1,0) -- (1.1,0) node[right]{$\phi_1$};
                        \draw [-latex] (0,-1.2) -- (0,11.2) node[above]{$\chi$};

                        \footnotesize
                        \node [below left] {0};
                        \foreach \x in {0.25,0.5,0.75,1} {
                            \draw (\x,0.4) -- ++(0,-0.8) node[below]{$\x$};
                        }
                        \foreach \y in {2,4,...,10} {
                            \draw (0.033,\y) -- ++(-0.066,0) node[left]{$\y$};
                        }

                        \def\None{1}
                        \def\Ntwo{1000}
                        \draw [rex,thick,/pgf/fpu,/pgf/fpu/output format=fixed] plot [domain=0.05:0.97,smooth] (\x,{1/(2*\None*\x)+1/(2*\Ntwo*(1-\x))});
                        \draw [rex,thick,/pgf/fpu,/pgf/fpu/output format=fixed] plot [domain=0.97:0.99999,smooth,samples=146] (\x,{1/(2*\None*\x)+1/(2*\Ntwo*(1-\x))});

                        \normalsize
                        \node at (0.5,4) {2 phases};
                    \end{tikzpicture}
                \end{center}\vspace{-1em}
            \end{proof}
            \item $N_1=N_2=1000$.
            \begin{proof}
                {\color{white}hi}\\[-0.8em]
                \begin{center}
                    \begin{tikzpicture}[xscale=3,yscale=0.25]
                        \small
                        \draw [-latex] (-0.1,0) -- (1.1,0) node[right]{$\phi_1$};
                        \draw [-latex] (0,-1.2) -- (0,11.2) node[above]{$\chi$};

                        \footnotesize
                        \node [below left] {0};
                        \foreach \x in {0.25,0.5,0.75,1} {
                            \draw (\x,0.4) -- ++(0,-0.8) node[below]{$\x$};
                        }
                        \foreach \y in {2,4,...,10} {
                            \draw (0.033,\y) -- ++(-0.066,0) node[left]{$\y$};
                        }

                        \def\None{1000}
                        \def\Ntwo{1000}
                        \draw [rex,thick,/pgf/fpu,/pgf/fpu/output format=fixed] plot [domain=0.03:0.98,smooth] (\x,{1/(2*\None*\x)+1/(2*\Ntwo*(1-\x))});
                        \draw [rex,thick,/pgf/fpu,/pgf/fpu/output format=fixed] plot [domain=0.00005:0.03,smooth,samples=146] (\x,{1/(2*\None*\x)+1/(2*\Ntwo*(1-\x))});
                        \draw [rex,thick,/pgf/fpu,/pgf/fpu/output format=fixed] plot [domain=0.97:0.99999,smooth,samples=146] (\x,{1/(2*\None*\x)+1/(2*\Ntwo*(1-\x))});

                        \normalsize
                        \node at (0.5,4) {2 phases};
                    \end{tikzpicture}
                \end{center}\vspace{-1em}
            \end{proof}
            \item $N_1=10$ and $N_2=1000$.
            \begin{proof}
                {\color{white}hi}\\[-0.8em]
                \begin{center}
                    \begin{tikzpicture}[xscale=3,yscale=0.25]
                        \small
                        \draw [-latex] (-0.1,0) -- (1.1,0) node[right]{$\phi_1$};
                        \draw [-latex] (0,-1.2) -- (0,11.2) node[above]{$\chi$};

                        \footnotesize
                        \node [below left] {0};
                        \foreach \x in {0.25,0.5,0.75,1} {
                            \draw (\x,0.4) -- ++(0,-0.8) node[below]{$\x$};
                        }
                        \foreach \y in {2,4,...,10} {
                            \draw (0.033,\y) -- ++(-0.066,0) node[left]{$\y$};
                        }

                        \def\None{10}
                        \def\Ntwo{1000}
                        \draw [rex,thick,/pgf/fpu,/pgf/fpu/output format=fixed] plot [domain=0.005:0.1,smooth] (\x,{1/(2*\None*\x)+1/(2*\Ntwo*(1-\x))});
                        \draw [rex,thick,/pgf/fpu,/pgf/fpu/output format=fixed] plot [domain=0.095:0.97,smooth] (\x,{1/(2*\None*\x)+1/(2*\Ntwo*(1-\x))});
                        \draw [rex,thick,/pgf/fpu,/pgf/fpu/output format=fixed] plot [domain=0.97:0.99999,smooth,samples=146] (\x,{1/(2*\None*\x)+1/(2*\Ntwo*(1-\x))});

                        \normalsize
                        \node at (0.5,4) {2 phases};
                    \end{tikzpicture}
                \end{center}\vspace{-1em}
            \end{proof}
        \end{enumerate}
        \item Please qualitatively explain the differences in the curves for the cases in part (c).
        \begin{proof}
            For case (i), we have a regular solution of two small molecules. Since the components are chemically similar, we obtain a symmetric graph. The critical point falls at $\chi_c=2$, so super high temperatures are not necessary to guarantee mixing.\par
            In case (ii), demixing is much more likely the more species 1 we have. Thus, we have an asymmetric graph. Indeed, full mixing is only possible if we either have very little of the polymeric component 2, or if the temperature is very high. This is because when we have polymers in solution at $c<c^*$, the molecules tend to coil up in localized areas, leading to out-of-equilibrium concentration domains throughout the solution. Unless we add enough thermal energy to access mostly elongated polymers, we will not be able to distribute the concentration of monomers evenly in the solution. Mathematically, this behavior is also consistent with the greater entropy gain for the system when $\phi_1$ is increased from zero as the $\phi_1/N_1=\phi_1/1$ prefactor is larger than the $\phi_2/N_2=\phi_2/1000$ prefactor.\par
            From case (iii), we can see that it is very hard to form a polymer blend at any concentration. Here, mixing is almost entirely enthalpically controlled. This is because there is a very small increase in entropy when we mix polymer chains that are mostly covalently bonded together (i.e., highly ordered). Thus, whether we form one phase or two will really only depend on how much the polymers like each other vs. themselves.\par
            And case (iv) is somewhere in between cases (ii) and (iii). In particular, more $\phi_1$ will still make mixing more difficult, and yet whether or not we get mixing is pretty much just down to enthalpy.
        \end{proof}
        \item From part (b), continue to derive the table for the critical composition and critical interaction parameter shown in class. In other words, find the critical $\chi$ parameter and corresponding volume fraction for the general case, and evaluate it for two low-molecular weight liquids, a solvent-polymer blend, and a symmetric polymer-polymer blend. Show your work.
        \begin{proof}
            Among the family of curves discussed in part (b), the critical interaction parameter $\chi_c$ occurs on the curve that switches from being always concave up (for small or negative $\chi$) to having concave and convex regions (for large or positive $\chi$). While curves with lower $\chi$ than this one have no inflection points and curves with greater $\chi$ than this one have two inflection points, this curve will have just \emph{one} inflection point. But if it has only one inflection point, this means that $\pdv*[2]{\Delta G_M}{\phi_1}$ must have only one root.\par
            When $\pdv*[2]{\Delta G_M}{\phi_1}$ has two roots, the tangent line at each of them has nonzero slope. However, when $\pdv*[2]{\Delta G_M}{\phi_1}$ has only one root, the tangent line has \emph{zero} slope (the original curve is concave up, then just barely neither concave up nor down, then concave up again). Therefore, among all $(\phi_1,\Delta G_M)$ satisfying the spinodal criterion for various $\chi$, $\phi_{1,c}$ satisfies
            \begin{equation*}
                0 = \eval{\pdv{\phi_1}(\pdv[2]{\Delta G_M}{\phi_1})}_{\phi_1=\phi_{1,c}} = \eval{\pdv[3]{\Delta G_M}{\phi_1}}_{\phi_1=\phi_{1,c}}
            \end{equation*}
            Continuing from part (b), we have that
            \begin{equation*}
                \pdv[3]{\Delta G_M}{\phi_1} = \frac{X_0\kB T}{N_1N_2}\left[ -N_2\phi_1^{-2}+N_1(1-\phi_1)^{-2} \right]
            \end{equation*}
            Thus,
            \begin{align*}
                0 &= -N_2\phi_{1,c}^{-2}+N_1(1-\phi_{1,c})^{-2}\\
                N_2(1-\phi_{1,c})^2 &= N_1\phi_{1,c}^2\\
                N_2-2N_2\phi_{1,c}+(N_2-N_1)\phi_{1,c}^2 &= 0\\
                \phi_{1,c} &= \frac{2N_2\pm\sqrt{4N_2^2-4(N_2-N_1)N_2}}{2(N_2-N_1)}\\
                &= \frac{N_2\pm\sqrt{N_1N_2}}{N_2-N_1}
            \end{align*}
            Suppose for the sake of contradiction that $\phi_{1,c}=(N_2+\sqrt{N_1N_2})/(N_2-N_1)$. Then since --- by definition --- $\phi_{1,c}\leq 1$ for all $\{N_1,N_2\in\mathbb{N}\mid N_1\leq N_2\}$, we have that
            \begin{align*}
                \frac{N_2+\sqrt{N_1N_2}}{N_2-N_1} &\leq 1\\
                N_2+\sqrt{N_1N_2} &\leq N_2-N_1\\
                \sqrt{N_1N_2} &\leq -N_1 < 0
            \end{align*}
            But the square root of a natural number cannot be less than zero, a contradiction. Therefore,
            \begin{equation*}
                \phi_{1,c} = \frac{N_2-\sqrt{N_1N_2}}{N_2-N_1}
            \end{equation*}
            Simplifying the above expression yields
            \begin{align*}
                \phi_{1,c} &= \frac{N_2-\sqrt{N_1N_2}}{N_2-N_1}\\
                &= \frac{\sqrt{N_2}(\sqrt{N_2}-\sqrt{N_1})}{(\sqrt{N_2}+\sqrt{N_1})(\sqrt{N_2}-\sqrt{N_1})}\\
                \Aboxed{\phi_{1,c} &= \frac{\sqrt{N_2}}{\sqrt{N_1}+\sqrt{N_2}}}
            \end{align*}
            as desired. Note that an alternative way to derive the above expression is by noting that the critical concentration occurs at the bottom of the spinodal, and thus we could solve $0=\pdv*{\chi}{\phi_1}$.\par
            Given the critical composition, we can plug into the result from part (b) to find the critical interaction parameter $\chi_c$, as follows.
            \begin{align*}
                \chi_c &= \frac{1}{2N_1\phi_{1,c}}+\frac{1}{2N_2(1-\phi_{1,c})}\\
                &= \frac{1}{2}\left[ \frac{\sqrt{N_1}+\sqrt{N_2}}{N_1\sqrt{N_2}}+\frac{1}{N_2\left( 1-\frac{\sqrt{N_2}}{\sqrt{N_1}+\sqrt{N_2}} \right)} \right]\\
                &= \frac{1}{2}\left[ \frac{\sqrt{N_1}+\sqrt{N_2}}{N_1\sqrt{N_2}}+\frac{\sqrt{N_1}+\sqrt{N_2}}{N_2\sqrt{N_1}} \right]\\
                &= \frac{1}{2}\left[ \frac{\sqrt{N_2}(\sqrt{N_1}+\sqrt{N_2})+\sqrt{N_1}(\sqrt{N_1}+\sqrt{N_2})}{N_1N_2} \right]\\
                &= \frac{1}{2}\left[ \frac{(\sqrt{N_1}+\sqrt{N_2})^2}{(\sqrt{N_1N_2})^2} \right]\\
                \Aboxed{\chi_c &= \frac{1}{2}\left( \frac{1}{\sqrt{N_1}}+\frac{1}{\sqrt{N_2}} \right)^2}
            \end{align*}
            To confirm that $(\phi_{1,c},\chi_c)$ satisfies the spinodal criterion, we can see that
            \begin{align*}
                \eval{\pdv[2]{\Delta G_M}{\phi_1}}_{(\phi_{1,c},\chi_c)} &\propto -2N_1N_2\chi_c+\frac{N_2}{\phi_{1,c}}+\frac{N_1}{1-\phi_{1,c}}\\
                &= -N_1N_2\left( \frac{1}{\sqrt{N_1}}+\frac{1}{\sqrt{N_2}} \right)^2+\frac{N_2(\sqrt{N_1}+\sqrt{N_2})}{\sqrt{N_2}}+\frac{N_1(\sqrt{N_1}+\sqrt{N_2})}{\sqrt{N_1}}\\
                &= -\left( \sqrt{N_1}+\sqrt{N_2} \right)^2+\left( \sqrt{N_1}+\sqrt{N_2} \right)^2\\
                &= 0
            \end{align*}
            as expected.\par
            \pagebreak
            We now use the two expressions derived above fill out the table shown in class. Evaluating the case of two low molecular weight liquids (i.e., $N_1=N_2=1$), we obtain
            \begin{align*}
                \phi_{1,c} &= \frac{\sqrt{1}}{\sqrt{1}+\sqrt{1}}&
                    \chi_c &= \frac{1}{2}\left( \frac{1}{\sqrt{1}}+\frac{1}{\sqrt{1}} \right)^2\\
                \Aboxed{\phi_{1,c}(N_1=N_2=1) &= 0.5}&
                    \Aboxed{\chi_c(N_1=N_2=1) &= 2}
            \end{align*}
            Evaluating the case of a solvent-polymer blend (i.e., $N_1=1<N_2$), we obtain
            \begin{align*}
                \phi_{1,c} &= \frac{\sqrt{N_2}}{\sqrt{1}+\sqrt{N_2}}&
                    \chi_c &= \frac{1}{2}\left( \frac{1}{\sqrt{1}}+\frac{1}{\sqrt{N_2}} \right)^2\\
                \Aboxed{\phi_{1,c}(N_1=1<N_2) &= \frac{\sqrt{N_2}}{1+\sqrt{N_2}}}&
                    \Aboxed{\chi_c(N_1=1<N_2) &= \frac{1}{2}\left( 1+\frac{1}{\sqrt{N_2}} \right)^2}
            \end{align*}
            Evaluating the case of a symmetric polymer-polymer blend (i.e., $N_1=N_2=N$), we obtain
            \begin{align*}
                \phi_{1,c} &= \frac{\sqrt{N}}{\sqrt{N}+\sqrt{N}}&
                    \chi_c &= \frac{1}{2}\left( \frac{1}{\sqrt{N}}+\frac{1}{\sqrt{N}} \right)^2\\
                \Aboxed{\phi_{1,c}(N_1=1<N_2) &= 0.5}&
                    \Aboxed{\chi_c(N_1=1<N_2) &= \frac{2}{N}}
            \end{align*}
        \end{proof}
    \end{enumerate}
    \pagebreak
    \item 
    \begin{enumerate}
        \item Estimate the Flory interaction parameter $\chi$ between polystyrene and polybutadiene at room temperature if the solubility parameter of polystyrene is $\delta_\text{PS}=\SI{18.7}{\raiseto{1/2}\mega\pascal}$ and the solubility parameter for polybutadiene is $\delta_\text{PB}=\SI{16.2}{\raiseto{1/2}\mega\pascal}$. For simplicity assume $v_0=\SI{100}{\cubic\angstrom}$.
        \begin{proof}
            We have from class that
            \begin{align*}
                \chi &= \frac{v_0}{\kB T}(\delta_1-\delta_2)^2\\
                &\approx \frac{(\SI{6.02e23}{\per\mole})(\SI{100}{\cubic\angstrom})(\SI{18.7}{\raiseto{1/2}\mega\pascal}-\SI{16.2}{\raiseto{1/2}\mega\pascal})^2}{(\SI{8.31}{\joule\per\mole\per\kelvin})(\SI{298}{\kelvin})}\\
                &= \frac{(\SI{6.02e-5}{\cubic\meter\per\mole})(\SI{6.25e6}{\pascal})}{\SI{2476.38}{\joule\per\mole}}\\
                \Aboxed{\chi &\approx 0.152}
            \end{align*}
        \end{proof}
        \item Show explicitly that Flory's $\chi$ parameter is always positive for nonpolar moleucles whose interaction can be described by the London dispersion potential (van der Waals interaction) given by
        \begin{equation*}
            \varepsilon_{ij} = -\frac{k\alpha_i\alpha_j}{r^6}
        \end{equation*}
        where $\alpha_i$ is the polarizability of molecule $i$, $k$ is a universal constant, and $r$ is the distance between molecules. Explain your assumptions.
        \begin{proof}
            By definition,
            \begin{equation*}
                \chi = \frac{z}{\kB T}\left[ \varepsilon_{12}-\frac{1}{2}(\varepsilon_{11}+\varepsilon_{22}) \right]
            \end{equation*}
            It follows that
            \begin{align*}
                \chi &= \frac{z}{\kB T}\left[ -\frac{k\alpha_1\alpha_2}{r^6}-\frac{1}{2}\left( -\frac{k\alpha_1^2}{r^6}-\frac{k\alpha_2^2}{r^6} \right) \right]\\
                &= \frac{zk}{2\kB Tr^6}\left[ -2\alpha_1\alpha_2+(\alpha_1^2+\alpha_2^2) \right]\\
                &= \frac{zk}{2\kB Tr^6}(\alpha_1-\alpha_2)^2\\
                &> 0
            \end{align*}
            as desired.\par
            I assume that $z,k>0$; unless we are in a gas, $z>0$ is safe to assume, and I will have to take $k>0$ for granted since I have no other information about it. I also assume $\alpha_1\neq\alpha_2$; if they are equal, then the enthalpy of mixing will be zero (no greater forces than the ones already present) and mixing will be purely entropically driven. Lastly, I make all of the assumptions used in the derivation of the Flory $\chi$ parameter, e.g., mean field.
        \end{proof}
    \end{enumerate}
    \pagebreak
    \item 
    \begin{enumerate}
        \item What is the critical value of $\chi$ required for high molecular mass polymers to dissolve in a solvent in all proportions?
        \begin{proof}
            Recall from Q1e that
            \begin{equation*}
                \chi_c = \frac{1}{2}\left( \frac{1}{\sqrt{N_1}}+\frac{1}{\sqrt{N_2}} \right)^2
            \end{equation*}
            We are interested in the specific case of high molecular mass polymers ($N_2\to\infty$) dissolving in a solvent ($N_1=1$). Thus, taking these limits in the above equation, the $\chi$ value below which \emph{any} high MW polymer will dissolve in a solvent must be
            \begin{align*}
                \chi_c &= \frac{1}{2}\left( \frac{1}{\sqrt{1}}+\frac{1}{\sqrt{\infty}} \right)^2\\
                &= \frac{1}{2}(1+0)^2\\
                \Aboxed{\chi_c &= \frac{1}{2}}
            \end{align*} 
        \end{proof}
        \item For a polystyrene-polybutadiene blend, explain under what temperature conditions you expect to find a homogeneously mixed system if the molecular weight of polystyrene is \SI[per-mode=symbol]{e5}{\gram\per\mole} and polybutadiene is \SI[per-mode=symbol]{e4}{\gram\per\mole}. The structures of these two polymers can be found in the literature/online. Describe these temperature conditions at all compositions --- in other words, at what temperature can you guarantee a homogeneously mixed system for any composition? Assume that $\chi$ can be expressed as
        \begin{equation*}
            \chi = \frac{A}{T}
        \end{equation*}
        where $A$ is a constant you need to determine. Note that the solubility parameters for this system are given in Q2.
        \begin{proof}
            From Q2a, we know that a polystyrene-polybutadiene blend at room temperature has $\chi\approx 0.152$. Thus,
            \begin{equation*}
                A = \chi T
                \approx (0.152)(\SI{298}{\kelvin})
                \approx \SI{45.3}{\kelvin}
            \end{equation*}
            The molecular weight of styrene is \SI[per-mode=symbol]{104.15}{\gram\per\mole} and the molecular weight of butadiene is \SI[per-mode=symbol]{54.09}{\gram\per\mole}. Thus, the degree of polymerization of each component is
            \begin{align*}
                N_\text{PS} &= \frac{10^5}{104.15} \approx 1000&
                N_\text{PB} &= \frac{10^4}{54.09} \approx 200
            \end{align*}
            Therefore, the critical temperature $T_c$ at and above which we can guarantee a homogeneously mixed system for any composition is
            \begin{align*}
                T_c &= \frac{A}{\chi_c}\\
                &= \frac{2A}{\left( N_\text{PS}^{-1/2}+N_\text{PB}^{-1/2} \right)^2}\\
                &= \frac{2(\SI{45.3}{\kelvin})}{\left( 1000^{-1/2}+200^{-1/2} \right)^2}\\
                T_c &\approx \SI{9000}{\kelvin}
            \end{align*}
        \end{proof}
        \item Polymer solutions are not well described by the mean-field theory because the connectivity of the chain keeps monomers from being uniformly distributed in solution (particularly at low polymer concentrations). An empirical form that better relates $\chi$ to the Hildebrand solubility parameter in polymer solutions is widely used with an entropic part of $\chi$ of 0.34, as follows.
        \begin{equation*}
            \chi = 0.34+\frac{v_0}{\kB T}(\delta_\text{A}-\delta_\text{B})^2
        \end{equation*}
        Using this formula and the following table, decide which solvents will dissolve poly(dimethyl siloxane), for which $\delta_\text{PDMS}=\SI{14.9}{\raiseto{1/2}\mega\pascal}$ and which will dissolve polystyrene ($\delta_\text{PS}=\SI{18.7}{\raiseto{1/2}\mega\pascal}$) at room temperature.
        \begin{center}
            \small
            \renewcommand{\arraystretch}{1.2}
            \begin{tabular}{c|S|S|S|S|S}
                \toprule
                \textbf{Solvent} & {\emph{n}-Heptane} & {Cyclohexane} & {Benzene} & {Chloroform} & {Acetone}\\
                \midrule
                \textbf{Molar volume ($\textbf{cm}^{\bm{3}}\bm{/}\textbf{mol}$)} & 195.9 & 108.5 & 29.4 & 80.7 & 74.0\\
                \textbf{Solubility parameter ($\textbf{MPa}^{\bm{1/2}}$)} & 15.1 & 16.8 & 18.6 & 19.0 & 20.3\\
                \bottomrule
            \end{tabular}
        \end{center}
        \begin{proof}
            From Q3a, $\chi_c\to 1/2^+$ for a solvent-polymer blend as $N_2\to\infty$. Thus, if our calculated values of $\chi\leq 1/2$, we can be confident that mixing will occur in all proportions. Additionally, note that the units in the above table are such that if we plug the numbers in directly, everything will cancel. Let's begin.\par
            For a mixture of PDMS in \emph{n}-heptane at room temperature, we have that
            \begin{align*}
                \chi &= 0.34+\frac{V_m(\ce{HpH})}{RT}(\delta_\text{PDMS}-\delta_{\ce{HpH}})^2\\
                &= 0.34+\frac{\SI{195.9}{\centi\meter\cubed\per\mole}}{\SI{2476.38}{\joule\per\mole}}(\SI{14.9}{\raiseto{1/2}\mega\pascal}-\SI{15.1}{\raiseto{1/2}\mega\pascal})^2\\
                \chi &\approx 0.34 \leq \frac{1}{2}
            \end{align*}
            and hence \fbox{PDMS dissolves in \emph{n}-heptane.}\par
            For a mixture of PS in \emph{n}-heptane at room temperature, we have that
            \begin{align*}
                \chi &= 0.34+\frac{V_m(\ce{HpH})}{RT}(\delta_\text{PS}-\delta_{\ce{HpH}})^2\\
                &= 0.34+\frac{\SI{195.9}{\centi\meter\cubed\per\mole}}{\SI{2476.38}{\joule\per\mole}}(\SI{18.7}{\raiseto{1/2}\mega\pascal}-\SI{15.1}{\raiseto{1/2}\mega\pascal})^2\\
                \chi &\approx 1.4 > \frac{1}{2}
            \end{align*}
            and hence \fbox{PS does not dissolve in \emph{n}-heptane.}\par
            For a mixture of PDMS in cyclohexane at room temperature, we have that
            \begin{align*}
                \chi &= 0.34+\frac{V_m(\ce{CyH})}{RT}(\delta_\text{PDMS}-\delta_{\ce{CyH}})^2\\
                &= 0.34+\frac{\SI{108.5}{\centi\meter\cubed\per\mole}}{\SI{2476.38}{\joule\per\mole}}(\SI{14.9}{\raiseto{1/2}\mega\pascal}-\SI{16.8}{\raiseto{1/2}\mega\pascal})^2\\
                \chi &\approx 0.50 \approx \frac{1}{2}
            \end{align*}
            and hence \fbox{PDMS may dissolve in cyclohexane (within error).}\par
            \pagebreak
            For a mixture of PS in cyclohexane at room temperature, we have that
            \begin{align*}
                \chi &= 0.34+\frac{V_m(\ce{CyH})}{RT}(\delta_\text{PS}-\delta_{\ce{CyH}})^2\\
                &= 0.34+\frac{\SI{108.5}{\centi\meter\cubed\per\mole}}{\SI{2476.38}{\joule\per\mole}}(\SI{18.7}{\raiseto{1/2}\mega\pascal}-\SI{16.8}{\raiseto{1/2}\mega\pascal})^2\\
                \chi &\approx 0.50 \approx \frac{1}{2}
            \end{align*}
            and hence \fbox{PS may dissolve in cyclohexane (within error).}\par
            For a mixture of PDMS in benzene at room temperature, we have that
            \begin{align*}
                \chi &= 0.34+\frac{V_m(\ce{PhH})}{RT}(\delta_\text{PDMS}-\delta_{\ce{PhH}})^2\\
                &= 0.34+\frac{\SI{29.4}{\centi\meter\cubed\per\mole}}{\SI{2476.38}{\joule\per\mole}}(\SI{14.9}{\raiseto{1/2}\mega\pascal}-\SI{18.6}{\raiseto{1/2}\mega\pascal})^2\\
                \chi &\approx 0.50 \approx \frac{1}{2}
            \end{align*}
            and hence \fbox{PDMS may dissolve in benzene (within error).}\par
            For a mixture of PS in benzene at room temperature, we have that
            \begin{align*}
                \chi &= 0.34+\frac{V_m(\ce{PhH})}{RT}(\delta_\text{PS}-\delta_{\ce{PhH}})^2\\
                &= 0.34+\frac{\SI{29.4}{\centi\meter\cubed\per\mole}}{\SI{2476.38}{\joule\per\mole}}(\SI{18.7}{\raiseto{1/2}\mega\pascal}-\SI{18.6}{\raiseto{1/2}\mega\pascal})^2\\
                \chi &\approx 0.34 \leq \frac{1}{2}
            \end{align*}
            and hence \fbox{PS dissolves in benzene.}\par
            For a mixture of PDMS in chloroform at room temperature, we have that
            \begin{align*}
                \chi &= 0.34+\frac{V_m(\ce{CHCl3})}{RT}(\delta_\text{PDMS}-\delta_{\ce{CHCl3}})^2\\
                &= 0.34+\frac{\SI{80.7}{\centi\meter\cubed\per\mole}}{\SI{2476.38}{\joule\per\mole}}(\SI{14.9}{\raiseto{1/2}\mega\pascal}-\SI{19.0}{\raiseto{1/2}\mega\pascal})^2\\
                \chi &\approx 0.89 > \frac{1}{2}
            \end{align*}
            and hence \fbox{PDMS does not dissolve in chloroform.}\par
            For a mixture of PS in chloroform at room temperature, we have that
            \begin{align*}
                \chi &= 0.34+\frac{V_m(\ce{CHCl3})}{RT}(\delta_\text{PS}-\delta_{\ce{CHCl3}})^2\\
                &= 0.34+\frac{\SI{80.7}{\centi\meter\cubed\per\mole}}{\SI{2476.38}{\joule\per\mole}}(\SI{18.7}{\raiseto{1/2}\mega\pascal}-\SI{19.0}{\raiseto{1/2}\mega\pascal})^2\\
                \chi &\approx 0.34 \leq \frac{1}{2}
            \end{align*}
            and hence \fbox{PS dissolves in chloroform.}\par
            \pagebreak
            For a mixture of PDMS in acetone at room temperature, we have that
            \begin{align*}
                \chi &= 0.34+\frac{V_m(\text{Ace})}{RT}(\delta_\text{PDMS}-\delta_\text{Ace})^2\\
                &= 0.34+\frac{\SI{74.0}{\centi\meter\cubed\per\mole}}{\SI{2476.38}{\joule\per\mole}}(\SI{14.9}{\raiseto{1/2}\mega\pascal}-\SI{20.3}{\raiseto{1/2}\mega\pascal})^2\\
                \chi &\approx 1.2 > \frac{1}{2}
            \end{align*}
            and hence \fbox{PDMS does not dissolve in acetone.}\par
            For a mixture of PS in acetone at room temperature, we have that
            \begin{align*}
                \chi &= 0.34+\frac{V_m(\text{Ace})}{RT}(\delta_\text{PS}-\delta_\text{Ace})^2\\
                &= 0.34+\frac{\SI{74.0}{\centi\meter\cubed\per\mole}}{\SI{2476.38}{\joule\per\mole}}(\SI{18.7}{\raiseto{1/2}\mega\pascal}-\SI{20.3}{\raiseto{1/2}\mega\pascal})^2\\
                \chi &\approx 0.42 \leq \frac{1}{2}
            \end{align*}
            and hence \fbox{PS dissolves in acetone.}
        \end{proof}
    \end{enumerate}
    \item The area per chain $\Sigma$ of polymers attached to a surface or interface is an important quantity that influences the thickness of the polymer layer in any resulting morphologies. In general, when the area per chain of the polymer attached to an interface is small enough to induce elongation of the polymer away from that interface, we call this stretched morphology a \textbf{polymer brush}. Microphase separated diblock copolymers can be thought of as similar to polymer brushes, where each block is "attached" to the IMDS and elongates away from it.
    \begin{enumerate}
        \item For a microphase-separated compositionally symmetric diblock copolymer, qualitatively explain the free energy contributions that determine the optimal elongation of polymer chains away from the IMDS.
        \begin{proof}
            For such a system, there will be an enthalpic and entropic contribution to the free energy functional. The entropic term deals with the fact that to maximize entropy, the homogeneous strands would like to coil up on both sides of the interface (entropy is minimized when the strands are perfectly straight). The enthalpic term deals with the fact that there is less interaction energy between the two phases when everything is straight because only a few unlike monomers are near each other. Thus, these forces compete with each other and seek a happy medium.
        \end{proof}
        \pagebreak
        \item Consider a linear ABC terpolymer and a 3 arm ABC star terpolymer, each consisting of 33.3\% of the respective A, B and C blocks. Where do you think the center of the star polymer will be pinned?\par
        \begin{proof}
            I think the morphology of choice will be parallel cylinders in an almost hexagonal grid, with each junction in the grid being the center of a star polymer. This would allow each cylinder to be composed of the like strands of three different molecules.
        \end{proof}
        Carefully draw a to-scale representation of the expected morphology in the melt state for each type of pure macromolecule for\dots
        \begin{enumerate}
            \item A system where all the $\chi$ parameters are positive and equal and the degree of polymerization of each arm is sufficiently large so that microphase separation occurs;
            \begin{proof}
                A positive $\chi$ parameter likely corresponds to demixing and microphase separation, as the problem statement indicates.\par
                For the ABC triblock copolymer, no particular component has a preference for any other component, so the system will relax to the configurations that mimimize deformation of the chain. This will orient chains roughly parallel to each other and form lamellae.
                \begin{center}
                    \begin{tikzpicture}[xscale=0.6]
                        \fill [blx] (0,0) rectangle ++(1,3);
                        \fill [rex] (1,0) rectangle ++(1,3);
                        \fill [grx] (2,0) rectangle ++(1,3);
                        \fill [rex] (3,0) rectangle ++(1,3);
                        \fill [blx] (4,0) rectangle ++(1,3);
                    \end{tikzpicture}
                \end{center}
                For the star polymer, as mentioned, we will form a kind of hexagonal lattice.
                \begin{center}
                    \begin{tikzpicture}
                        \fill [blx] (0,0) -- ++(30:0.4) -- ++(90:0.4) -- ++(150:0.4) -- ++(210:0.4) -- ++(270:0.4) -- cycle;
                        \fill [rex] (-60:{0.4*3^0.5}) -- ++(30:0.4) -- ++(90:0.4) -- ++(150:0.4) -- ++(210:0.4) -- ++(270:0.4) -- cycle;
                        \fill [grx] (0:{0.4*3^0.5}) -- ++(30:0.4) -- ++(90:0.4) -- ++(150:0.4) -- ++(210:0.4) -- ++(270:0.4) -- cycle;
                        \fill [rex] (60:{0.4*3^0.5}) -- ++(30:0.4) -- ++(90:0.4) -- ++(150:0.4) -- ++(210:0.4) -- ++(270:0.4) -- cycle;
                        \fill [grx] (120:{0.4*3^0.5}) -- ++(30:0.4) -- ++(90:0.4) -- ++(150:0.4) -- ++(210:0.4) -- ++(270:0.4) -- cycle;
                        \fill [rex] (180:{0.4*3^0.5}) -- ++(30:0.4) -- ++(90:0.4) -- ++(150:0.4) -- ++(210:0.4) -- ++(270:0.4) -- cycle;
                        \fill [grx] (240:{0.4*3^0.5}) -- ++(30:0.4) -- ++(90:0.4) -- ++(150:0.4) -- ++(210:0.4) -- ++(270:0.4) -- cycle;
                    \end{tikzpicture}
                \end{center}
            \end{proof}
            \item A system where $\chi_\text{AB}=\chi_\text{BC}>0$ while $\chi_\text{AC}$ is negative.
            \begin{proof}
                What we essentially have here is a biphasic system, where the A and C components will form one phase and component B will form a different phase. Thus, from the perspective of the B chain, the A/C mixed phase will be at one end regardless of architecture. (Indeed, the negative $\chi_\text{AC}$ parametter will lead the system to pay the deformation energy in order ot maximize A-C contacts.) Thus, for both architectures, the expected morphology for a 33\% distribution is cylinders, as follows. Note that the A/C blue/green mixed phase has been colored cyan.
                \begin{center}
                    \begin{tikzpicture}
                        \fill [cyan] (0,0) rectangle (3,3);
                        \fill [rex] (0,1) circle (3mm);
                        \fill [rex] (0,2) circle (3mm);
                        \fill [rex] (1,1) circle (3mm);
                        \fill [rex] (1,2) circle (3mm);
                        \fill [rex] (2,1) circle (3mm);
                        \fill [rex] (2,2) circle (3mm);
                        \fill [rex] (3,1) circle (3mm);
                        \fill [rex] (3,2) circle (3mm);
                    \end{tikzpicture}
                \end{center}
            \end{proof}
        \end{enumerate}
    \end{enumerate}
    \pagebreak
    \item A common example of a polymer brush in real systems consists of homopolymers grafted to a solid surface. Think of this as hairs with a sticky end that can bind to a scalp. If each chain grafted to the surface gains in enthalpy $-\varepsilon$ by sticking to the surface\dots
    \begin{enumerate}
        \item Describe what grafting density is necessary for a grafted polymer system to be considered a brush (i.e., when chains overlap and stretch away from surface due to excluded volume effects). Qualitatively, under what condition would brush formation be thermodynamically favorable? \emph{Hint}: Think about overlap of polymer chains.
        \begin{proof}
            A grafted polymer system becomes a \emph{brush} when the density of chains on the surface surpasses the density within a lone coil. Since the area of a coil goes as $R_g^2$, the critical grafting density will be when there is 1 chain per every $R_g^2$ units of area on the surface, or when
            \begin{equation*}
                \boxed{\rho = \frac{1}{R_g^2}}
            \end{equation*}
            Brush formation would be thermodynamically favorable under the condition that the decrease in energy caused by the sticking enthalpy is greater than increase in entropy caused by not being constrained to a rigid surface.
        \end{proof}
        \item Find the optimal grafting density in chains/area for a polymer brush system in terms of $\varepsilon$. Define any variables you use and remember the effect of steric (excluded volume) interactions.
        \begin{proof}
            To answer this question, we will mimick the full analysis of the free energy change for diblock copolymers carried out in class. Let's begin.\par
            Let State 1 denote $n$ pure polymer chains in a melt, and let State 2 denote those $n$ polymer chains bound to a surface of total area $A$. Recall from Q5a that for State 2 to be a brush, the density $\rho=n/A$ will have to be large enough for chains to overlap and stretch away from the surface. Note that even if there are more than $n$ chains in the melt, the change in free energy is only dependent on the change in state of the $n$ chains that make the transition to State 2, so we can well consider only these $n$ chains to be our system under study. The change $\Delta G$ in free energy \emph{per polymer chain} upon transitioning from State 1 to State 2 is
            \begin{equation*}
                \Delta G = \Delta H-T(S_2-S_1)
            \end{equation*}
            As in class, $S=-3\kB R^2/2Na^2$, where $R$ is the end-to-end distance, $N$ is the number of segments in the polymer chain, and $a$ is the step length. Additionally, from the problem statement, $\Delta H=-\varepsilon$. We can think of this change in energy as analogous to the interfacial energy $\gamma_\text{AB}\Sigma$, and observe that there is no mixing energy because we only have one component in both states (it's just this component free, vs. this component bound to a surface). One additional complication is the excluded volume effects that occur when polymers become attached to the surface. From class, this adds an additional $\kB TvN^2/2HD^2$ term to the change in enthalpy, where $v$ is the excluded volume, $H$ is the average height to which the polymers extend away from the surface in State 2, and $D$ is the average distance between polymer chains on the surface. Thus, we have
            \begin{align*}
                \Delta G &= -\varepsilon+\frac{\kB TvN^2}{2HD^2}-T\left[ -\frac{3\kB H^2}{2Na^2}+\frac{3\kB(N^{1/2}a)^2}{2Na^2} \right]\\
                &= -\varepsilon+\frac{\kB T}{2}\left( \frac{vN}{a^3}+\frac{3H^2}{Na^2}-3 \right)
            \end{align*}
            Now the decrease in free energy for any individual chain will obviously be greatest when the chain only benefits from the sticking enthalpy and doesn't have to stretch away from it's entropically ideal conformation. However, this may not be best for the system, since this would limit the number of chains that can gain the sticking enthalpy \emph{for the system}. Thus, before optimizing, we need to switch to evaluating the free energy change of the whole system. It follows that the total change in free energy of the system (i.e., once all $n$ chains have bonded to the surface) is
            \begin{equation*}
                \Delta G_\text{sys} = -n\varepsilon+\frac{n\kB T}{2}\left( \frac{vN}{a^3}+\frac{3H^2}{Na^2}-3 \right)
            \end{equation*}
            At this point, we have expressed the free energy change in terms of two variables: The number of chains $n$, and the height $H$. However, these variables are not independent of one another, since they are related through $n/A=\rho=H/Na^3$. Thus, we could express the above equation in terms of either $n$ or $H$ and optimize over that\dots or we could express the above equation in terms of the variable we're actually interested in (the optimal grafting density $\rho$), since it also happens to be related to both $n$ and $H$! Thus, let's change variables with the substitutions $H=\rho Na^3$ and $n=\rho A$.
            \begin{align*}
                \Delta G_\text{sys} &= -\rho A\varepsilon+\frac{\rho A\kB T}{2}\left[ \frac{vN}{a^3}+\frac{3(\rho Na^3)^2}{Na^2}-3 \right]\\
                &= -\rho A\varepsilon+\frac{A\kB T}{2}\left( \frac{vN\rho}{a^3}+3Na^4\rho^3-3\rho \right)
            \end{align*}
            It follows that the optimal grafting density is
            \begin{align*}
                0 &= \pdv{\Delta G}{\rho}\\
                &= -A\varepsilon+\frac{A\kB T}{2}\left( \frac{vN}{a^3}+9Na^4\rho^2-3 \right)\\
                \frac{9}{2}\kB TNa^4\rho^2 &= \varepsilon-\frac{\kB TvN}{2a^3}+\frac{3\kB T}{2}\\
                \kB TNa^4\rho^2 &\approx \varepsilon-\frac{\kB TvN}{a^3}\\
                \Aboxed{\rho &\approx \left( \frac{\varepsilon}{\kB TNa^4}-\frac{v}{a^7} \right)^{1/2}}
            \end{align*}
            where from the third to the fourth line, we have made the approximation that we can ignore factors on the order of unity.
        \end{proof}
        \item What is the scaling of the brush height with degree of polymerization in the dense brush regime from part (b)?
        \begin{proof}
            Substituting into the above, we have
            \begin{align*}
                \frac{H}{Na^3} &\approx \left( \frac{\varepsilon}{\kB TNa^4}-\frac{v}{a^7} \right)^{1/2}\\
                H &\approx \left( \frac{Na^2\varepsilon}{\kB T}-\frac{vN^2}{a} \right)^{1/2}
            \end{align*}
            Thus, \fbox{$H\propto N^\nu$ for $\nu\in(0.5,1)$.}
        \end{proof}
    \end{enumerate}
\end{enumerate}




\end{document}