\documentclass[../psets.tex]{subfiles}

\pagestyle{main}
\renewcommand{\leftmark}{Problem Set \thesection}

\begin{document}




\section{The Macromolecule}
\noindent Name: Steven Labalme\\
\noindent Email: \href{mailto:labalme@mit.edu}{labalme@mit.edu}\\
\noindent Collaborators: \emph{None}\\
\begin{enumerate}
    \item \marginnote{9/18:}(25 pts) Consider the following distribution of polymer chains with repeat unit molecular weight $M_0$:
    \begin{center}
        10 chains with degree of polymerization 100;\\
        100 chains with degree of polymerization \num{1000};\\
        10 chains with degree of polymerization \num{10000}.
    \end{center}
    \begin{enumerate}
        \item Calculate the number average molecular weight of this distribution.
        \item Calculate the weight average molecular weight of this distribution.
        \item What is the polydispersity index of this distribution?
        \item If you randomly chose a monomer in the solution, what is the chance that is belongs to a chain with degree of polymerization \num{10000}?
    \end{enumerate}
    \item (25 pts) The root mean squared end-to-end distance $\prb{R^2}^{1/2}$ of a poly(methyl methacrylate) (PMMA) molecule ($M_W=\SI[per-mode=symbol]{e7}{\gram\per\mole}$) in tetrahydrofuran (THF) at its Theta temperature was found to be \SI{200}{\nano\meter}.
    \begin{enumerate}
        \item What is the degree of polymerization for this polymer?
        \item Estimate $C_\infty$ for this polymer.
        \item Estimate the Kuhn length and number of Kuhn segments for this polymer.
        \item Estimate the persistence length for this polymer.
        \item Is this PMMA a "flexible," "semi-flexible," or "rod-like" polymer?
    \end{enumerate}
    \item (25 pts) Consider a linear copolymer with $N_A$ steps of Kuhn length $\ell_A$ and $N_B$ steps of Kuhn length $\ell_B$. The solvent is such that excluded volume effects are negligible (theta solvent).
    \begin{enumerate}
        \item Calculate the mean squared end-to-end distance $\prb{R^2}$ of the chain for a linear diblock architecture (all A monomers connected to all B monomers).
        \item Will your answer in (a) change if the chain has a random distribution of monomers A and B? Explain why or why not.
    \end{enumerate}
    \item (25 pts) Consider a linear polymer with $N$ Kuhn steps of length $\ell_k$ restricted to a two-dimensional interface (e.g. air/water interface). The Flory free energy terms for chain extension and excluded volume are
    \begin{equation*}
        \frac{F}{\kB T} = \frac{1}{2}\frac{aN^2}{R^2}+\frac{R^2}{N\ell_k^2}
    \end{equation*}
    where $a$ is the effective excluded area per monomer (analogue to excluded volume in 3D). The first term on the right-hand side is related to excluded volume and the second term related to stretching.
    \begin{enumerate}
        \item Calculate the scaling for $R$.
    \end{enumerate}
    \item (25 pts) Consider monomers which interact via the following pairwise interaction potential, which is called a square well interaction:
    \begin{align*}
        U &= \infty\tag{$r<b$}\\
        U &= -\varepsilon\tag{$b\leq r\leq\lambda b$}\\
        U &= 0\tag{$r>b$}
    \end{align*}
    \begin{enumerate}
        \item Calculate the excluded volume $B$.
        \item Simplify your expression for $B$ by assuming $\varepsilon/\kB T\ll 1$. \emph{Hint}: Expand the exponential.
        \item Using your answer in (b), determine the Theta temperature for this system.
    \end{enumerate}
\end{enumerate}




\end{document}