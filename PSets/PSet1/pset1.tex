\documentclass[../psets.tex]{subfiles}

\pagestyle{main}
\renewcommand{\leftmark}{Problem Set \thesection}

\begin{document}




\section{The Macromolecule}
\noindent Name: Steven Labalme\\
\noindent Email: \href{mailto:labalme@mit.edu}{labalme@mit.edu}\\
\noindent Collaborator(s): Isa Pan, Jordan Gray\\
\begin{enumerate}
    \item \marginnote{9/18:}(25 pts) Consider the following distribution of polymer chains with repeat unit molecular weight $M_0$:
    \begin{center}
        10 chains with degree of polymerization 100;\\
        100 chains with degree of polymerization \num{1000};\\
        10 chains with degree of polymerization \num{10000}.
    \end{center}
    \begin{enumerate}
        \item Calculate the number average molecular weight of this distribution.
        \begin{proof}
            \begin{align*}
                \Mn &= M_0\cdot\frac{\sum_iin_i}{\sum_in_i}\\
                &= M_0\cdot\frac{100\cdot10+\num{1000}\cdot 100+\num{10000}\cdot 10}{10+100+10}\\
                \Aboxed{\Mn &= 1675M_0}
            \end{align*}
        \end{proof}
        \item Calculate the weight average molecular weight of this distribution.
        \begin{proof}
            \begin{align*}
                \Mw &= M_0\cdot\frac{\sum_ii^2n_i}{\sum_iin_i}\\
                &= M_0\cdot\frac{100^2\cdot 10+1000^2\cdot 100+\num{10000}^2\cdot 10}{100\cdot10+\num{1000}\cdot 100+\num{10000}\cdot 10}\\
                \Aboxed{\Mw &\approx 5473M_0}
            \end{align*}
        \end{proof}
        \item What is the polydispersity index of this distribution?
        \begin{proof}
            \begin{align*}
                \Dstroke &= \frac{\Mw}{\Mn}\\
                &\approx \frac{5473M_0}{1675M_0}\\
                \Aboxed{\Dstroke &\approx 3.27}
            \end{align*}
        \end{proof}
        \item If you randomly chose a monomer in the solution, what is the chance that is belongs to a chain with degree of polymerization \num{10000}?
        \begin{proof}
            This probability would be equal to the sum of the number of monomers in chains with $N=\num{10000}$ divided by the total number of monomers in solution. But this is just the weight fraction of a \num{10000}-mer! Symbolically, we have
            \begin{equation*}
                \frac{\num{10000}\cdot n_{\num{10000}}}{\sum_iin_i} = w_{\num{10000}} = \frac{\num{10000}\cdot 10}{100\cdot10+\num{1000}\cdot 100+\num{10000}\cdot 10} = \boxed{49.8\%}
            \end{equation*}
        \end{proof}
    \end{enumerate}
    \item (25 pts) The root mean squared end-to-end distance $\prb{R^2}^{1/2}$ of a poly(methyl methacrylate) (PMMA) molecule ($\text{MW}=\SI[per-mode=symbol]{e7}{\gram\per\mole}$) in tetrahydrofuran (THF) at its Theta temperature was found to be \SI{200}{\nano\meter}.
    \begin{enumerate}
        \item What is the degree of polymerization for this polymer?
        \begin{proof}
            The molar mass of a methyl methacrylate (MMA) monomer is $M=\SI[per-mode=symbol]{100.12}{\gram\per\mole}$. Thus,
            \begin{align*}
                N &= \frac{\text{MW}}{M}\\
                \Aboxed{N &\approx 99880}
            \end{align*}
        \end{proof}
        \item Estimate $C_\infty$ for this polymer.
        \begin{proof}
            Since each MMA monomer introduces 2 \ce{C-C} bonds to the polymer chain, the number $n$ of \ce{C-C} bonds in the polymer is
            \begin{equation*}
                n = 2N \approx 199760
            \end{equation*}
            Additionally, the length $l$ of a typical \ce{C-C} bond is \SI{0.154}{\nano\meter}. Thus, from the symmetric hindered rotations model, the characteristic ratio $C_\infty$ is
            \begin{align*}
                C_\infty &= \frac{\prb{R^2}}{nl^2}\\
                &= \frac{(200)^2}{(199760)(0.154)^2}\\
                \Aboxed{C_\infty &\approx 8.44}
            \end{align*}
        \end{proof}
        \item Estimate the Kuhn length and number of Kuhn segments for this polymer.
        \begin{proof}
            For a polyolefin like PMMA, the straight-chain length $R_\text{max}$ is given by
            \begin{equation*}
                R_\text{max} = nl\cos(70.5/2)
                = (199760)(\SI{0.154}{\nano\meter})\cos(70.5/2)
                \approx \SI{25122}{\nano\meter}
            \end{equation*}
            Thus, the Kuhn length $l_k$ is
            \begin{align*}
                l_k &= \frac{\prb{R^2}}{R_\text{max}}\\
                &= \frac{(200)^2}{25122}\\
                \Aboxed{l_k &\approx \SI{1.59}{\nano\meter}}
            \end{align*}
            It follows that the number $N_k$ of Kuhn steps is
            \begin{align*}
                N_k &= \frac{R_\text{max}}{l_k}\\
                &= \frac{25122}{1.59}\\
                \Aboxed{N_k &\approx 15800}
            \end{align*}
        \end{proof}
        \pagebreak
        \item Estimate the persistence length for this polymer.
        \begin{proof}
            Since the Kuhn length is twice the persistence length $a$,
            \begin{align*}
                a &= \frac{l_k}{2}\\
                &= \frac{1.59}{2}\\
                \Aboxed{a &\approx \SI{0.795}{\nano\meter}}
            \end{align*}
        \end{proof}
        \item Is this PMMA a "flexible," "semi-flexible," or "rod-like" polymer?
        \begin{proof}
            Since the chain length is $R_\text{max}=\SI{25122}{\nano\meter}$ and the persistence length is $a=\SI{0.795}{\nano\meter}$, the chain length is clearly much greater than the persistence length (indeed, by nearly five orders of magnitude). Therefore, PMMA is a \fbox{flexible} polymer.
        \end{proof}
    \end{enumerate}
    \item (25 pts) Consider a linear copolymer with $N_A$ steps of Kuhn length $\ell_A$ and $N_B$ steps of Kuhn length $\ell_B$. The solvent is such that excluded volume effects are negligible (theta solvent).
    \begin{enumerate}
        \item Calculate the mean squared end-to-end distance $\prb{R^2}$ of the chain for a linear diblock architecture (all A monomers connected to all B monomers).
        \begin{proof}
            By the definition of $\prb{R^2}$ for a freely jointed chain --- regardless of whether or not all steps are the same length --- we have
            \begin{align*}
                \prb{R^2} &= \prb{\sum_{i=1}^{N_A+N_B}\mathbf{l}_i\cdot\sum_{j=1}^{N_A+N_B}\mathbf{l}_j}\\
                &= \sum_{i=1}^{N_A+N_B}\sum_{j=1}^{N_A+N_B}\prb{\mathbf{l}_i\cdot\mathbf{l}_j}\\
                &= \sum_{i=1}^{N_A+N_B}\prb{\mathbf{l}_i\cdot\mathbf{l}_i}+\underbrace{\sum_{i=1}^{N_A+N_B}\sum_{\substack{j=1\\j\neq i}}^{N_A+N_B}\prb{\mathbf{l}_i\cdot\mathbf{l}_j}}_0\\
                &= \sum_{i=1}^{N_A}\prb{\mathbf{l}_i\cdot\mathbf{l}_i}+\sum_{i=N_A+1}^{N_B}\prb{\mathbf{l}_i\cdot\mathbf{l}_i}\\
                \Aboxed{\prb{R^2} &= N_A\ell_A^2+N_B\ell_B^2}
            \end{align*}
            Note that the underbracketed term still goes to zero because for each expected dot product of uncorrelated vectors, regardless of whether the vectors are the same length or different, every possible orientation of $\mathbf{l}_i$ and $\mathbf{l}_j$ (and hence their dot product) is in one-to-one correspondence with an equally probable opposing orientation.
        \end{proof}
        \item Will your answer in (a) change if the chain has a random distribution of monomers A and B? Explain why or why not.
        \begin{proof}
            As long as the polymer can still be described by the Kuhn steps laid out in the original problem statement, the commutativity principle postulates that it does not matter in which order we add up the dot products. Thus, the answer in (a) will \fbox{not} change. However, if the new bonding changes the Kuhn steps and length, then we may well obtain a different answer.
        \end{proof}
    \end{enumerate}
    \item (25 pts) Consider a linear polymer with $N$ Kuhn steps of length $\ell_k$ restricted to a two-dimensional interface (e.g. air/water interface). The Flory free energy terms for chain extension and excluded volume are
    \begin{equation*}
        \frac{F}{\kB T} = \frac{1}{2}\frac{aN^2}{R^2}+\frac{R^2}{N\ell_k^2}
    \end{equation*}
    where $a$ is the effective excluded area per monomer (analogue to excluded volume in 3D). The first term on the right-hand side is related to excluded volume and the second term related to stretching.
    \begin{enumerate}
        \item Calculate the scaling for $R$.
        \begin{proof}
            The polymer will be at its optimal size when there is no force either pushing it apart or pulling it together any more. This condition of zero force is equivalent to the change in free energy with respect to distance being equal to zero, or $\pdv*{F}{R}=0$. But this condition implies that
            \begin{align*}
                0 &= \pdv{F}{R}\\
                &= \kB T\left( -\frac{aN^2}{R^3}+\frac{2R}{N\ell_k^2} \right)\\
                \frac{aN^2}{R^3} &= \frac{2R}{N\ell_k^2}\\
                \frac{aN^3\ell_k^2}{2} &= R^4\\
                \Aboxed{R &\approx a^{1/4}N^{3/4}\ell_k^{1/2}}
            \end{align*}
        \end{proof}
    \end{enumerate}
    \item (25 pts) Consider monomers which interact via the following pairwise interaction potential, which is called a square well interaction:
    \begin{align*}
        U &= \infty\tag{$r<b$}\\
        U &= -\varepsilon\tag{$b\leq r\leq\lambda b$}\\
        U &= 0\tag{$r>\lambda b$}
    \end{align*}
    \begin{enumerate}
        \item Calculate the excluded volume $B$.
        \begin{proof}
            The excluded volume $B$ is given by
            \begin{align*}
                B &= -4\pi\int_0^\infty r^2\left[ \exp(-\frac{U(r)}{\kB T})-1 \right]\dd{r}\\
                &= -4\pi\left\{ \int_0^br^2[-1]\dd{r}+\int_b^{\lambda b}r^2\left[ \exp(\frac{\varepsilon}{\kB T})-1 \right]\dd{r}+\int_{\lambda b}^\infty r^2[0]\dd{r} \right\}\\
                &= -4\pi\left\{ -\int_0^br^2\dd{r}+\left[ \exp(\frac{\varepsilon}{\kB T})-1 \right]\int_b^{\lambda b}r^2\dd{r}+0 \right\}\\
                &= -4\pi\left\{ -\frac{b^3}{3}+\left[ \exp(\frac{\varepsilon}{\kB T})-1 \right]\cdot\frac{b^3}{3}(\lambda^3-1) \right\}\\
                % &= \frac{4}{3}\pi b^3\left\{ 1-\left[ \exp(\frac{\varepsilon}{\kB T})-1 \right]\cdot(\lambda-1) \right\}\\
                &= \frac{4}{3}\pi b^3\left\{ 1-\left[ \lambda^3\cdot\exp(\frac{\varepsilon}{\kB T})-\exp(\frac{\varepsilon}{\kB T})-\lambda^3+1 \right] \right\}\\
                &= \frac{4}{3}\pi b^3\left[ -\lambda^3\cdot\exp(\frac{\varepsilon}{\kB T})+\exp(\frac{\varepsilon}{\kB T})+\lambda^3 \right]\\
                \Aboxed{B &= \frac{4}{3}\pi b^3\left[ (1-\lambda^3)\exp(\frac{\varepsilon}{\kB T})+\lambda^3 \right]}
            \end{align*}
        \end{proof}
        \item Simplify your expression for $B$ by assuming $\varepsilon/\kB T\ll 1$. \emph{Hint}: Expand the exponential.
        \begin{proof}
            The Taylor series expansion for $\e[x]$ is
            \begin{equation*}
                \e[x] = 1+x+\frac{x^2}{2!}+\cdots+\frac{x^n}{n!}+\cdots
            \end{equation*}
            Given that $x=\varepsilon/\kB T$ is very small, it is then clear that the higher-order terms in the Taylor series become negligible very quickly. Thus, it is not a bad approximation to say that $\e[x]\approx 1+x$ for small $x$. Applying this line of reasoning to the result from part (a) and simplifying then yields
            \begin{align*}
                B &= \frac{4}{3}\pi b^3\left[ (1-\lambda^3)\left( 1+\frac{\varepsilon}{\kB T} \right)+\lambda^3 \right]\\
                &= \frac{4}{3}\pi b^3\left[ \left( 1+\frac{\varepsilon}{\kB T}-\lambda^3-\lambda^3\cdot\frac{\varepsilon}{\kB T} \right)+\lambda^3 \right]\\
                &= \frac{4}{3}\pi b^3\left[ 1+\frac{\varepsilon}{\kB T}-\lambda^3\cdot\frac{\varepsilon}{\kB T} \right]\\
                \Aboxed{B &= \frac{4}{3}\pi b^3\left[ 1+(1-\lambda^3)\frac{\varepsilon}{\kB T} \right]}
            \end{align*}
        \end{proof}
        \item Using your answer in (b), determine the Theta temperature for this system.
        \begin{proof}
            The theta temperature for a system is the one at which excluded volume effects are negligible, i.e., $B=0$. Using this constraint and solving the result from part (b) for $T$ yields
            \begin{align*}
                0 &= \frac{4}{3}\pi b^3\left[ 1+(1-\lambda^3)\frac{\varepsilon}{\kB T} \right]\\
                -1 &= (1-\lambda^3)\frac{\varepsilon}{\kB T}\\
                \Aboxed{T &= \frac{\varepsilon}{\kB}(\lambda^3-1)}
            \end{align*}
        \end{proof}
    \end{enumerate}
\end{enumerate}




\end{document}