\documentclass[../notes.tex]{subfiles}

\pagestyle{main}
\renewcommand{\chaptermark}[1]{\markboth{\chaptername\ \thechapter\ (#1)}{}}

\begin{document}




\chapter{The Macromolecule}
\section{The Macromolecule}\label{sse:2}
\begin{itemize}
    \item \marginnote{9/9:}Pat Doyle begins; he will teach the next three lectures.
    \begin{itemize}
        \item They've broken the class up into modules taught alternatingly.
        \item Aside: Alfredo has taught this course 10 times; Doyle never has (he's replacing Greg Rutledge this year).
    \end{itemize}
    \item Announcements.
    \begin{itemize}
        \item Slides and homework 1 have been posted.
        \item Slides should be posted before lecture, but may not be posted much before.
    \end{itemize}
    \item Lecture outline.
    \begin{itemize}
        \item Connectivity: Thermodynamic limit, architecture, and molecular weight.
        \item Configurations: Structural, chemical, stereo, and geometrical isomerism.
        \item Conformations: Rotational isomeric states.
    \end{itemize}
    \item \textbf{Connectivity}: The joining of small parts (monomers) into larger molecules (polymers).
    \item \textbf{Degree of polymerization}: The number of repeat units in a polymer. \emph{Denoted by} $\bm{N}$.
    \begin{itemize}
        \item Derivable from the molecular weight(s).
    \end{itemize}
    \item Example: Polyethylene.
    \begin{itemize}
        \item In this course, vinyl examples will be our workhorses, but we will "riff off of them" to other polymer types.
        \item The \textbf{repeat unit} here is \ce{CH2CH2}, consistent with the IUPAC nomenclature of poly\emph{ethylene}.
        \item Ethylene is also the monomer.
        \item The end groups do look different, but the \textbf{thermodynamic limit} addresses them.
    \end{itemize}
    \item \textbf{Repeat unit}: A part of a polymer whose repetition would produce the complete polymer chain (except for the end groups) by linking the repeat units together successively along a chain.
    \item \textbf{Thermodynamic limit}: The finding that as $N\to\infty$, the end group chemistry matters less. \emph{Also known as} \textbf{polymer limit}.
    \begin{itemize}
        \item The thermodynamic limit is also sometimes discussed in the context of statistical mechanics, where collective behavior also matters more than individual or picoscale.
    \end{itemize}
    \item \textbf{Glass transition temperature}: The temperature at which a substance will go from brittle to kind of rubbery. \emph{Denoted by} $\bm{T_\textbf{g}}$.
    \item \textbf{Flory-Fox correlation}: The simple model that the $\Tg$ of a polymer asymptotically approaches a limit $\Tg(M\to\infty)$ for higher and higher molecular weights at some empirically derived rate $A$. \emph{Given by}
    \begin{equation*}
        \Tg(\Mn) := \Tg(M\to\infty)-\frac{A}{\Mn}
    \end{equation*}
    \begin{itemize}
        \item Alkane series often obey this simple $1/x$ relation.
    \end{itemize}
    \item Another example of thermodynamic limits: Regardless of polymer structure, a power law defines polymer viscosity.
    \begin{figure}[h!]
        \centering
        \includegraphics[width=0.3\linewidth]{viscosityScaling.png}
        \caption{Polymer viscosity scales universally by power laws.}
        \label{fig:viscosityScaling}
    \end{figure}
    \begin{itemize}
        \item Namely, polymer viscosity increases up to a critical concentration $c^*$ at a slope of 1, then to an entanglement concentration $c_e$ at a slope of 2, and then at a slope of $14/3$ past that.
        \item Thus, there are three universal scaling regimes.
        \item A log-log plot is used to show power-law scalings, like in high school trigonometry!
        \item Aside: Power laws are present everywhere once you get big enough, down to the volume of our lungs relative to our bodies in bigger and bigger animals.
    \end{itemize}
    \item This class isn't super stringent on nomenclature, but it's good to know terms for when we read papers (Table \ref{tab:copolymerNomenclature}).
    \begin{table}[h!]
        \centering
        \includegraphics[width=0.9\linewidth]{copolymerNomenclature.png}
        \caption{IUPAC nomenclature of copolymers.}
        \label{tab:copolymerNomenclature}
    \end{table}
    \begin{itemize}
        \item Alfredo will talk about block copolymers a good bit!
    \end{itemize}
    \item Polymer architectures.
    \begin{itemize}
        \item Linear polymers.
        \begin{itemize}
            \item Polyrotoxanes and other supramolecular assemblies can have interesting properties. Example: Catenated DNA!
        \end{itemize}
        \item Cross-linked systems (nice gelation).
        \item Branched polymers can have a single monomer, or multiple as in graft (Table \ref{tab:copolymerNomenclature}).
        \item Dendritic polymers have different generations with regular branching for very dense structures.
        \item There are a few more classes, as well.
    \end{itemize}
    \item We'll now discuss some nomenclature on molecular weight.
    \item Aside: Other than nature, synthetic chemists cannot make dispersity 1 polymers; "polymer chemists aren't gods, despite some thinking they are."
    \item \textbf{$\bm{i}$-mer}: A segment of a polymer with degree of polymeriation $i$.
    \item $\bm{M_i}$: The molecular weight of the $i$-mer. \emph{Given by}
    \begin{equation*}
        M_i := iM_0
    \end{equation*}
    \item $\bm{M_0}$: The molecular weight of the repeat unit in a polymer chain.
    \item $\bm{n_i}$: The number of $i$-mers.
    \item \textbf{Number fraction} (of an $i$-mer): The probability of picking an $i$-mer out of solution when picking a chain. \emph{Denoted by} $\bm{x_i}$. \emph{Given by}
    \begin{equation*}
        x_i := \frac{n_i}{\sum_in_i}
    \end{equation*}
    \item \textbf{Weight fraction} (of an $i$-mer): The probability that a repeat unit picked out of solution belongs to an $i$-mer. \emph{Also known as} \textbf{mass fraction}. \emph{Denoted by} $\bm{w_i}$. \emph{Given by}
    \begin{equation*}
        w_i := \frac{n_iM_i}{\sum_in_iM_i}
        = \frac{n_i(iM_0)}{\sum_in_i(iM_0)}
        = \frac{in_i}{\sum_iin_i}
    \end{equation*}
    \begin{itemize}
        \item Note that $in_i$ is the total number of monomers in the $i$-mer, and $\sum_iin_i$ is the total number of monomers in solution.
    \end{itemize}
    \item \textbf{Number-average molecular weight}: The arithmetic mean of the molecular masses of the individual macromolecules. \emph{Denoted by} $\bm{M_\textbf{n}}$. \emph{Given by}
    \begin{equation*}
        \Mn := \sum_ix_iM_i = M_0\cdot\frac{\sum_iin_i}{\sum_in_i}
    \end{equation*}
    \item \textbf{Weight average molecular weight}: A mesurement of molecular weight that gives more contribution to higher-weight molecules. \emph{Denoted by} $\bm{M_\textbf{w}}$. \emph{Given by}
    \begin{equation*}
        \Mw := \sum_iw_iM_i = M_0\cdot\frac{\sum_ii^2n_i}{\sum_iin_i}
    \end{equation*}
    \item Both $\Mn$ and $\Mw$ look like moments of a distribution (i.e., first and second moment).
    \begin{itemize}
        \item We could generalize even more, but we don't need to.
        \item However, to figure out if we have a tight or wide distribution, we often look at ratios of our moments. This leads to the following definition.
    \end{itemize}
    \item \textbf{Dispersity}: A measure of the breadth of the distribution of fragment molecular weights in a polymer sample. \emph{Also known as} \textbf{polydispersity index}, \textbf{PDI}. \emph{Denoted by} $\tikz{
        \node[inner sep=0pt]{$\bm{D}$};
        \draw [thick] (-0.12,0) -- ++(0.12,0);
    }$. \emph{Given by}
    \begin{equation*}
        \Dstroke := \frac{M_w}{M_n} = \frac{\text{second moment}}{\text{first moment}}
    \end{equation*}
    \item In \textcite{bib:HiemenzLodge}, they also derive the \textbf{variance}. You don't need to worry about the math, though.
    \item \textbf{Variance}: Another measure of the breadth of the distribution of fragment molecular weights. \emph{Denoted by} $\bm{\sigma^2}$. \emph{Given by}
    \begin{equation*}
        \sigma^2 := M_n^2[\Dstroke-1]
    \end{equation*}
    \item \textbf{Schulz-Zimm distribution}: An idealized mathematical model for polymer molecular weight distribution.
    \item With these definitions, we can now do homework problem number 1!
    \item Experimental techniques to measure molecular weight.
    \begin{itemize}
        \item Size exclusion chromatography.
        \item Osmotic pressure.
        \item End group analysis.
        \item Light scattering techniques.
        \begin{itemize}
            \item More sensitive to higher moments.
        \end{itemize}
    \end{itemize}
    \item We now move onto polymer configurations.
    \item \textbf{Configurations}: The way things are connected or bonded together.
    \begin{itemize}
        \item Physicists and chemists have many competing definitions of "configurations," but the one above is what we'll use in this class. Essentially, think of it as a synonym for constitutional isomerism.
        \item Under this definition, you have to \emph{break bonds} to create a new configuration.
        \item We are \emph{not} yet talking about rotamers (what we'll call \textbf{conformations}). As you make your polymers longer and longer, the conformational space you can explore gets bigger.
    \end{itemize}
    \item There are 3 main types of structural isomers (i.e., configurations): \textbf{Positional}, \textbf{stereo}, and \textbf{geometric} isomers.
    \item \textbf{Positional} (isomers): Changing connectivity.
    \item \textbf{Stereoisomers}: Related to chiral centers.
    \item \textbf{Geometric} (isomers): Related to double bonds.
    \item More on positional isomers.
    \begin{figure}[H]
        \centering
        \includegraphics[width=0.6\linewidth]{headTail.png}
        \caption{Head-head and head-tail monomer addition.}
        \label{fig:headTail}
    \end{figure}
    \begin{itemize}
        \item \textbf{Head-head} vs. \textbf{head-tail} bonding in vinyl monomers.
        \item Head-tail is more common, and differences can change the $\Tg$ substantially.
    \end{itemize}
    \item \textbf{Head-head} (orientation): Monomer addition wherein the substituted carbon attaches to the growing chain end. \emph{Also known as} \textbf{head-to-head}.
    \item \textbf{Head-tail} (orientation): Monomer addition wherein the \emph{un}substituted carbon attaches to the growing chain end. \emph{Also known as} \textbf{head-to-tail}.
    \item More on stereoisomers.
    \item Doyle reviews \textbf{chirality}, \textbf{rectus} vs. \textbf{sinister}, and the \textbf{Cahn-Ingold-Prelog nomenclature}.
    \item \textbf{Pseudochiral} (center): A chiral center where two of the substituents are identical \emph{except} for chirality.
    \begin{itemize}
        \item In this case, CIP nomenclature gives priority to the branch with more R chiral centers.
    \end{itemize}
    \item Chiral centers in polymers give rise to \textbf{tacticity}.
    \begin{figure}[h!]
        \centering
        \includegraphics[width=0.7\linewidth]{tacticity.png}
        \caption{Tacticity.}
        \label{fig:tacticity}
    \end{figure}
    \begin{itemize}
        \item There are \textbf{meso diads} and \textbf{racemic diads}; which ones you have determine if the polymer is \textbf{isotactic}, \textbf{syndiotactic}, or \textbf{atactic}.
    \end{itemize}
    \item \textbf{Meso} (diad): Two adjacent chiral centers with a local plane of symmetry halfway between them.
    \item \textbf{Racemic} (diad): Two adjacent chiral centers with\emph{out} a local plane of symmetry halfway between them.
    \item \textbf{Isotactic} (polymer): A polymer containing only meso diads.
    \item \textbf{Syndiotactic} (polymer): A polymer containing only racemic diads.
    \item \textbf{Atactic} (polymer): A polymer containing both meso and racemic diads.
    \item Example: Atactic polystyrene tends to be more amorphous than syndiotactic or isotactic polymers, which can be semicrystalline.
    \begin{itemize}
        \item Nobel prize (1963) to Ziegler and Natta for a catalyst generating isotactic polystyrene (PS-it).
        \begin{itemize}
            \item Note that the initial Ziegler-Natta catalysts weren't metallocenes! The introduction of these types only came later.
        \end{itemize}
        \item Syndiotactic polystyrene (PS-st) came later in 1986 and had superior properties.
        \begin{itemize}
            \item PS-st crystallizes an order of magnitude faster than PS-it; has half the entanglement molecular weight; and is commonly used today in auto parts, electronics, and medical equipment.
        \end{itemize}
    \end{itemize}
    \item Tacticity is often measured by certain splittings (or their absence) in \ce{{}^1H} NMR.
    \begin{itemize}
        \item Isotactic polymers put the geminal methylene protons into distinct chemical environments; syndiotactic polymers do not. Atactic polymers will have a mix of both, and the mix can be quantified with integration.
        \item \ce{{}^13C} NMR can be used, too.
    \end{itemize}
    \item More on geometric isomers.
    \begin{itemize}
        \item Example: Polybutadiene can be formed \emph{trans} or \emph{cis}, as guided by a catalyst.
        \item Natural rubber is \emph{cis}-1,4-polyisoprene. Other types of tree sap can give \emph{trans}-1,4-polyisoprene.
    \end{itemize}
    \item We now move onto polymer conformations.
    \item \textbf{Conformations}: The spacial arrangements possible (or "probable," taking energy into account) for a polymer.
    \begin{itemize}
        \item No bonds are \emph{broken} here, just rotated.
    \end{itemize}
    \item Reviews \textbf{Newman projections}.
    \begin{itemize}
        \item \textbf{Staggered} configuration is defined as \ang{0}.
        \item \textbf{Eclipsed} configuration then starts at \ang{60}.
        \item There are energy pentalties to being in different conformations.
        \begin{itemize}
            \item As one example, eclipsed is higher energy than staggered due to sterics.
            \item Generally sinusoidal relation in a plot of potential energy $V$ against dihedral angle $\theta$.
            \item The energy difference between rotamers is approximately $3\kB T$, which is not huge but big enough that the system will spend most of its time in the valleys. Each "valley" is a \textbf{conformer}.
            \item So then since probability is proportional to $\e[-V/\kB T]$, the probability that a molecule will be staggered is greater than that it will be eclipsed.
        \end{itemize}
        \item In molecules longer than ethane, we break degeneracy of the valleys.
        \item The rapid growth of conformers: Ethane has 3 conformers. Propane has $3^2$. Butane has $3^3$. Decane has $3^{10}$. Polyethylene with $N=10^5$ already has on the order of \num{e47000} possible conformers, a huge conformational space.
        \begin{itemize}
            \item This is because each bond has 3 valleys!
            \item Many of the models we'll develop are ways of enumerating these conformations in relation to some higher-order measurement of the polymer, such as the \textbf{n-band difference}.
            \item Polymers are indeed often moderately sized coils rather than fully stretched out rods.
            \item Example: 166 kbp DNA (approximately 684 \textbf{Kuhn steps}, discussed next lecture) can be videotaped moving around, and it never fully elongates.
            \item Stretched out polymers shrink back over some characteristic time.
        \end{itemize}
        \item Polymers with high degrees of polymerization result in many possible conformations without breaking bonds --- this is what we'll discuss in the next two lectures!
    \end{itemize}
    \item A good converions to keep in mind: $\SI[per-mode=symbol]{2.5}{\kilo\joule\per\mole}\approx 1\kB T$.
    \item Today is probably the most jampacked bits and pieces day; other lectures will be more focused, but this is important background.
\end{itemize}



\section{Conformation: Ideal Chains}\label{sse:3}
\begin{itemize}
    \item \marginnote{9/11:}Lecture outline.
    \begin{itemize}
        \item Conformation (degrees of freedom).
        \item Ideal chain models.
        \item Entropic elasticity.
    \end{itemize}
    \item Many material properties we care about are correlated with the size of the polymer.
    \item This size is measured by a vector $\mathbf{R}$ that goes from one end of the polymer to the other.
    \begin{figure}[h!]
        \centering
        \includegraphics[width=0.2\linewidth]{Rvector.png}
        \caption{End-to-end polymer vector.}
        \label{fig:Rvector}
    \end{figure}
    \begin{itemize}
        \item We define $\mathbf{R}$ as the sum of all consituent bond vectors $\mathbf{l}$ (end-to-end of each chemical bond along the backbone). Symbolically,
        \begin{equation*}
            \mathbf{R} := \sum_{i=1}^n\mathbf{l}_i
        \end{equation*}
        \item Note that $\mathbf{R}$ has length $|\mathbf{R}|=R$.
        \item Today, we will look at several models that can be used to calculate the expected length of this vector, $\prb{R}$.
    \end{itemize}
    \pagebreak
    \item Freely Jointed Chain (FJC) model.
    \begin{figure}[h!]
        \centering
        \includegraphics[width=0.15\linewidth]{joint.png}
        \caption{A joint in a polymer chain.}
        \label{fig:joint}
    \end{figure}
    \begin{itemize}
        \item In this model, there are no restrictions on how adjacent bonds rotate relative to each other. Rather, this is just a random walk. In effect, this means that there are no energy barriers and no excluded volume.
        \begin{itemize}
            \item It follows that $\prb{R}=0$.
        \end{itemize}
        \item However, while $\prb{R}=0$, we have
        \begin{align*}
            \prb{R^2} &= \prb{\sum_{i=1}^n\mathbf{l}_i\cdot\sum_{j=1}^n\mathbf{l}_j}\\
            &= \sum_{i=1}^n\sum_{j=1}^n\prb{\mathbf{l}_i\cdot\mathbf{l}_j}\\
            &= \sum_{i=1}^n\prb{\mathbf{l}_i\cdot\mathbf{l}_i}+\sum_{i=1}^n\sum_{\substack{j=1\\j\neq i}}^n\prb{\mathbf{l}_i\cdot\mathbf{l}_j}\\
            &= nl^2+0
        \end{align*}
        \begin{itemize}
            \item The second term goes to zero as $n\to\infty$ because there is no correlation among segments (i.e., they are randomly oriented).
            \item This gives us the following important scaling law.
            \begin{equation*}
                \prb{R^2}^{1/2} \propto n^{1/2}
            \end{equation*}
            \item This means that scaling is consistent with the polymer taking a "coil-like" conformation.
        \end{itemize}
        \item This combined with the fact that the polymer's fully stetched length is $nl$ gives us the following expression for the number of backbone bonds in a polymer.
        \begin{equation*}
            \frac{(\text{fully stretched length})^2}{\prb{R^2}} = n
        \end{equation*}
    \end{itemize}
    \item Note that \textcite{bib:HiemenzLodge} uses $\theta$ for the \emph{complement} of bond angle, as in Figure \ref{fig:joint}. Other texts may use a different convention.
    \item Now, let's refine the FJC by accounting for nearest neighbor correlations.
    \item First, we'll look at a polymer with only two segments (i.e., $n=2$)
    \begin{itemize}
        \item Suppose every joint is \emph{fixed} at complementary angle $\theta$, but there is no energy penalty to rotate in $\phi$.
        \item From Figure \ref{fig:joint}, trigonometry tells us that
        \begin{equation*}
            \mathbf{l}_i\cdot\mathbf{l}_{i+1} = l^2\cos\theta
        \end{equation*}
        \item Thus, under the conditons of this model,
        \begin{align*}
            \prb{R^2} &= 2l^2+\sum_{i=1}^2\sum_{\substack{j=1\\j\neq i}}^2\prb{\mathbf{l}_i\cdot\mathbf{l}_j}\\
            &= 2l^2+2\prb{l^2\cos\theta}\\
            &= nl^2(1+\cos\theta)
        \end{align*}
        \item The important takeaway is that with this chemical realism, the chain is bigger than in the previous model!
        \begin{itemize}
            \item It is also noteworthy that the $nl^2$ scaling relation is retained.
        \end{itemize}
        \item This is a precursor to the freely rotating chain, where we can rotate in $\phi$ but not in $\theta$.
    \end{itemize}
    \item Let's now look at the full Freely Rotating Chain (FRC) model.
    \begin{itemize}
        \item As we elongate the chain, there is a slow decay of "memory of correlation" since $\phi$ rotates freely. Eventually (see \textcite[239-40]{bib:HiemenzLodge} for the derivation), we asymptote to
        \begin{equation*}
            \prb{R^2} = nl^2\underbrace{\left( \frac{1+\cos\theta}{1-\cos\theta} \right)}_{C_n}
        \end{equation*}
    where $C_n$ may be an empirically derived \textbf{characteristic ratio}.
    \end{itemize}
    \item Model 3: Symmetric hindered rotations.
    \begin{itemize}
        \item Recall from last class that certain rotational conformations have lower energies than others.
        \item As such, we can give a Boltzman weighting to the energetic valleys.
        \item Thus, we reevaluate our hindered rotations with a Boltzmann weighting and the following expression crashes out of the math.
        \begin{equation*}
            \prb{R^2} = nl^2\underbrace{\left( \frac{1+\cos\theta}{1-\cos\theta} \right)\left( \frac{1+\prb{\cos\phi}}{1-\prb{\cos\phi}} \right)}_{C_\infty}
        \end{equation*}
        \begin{itemize}
            \item Note that it's still just $nl^2$ times a constant!
        \end{itemize}
        \item The \textbf{characteristic ratio} $C_\infty$ can be calculated for models or obtained from experiments. The following rearranged definition is also important.
        \begin{equation*}
            C_\infty := \frac{\prb{R^2}_0}{nl^2}
        \end{equation*}
        \item Alert: Be aware of sign changes due to different conventions for $\theta$ and $\phi$ in different texts!
    \end{itemize}
    \item There are tables of characteristic ratios in both \textcite{bib:HiemenzLodge} and \textcite{bib:RubinsteinColby}.
    \begin{table}[H]
        \centering
        \includegraphics[width=0.65\linewidth]{CinftyCommon.png}
        \caption{$C_\infty$ values for common polymers at \SI{413}{\kelvin}.}
        \label{tab:CinftyCommon}
    \end{table}
    \begin{itemize}
        \item $C_\infty$ gets bigger with bigger side chains.
        \item Typical range is 5-10; can go up to 20 or higher, though.
    \end{itemize}
    \item Example: What is the size of a polyethylene molecule at \SI{413}{\kelvin} and having molecular weight $\SI[per-mode=symbol]{e4}{\gram\per\mole}$?
    \begin{itemize}
        \item Approach: We want to find $\prb{R^2}$ and take its square root; that will be our answer.
        \item We can look up that the length $l$ of a typical \ce{C-C} bond is \SI{0.154}{\nano\meter}.
        \item Based on the molecular weight and the known weight of the ethylene (\ce{CH2CH2}) repeat unit,
        \begin{equation*}
            N = \frac{\SI[per-mode=symbol]{e4}{\gram\per\mole}}{\SI[per-mode=symbol]{28}{\gram\per\mole}} \approx 357.1
        \end{equation*}
        \item Because there are two carbon-carbon bonds per repeat unit, $n=2N$.
        \item Thus,
        \begin{align*}
            \prb{R} &= \prb{R^2}^{1/2}\\
            &= [nl^2C_\infty]^{1/2}\\
            &= \left[ (2\cdot 357.1)(\SI{0.154}{\nano\meter})^2(7.4) \right]^{1/2}\\
            \prb{R} &\approx \SI{11}{\nano\meter}
        \end{align*}
    \end{itemize}
    \item Let's compare the answer in the above example to the straight chain estimation.
    \begin{figure}[h!]
        \centering
        \includegraphics[width=0.22\linewidth]{straightChainCC.png}
        \caption{Straight chain estimation of a polyolefin.}
        \label{fig:straightChainCC}
    \end{figure}
    \begin{itemize}
        \item In this case,
        \begin{equation*}
            R_\text{max} = nl\cos(\theta/2)
        \end{equation*}
        \item We can look up that for a typical \ce{C-C} bond, $\theta=\ang{180}-\ang{109.5}=\ang{70.5}$.
        \item Thus,
        \begin{align*}
            R_\text{max} &= nl\cos(\theta/2)\\
            &= (2\cdot 357.1)(\SI{0.154}{\nano\meter})\cos(70.5/2)\\
            R_\text{max} &\approx \SI{90}{\nano\meter}
        \end{align*}
    \end{itemize}
    \item This calculation is a bit tedious, so Kuhn refined the coarse grained FJC model to be computationally simpler.
    \item Model 4: Kuhn's Equivalent Chain.
    \begin{figure}[H]
        \centering
        \includegraphics[width=0.3\linewidth]{KuhnSteps.png}
        \caption{Kuhn steps.}
        \label{fig:KuhnSteps}
    \end{figure}
    \begin{itemize}
        \item Building off of models 3 and 1, let's postulate the existence of an FJC along our polymer. This FJC will have $N$ steps of length $l_k$, where $l_k$ is a Kuhn step.
        \item This gives us two variables to define. We thus need two equations by which to define them.
        \begin{itemize}
            \item Equation 1: By the universal scaling law,
            \begin{equation*}
                Nl_k^2 = \prb{R^2} = C_\infty nl^2
            \end{equation*}
            \item Equation 2: If the total length of the chain is $R_\text{max}$, and the chain is being broken up into $N$ Kuhn steps each of length $l_k$, then
            \begin{equation*}
                R_\text{max} = Nl_k
            \end{equation*}
        \end{itemize}
        \item By solving this system of equations, we can then define $N$ and $l_k$ purely in terms of prevoiusly derived variables.
    \end{itemize}
    \item \textbf{Kuhn step}: A subsegment of a polymer chain with length defined as follows. \emph{Denoted by} $\bm{l_k}$. \emph{Given by}
    \begin{equation*}
        l_k := \frac{\prb{R^2}}{R_\text{max}} = \frac{C_\infty nl^2}{R_\text{max}}
    \end{equation*}
    \item \textbf{Number of Kuhn steps}: The number of Kuhn steps in a polymer chain. \emph{Denoted by} $\bm{N}$. \emph{Given by}
    \begin{equation*}
        N := \frac{R_\text{max}}{l_k} = \frac{R_\text{max}^2}{C_\infty nl^2}
    \end{equation*}
    \item Example: In the case of a fully elongated carbon-carbon chain (Figure \ref{fig:straightChainCC}), the number of Kuhn steps is
    \begin{equation*}
        l_k = \frac{C_\infty nl^2}{R_\text{max}}
        = \frac{C_\infty nl^2}{nl\cos(\theta/2)}
        = \frac{C_\infty l}{\cos(\theta/2)}
    \end{equation*}
    \item Note that the symbols $R_\text{max}$ and $L$ will be used interchangeably for the straight-chain length of a polymer.
    \item $R_\text{max}$ largely depends on the chemistry of the polymer (e.g., specific atoms' bond angles).
    \item We'll now look at some models for "stiff" chains, such as dsDNA or microtubules.
    \begin{itemize}
        \item These tend to have even higher $C_\infty$ values.
        \item Note that twists in the chain are \emph{much} bigger than individual nucleobases.
        \item Molecular simulations of 75 bp dsDNA shows barely any bending. Indeed, there is a high correlation between end vectors even though they are very far away.
    \end{itemize}
    \item A model for very stiff polymers: The Worm-Like Chain (WLC).
    \begin{figure}[H]
        \centering
        \includegraphics[width=0.32\linewidth]{WLC.png}
        \caption{Worm-like chain.}
        \label{fig:WLC}
    \end{figure}
    \begin{itemize}
        \item Represent a segment as an infinitesimal elastic rod.
        \item This rod has a contour you can be some point $s$ along, where $s\in[0,L]$.
        \item The direction of the rod is defined by a tangent vector $\mathbf{t}(s)$.
        \item We also give the rod some bending energy $U_b$, related to how the tangent changes as we go along $s$. But physically, this is just curvature squared. To get the right units, we throw in a bending stiffness parameter $\kappa_b$, which incorporates some chemical / molecular details. Throw in a $1/2$ from mechanics definitions, and we get
        \begin{equation*}
            U_b = \frac{1}{2}\kappa_b\int_0^L\left( \pdv{\mathbf{t}}{s} \right)^2\dd{s}
        \end{equation*}
        \item An implication of this is that correlation decays exponentially per
        \begin{equation*}
            \prb{\mathbf{t}(0)\cdot\mathbf{t}(s)} = \e[-s\kB T/\kappa_b]
        \end{equation*}
        \begin{itemize}
            \item I.e., you use memory of how oriented you are in one part of the chain when you get farther away from that point.
        \end{itemize}
        \item Since $\kappa_b/\kB T$ has units of length by dimensional analysis, we can define it to be the \textbf{persistence length}.
        \item From this tangent correlation function, we can calculate many interesting properties --- including the mean squared end-to-end distance!
        \begin{equation*}
            \prb{R^2} = 2aL\left[ 1-\frac{a}{L}\left( 1-\e[-L/a] \right) \right]
        \end{equation*}
        \begin{itemize}
            \item The calculation is complicated, so Dolye skips it.
        \end{itemize}
        \item This equation reveals some interesting polymer behavior in two limits: That of \emph{long} and \emph{short} stiff polymers.
        \begin{itemize}
            \item When the polymer gets long, $a/L\to 0$ and $\prb{R^2}\to 2aL$.
            \item When the polymer gets short, $a/L\to\infty$ and $\prb{R^2}\to L^2$.
            \begin{itemize}
                \item This makes intuitive sense as if it's short, it should be roughtly straight and have end-to-end distance approximately equal to its length!
                \item Note that in real life, there \emph{are} polymers where the persistence length is longer than the length of the polymer! These behave like rigid rods.
            \end{itemize}
        \end{itemize}
        \item Lastly, it follows from definition of Kuhn steps that
        \begin{equation*}
            2aL = \prb{R^2} = Nl_k^2 = \frac{R_\text{max}}{l_k}\cdot l_k^2 = Ll_k
        \end{equation*}
        \begin{itemize}
            \item Thus, the Kuhn length is twice the persistence length! Symbolically,
            \begin{equation*}
                2a = l_k
            \end{equation*}
        \end{itemize}
    \end{itemize}
    \item \textbf{Persistence length}: A characteristic length over which a stiff polymer loses memory of its orientation along other parts of the chain. \emph{Denoted by} $\bm{a}$. \emph{Given by}
    \begin{equation*}
        a := \frac{\kappa_b}{\kB T}
    \end{equation*}
    \item In conclusion, two biggest models to remember: Kuhn model (rigid steps) and WLC (continuum approximation with persistence length for semi-rigid chains).
    \item Example: Actin has a persistence length of \SI{10}{\micro\meter}; since most cells are smaller than this, actin is funtionally a rigid rod within a cellular context.
    \item Example: Measuring persistence length of dsDNA.
    \begin{itemize}
        \item Adsorb DNA onto a surface that loosely binds it, so that it can still move around but won't fall off.
        \item Then look at thousands of strands next to each other and calculate tangent lengths!
        \item Reference: \textcite{bib:DNApersisLength}.
    \end{itemize}
    \item Example: Bottlebrush polymers in cartilage.
    \begin{itemize}
        \item These have highly charged side chains, but fewer with age.
        \item This causes more bending.
        \item We can observe this with atomic force microscopy.
    \end{itemize}
    \item Example: As you increase the concentration of salt in solution, you shrink the \textbf{Debye length} and also the persistence length.
    \begin{itemize}
        \item This modifies the effect of charges on dsDNA.
    \end{itemize}
    \item Example: Actin cytoskeleton filaments.
    \begin{itemize}
        \item Made out of polymerized protein subunits.
        \item A very thin polymer, biologicaly speaking.
        \item Very long persistence length, as mentioned earlier.
        \item As cells move, they push actin against the cell membrane to distort it! This works because actin is a very rigid rod, so rigid that it can overcome the membrane pressure.
    \end{itemize}
    \item Example: Conjugated polymers.
    \begin{itemize}
        \item They calculated the persistence length using DFT, and then measured it experimentally.
        \item Some of their polymers are stiffer tha dsDNA!
        \item Rotation around one particular engineered bond is used to estimate persistence length.
    \end{itemize}
    \item \textbf{Flexible} (polymer): A polymer for which chain length is much greater than persistence length.
    \item \textbf{Semi-flexible} (polymer): A polymer for which chain length is approximately equal to persistence length.
    \item \textbf{Rod-like} (polymer): A polymer for which chain length is much shorter than persistence length.
    \item Summary of ideal chains.
    \begin{itemize}
        \item All chains show a similar, universal scaling relation that $R\approx N^{1/2}$.
        \item For ideal chains, local interactions set the rigidity length scale and excluded volume is not significant.
        \item This approximation is ok for dilute solutions at \textbf{theta conditions} and polymer \textbf{melts}.
    \end{itemize}
    \item \textbf{Theta condition}: When you've essentially turned off excluded volume for the chain.
    \item \textbf{Polymer melt}: A condition in which the polymer is essentially in a solution of itself.
    \item Nomenclature for polymer solution regimes.
    \begin{figure}[H]
        \centering
        \includegraphics[width=0.2\linewidth]{hardPervaded.png}
        \caption{The volume occupied by polymers.}
        \label{fig:hardPervaded}
    \end{figure}
    \item \textbf{Hard volume} (of a polymer): The volume occupied by the chain, where each repeat unit is considered to occupy a sphere with radius equal to the bond length and the polymer is the sum of these "beads" touching each other. \emph{Also known as} \textbf{occupied volume}. \emph{Denoted by} $\bm{v}$. \emph{Given by}
    \begin{equation*}
        v \propto nl^3
    \end{equation*}
    \begin{itemize}
        \item The volume of each bead is thus on the order of the bond length.
    \end{itemize}
    \item \textbf{Pervaded volume} (of a polymer): The sphere encapsulating the volume in which the polymer chain is \emph{expected} to move around. \emph{Denoted by} $\bm{V}$, $\bm{V_\text{p}}$. \emph{Given by}
    \begin{equation*}
        V \propto \prb{\mathbf{R}^2}^{3/2} \propto n^{3/2}l^3
    \end{equation*}
    \item From this, we can see that
    \begin{equation*}
        \frac{v}{V} \propto n^{-1/2}
    \end{equation*}
    \begin{itemize}
        \item It follows that the pervaded volume is mostly empty as $n$ becomes large.
    \end{itemize}
    \item \textbf{Critical concentration}: The concentration at which all polymers in solution can "see" each other. \emph{Also known as} \textbf{coil overlap concentration}. \emph{Denoted by} $\bm{c^*}$.
    \begin{figure}[h!]
        \centering
        \includegraphics[width=0.4\linewidth]{concentrationRegimes.png}
        \caption{Concentration regimes.}
        \label{fig:concentrationRegimes}
    \end{figure}
    \begin{itemize}
        \item In dilute solutions, polymers will all be coiled up within their own pervaded volume and will not interact.
        \item When they reach the critical concentration, all of the pervaded volumes are essentially touching each other.
        \item Past the critical concentration, we get entanglement and no longer see individual polymer coils in their own pervaded volume.
        \item This is the same $c^*$ as in Figure \ref{fig:viscosityScaling}!
        \begin{itemize}
            \item Unifying implication: These concentrations are related to pervaded volume, which is related to expected size.
        \end{itemize}
    \end{itemize}
    \item Next time.
    \begin{itemize}
        \item Adding in excluded volume, which will wrap up our discussion on chains.
        \item Then Alfredo on thermodynamics of interactions.
    \end{itemize}
\end{itemize}



\section{Conformation: Real Chains}\label{sse:4}
\begin{itemize}
    \item \marginnote{9/16:}Announcements.
    \begin{itemize}
        \item PSet 1 due Thursday. We can do 1-2 right now; 3-4 will be possible after today.
        \item Reach out if we have questions!
    \end{itemize}
    \item Lecture outline.
    \begin{itemize}
        \item Ideal chains.
        \begin{itemize}
            \item Gaussian chains.
            \item Entropic elasticity.
        \end{itemize}
        \item Real chains.
        \begin{itemize}
            \item Excluded volume.
            \item Solvent quality.
            \item Expansion factor $\alpha$.
        \end{itemize}
    \end{itemize}
    \item Review from last lecture.
    \begin{itemize}
        \item Our workhorse model for polymer chain length is Kuhn's equivalent chain, which is in turn based on the FJC random walk.
        \item This is a coarse-grained model because atoms don't matter, and we're neglecting to consider excluded volume and energy.
    \end{itemize}
    \item Aside: Another measure of polymer size worth being aware of is the Radius of Gyration.
    \begin{itemize}
        \item The radius of gyration $R_g$ is defined as the 2nd moment of the monomers relative to their center of mass $\mathbf{R}_\text{cm}$.
        \begin{equation*}
            R_g^2 = \frac{1}{N}\sum_{i=1}^N(\mathbf{R}_i-\mathbf{R}_\text{cm})^2
        \end{equation*}
        \item Since the center of mass is defined as
        \begin{equation*}
            \mathbf{R}_\text{cm} = \frac{1}{N}\sum_{j=1}^N\mathbf{R}_j
        \end{equation*}
        we can do some algebra to learn that
        \begin{equation*}
            \prb{R_g^2} = \frac{1}{6}\prb{R^2}
        \end{equation*}
        \item Covered in the lecture reading.
    \end{itemize}
    \item Question: How many monomers before Kuhn steps have statistical meaning?
    \begin{itemize}
        \item About 100 before you converge to a Gaussian distribution.
    \end{itemize}
    \item Before we relate Kuhn's equivalent chain (a 3D model) to polymer size, let's look at the 1D analog: The full probability distribution of a 1D random walk.
    \begin{figure}[H]
        \centering
        \includegraphics[width=0.2\linewidth]{1Dwalk.png}
        \caption{Progression of a 1D random walk over time.}
        \label{fig:1Dwalk}
    \end{figure}
    \begin{itemize}
        \item In 1D, you can either go in the positive or negative direction by a distance $b$.
        \item Imagine you keep doing this for $n$ steps, after which we'll say you've gone a distance $x$ from your starting point.
        \item If $n_1$ is the number of steps right and $n_2$ is the number of steps left, then
        \begin{equation*}
            x = b(n_1-n_2)
        \end{equation*}
        \item Let's now analyze this problem statistically.
        \item From the perspective of statistics, the 1D random walk is equivalent to the coin flipping problem. Specifically, the total number $\Omega$ of ways (microstates) in which you can move a distance $x$ is
        \begin{equation*}
            \Omega = \frac{n!}{n_1!n_2!} = \binom{n}{n_1} = \binom{n}{n_2}
        \end{equation*}
        \begin{itemize}
            \item $\Omega$ is then a binomial, and as such is maximized when $n_1$ is close to $n_2$. But when $n_1\approx n_2$, $x=b(n_1-n_2)$ is small.
            \item The overall implication is that you are most likely to end up back where you started!
            \item Aside: Why is $\Omega$ a binomial?
            \begin{itemize}
                \item Binomials mathematically originate from "$n$ choose $k$" counting of unordered \emph{combinations}. For example, if you have 4 objects A, B, C, and D, there are $\binom{4}{2}=\frac{4!}{(4-2)!2!}=6$ ways you can choose two of them: AB, AC, AD, BC, BD, and CD. Note that \emph{choosing} AB is the same as choosing BA, i.e., order does not matter. This situation is related to a binomial because the "first" time we choose an object out of ABCD, we have 4 options. Suppose we choose B. Then when we go back to choose a second object, we have 3 remaining choices: ACD. Thus, there are $4\cdot 3=12=\frac{4!}{(4-2)!}$ paths to two objects (aka \emph{permutations}). But choosing A then B is the same as choosing B then A when it comes to combinations, so we have overcounted by exactly the number of ways there are to arrange 2 objects, which is $2!$. Thus, we must divide by $2!$ again to get $\frac{4!}{(4-2)!2!}=6$.
                \item This established, where in the context of the 1D random walk are we "choosing" objects, and what objects are we choosing?
                \item Suppose we are taking $n=4$ steps, and we want to identify the total number $\Omega$ of ways we can move a distance $x=0$. By solving the system of equations $4=n_1+n_2$ and $0=b(n_1-n_2)$, we can learn that $n_1=n_2=2$. This means that we must take $n_1=2$ steps to the right, and $n_2=2$ steps to the left in order to end up back at zero. But when do we take these steps? Diagramming this out, there are four timeslots at which we can take a step (\verb|____|), and two of those must eventually be filled by steps to the right \verb|R|. The first time we \emph{choose} a timeslot, we have 4 options: \verb|R___|, \verb|_R__|, \verb|__R_|, and \verb|___R|. Then the second time we choose, we have three remaining options; moreover, if we fill (for example) the first and then the second timeslot, that's equivalent to filling the second and then the first! This is how we relate the 1D random walk to "$n$ choose $k$" probability.
            \end{itemize}
        \end{itemize}
        \item Now that we know the number of ways we can move a distance $x$, we can calculate the probability $P(x)$ of moving a distance $x$ by dividing by the total number of possible paths.
        \begin{equation*}
            P(x) = \frac{\Omega}{\Omega_\text{tot}} = \frac{n!}{n_1!n_2!2^n}
        \end{equation*}
        \begin{itemize}
            \item Moreover, because $\Omega$ is a \emph{binomial}, $P(x)$ is a \emph{binomial distribution}.
        \end{itemize}
        \item \textbf{Stirling's approximation} allows us to expand the factorials for large $n$, and subsequently prove that the above binomial distribution converges to the following Gaussian distribution.
        \begin{equation*}
            P(x,n) = \left( \frac{1}{2\pi nb^2} \right)^{1/2}\exp(\frac{-x^2}{2nb^2})
        \end{equation*}
        \item This allows us to then get back to the following familiar expression.
        \begin{equation*}
            \prb{x^2} = nb^2
        \end{equation*}
    \end{itemize}
    \item From this 1D model, we can generalize to the Gaussian distribution of an end-to-end vector in 3D.
    \begin{equation*}
        P(\mathbf{R},N) = \left( \frac{3}{2\pi Nb^2} \right)^{3/2}\exp(-\frac{3R^2}{2Nb^2})
    \end{equation*}
    \begin{itemize}
        \item Note the units of reciprocal volume, as we'd expect for a spatial probability.
        \item Using spherical coordinates gets us to a familiar expression again, as follows.
        \begin{equation*}
            \prb{R^2} = \int_0^\infty R^2P(\mathbf{R},N)4\pi R^2\dd{R} = Nb^2
        \end{equation*}
        \begin{itemize}
            \item Note that $R^2$ appears twice. The first time, it is because $R^2$ is what we're taking the expected value of over all space. The second time, it is because an easy way to take the intergral of a radial probability distribution is in spherical coordinates.
            \item $P(\mathbf{R},N)$ is radial because the random walk does not have a particular direction in mind in 3D space; you are equally likely to end up $x$ units from the origin in the $+x$ direction, as you are in the $-x$ direction, $-z$ direction, or any other direction you can imagine.
        \end{itemize}
    \end{itemize}
    \item This concludes our discussion of Gaussian chains.
    \item We'll now discuss entropic elasticity.
    \begin{itemize}
        \item Recall from thermodynamics that
        \begin{equation*}
            F = U-TS
        \end{equation*}
        where $F$ is free energy, $U$ is internal potential energy, $T$ is temperature, and $S$ is entropy.
        \item For an FJC, $U=0$ (by definition) and $S$ is the only thing we have to consider.
        \item Recall from Boltzmann that
        \begin{equation*}
            S = \kB\ln[\Omega(\mathbf{R},N)]
        \end{equation*}
        \item The generalization of our above definition of probability is the following.
        \begin{equation*}
            P(\mathbf{R},N) = \frac{\Omega(\mathbf{R},N)}{\int_{\R^3}\Omega(\mathbf{R},N)\dd{\mathbf{R}}}
        \end{equation*}
        \item Letting $\Omega_N$ denote the denominator (to emphasize that it's a function of $N$, not $\mathbf{R}$) and substituting in our previous results affords
        \begin{align*}
            S(\mathbf{R},N) = \kB\ln[P(\mathbf{R},N)]+\kB\ln(\Omega_N)
            = -\frac{3\kB R^2}{2Nb^2}+\underbrace{\kB\ln(\frac{3}{2\pi Nb^2})^{3/2}+\kB\ln(\Omega_N)}_{S_N}
        \end{align*}
        \begin{itemize}
            \item The second two terms become $S_N$ because they're less interesting (as functions of $N$, not $\mathbf{R}$).
        \end{itemize}
        \item By substituting the above back into $F=U-TS$, it follows that the FJC's entropic free energy is
        \begin{equation*}
            F = \frac{3\kB TR^2}{2Nb^2}+S_N
        \end{equation*}
        \begin{itemize}
            \item In this context, we can think of $S_N$ as an additive normalization constant.
        \end{itemize}
        \item With this result, we can determine how to minimize $F$ relative to $R$.
        \begin{itemize}
            \item Minimizing $F$ relative to $R$ is a goal because the radius at which the free energy is minimal will be the radius of polymer coils in solution (everything wants to minimize energy).
        \end{itemize}
    \end{itemize}
    \item Using this (FJC) model, we can calculate an entropic spring force $\mathbf{F}(\mathbf{R},N)$.
    \begin{itemize}
        \item We know that
        \begin{equation*}
            \mathbf{F}(\mathbf{R},N) = \pdv{F}{\mathbf{R}}
            = \pdv{F}{R}\hat{\mathbf{R}}+\pdv{F}{\theta}\hat{\theta}+\pdv{F}{\phi}\hat{\phi}
            = \frac{3\kB T}{Nb^2}\hat{\mathbf{R}}
        \end{equation*}
        \item It follows that at a molecular level, a polymer constitutes a linear Hookean spring ($F=kx$), where the whole prefactor to $\hat{\mathbf{R}}$ is our spring constant $k$.
        \item Rearranging, we can express the above equation with a relative extension.
        \begin{equation*}
            \mathbf{F}(\mathbf{R},N) = \frac{3\kB T}{b}\left( \frac{\hat{\mathbf{R}}}{Nb} \right)
        \end{equation*}
        \begin{itemize}
            \item $Nb$ is the length of the fully extended freely jointed chain, and $\hat{\mathbf{R}}$ is a partial extension. Thus, their the fraction is a \textbf{relative extension}.
        \end{itemize}
        \item It has been experimentally shown that up to about 30\% extension, these force laws hold!\footnote{Thus, Kuhn steps are like a higher level of theory to rationalize an experimentally observed result!}
    \end{itemize}
    \item We'll now begin discussing \textbf{real chains}.
    \begin{itemize}
        \item Up to now, we've been discussing \textbf{ideal chains}, which occur at theta conditions and in melts.
        \item Monomers in ideal chains also only interact via local effects (e.g., hindered rotations in the symmetric hindered rotations model).
    \end{itemize}
    \item Real chains.
    \begin{figure}[H]
        \centering
        \includegraphics[width=0.3\linewidth]{LocNon.JPG}
        \caption{Local and non-local interactions in a polymer.}
        \label{fig:LocNon}
    \end{figure}
    \begin{itemize}
        \item Local chemistry is coarse-grained out by Kuhn lengths.
        \item Particles farther away from each other interact via \textbf{excluded volume}. Essentially, monomers may have some interaction (favorable or unfavorable) when they float near each other in solution.
    \end{itemize}
    \pagebreak
    \item To ground our mathematical/physical analysis, we will approximate monomers in a polymer as an ideal "gas" of monomers in a volume $R$ (Figure \ref{fig:monomerGasa}).
    \begin{figure}[h!]
        \centering
        \begin{subfigure}[b]{0.3\linewidth}
            \centering
            \includegraphics[width=0.6\linewidth]{monomerGasa.JPG}
            \caption{Gas in a volume $R$.}
            \label{fig:monomerGasa}
        \end{subfigure}
        \begin{subfigure}[b]{0.2\linewidth}
            \centering
            \includegraphics[width=0.65\linewidth]{monomerGasb.JPG}
            \caption{Excluded volume.}
            \label{fig:monomerGasb}
        \end{subfigure}
        \begin{subfigure}[b]{0.3\linewidth}
            \centering
            \includegraphics[width=0.3\linewidth]{monomerGasc.JPG}
            \caption{Distance between monomers.}
            \label{fig:monomerGasc}
        \end{subfigure}
        \caption{Monomer gas model of a polymer's excluded volume.}
        \label{fig:monomerGas}
    \end{figure}
    \begin{itemize}
        \item As a first stab at visualizing "nonlocal" interactions, consider a gas of hard-sphere monomers.
        \begin{itemize}
            \item Given two hard-sphere monomers with center-to-center distance $\delta$, Figure \ref{fig:monomerGasb} shows us that the volume that one of them would prevent the other from exploring is
            \begin{equation*}
                V_\text{ex} = \frac{4}{3}\pi\delta^3
            \end{equation*}
        \end{itemize}
        \item Let's generalize: Per Boltzmann, the probability of finding two monomers at a distance $r$ from each other is proportional to $\e[-U(r)/\kB T]$. This allows us to define the \textbf{Mayer-$\bm{f}$ function}.
        \item The Mayer-$f$ function then weights the extent to which a given particle will exclude other particles from each infinitesimal volume of space. Thus, integrating over all space --- see \textcite{bib:RubinsteinColby} for details --- we can derive the below definition of excluded volume.
    \end{itemize}
    \item \textbf{Mayer-$\bm{f}$ function}: The probability of finding two particles close to each other relative to finding them infinitely far apart, where the particles interact via a spherically symmetric distance-dependent potential $U(r)$. \emph{Denoted by} $\bm{f(r)}$. \emph{Given by}
    \begin{equation*}
        f(r) := \exp(-\frac{U(r)}{\kB T})-\exp\bigg( -\frac{\overbrace{U(\infty)}^0}{\kB T} \bigg)
        = \exp(-\frac{U(r)}{\kB T})-1
    \end{equation*}
    \begin{itemize}
        \item Notice that it is equal to zero when there is no interaction (i.e., when $U=0$) because in that case, all $r$ are equally probable.
    \end{itemize}
    \item \textbf{Excluded volume}: The volume a given monomer would occupy were it a hard sphere, on account of its \emph{energetic} interactions with other monomers. \emph{Denoted by} $\bm{V_\textbf{ex}},\bm{v},\bm{B}$. \emph{Given by}
    \begin{equation*}
        V_\text{ex} := -4\pi\int_0^\infty r^2f(r)\dd{r}
    \end{equation*}
    \begin{itemize}
        \item This is the volume that a given monomer tries to \emph{exclude} other monomers from entering.
        \item In the case of a negative excluded volume, monomers like each other so much that they'd rather be abnormally close together.
        \item Notice that we're in polar coordinates.
        \item The monomer-to-monomer interaction $U(r)$ also incorporates the solvent since the energy is not that of two bare monomers, but those monomers with a medium between them (e.g., two styrene monomers in a sea of toluene).
    \end{itemize}
    \item Sanity check: Evaluating the definition of excluded volume for the hard sphere potential (with the switchover occurring at $\delta$) yields $V_\text{ex}=4\pi\delta^3/3$!
    \item The Mayer-$f$ function can be visualized through plots and integrals.
    \begin{figure}[H]
        \centering
        \begin{subfigure}[b]{0.2\linewidth}
            \centering
            \includegraphics[width=0.6\linewidth]{MayerFa.png}
            \caption{Good solvent.}
            \label{fig:MayerFa}
        \end{subfigure}
        \begin{subfigure}[b]{0.2\linewidth}
            \centering
            \includegraphics[width=0.6\linewidth]{MayerFb.png}
            \caption{Bad solvent.}
            \label{fig:MayerFb}
        \end{subfigure}
        \begin{subfigure}[b]{0.2\linewidth}
            \centering
            \includegraphics[width=0.6\linewidth]{MayerFc.png}
            \caption{$\theta$-solvent.}
            \label{fig:MayerFc}
        \end{subfigure}\\[2em]
        \begin{subfigure}[b]{0.3\linewidth}
            \centering
            \includegraphics[width=0.8\linewidth]{MayerFd.png}
            \caption{Hard sphere particles.}
            \label{fig:MayerFd}
        \end{subfigure}
        \begin{subfigure}[b]{0.3\linewidth}
            \centering
            \includegraphics[width=0.8\linewidth]{MayerFe.png}
            \caption{Lennard-Jones particles.}
            \label{fig:MayerFe}
        \end{subfigure}
        \caption{Visualizing the Mayer-$f$ function.}
        \label{fig:MayerF}
    \end{figure}
    \begin{itemize}
        \item Depending on $U$, we have three possibilities.
        \begin{itemize}
            \item In a good solvent, the coils unwind / volume expands. Repulsion dominates (Figure \ref{fig:MayerFa}).
            \item In a bad solvent, the coils bunch up. Attraction dominates (Figure \ref{fig:MayerFb}).
            \item In a $\theta$-solvent, repulsion balances attraction (Figure \ref{fig:MayerFc}).
        \end{itemize}
        \item We can also plot potentials and mathematical forms relevant to the Mayer-$f$ function.
        \begin{itemize}
            \item Thus, we can see that in the hard-sphere potential (Figure \ref{fig:MayerFd}), there is only a repulsive contribution and thus no $\theta$-temperature or $\theta$-solvent.
            \item In contrast, for the Lennard-Jones potential (Figure \ref{fig:MayerFe}), there are both repulsive and attractive contributions. Thus, there \emph{will} be a $\theta$-temperature in each $\theta$-solvent.
        \end{itemize}
    \end{itemize}
    \item In a theta solvent, $V_\text{ex}(T)=0$.
    \begin{itemize}
        \item The temperature $T$ at which $V_\text{ex}=0$ is called the \textbf{Boyle temperature}; it is the temperature at which excluded volume is zero. Basically, you always have excluded volume, but if that excluded volume is balanced by a bit of attraction, you can have something that looks ideal.
        \item PSet 1, Q5 explores the relation between $V_\text{ex}$ and $T$ more explicitly!
    \end{itemize}
    \item The heart of polymer physics is Flory mean-field approximations.
    \begin{figure}[H]
        \centering
        \includegraphics[width=0.2\linewidth]{compressPenalty.JPG}
        \caption{A post-facto compression penalty.}
        \label{fig:compressPenalty}
    \end{figure}
    \begin{itemize}
        \item In a $\theta$-solvent, $F=-TS$ (because $U=0$).
        \item As we derived above, for this situation, the entropic free energy $F_\text{ent}$ is
        \begin{equation*}
            F_\text{ent} = \frac{3\kB T}{2}\frac{R^2}{Nb^2}
        \end{equation*}
        and
        \begin{equation*}
            \pdv{F}{R} = \frac{3\kB TR}{Nb^2}
        \end{equation*}
        \item At equilibrium, $\pdv*{F}{R}=0$. It follows that
        \begin{align*}
            \frac{3\kB TR}{Nb^2} &= 0\\
            R &= 0
        \end{align*}
        \begin{itemize}
            \item But this is inconsistent with the finding that $\prb{R^2}=Nb^2$.
        \end{itemize}
        \item We will address this with a post facto compression penalty. Indeed, if we set
        \begin{equation*}
            F_\text{ent} = \frac{3\kB T}{2}\left( \frac{R^2}{Nb^2}+\frac{Nb^2}{R^2} \right)
        \end{equation*}
        then $\pdv*{F}{R}=0$ implies
        \begin{align*}
            R &= N^{1/2}b\\
            R^2 &= Nb^2
        \end{align*}
        \item The nice thing about this compression penalty is that the graph of $1/R^2+R^2$ forms a nice energy well, dictated by the rates at which one or the other increases. See Figure \ref{fig:compressPenalty}.
    \end{itemize}
    \item Now analogously to a \textbf{Virial expansion} (from thermodynamics), we are going to do another expansion.
    \begin{equation*}
        \frac{F_\text{int}}{V\kB T} = \frac{1}{2!}vc^2+\frac{1}{3!}wc^3+\cdots
    \end{equation*}
    \begin{itemize}
        \item This takes the perspective that we disregard monomer connectivity, and evaluate the enthalpic interaction energy $F_\text{int}$ as if it was a gas of monomers in a spherical container of radius $R$.
        \item $c=N/R^3$ is the concentration of the monomers, and $V\propto R^3$.
        \begin{itemize}
            \item We are throwing out prefactors like $4/3$ and $\pi$.
        \end{itemize}
        \item This gives us a term for binary interactions, ternary interactions, etc.
    \end{itemize}
    \item In a good solvent, $R>Nb^2$ because the polymer swells due to $V_\text{ex}$.
    \begin{itemize}
        \item Free energy is then
        \begin{equation*}
            F = \frac{3\kB TR^2}{2Nb^2}+\frac{1}{2}V_\text{ex}\frac{N^2}{R^3}\kB T
        \end{equation*}
        \item Then $\pdv*{F}{R}=0$ implies
        \begin{align*}
            0 &= \frac{\kB TR}{Nb^2}-\frac{3V_\text{ex}N^2\kB T}{2R^4}\\
            R^5 &= \frac{1}{2}N^3b^2V_\text{ex}\\
            R &\approx b^{2/5}V_\text{ex}^{1/5}N^{3/5}
        \end{align*}
        \begin{itemize}
            \item Taking everything into account with renormalized group theory yields something pretty close!
            \item This stands in contrast to the ideal $R\approx N^{1/2}b$.
        \end{itemize}
    \end{itemize}
    \item In a bad solvent, enthalpic interactions matter more than entropic interactions because the polymer is compressed into a globule regardless.
    \begin{itemize}
        \item Thus, we have to balance the first two terms in $F_\text{int}$ to determine the scaling.
        \begin{align*}
            0 &= \pdv{F_\text{int}}{R}\\
            &= \pdv{R}\left\{ \kB TR^3\left[ \frac{1}{2}v\left( \frac{N}{R^3} \right)^2+\frac{1}{6}w\left( \frac{N}{R^3} \right)^3 \right] \right\}\\
            &= \kB T\pdv{R}(\frac{vN^2}{2R^3}+\frac{wN^3}{6R^6})\\
            0 &= -\frac{3vN^2}{2R^4}-\frac{wN^3}{R^7}\\
            R^3 &= -\frac{2wN}{3v}\\
            R &\propto \left( \frac{w}{-v} \right)^{1/3}N^{1/3}
        \end{align*}
    \end{itemize}
    \item Combining the enthalpic and entropic terms, the full Flory free energy has many terms that all balance; you play with the ingredients to see what fits.
    \begin{equation*}
        \frac{F}{\kB T} = \frac{V_\text{ex}N^2}{2R^3}+\frac{wN^3}{6R^6}+\frac{3R^2}{2Nb^2}+\frac{3Nb^2}{2R^2}
    \end{equation*}
    \item The chain or polymer must be long enough for excluded volume to factor in.
    \begin{itemize}
        \item In $F_\text{ev}\geq\kB T$, we have a good solvent.
        \item We then do some more math and get
        \begin{equation*}
            z = \frac{V_\text{ex}N^{1/2}}{b^3} < 1
        \end{equation*}
    \end{itemize}
    \item There is additional material in the slides.
    \begin{itemize}
        \item Importantly, as you go away from $\theta$-conditions, you still have an $N^{1/2}$ relation.
        \item $1-\theta/T$ is related to the homework; it's a perturbation of a Taylor series.
    \end{itemize}
    \item The \textbf{expansion factor} $\alpha$ is just the ratio of the real volume to the ideal.
\end{itemize}



\section{Chapter 1: Introduction to Chain Molecules}
\emph{From \textcite{bib:HiemenzLodge}.}
\begin{itemize}
    \item \marginnote{9/14:}Begins with the uses and history of polymers.
    \item \textbf{Macromolecule}: A large or long molecule.
    \item \textbf{Polymer}: A macromolecule made by repeating one (or a few) chemical units many times.
    \begin{itemize}
        \item All polymers are macromolecules, but not all macromolecules are polymers.
    \end{itemize}
    \item \textcite{bib:HiemenzLodge} will often write molecular weights without the unit "\si[per-mode=symbol]{\gram\per\mole}," but we should write them with this unit.
    \item Some types of polymerization involve only the joining of monomers; others involve this joining, but with the elimination of a small molecule byproduct (like \ce{H2O} or \ce{HCl}).
    \item Figure \ref{fig:straightChainCC}'s derivation uses the \textbf{law of cosines} from high school trigonometry.
    \item Polyolefin size: Based on typical \ce{C-C} bond angles and lengths\dots
    \begin{itemize}
        \item The straight-chain length is approximately $N\cdot\SI{0.25}{\nano\meter}$;
        \item The actual end-to-end distance is approximately $N^{1/2}\cdot\SI{0.25}{\nano\meter}$.
    \end{itemize}
    \item \textbf{Oligomer}: A molecule made of repeat units but for which $N<10$.
    \item Note that \emph{polymers} typically have $M\geq\SI[per-mode=symbol]{1000}{\gram\per\mole}$, but this cutoff is arbitrary (and probably on the low side).
    \item Polymer architectures.
    \item \textbf{Branched} (polymer): A linear molecule with additional polymeric chains issuing from its (linear) backbone.
    \item \textbf{Short-chain branch}: A small substituent (such as a methyl or phenyl group) on the repeat unit.
    \begin{itemize}
        \item These are generally not considered branches.
    \end{itemize}
    \item Branching can occur \emph{by design}, \emph{adventitiously}, or through \textbf{grafting}.
    \begin{itemize}
        \item By design, we may introduce polyfunctional junctions into the formulation.
        \item Adventitiously, "an atom [may be] abstracted from the original linear molecule, with chain growth occurring from the resulting active site" \parencite[7]{bib:HiemenzLodge}.
        \begin{itemize}
            \item This commonly happens with polyethylene!
        \end{itemize}
    \end{itemize}
    \item \textbf{Graft to} (polymerization): Pre-formed but still reactive polymer chains can be added to sites along an existing backbone.
    \item \textbf{Graft from} (polymerization): Multiple initiation sites along a chain can be exposed to monomer.
    \item Note: \textbf{Graft through} polymerization is not covered.
    \item For simple branching (no loops), a molecule with $v$ branches has $v+2$ chain ends.
    \item \textbf{Comb} (polymer): A polymer in which a series of relatively uniform branches emanate from along the length of a common backbone.
    \item \textbf{Star} (polymer): A polymer in which all branches radiate from a central junction.
    \item \textbf{Cross-linked} (polymer): A polymer with such extensive branching that the macroscopic object may be considered to consist of essentially one molecule.
    \begin{itemize}
        \item These polymers are given cohesiveness by covalent bonds instead of intermolecular forces, and thus have very different properties from non-cross-linked polymers.
    \end{itemize}
    \item \textbf{Hyperbranched} (polymer): A polymer that is highly branched, but in which the components remain as discrete entities.
    \item \textbf{Dendrimer}: A class of hyperbranched polymers that are tree-like, in that they have completely regular structures formed by successive condensation of branched monomers.
    \begin{itemize}
        \item Start with a \ce{B3} junction (generation 0), condense on three \ce{AB$'$2} monomers and deprotect (generation 1), condense on six \ce{AB$'$2} monomers and deprotect (generation 2), etc.
        \item At generation 6 or 7, the surface of the molecule becomes so congested that adding further complete generations is impossible.
    \end{itemize}
    \item \textbf{Cycle} (polymer): A polymer in which the two chain ends react to close the loop. \emph{Also known as} \textbf{ring}.
    \item \textbf{Homopolymer}: A polymer with only a single kind of repeat unit.
    \begin{itemize}
        \item Note that polymers made with two different monomers can still be homopolymers. For example, a polyester condensed from diacid and diol \emph{monomers} still only has one kind of \emph{repeat unit}.
    \end{itemize}
    \item \textbf{Copolymer}: A polymer with two kinds of repeat units.
    \item \textbf{Terpolymer}: A polymer with three kinds of repeat units.
    \item \textbf{Multicomponent} (polymer): A polymer with four or more kinds of repeat units.
    \item \textbf{Random} (copolymer): A copolymer in which the \ce{A}-\ce{B} sequence is governed strictly by chance, subject only to the relative abundances of repeat units.
    \item \textbf{Statistical} (copolymer): A copolymer in which monomer addition at a growing chain end may depend on the preceding monomer(s), according to a statistical law.\footnote{The copolymers we make with CCs are all statistical, since addition is governed by probability-based reactivity ratios.}
    \item \textbf{Alternating} (copolymer): A copolymer containing a regular pattern of alternating repeat units.
    \item \textbf{Block} (copolymer): A copolymer containing long, uninterrupted sequences of each monomer.
    \item \textbf{$\bm{n}$-block} (copolymer): A block copolymer containing $n$ uninterrupted sequences of each monomer.
    \begin{itemize}
        \item For small $n$, the terms \textbf{diblock}, \textbf{triblock}, and \textbf{tetrablock} are commonly used.
        \item If a triblock copolymer consists of a sequence of monomer \ce{A}, followed by monomer \ce{B}, followed by monomer \ce{A}, we may call it an \textbf{ABA triblock copolymer}.
    \end{itemize}
    \item \marginnote{9/16:}In addition to classifying polymers by architecture, we can classify them by the reactions used to make them.
    \item \textbf{Addition} and \textbf{condensation} polymerization are the most common, but there are more.
    \item \textbf{Addition} (polymerization): A polymerization for which the following three statements apply. \emph{Also known as} \textbf{chain-growth}. \emph{Defined by}
    \begin{enumerate}
        \item The repeat unit in the polymer and monomer have the same composition (although bonding is different in each).
        \item The polymerization proceeds through a \textbf{chain reaction} mechanism, with either free radicals or ionic groups responsible for propagating the chain reaction.
        \item The product molecules \emph{often} have an all \ce{C-C} backbone, with pendant substituent groups.
    \end{enumerate}
    \item \textbf{Condensation} (polymerization): A polymerization for which the following three statements apply. \emph{Also known as} \textbf{step-growth}. \emph{Defined by}
    \begin{enumerate}
        \item The polymer repeat unit arises from the reaction of two different functional groups, which \emph{usually} originate on different monomers. In this case, the repeat unit is ifferent from either monomer. In addition, small molecules are \emph{often} eliminated during the condensation reaction.
        \item Mechanistically, the reactions occur in steps; in other words, the formation of the linkage functional group between two small molecules is not essentially different from that between one of these growing polymers and a monomer.
        \item The product molecules have the functional groups formed by the condensation reactins interspersed regularly along the backbone of the polymer molecule.
    \end{enumerate}
    \item Properties of condensation polymerizations.
    \begin{itemize}
        \item Either two difunctional monomers with different functional groups, or one difunctional monomer with a functional group of each kind.
        \begin{itemize}
            \item Beware the creation of loops!
        \end{itemize}
        \item Very sensitive to impurities.
        \begin{itemize}
            \item Example: Trace methanol in a polyester condensation could cap some carbonyl derivatives as methyl esters, preventing them from reacting further.
        \end{itemize}
        \item Functionality greater than 2 can introduce branching.
        \item Introduction of reagents with varying functionalities in carefully controlled amounts affords control over polymer size and geometry.
    \end{itemize}
    \item The physics in this text apply equally well to organic and inorganic (e.g., PDMS) polymers.
    \item Biopolymers are briefly introduced.
    \item Polymer nomenclature.
    \item IUPAC recommendation.
    \begin{itemize}
        \item Polymers formed from a single monomer.
        \begin{itemize}
            \item Take the IUPAC name of the monomer, enclose it in parentheses, and add the prefix "poly".
            \item Example: poly(1-chloroethylene) for PVC.
            \item Polymers need not be synthesized from the monomer named. For example, poly(1-hydroxyethylene) is synthesized from the hydrolysis of poly(1-acetoxyethylene).
        \end{itemize}
        \item Polymers formed from multiple monomers.
        \begin{itemize}
            \item Apply the preceding rules to the repeat unit.
            \item Example: poly(hexamethylene adipamide) is nylon-6,6.
        \end{itemize}
        \item Commercially important cross-linked polymers.
        \begin{itemize}
            \item Typically go without names, or at best, by specifying the monomers that go into it.
            \item Example: "Phenol-formaldehyde resin" for bakelite.
        \end{itemize}
    \end{itemize}
    \item Notes on common names.
    \begin{itemize}
        \item Many polymers have a common/outdated chemical name, a trade name, and/or an acronym.
        \item Examples: polyethylene glycol, Teflon, PVC.
    \end{itemize}
    \item On positional isomerism.
    \begin{itemize}
        \item Head-to-head addition is more common (1) at higher temperatures and (2) with halogenated monomers.
        \item Head-to-tail is dominant because of (1) resonance stabilization at the head positoin and (2) steric exposure of the tail.
        \item Polymer cleavage can provide information about head-head vs. head-tail polymerization: Since diols are cleavable by periodate, poly(vinyl alcohol) can have its weight measured, be subject to cleavage conditions, and have the resultant fragments' weight remeasured to calculate the average frequencey of diols.
    \end{itemize}
    \item Stereoisomerism reviewed.
    \item On geometric isomerism.
    \begin{itemize}
        \item Not only can you have \emph{cis}- or \emph{trans}-backbones, but isoprene (for example) can polymerize through either --- or both! --- of its double bonds.
    \end{itemize}
    \item \marginnote{10/1:}Molecular weights and molecular weight averages.
    \item \textbf{Mass concentration} (of an $i$-mer): The number of grams per mole of $i$-mer per unit volume. \emph{Denoted by} $\bm{c_i}$. \emph{Given by}
    \begin{equation*}
        c_i := \frac{n_iM_i}{V}
    \end{equation*}
    \item $\Mn$ and $\Mw$ are by far the most important and most common measurements of polymer average weight, but there are also others such as the \textbf{$\bm{z}$-average molecular weight}.
    \item \textbf{$\bm{z}$-average molecular weight}: The third moment of the weight distribution. \emph{Denoted by} $\bm{M_\textbf{z}}$. \emph{Given by}
    \begin{equation*}
        M_\text{z} := M_0\frac{\sum_ii^3n_i}{\sum_ii^2n_i}
    \end{equation*}
    \item \textbf{Monodisperse} (sample): A polymer sample for which $\Dstroke=1$.
    \item \textbf{Narrow} (distribution): A polymer weight distribution for which $\Dstroke<1.5$.
    \item \textbf{Broad} (distribution): A polymer weight distribution for which $\Dstroke>2$.
    \item \textcite{bib:HiemenzLodge} derive the variance expression given in class.
    \item \textbf{Mean} (of a distribution $i$): The numerical average of the distribution. \emph{Denoted by} $\bm{\prb{i}}$. \emph{Given by}
    \begin{align*}
        \prb{i} &:= \frac{\sum_iin_i}{\sum_in_i} = \sum_iix_i&
        \prb{i} &:= \int_0^\infty iP(i)\dd{i}
    \end{align*}
    \item \textbf{Normalized} (distribution): A discrete distribution $x_i$ or continuous distribution $P(i)$ satisfying the following respective criterion. \emph{Constraints}
    \begin{align*}
        \sum_ix_i &= 1&
        \int_0^\infty P(i)\dd{i} &= 1
    \end{align*}
    \item \textbf{$\bm{k}$-th moment} (of a normalized distribution): The number defined as follows. \emph{Denoted by} $\bm{\mu_k}$. \emph{Given by}
    \begin{align*}
        \mu_k &:= \sum_ix_ii^k&
        \mu_k &:= \int_0^\infty i^kP(i)\dd{i}
    \end{align*}
    \begin{itemize}
        \item The mean is therefore the first moment of a distribution!
        \item $\Mw$ and $M_\text{z}$ are proportional to the ratios of the 2nd to the 1st moment and the 3rd to the 2nd moment, respectively.
    \end{itemize}
    \item \textbf{$\bm{k}$-th moment about the mean} (of a normalized distribution): The number defined as follows. \emph{Denoted by} $\bm{v_k}$. \emph{Given by}
    \begin{equation*}
        v_k := \sum_ix_i(i-\prb{i})^k
    \end{equation*}
    \begin{itemize}
        \item It follows that the variance is the second moment about the mean.
    \end{itemize}
    \item \textbf{Schulz-Zimm distribution}: A one-parameter mathematical model for polymer weight distributions, where varying the one parameter affords reasonable descriptions for typical narrow or moderately broad samples. \emph{Given by}
    \begin{equation*}
        P(M_i) := \frac{z^{z+1}}{\Gamma(z+1)}\frac{M_i^{z-1}}{\Mn^z}\e[-zM_i/\Mn]
    \end{equation*}
    \begin{itemize}
        \item $\Gamma$ denotes the \textbf{gamma function}, a popular extension of the factorial function.
        \item \textcite{bib:HiemenzLodge} extend their discussion of this distribution quite a bit and graph some examples of it. One such graph is included in the lecture 2 slides.
    \end{itemize}
    \item Measurement of molecular weight.
    \item \textbf{Size exclusion chromatography}: A method of measuring polymer molecular weight that can provide information about the full distribution of weights. \emph{Also known as} \textbf{SEC}.
    \begin{itemize}
        \item Benefits.
        \begin{itemize}
            \item The dominant method today; almost all polymer chemistry laboratories have SEC capabilities.
            \item Automated analysis of a few milligrams of sample in a good solvent can be achieved in half an hour.
        \end{itemize}
        \item Limitations.
        \begin{itemize}
            \item Poor resolution.
            \item Reliance on standards.
        \end{itemize}
    \end{itemize}
    \item \textbf{Matrix-assisted laser desorption/ionization mass spectrometry}: A method of measuring polymer molecular weight that can provide information about the full distribution of weights. \emph{Also known as} \textbf{MALDI}.
    \begin{itemize}
        \item Benefits.
        \begin{itemize}
            \item High resolution.
        \end{itemize}
        \item Limitations.
        \begin{itemize}
            \item Relatively new and still being expanded in scope.
            \item Resolution is diminished as $M$ increases.
            \item Sensitivity is diminished as $M$ increases: Higher MW polymers are just harder to get into the gas phase.
            \begin{itemize}
                \item Sensitivity is also a heretofore unknown function of molecular weight, so extracting $\Mn$ and $\Mw$ quantitatively is unreliable.
                \item Microscopic differences in drops and their structure add further uncertainty.
            \end{itemize}
            \item Works better for more polar polymers (PE is almost impossible).
            \item More highly charged species confound data.
        \end{itemize}
        \item Section 1.8.3 covers the basics of how MALDI works.
        \begin{itemize}
            \item "A great deal remains to be learned about both the desorption and ionization processes, and standard practice is to follow particular recipes (matrix and salt) that have been found to be successful for a given polymer" \parencite[35]{bib:HiemenzLodge}.
        \end{itemize}
    \end{itemize}
    \item If SEC and MALDI can be done reliably, accurately, and conveniently, then there is little need for any other technique.
    \item \textbf{Colligative} (property): A property of a solution that depends on the the \emph{number} of solute particles alone, with no consideration of their nature.
    \item Some techniques provide information on \emph{only} $\Mn$ by measuring colligative properties.
    \begin{itemize}
        \item Osmotic pressure (most common), freezing point depression, boiling point elevation, light scattering, end group analysis, etc.
        \item Osmotic pressure is based on equilibrium thermodynamics, and thus affords absolute measurements \emph{without} calibration!
        \item End group analysis relies on the fact that if a molecule only has two ends, counting the number of ends is equivalent to counting the number of molecules.
        \begin{itemize}
            \item Common techniques: Acid-base titration (acidic or basic end groups), NMR (esp. \ce{{}^1H} qNMR).
        \end{itemize}
    \end{itemize}
    \item Some techniques provide information on \emph{only} $\Mw$.
    \begin{itemize}
        \item Light scattering is also based on equilibrium thermodynamics (so absolute and no calibration needed).
    \end{itemize}
    \item Misc. techniques: Sedimentation, gel electrophoresis, and intrinsic viscosity.
    \item End group analysis principles.
    \begin{enumerate}
        \item "The chemical structure of the end group must be sufficiently different from that of the repeat unit for the chosen analytical technique to resolve the two clearly" \parencite[32]{bib:HiemenzLodge}.
        \item "There must be a well-defined number of end groups per polymer, at least on average. For a linear polymer, there will be two and only two end groups per molecule, which may or may not be distinct from each other. For branched polymers, the relation of the number of end groups to the number of polymers is ambiguous, unless the total number of branching points is also known" \parencite[32]{bib:HiemenzLodge}.
        \item "The technique is limited to relatively low molecular weights, as the end groups become more and more dilute as $N$ increases\dots As a general rule end groups present at the 1\% level (corresponding to degrees of polymerization of 100 for a single end group, 200 for both end groups) can be reliably determined; those at the 0.1\% level cannot" \parencite[32-33]{bib:HiemenzLodge}.
    \end{enumerate}
\end{itemize}



\section{Chapter 6: Polymer Conformations}
\emph{From \textcite{bib:HiemenzLodge}.}
\begin{itemize}
    \item \textbf{Globule}: A compressed conformation of a polymer similar to a dense ball.
    \item \textbf{Random coil}: A conformation of a polyemer where the monomer subunits are oriented randomly while still being bonded to adjacent units.
    \item The approximately $0.7\kB T$ ($3\kB T$ from class appears to be wrong) energy difference between the trans and either gauche state implies that the gauche states will be about $\e[-0.7]\approx 0.5$ times as populated as the trans states.
    \item \textbf{Ergodic} (system): A system for which the time average of its states is equivalent to the ensemble average.
    \begin{itemize}
        \item For example, polymer conformations are ergodic because the end-to-end distance of a polymer averaged over time as it wiggles around is the same as its end-to-end distance averaged over a large collection of structurally identical chains at a given instant in time.
    \end{itemize}
    \item \textbf{Isotropic} (quantity): A quantity that is not biased in any particular direction.
    \begin{itemize}
        \item For example, the polymer end-to-end vector should not orient in any particular direction in the absence of external bias. This is why $\prb{\mathbf{R}}=0$, as discussed in class.
    \end{itemize}
    \item A more rigorous basis for why the cross terms in the FJC expansion of $\prb{R^2}$ equal zero.
    \begin{itemize}
        \item For any two vectors $\mathbf{l}_i,\mathbf{l}_j$, we have
        \begin{equation*}
            \prb{\mathbf{l}_i\cdot\mathbf{l}_j} = l^2\prb{\cos\theta}
        \end{equation*}
        where $\theta$ is the angle between them when translated through space to be tail-to-tail.
        \item Since we are assuming lack of correlation, $\theta$ is equally likely to be anywhere along its principle branch of $[0,\pi)$. Thus,
        \begin{equation*}
            \prb{\cos\theta} = \int_0^\pi\cos\theta\cdot\frac{\dd\theta}{\pi-0} = 0
        \end{equation*}
        \item This zeroes out the cross term.
    \end{itemize}
    \item \textcite[239-40]{bib:HiemenzLodge} explicitly derive the full FRC result for $\prb{R^2}$.
    \begin{itemize}
        \item It is an exact, analytical derivation; the only assumptions are that $n\to\infty$ (generally justified within measurement error for $n>100$) and $\theta\neq 0$ (generally justified because most bonds are not straight from one to the next).
    \end{itemize}
    \item Lessons from the FRC.
    \begin{itemize}
        \item $\prb{R^2}$ grows relative to the FJC approximation. This is reasonable because we have less doubling back.
        \item We still have $\prb{R^2}\propto nl^2$, despite the increase in complexity of the model.
        \item When $\theta\approx\ang{70.5}$ (as for a \ce{C-C} bond), $\prb{R^2}\approx 2nl^2$ under this model.
    \end{itemize}
    \item \textcite{bib:Flory2} derives $\prb{R^2}$ for the symmetric hindered rotations model.
    \item Theorem: Under the assumptions that $n\to\infty$ and polymer chains can freely pass through themselves, $\prb{R^2}=Cnl^2$ where $C$ is a numerical constant that depends only on local constraints and not on $n$.
    \item The principle of Kuhn's equivalent chain follows from this theorem: Memory of orientation is lost at a far enough distance away from the original link, making it so that "for any chain of $n$ links whose relative orientations are constrained, we can always generate an equivalent chain with a new (bigger) link that is freely jointed, so that the original chain and the new chain have the same [$\prb{R^2}$]" \parencite[241]{bib:HiemenzLodge}.
    \item Common backbone bond lengths.
    \begin{table}[h!]
        \centering
        \small
        \renewcommand{\arraystretch}{1.2}
        \begin{tabular}{c|ccccc}
            \toprule
            \textbf{Bond Type} & \ce{C-C} & \ce{C=C} & \ce{C-O} & \ce{C-N} & \ce{Si-O}\\
            \textbf{Bond Length (\si[detect-weight]{\angstrom})} & 1.53 & 1.34 & 1.43 & 1.47 & 1.6\\
            \bottomrule
        \end{tabular}
        \caption{Common polymer backbone bond lengths.}
        \label{tab:bondLength}
    \end{table}
    \begin{itemize}
        \item For polymers with multiple backbone bonds, add $n_1l_1^2+n_2l_2^2+\cdots$ for each type of backbone bond.
    \end{itemize}
    \item Larger $C_\infty$ implies stiffer chains.
    \item Characteristic ratio terminology.
    \begin{itemize}
        \item $C_n$ describes the characteristic ratio for a type of chain with \emph{exactly} $n$ bonds.
        \item $C_\infty$ is $C_n$, but specifically in the large $n$ limit where $n\to\infty$.
    \end{itemize}
    \item Common monomer parameters ($C_\infty$, Kuhn length, monomer volume, etc.) are compiled in a table on \textcite[243]{bib:HiemenzLodge}, sourced from \textcite{bib:monomerParameters}. This significantly extends Table \ref{tab:CinftyCommon}.
    \item Although the Kuhn length varies monotonically with $C_\infty$, is is not as good a measure of flexibility.
    \begin{itemize}
        \item Example: The Kuhn length for polystyrene is just a bit longer than for polyisoprene, but polystyrene is much stiffer (as reflected by a $C_\infty$ more than double polyisoprene's).
        \item See Problem 6.4 for more.
    \end{itemize}
    \item \textbf{Monomer volume}: The volume of a single monomer of interest. \emph{Denoted by} $\bm{v_0}$.
    \item A good visual for semiflexible, worm-like chains: A garden hose ($a$ is approximately 1 foot).
    \item The persistence length is equal to the end-to-end vector onto the direction of the first bond?? $a=\prb{\hat{\mathbf{l}}_1\cdot\mathbf{R}}$?? See \textcite[246]{bib:HiemenzLodge}.
    \begin{itemize}
        \item It seems that all subsequent results about the WLC differ significantly from our treatment in class.
    \end{itemize}
    \item Radius of gyration content and derivations of the in-class equation, as well as relation to the textbook's definition of persistence length.
    \item The distribution of most probable sizes is Gaussian.
    \begin{itemize}
        \item Derivation of the binomial is \emph{exactly} as I rationalized!
        \item \textcite[255-56]{bib:HiemenzLodge} also derives the 3D Gaussian from the binomial using only first principles.
        \item Problem 6.20 shows that the Gaussian is useful for $N$ as small as 10.
        \item It is important to keep in mind that the Gaussian is a continuous approximation for a discrete function; thus, its values can never be \emph{exactly} accurate, and some (e.g., finite probabilities at lengths greater than the contour length) may be meaningless.
        \item The pure Gaussian tells us that the most probable \emph{vector} value $\mathbf{R}$ is 0, but multiplying by $4\pi R^2$ reveals that the most probable \emph{scalar} value $R$ is finite.
        \begin{itemize}
            \item Analogous to normalization of the \emph{s} electrons in the hydrogen atom and molecular speeds in the Maxwell-Boltzmann distribution.
        \end{itemize}
    \end{itemize}
    \item \textbf{Dilute} (solution): A solution for which the concentration $c$ is much less than the critical concentration $c^*$.
    \item A note on excluded volume.
    \begin{itemize}
        \item Tends to \emph{expand} the coil, because the polymer feels like it has less space available to it.
        \item Two cases where excluded volume disappears: Polymer melt and $\theta$ solvent.
        \begin{itemize}
            \item Polymer melt: Two monomers still cannot occupy the same space, but there is no benefit to expanding the coil because the adjacent space is \emph{already} surrounded by monomers.
            \begin{itemize}
                \item Conjectured by Flory long ago, and confirmed by SANS in the 1970s.
            \end{itemize}
            \item $\theta$ solvent: A non-very-good solvent that makes monomers prefer to be near each other than far away.
        \end{itemize}
    \end{itemize}
\end{itemize}



\section{Chapter 7: Thermodynamics of Polymer Mixtures}
\emph{From \textcite{bib:HiemenzLodge}.}
\begin{itemize}
    \item \marginnote{10/12:}\textcite[310-14]{bib:HiemenzLodge} has some additional information on the derivation of the full Flory free energy and its four constituent terms.
\end{itemize}




\end{document}