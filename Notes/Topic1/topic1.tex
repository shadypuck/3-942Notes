\documentclass[../notes.tex]{subfiles}

\pagestyle{main}
\renewcommand{\chaptermark}[1]{\markboth{\chaptername\ \thechapter\ (#1)}{}}

\begin{document}




\chapter{The Macromolecule}
\section{The Macromolecule}
\begin{itemize}
    \item \marginnote{9/9:}Pat Doyle begins; he will teach the next three lectures.
    \begin{itemize}
        \item They've broken the class up into modules taught alternatingly.
        \item Aside: Alfredo has taught this course 10 times; Doyle never has (he's replacing Greg Rutledge this year).
    \end{itemize}
    \item Announcements.
    \begin{itemize}
        \item Slides and homework 1 have been posted.
        \item Slides should be posted before lecture, but may not be posted much before.
    \end{itemize}
    \item Lecture outline.
    \begin{itemize}
        \item Connectivity: Thermodynamic limit, architecture, and molecular weight.
        \item Configurations: Structural, chemical, stereo, and geometrical isomerism.
        \item Conformations: Rotational isomeric states.
    \end{itemize}
    \item \textbf{Connectivity}: The joining of small parts (monomers) into larger molecules (polymers).
    \item \textbf{Degree of polymerization}: The number of repeat units in a polymer. \emph{Denoted by} $\bm{N}$.
    \begin{itemize}
        \item Derivable from the molecular weight(s).
    \end{itemize}
    \item Example: Polyethylene.
    \begin{itemize}
        \item In this course, vinyl examples will be our workhorses, but we will "riff off of them" to other polymer types.
        \item The \textbf{repeat unit} here is \ce{CH2CH2}, consistent with the IUPAC nomenclature of poly\emph{ethylene}.
        \item Ethylene is also the monomer.
        \item The end groups do look different, but the \textbf{thermodynamic limit} addresses them.
    \end{itemize}
    \item \textbf{Repeat unit}: A part of a polymer whose repetition would produce the complete polymer chain (except for the end groups) by linking the repeat units together successively along a chain.
    \item \textbf{Thermodynamic limit}: The finding that as $N\to\infty$, the end group chemistry matters less. \emph{Also known as} \textbf{polymer limit}.
    \begin{itemize}
        \item The thermodynamic limit is also sometimes discussed in the context of statistical mechanics, where collective behavior also matters more than individual or picoscale.
    \end{itemize}
    \item \textbf{Glass transition temperature}: The temperature at which a substance will go from brittle to kind of rubbery. \emph{Denoted by} $\bm{T_\textbf{g}}$.
    \item \textbf{Flory-Fox correlation}: The simple model that the $\Tg$ of a polymer asymptotically approaches a limit $\Tg(M\to\infty)$ for higher and higher molecular weights at some empirically derived rate $A$. \emph{Given by}
    \begin{equation*}
        \Tg(\Mn) := \Tg(M\to\infty)-\frac{A}{\Mn}
    \end{equation*}
    \begin{itemize}
        \item Alkane series often obey this simple $1/x$ relation.
    \end{itemize}
    \item Another example of thermodynamic limits: Regardless of polymer structure, a power law defines polymer viscosity.
    \begin{figure}[h!]
        \centering
        \includegraphics[width=0.3\linewidth]{viscosityScaling.png}
        \caption{Polymer viscosity scales universally by power laws.}
        \label{fig:viscosityScaling}
    \end{figure}
    \begin{itemize}
        \item Namely, polymer viscosity increases up to a critical concentration $c^*$ at a slpe of 1, then to an entanglement concentration $c_e$ at a slope of 2, and then at a slope of $14/3$ past that.
        \item Thus, there are three universal scaling regimes.
        \item A log-log plot is used to show power-law scalings, like in high school trigonometry!
        \item Aside: Power laws are present everywhere once you get big enough, down to the volume of our lungs relative to our bodies in bigger and bigger animals.
    \end{itemize}
    \item This class isn't super stringent on nomenclature, but it's good to know terms for when we read papers (Table \ref{tab:copolymerNomenclature}).
    \begin{table}[h!]
        \centering
        \includegraphics[width=0.9\linewidth]{copolymerNomenclature.png}
        \caption{IUPAC nomenclature of copolymers.}
        \label{tab:copolymerNomenclature}
    \end{table}
    \begin{itemize}
        \item Alfredo will talk about block copolymers a good bit!
    \end{itemize}
    \item Polymer architectures.
    \begin{itemize}
        \item Linear polymers.
        \begin{itemize}
            \item Polyrotoxanes and other supramolecular assemblies can have interesting properties. Example: Catenated DNA!
        \end{itemize}
        \item Cross-linked systems (nice gelation).
        \item Branched polymers can have a single monomer, or multiple as in graft (Table \ref{tab:copolymerNomenclature}).
        \item Dendritic polymers have different generations with regular branching for very dense structures.
        \item There are a few more classes, as well.
    \end{itemize}
    \item We'll now discuss some nomenclature on molecular weight.
    \item Aside: Other than nature, synthetic chemists cannot make dispersity 1 polymers; "polymer chemists aren't gods, despite some thinking they are."
    \item \textbf{$\bm{i}$-mer}: A segment of a polymer with degree of polymeriation $i$.
    \item $\bm{M_i}$: The molecular weight of the $i$-mer. \emph{Given by}
    \begin{equation*}
        M_i := iM_0
    \end{equation*}
    \item $\bm{M_0}$: The molecular weight of the repeat unit in a polymer chain.
    \item $\bm{n_i}$: The number of $i$-mers.
    \item \textbf{Number fraction} (of an $i$-mer): The probability of picking an $i$-mer out of solution when picking a chain. \emph{Denoted by} $\bm{x_i}$. \emph{Given by}
    \begin{equation*}
        x_i := \frac{n_i}{\sum_in_i}
    \end{equation*}
    \item \textbf{Weight fraction} (of an $i$-mer): The probability that a repeat unit picked out of solution belongs to an $i$-mer. \emph{Also known as} \textbf{weight fraction}. \emph{Denoted by} $\bm{w_i}$. \emph{Given by}
    \begin{equation*}
        w_i := \frac{n_iM_i}{\sum_in_iM_i}
        = \frac{n_i(iM_0)}{\sum_in_i(iM_0)}
        = \frac{in_i}{\sum_iin_i}
    \end{equation*}
    \begin{itemize}
        \item Note that $in_i$ is the total number of monomers in the $i$-mer, and $\sum_iin_i$ is the total number of monomers in solution.
    \end{itemize}
    \item \textbf{Number-average molecular weight}: The arithmetic mean of the molecular masses of the individual macromolecules. \emph{Denoted by} $\bm{M_\textbf{n}}$. \emph{Given by}
    \begin{equation*}
        \Mn := \sum_ix_iM_i = M_0\cdot\frac{\sum_iin_i}{\sum_in_i}
    \end{equation*}
    \item \textbf{Weight average molecular weight}: A mesurement of molecular weight that gives more contribution to higher-weight molecules. \emph{Denoted by} $\bm{M_\textbf{w}}$. \emph{Given by}
    \begin{equation*}
        \Mw := \sum_iw_iM_i = M_0\cdot\frac{\sum_ii^2n_i}{\sum_iin_i}
    \end{equation*}
    \item Both $\Mn$ and $\Mw$ look like moments of a distribution (i.e., first and second moment).
    \begin{itemize}
        \item We could generalize even more, but we don't need to.
        \item However, to figure out if we have a tight or wide distribution, we often look at ratios of our moments. This leads to the following definition.
    \end{itemize}
    \item \textbf{Dispersity}: A measure of the breadth of the distribution of fragment molecular weights in a polymer sample. \emph{Also known as} \textbf{polydispersity index}, \textbf{PDI}. \emph{Denoted by} $\tikz{
        \node[inner sep=0pt]{$\bm{D}$};
        \draw [thick] (-0.12,0) -- ++(0.12,0);
    }$. \emph{Given by}
    \begin{equation*}
        \Dstroke := \frac{M_w}{M_n} = \frac{\text{second moment}}{\text{first moment}}
    \end{equation*}
    \item In \textcite{bib:HiemenzLodge}, they also derive the \textbf{variance}. You don't need to worry about the math, though.
    \item \textbf{Variance}: Another measure of the breadth of the distribution of fragment molecular weights. \emph{Denoted by} $\bm{\sigma^2}$. \emph{Given by}
    \begin{equation*}
        \sigma^2 := M_n^2[\Dstroke-1]
    \end{equation*}
    \item \textbf{Schultz-Zimm distribution}: An idealized mathematical model for polymer molecular weight distribution.
    \item With these definitions, we can now do homework problem number 1!
    \item Experimental techniques to measure molecular weight.
    \begin{itemize}
        \item Size exclusion chromatography.
        \item Osmotic pressure.
        \item End group analysis.
        \item Light scattering techniques.
        \begin{itemize}
            \item More sensitive to higher moments.
        \end{itemize}
    \end{itemize}
    \item We now move onto polymer configurations.
    \item \textbf{Configurations}: The way things are connected or bonded together.
    \begin{itemize}
        \item Physicists and chemists have many competing definitions of "configurations," but the one above is what we'll use in this class. Essentially, think of it as a synonym for constitutional isomerism.
        \item Under this definition, you have to \emph{break bonds} to create a new configuration.
        \item We are \emph{not} yet talking about rotamers (what we'll call \textbf{conformations}). As you make your polymers longer and longer, the conformational space you can explore gets bigger.
    \end{itemize}
    \item There are 3 main types of structural isomers (i.e., configurations): \textbf{Positional}, \textbf{stereo}, and \textbf{geometric} isomers.
    \item \textbf{Positional} (isomers): Changing connectivity.
    \item \textbf{Stereoisomers}: Related to chiral centers.
    \item \textbf{Geometric} (isomers): Related to double bonds.
    \item More on positional isomers.
    \begin{figure}[H]
        \centering
        \includegraphics[width=0.6\linewidth]{headTail.png}
        \caption{Head-head and head-tail monomer addition.}
        \label{fig:headTail}
    \end{figure}
    \begin{itemize}
        \item \textbf{Head-head} vs. \textbf{head-tail} bonding in vinyl monomers.
        \item Head-tail is more common, and differences can change the $\Tg$ substantially.
    \end{itemize}
    \item \textbf{Head-head} (orientation): Monomer addition wherein the substituted carbon attaches to the growing chain end. \emph{Also known as} \textbf{head-to-head}.
    \item \textbf{Head-tail} (orientation): Monomer addition wherein the \emph{un}substituted carbon attaches to the growing chain end. \emph{Also known as} \textbf{head-to-tail}.
    \item More on stereoisomers.
    \item Doyle reviews \textbf{chirality}, \textbf{rectus} vs. \textbf{sinister}, and the \textbf{Cahn-Ingold-Prelog nomenclature}.
    \item \textbf{Pseudochiral} (center): A chiral center where two of the substituents are identical \emph{except} for chirality.
    \begin{itemize}
        \item In this case, CIP nomenclature gives priority to the branch with more R chiral centers.
    \end{itemize}
    \item Chiral centers in polymers give rise to \textbf{tacticity}.
    \begin{figure}[h!]
        \centering
        \includegraphics[width=0.7\linewidth]{tacticity.png}
        \caption{Tacticity.}
        \label{fig:tacticity}
    \end{figure}
    \begin{itemize}
        \item There are \textbf{meso diads} and \textbf{racemic diads}; which ones you have determine if the polymer is \textbf{isotactic}, \textbf{syndiotactic}, or \textbf{atactic}.
    \end{itemize}
    \item \textbf{Meso} (diad): Two adjacent chiral centers with a local plane of symmetry halfway between them.
    \item \textbf{Racemic} (diad): Two adjacent chiral centers with\emph{out} a local plane of symmetry halfway between them.
    \item \textbf{Isotactic} (polymer): A polymer containing only meso diads.
    \item \textbf{Syndiotactic} (polymer): A polymer containing only racemic diads.
    \item \textbf{Atactic} (polymer): A polymer containing both meso and racemic diads.
    \item Example: Atactic polystyrene tends to be more amorphous than syndiotactic or isotactic polymers, which can be semicrystalline.
    \begin{itemize}
        \item Nobel prize (1963) to Ziegler and Natta for a catalyst generating isotactic polystyrene (PS-it).
        \begin{itemize}
            \item Note that the initial Ziegler-Natta catalysts weren't metallocenes! The introduction of these types only came later.
        \end{itemize}
        \item Syndiotactic polystyrene (PS-st) came later in 1986 and had superior properties.
        \begin{itemize}
            \item PS-st crystallizes an order of magnitude faster than PS-it; has half the entanglement molecular weight; and is commonly used today in auto parts, electronics, and medical equipment.
        \end{itemize}
    \end{itemize}
    \item Tacticity is often measured by certain splittings (or their absence) in \ce{{}^1H} NMR.
    \begin{itemize}
        \item Isotactic polymers put the geminal methylene protons into distinct chemical environments; syndiotactic polymers do not. Atactic polymers will have a mix of both, and the mix can be quantified with integration.
        \item \ce{{}^13C} NMR can be used, too.
    \end{itemize}
    \item More on geometric isomers.
    \begin{itemize}
        \item Example: Polybutadiene can be formed \emph{trans} or \emph{cis}, as guided by a catalyst.
        \item Natural rubber is \emph{cis}-1,4-polyisoprene. Other types of tree sap can give \emph{trans}-1,4-polyisoprene.
    \end{itemize}
    \item We now move onto polymer conformations.
    \item \textbf{Conformations}: The spacial arrangements possible (or "probable," taking energy into account) for a polymer.
    \begin{itemize}
        \item No bonds are \emph{broken} here, just rotated.
    \end{itemize}
    \item Reviews \textbf{Newman projections}.
    \begin{itemize}
        \item \textbf{Staggered} configuration is defined as \ang{0}.
        \item \textbf{Eclipsed} configuration then starts at \ang{60}.
        \item There are energy pentalties to being in different conformations.
        \begin{itemize}
            \item As one example, eclipsed is higher energy than staggered due to sterics.
            \item Generally sinusoidal relation in a plot of potential energy $V$ against dihedral angle $\theta$.
            \item The energy difference between rotamers is approximately $3\kB T$, which is not huge but big enough that the system will spend most of its time in the valleys. Each "valley" is a \textbf{conformer}.
            \item So then since probability is proportional to $\e[-V/\kB T]$, the probability that a molecule will be staggered is greater than that it will be eclipsed.
        \end{itemize}
        \item In molecules longer than ethane, we break degeneracy of the valleys.
        \item The rapid growth of conformers: Ethane has 3 conformers. Propane has $3^2$. Butane has $3^3$. Decane has $3^10$. Polyethylene with $N=10^5$ already has on the order of \num{e47000} possible conformers, a huge conformational space.
        \begin{itemize}
            \item This is because each bond has 3 valleys!
            \item Many of the models we'll develop are ways of enumerating these conformations in relation to some higher-order measurement of the polymer, such as the \textbf{n-band difference}.
            \item Polymers are indeed often moderately sized coils rather than fully stretched out rods.
            \item Example: 166 kbp DNA (approximately 684 \textbf{Kuhn steps}, discussed next lecture) can be videotaped moving around, and it never fully elongates.
            \item Stretched out polymers shrink back over some characteristic time.
        \end{itemize}
        \item Polymers with high degrees of polymerization result in many possible conformations without breaking bonds --- this is what we'll discuss in the next two lectures!
    \end{itemize}
    \item A good converions to keep in mind: $\SI[per-mode=symbol]{2.5}{\kilo\joule\per\mole}\approx 1\kB T$.
    \item Today is probably the most jampacked bits and pieces day; other lectures will be more focused, but this is important background.
\end{itemize}



\section{Conformation: Ideal Chains}
\begin{itemize}
    \item \marginnote{9/11:}Lecture outline.
    \begin{itemize}
        \item Conformation (degrees of freedom).
        \item Ideal chain models.
        \item Entropic elasticity.
    \end{itemize}
    \item Many material properties we care about are correlated with the size of the polymer.
    \item This size is measured by a vector $\mathbf{R}$ that goes from one end of the polymer to the other.
    \begin{figure}[h!]
        \centering
        \includegraphics[width=0.2\linewidth]{Rvector.png}
        \caption{End-to-end polymer vector.}
        \label{fig:Rvector}
    \end{figure}
    \begin{itemize}
        \item We define $\mathbf{R}$ as the sum of all consituent bond vectors $\mathbf{l}$ (end-to-end of each chemical bond along the backbone). Symbolically,
        \begin{equation*}
            \mathbf{R} := \sum_{i=1}^n\mathbf{l}_i
        \end{equation*}
        \item Note that $\mathbf{R}$ has length $|\mathbf{R}|=R$.
        \item Today, we will look at several models that can be used to calculate the expected length of this vector, $\prb{R}$.
    \end{itemize}
    \pagebreak
    \item Freely Jointed Chain (FJC) model.
    \begin{figure}[h!]
        \centering
        \includegraphics[width=0.15\linewidth]{joint.png}
        \caption{A joint in a polymer chain.}
        \label{fig:joint}
    \end{figure}
    \begin{itemize}
        \item In this model, there are no restrictions on how adjacent bonds rotate relative to each other. Rather, this is just a random walk. In effect, this means that there are no energy barriers and no excluded volume.
        \begin{itemize}
            \item It follows that $\prb{R}=0$.
        \end{itemize}
        \item However, while $\prb{R}=0$, we have
        \begin{align*}
            \prb{R^2} &= \prb{\sum_{i=1}^n\mathbf{l}_i\cdot\sum_{j=1}^n\mathbf{l}_j}\\
            &= \sum_{i=1}^n\sum_{j=1}^n\prb{\mathbf{l}_i\cdot\mathbf{l}_j}\\
            &= \sum_{i=1}^n\prb{\mathbf{l}_i\cdot\mathbf{l}_i}+\sum_{i=1}^n\sum_{\substack{j=1\\j\neq i}}^n\prb{\mathbf{l}_i\cdot\mathbf{l}_j}\\
            &= nl^2+0
        \end{align*}
        \begin{itemize}
            \item The second term goes to zero as $n\to\infty$ because there is no correlation among segments (i.e., they are randomly oriented).
            \item This gives us the following important scaling law.
            \begin{equation*}
                \prb{R^2}^{1/2} \propto n^{1/2}
            \end{equation*}
            \item This means that scaling is consistent with the polymer taking a "coil-like" conformation.
        \end{itemize}
        \item This combined with the fact that the polymer's fully stetched length is $nl$ gives us the following expression for the number of backbone bonds in a polymer.
        \begin{equation*}
            \frac{(\text{fully stretched length})^2}{\prb{R^2}} = n
        \end{equation*}
    \end{itemize}
    \item Note that \textcite{bib:HiemenzLodge} uses $\theta$ for the \emph{complement} of bond angle, as in Figure \ref{fig:joint}. Other texts may use a different convention.
    \item Now, let's refine the FJC by accounting for nearest neighbor correlations.
    \item First, we'll look at a polymer with only two segments (i.e., $n=2$)
    \begin{itemize}
        \item Suppose every joint is \emph{fixed} at complementary angle $\theta$, but there is no energy penalty to rotate in $\phi$.
        \item From Figure \ref{fig:joint}, trigonometry tells us that
        \begin{equation*}
            \mathbf{l}_i\cdot\mathbf{l}_{i+1} = l^2\cos\theta
        \end{equation*}
        \item Thus, under the conditons of this model,
        \begin{align*}
            \prb{R^2} &= 2l^2+\sum_{i=1}^2\sum_{\substack{j=1\\j\neq i}}^2\prb{\mathbf{l}_i\cdot\mathbf{l}_j}\\
            &= 2l^2+2\prb{l^2\cos\theta}\\
            &= nl^2(1+\cos\theta)
        \end{align*}
        \item The important takeaway is that with this chemical realism, the chain is bigger than in the previous model!
        \begin{itemize}
            \item It is also noteworthy that the $nl^2$ scaling relation is retained.
        \end{itemize}
        \item This is a precursor to the freely rotating chain, where we can rotate in $\phi$ but not in $\theta$.
    \end{itemize}
    \item Let's now look at the full Freely Rotating Chain (FRC) model.
    \begin{itemize}
        \item As we elongate the chain, there is a slow decay of "memory of correlation" since $\phi$ rotates freely. Eventually (see \textcite[239-40]{bib:HiemenzLodge} for the derivation), we asymptote to
        \begin{equation*}
            \prb{R^2} = nl^2\underbrace{\left( \frac{1+\cos\theta}{1-\cos\theta} \right)}_{C_n}
        \end{equation*}
    where $C_n$ may be an empirically derived \textbf{characteristic ratio}.
    \end{itemize}
    \item Model 3: Symmetric hindered rotations.
    \begin{itemize}
        \item Recall from last class that certain rotational conformations have lower energies than others.
        \item As such, we can give a Boltzman weighting to the energetic valleys.
        \item Thus, we reevaluate our hindered rotations with a Boltzmann weighting and the following expression crashes out of the math.
        \begin{equation*}
            \prb{R^2} = nl^2\underbrace{\left( \frac{1+\cos\theta}{1-\cos\theta} \right)\left( \frac{1+\prb{\cos\phi}}{1-\prb{\cos\phi}} \right)}_{C_\infty}
        \end{equation*}
        \begin{itemize}
            \item Note that it's still just $nl^2$ times a constant!
        \end{itemize}
        \item The \textbf{characteristic ratio} $C_\infty$ can be calculated for models or obtained from experiments. The following rearranged definition is also important.
        \begin{equation*}
            C_\infty := \frac{\prb{R^2}_0}{nl^2}
        \end{equation*}
        \item Alert: Be aware of sign changes due to different conventions for $\theta$ and $\phi$ in different texts!
    \end{itemize}
    \item There are tables of characteristic ratios in both \textcite{bib:HiemenzLodge} and \textcite{bib:RubinsteinColby}.
    \begin{table}[H]
        \centering
        \includegraphics[width=0.65\linewidth]{CinftyCommon.png}
        \caption{$C_\infty$ values for common polymers at \SI{413}{\kelvin}.}
        \label{tab:CinftyCommon}
    \end{table}
    \begin{itemize}
        \item $C_\infty$ gets bigger with bigger side chains.
        \item Typical range is 5-10; can go up to 20 or higher, though.
    \end{itemize}
    \item Example: What is the size of a polyethylene molecule at \SI{413}{\kelvin} and having molecular weight $\SI[per-mode=symbol]{e4}{\gram\per\mole}$?
    \begin{itemize}
        \item Approach: We want to find $\prb{R^2}$ and take its square root; that will be our answer.
        \item We can look up that the length $l$ of a typical \ce{C-C} bond is \SI{0.154}{\nano\meter}.
        \item Based on the molecular weight and the known weight of the ethylene (\ce{CH2CH2}) repeat unit,
        \begin{equation*}
            N = \frac{\SI[per-mode=symbol]{e4}{\gram\per\mole}}{\SI[per-mode=symbol]{28}{\gram\per\mole}} \approx 357.1
        \end{equation*}
        \item Because there are two carbon-carbon bonds per repeat unit, $n=2N$.
        \item Thus,
        \begin{align*}
            \prb{R} &= \prb{R^2}^{1/2}\\
            &= [nl^2C_\infty]^{1/2}\\
            &= \left[ (2\cdot 357.1)(\SI{0.154}{\nano\meter})^2(7.4) \right]^{1/2}\\
            \prb{R} &\approx \SI{11}{\nano\meter}
        \end{align*}
    \end{itemize}
    \item Let's compare the answer in the above example to the straight chain estimation.
    \begin{figure}[h!]
        \centering
        \includegraphics[width=0.22\linewidth]{straightChainCC.png}
        \caption{Straight chain estimation of a polyolefin.}
        \label{fig:straightChainCC}
    \end{figure}
    \begin{itemize}
        \item In this case,
        \begin{equation*}
            R_\text{max} = nl\cos(\theta/2)
        \end{equation*}
        \item We can look up that for a typical \ce{C-C} bond, $\theta=\ang{180}-\ang{109.5}=\ang{70.5}$.
        \item Thus,
        \begin{align*}
            R_\text{max} &= nl\cos(\theta/2)\\
            &= (2\cdot 357.1)(\SI{0.154}{\nano\meter})\cos(70.5/2)\\
            R_\text{max} &\approx \SI{90}{\nano\meter}
        \end{align*}
    \end{itemize}
    \item This calculation is a bit tedious, so Kuhn refined the coarse grained FJC model to be computationally simpler.
    \item Model 4: Kuhn's Equivalent Chain.
    \begin{figure}[H]
        \centering
        \includegraphics[width=0.3\linewidth]{KuhnSteps.png}
        \caption{Kuhn steps.}
        \label{fig:KuhnSteps}
    \end{figure}
    \begin{itemize}
        \item Building off of models 3 and 1, let's postulate the existence of an FJC along our polymer. This FJC will have $N$ steps of length $l_k$, where $l_k$ is a Kuhn step.
        \item This gives us two variables to define. We thus need two equations by which to define them.
        \begin{itemize}
            \item Equation 1: By the universal scaling law,
            \begin{equation*}
                Nl_k^2 = \prb{R^2} = C_\infty nl^2
            \end{equation*}
            \item Equation 2: If the total length of the chain is $R_\text{max}$, and the chain is being broken up into $N$ Kuhn steps each of length $l_k$, then
            \begin{equation*}
                R_\text{max} = Nl_k
            \end{equation*}
        \end{itemize}
        \item By solving this system of equations, we can then define $N$ and $l_k$ purely in terms of prevoiusly derived variables.
    \end{itemize}
    \item \textbf{Kuhn step}: A subsegment of a polymer chain with length defined as follows. \emph{Denoted by} $\bm{l_k}$. \emph{Given by}
    \begin{equation*}
        l_k := \frac{\prb{R^2}}{R_\text{max}} = \frac{C_\infty nl^2}{R_\text{max}}
    \end{equation*}
    \item \textbf{Number of Kuhn steps}: The number of Kuhn steps in a polymer chain. \emph{Denoted by} $\bm{N}$. \emph{Given by}
    \begin{equation*}
        N := \frac{R_\text{max}}{l_k} = \frac{R_\text{max}^2}{C_\infty nl^2}
    \end{equation*}
    \item Example: In the case of a fully elongated carbon-carbon chain (Figure \ref{fig:straightChainCC}), the number of Kuhn steps is
    \begin{equation*}
        l_k = \frac{C_\infty nl^2}{R_\text{max}}
        = \frac{C_\infty nl^2}{nl\cos(\theta/2)}
        = \frac{C_\infty l}{\cos(\theta/2)}
    \end{equation*}
    \item Note that the symbols $R_\text{max}$ and $L$ will be used interchangeably for the straight-chain length of a polymer.
    \item $R_\text{max}$ largely depends on the chemistry of the polymer (e.g., specific atoms' bond angles).
    \item We'll now look at some models for "stiff" chains, such as dsDNA or microtubules.
    \begin{itemize}
        \item These tend to have even higher $C_\infty$ values.
        \item Note that twists in the chain are \emph{much} bigger than individual nucleobases.
        \item Molecular simulations of 75 bp dsDNA shows barely any bending. Indeed, there is a high correlation between end vectors even though they are very far away.
    \end{itemize}
    \item A model for very stiff polymers: The Worm-Like Chain (WLC).
    \begin{figure}[H]
        \centering
        \includegraphics[width=0.32\linewidth]{WLC.png}
        \caption{Worm-like chain.}
        \label{fig:WLC}
    \end{figure}
    \begin{itemize}
        \item Represent a segment as an infinitesimal elastic rod.
        \item This rod has a contour you can be some point $s$ along, where $s\in[0,L]$.
        \item The direction of the rod is defined by a tangent vector $\mathbf{t}(s)$.
        \item We also give the rod some bending energy $U_b$, related to how the tangent changes as we go along $s$. But physically, this is just curvature squared. To get the right units, we throw in a bending stiffness parameter $\kappa_b$, which incorporates some chemical / molecular details. Throw in a $1/2$ from mechanics definitions, and we get
        \begin{equation*}
            U_b = \frac{1}{2}\kappa_b\int_0^L\left( \pdv{\mathbf{t}}{s} \right)^2\dd{s}
        \end{equation*}
        \item An implication of this is that correlation decays exponentially per
        \begin{equation*}
            \prb{\mathbf{t}(0)\cdot\mathbf{t}(s)} = \e[-s\kB T/\kappa_b]
        \end{equation*}
        \begin{itemize}
            \item I.e., you use memory of how oriented you are in one part of the chain when you get farther away from that point.
        \end{itemize}
        \item Since $\kappa_b/\kB T$ has units of length by dimensional analysis, we can define it to be the \textbf{persistence length}.
        \item From this tangent correlation function, we can calculate many interesting properties --- including the mean squared end-to-end distance!
        \begin{equation*}
            \prb{R^2} = 2aL\left[ 1-\frac{a}{L}\left( 1-\e[-L/a] \right) \right]
        \end{equation*}
        \begin{itemize}
            \item The calculation is complicated, so Dolye skips it.
        \end{itemize}
        \item This equation reveals some interesting polymer behavior in two limits: That of \emph{long} and \emph{short} stiff polymers.
        \begin{itemize}
            \item When the polymer gets long, $a/L\to 0$ and $\prb{R^2}\to 2aL$.
            \item When the polymer gets short, $a/L\to\infty$ and $\prb{R^2}\to L^2$.
            \begin{itemize}
                \item This makes intuitive sense as if it's short, it should be roughtly straight and have end-to-end distance approximately equal to its length!
                \item Note that in real life, there \emph{are} polymers where the persistence length is longer than the length of the polymer! These behave like rigid rods.
            \end{itemize}
        \end{itemize}
        \item Lastly, it follows from definition of Kuhn steps that
        \begin{equation*}
            2aL = \prb{R^2} = Nl_k^2 = \frac{R_\text{max}}{l_k}\cdot l_k^2 = Ll_k
        \end{equation*}
        \begin{itemize}
            \item Thus, the Kuhn length is twice the persistence length! Symbolically,
            \begin{equation*}
                2a = L
            \end{equation*}
        \end{itemize}
    \end{itemize}
    \item \textbf{Persistence length}: A characteristic length over which a stiff polymer loses memory of its orientation along other parts of the chain. \emph{Denoted by} $\bm{a}$. \emph{Given by}
    \begin{equation*}
        a := \frac{\kappa_b}{\kB T}
    \end{equation*}
    \item In conclusion, two biggest models to remember: Kuhn model (rigid steps) and WLC (continuum approximation with persistence length for semi-rigid chains).
    \item Example: Actin has a persistence length of \SI{10}{\micro\meter}; since most cells are smaller than this, actin is funtionally a rigid rod within a cellular context.
    \item Example: Measuring persistence length of dsDNA.
    \begin{itemize}
        \item Adsorb DNA onto a surface that loosely binds it, so that it can still move around but won't fall off.
        \item Then look at 1000s of strands next to each other and calculate tangent lengths!
        \item Reference: \textcite{bib:DNApersisLength}.
    \end{itemize}
    \item Example: Bottlebrush polymers in cartilage.
    \begin{itemize}
        \item These have highly charged side chains, but fewer with age.
        \item This causes more bending.
        \item We can observe this with atomic force microscopy.
    \end{itemize}
    \item Example: As you increase the concentration of salt in solution, you shrink the \textbf{Debye length} and also the persistence length.
    \begin{itemize}
        \item This modifies the effect of charges on dsDNA.
    \end{itemize}
    \item Example: Actin cytoskeleton filaments.
    \begin{itemize}
        \item Made out of polymerized protein subunits.
        \item A very thin polymer, biologicaly speaking.
        \item Very long persistence length, as mentioned earlier.
        \item As cells move, they push actin against the cell membrane to distort it! This works because actin is a very rigid rod, so rigid that it can overcome the membrane pressure.
    \end{itemize}
    \item Example: Conjugated polymers.
    \begin{itemize}
        \item They calculated the persistence length using DFT, and then measured it experimentally.
        \item Some of their polymers are stiffer tha dsDNA!
        \item Rotation around one particular engineered bond is used to estimate persistence length.
    \end{itemize}
    \item \textbf{Flexible} (polymer): A polymer for which chain length is much greater than persistence length.
    \item \textbf{Semi-flexible} (polymer): A polymer for which chain length is approximately equal to persistence length.
    \item \textbf{Rod-like} (polymer): A polymer for which chain length is much shorter than persistence length.
    \item Summary of ideal chains.
    \begin{itemize}
        \item All chains show a similar, universal scaling relation that $R\approx N^{1/2}$.
        \item For ideal chains, local interactions set the rigidity length scale and excluded volume is not significant.
        \item This approximation is ok for dilute solutions at \textbf{theta conditions} and polymer \textbf{melts}.
    \end{itemize}
    \item \textbf{Theta condition}: When you've essentially turned off excluded volume for the chain.
    \item \textbf{Polymer melt}: A condition in which the polymer is essentially in a solution of itself.
    \item Nomenclature for polymer solution regimes.
    \begin{figure}[H]
        \centering
        \includegraphics[width=0.2\linewidth]{hardPervaded.png}
        \caption{The volume occupied by polymers.}
        \label{fig:hardPervaded}
    \end{figure}
    \item \textbf{Hard volume} (of a polymer): The volume occupied by the chain, where each repeat unit is considered to occupy a sphere with radius equal to the bond length and the polymer is the sum of these "beads" touching each other. \emph{Denoted by} $\bm{v}$. \emph{Given by}
    \begin{equation*}
        v \propto nl^3
    \end{equation*}
    \begin{itemize}
        \item The volume of each bead is thus on the order of the bond length.
    \end{itemize}
    \item \textbf{Pervaded volume} (of a polymer): The sphere encapsulating the volume in which the polymer chain is \emph{expected} to move around. \emph{Denoted by} $\bm{V}$. \emph{Given by}
    \begin{equation*}
        V \propto \prb{\mathbf{R}^2}^{3/2} \propto n^{3/2}l^3
    \end{equation*}
    \item From this, we can see that
    \begin{equation*}
        \frac{v}{V} \propto n^{-1/2}
    \end{equation*}
    \begin{itemize}
        \item It follows that the pervaded volume is mostly empty as $n$ becomes large.
    \end{itemize}
    \item \textbf{Critical concentration}: The concentration at which all polymers in solution can "see" each other. \emph{Denoted by} $\bm{c^*}$.
    \begin{figure}[h!]
        \centering
        \includegraphics[width=0.4\linewidth]{concentrationRegimes.png}
        \caption{Concentration regimes.}
        \label{fig:concentrationRegimes}
    \end{figure}
    \begin{itemize}
        \item In dilute solutions, polymers will all be coiled up within their own pervaded volume and will not interact.
        \item When they reach the critical concentration, all of the pervaded volumes are essentially touching each other.
        \item Past the critical concentration, we get entanglement and no longer see individual polymer coils in their own pervaded volume.
        \item This is the same $c^*$ as in Figure \ref{fig:viscosityScaling}!
        \begin{itemize}
            \item Unifying implication: These concentrations are related to pervaded volume, which is related to expected size.
        \end{itemize}
    \end{itemize}
    \item Next time.
    \begin{itemize}
        \item Adding in excluded volume, which will wrap up our discussion on chains.
        \item Then Alfredo on thermodynamics of interactions.
    \end{itemize}
\end{itemize}




\end{document}