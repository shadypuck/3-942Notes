\documentclass[../notes.tex]{subfiles}

\pagestyle{main}
\renewcommand{\chaptermark}[1]{\markboth{\chaptername\ \thechapter\ (#1)}{}}
\setcounter{chapter}{2}

\begin{document}




\chapter{Dilute Solutions}
\section{Intrinsic Viscosity - 1}
\begin{itemize}
    \item \marginnote{10/7:}Announcements.
    \begin{itemize}
        \item Grade our HWs on Canvas.
        \item The profs are grading the first quiz.
        \item PSet 3 will be posted tomorrow morning.
        \begin{itemize}
            \item After today's lecture, we'll have a lot of fodder to get going on it!
        \end{itemize}
    \end{itemize}
    \item Overview of Topic 3.
    \begin{itemize}
        \item We've built up a lot of theory, but now we want to discuss how we measure the parameters we've introduced.
        \begin{itemize}
            \item Example: Measuring polymer size and conformation.
        \end{itemize}
        \item We'll also touch on why such parameters are important for various material properties.
        \begin{itemize}
            \item Examples of where polymer size and conformation are important: Non-entangled rubber elasticity, shear thickeners, elastic modulus of crosslinked networks, and electrical conductivity.
        \end{itemize}
    \end{itemize}
    \item Upcoming lectures by day.
    \begin{itemize}
        \item Today: Viscometry. What is the viscosity, intrisic viscosity, etc. of a polymer? How rheology tells us stuff about a polymer sample.
        \begin{itemize}
            \item Specifically, we'll relate the viscosity of a dilute polymer-solvent solution to the polymer molecular weight.
        \end{itemize}
        \item Thursday: Standard through state-of-the-art ways to measure the viscosity of a polymer solution.
        \item Next Tuesday: Fractionation (e.g., via GPC), and how it works (based on our theory). Also osmotic pressure.
        \begin{itemize}
            \item Fractionation techniques can provide us the full molecular weight distribution of a polymer sample.
            \item Osmotic pressure can provide information on molecular size and polymer-polymer interactions.
        \end{itemize}
        \item Next Thursday: A high-level discussion of light scattering. Different power sources (e.g., NMR), and how scattering tells us something about the polymer size.
    \end{itemize}
    \item Outline of today's lecture.
    \begin{itemize}
        \item Drag coefficient of a polymer.
        \begin{itemize}
            \item Also referred to (e.g., by the textbook) as the "friction factor."
            \item This number is related to a polymer's conformation.
        \end{itemize}
        \item Draining and free draining models.
        \item The diffusion coefficient, and how molecules randomly move in a solution due to thermal energy.
        \item Intrinsic viscisity.
        \begin{itemize}
            \item This is key, and the first three topics all build up to it.
            \item The intrinsic viscosity has a scaling with the diffusion coefficient?? Did we cover this today?
        \end{itemize}
        \item Mark-Houwink-Sakurada model.
    \end{itemize}
    \item Before we get to the viscosity of dilute solutions of polymers, let's introduce some basic concepts around just plain viscosity. This will basically serve as a "fluid dynamics 101."
    \item To help visualize the following definitions, consider this thought experiment.
    \begin{figure}[h!]
        \centering
        \includegraphics[width=0.3\linewidth]{viscosity.png}
        \caption{Viscosity visualization.}
        \label{fig:viscosity}
    \end{figure}
    \begin{itemize}
        \item Put a liquid between two plates, and move the top one.
        \item You can show (via the fluid equation or \textbf{Navier-Stokes equations}) that you will quickly develop a velocity profile. Molecular friction exerts a force, and the force is related to the \textbf{viscosity}.
        \item Consideration of the viscosity leads us to the other definitions below.
    \end{itemize}
    \item \textbf{Navier-Stokes equations}: The governing equations of fluids.
    \item \textbf{Viscosity}: A relationship which tells us something about when we exert stresses on surfaces as a result of flow, and the relation to how fast we're moving a fluid. \emph{Also known as} \textbf{coefficient of viscosity}, \textbf{shear viscosity}. \emph{Units} \textbf{\si[text-series-to-math]{\pascal\second}}. \emph{Denoted by} $\bm{\eta}$. \emph{Given by}
    \begin{equation*}
        \eta := \frac{\text{shear stress}}{\text{rate of shear}}
    \end{equation*}
    \item \textbf{Shear rate}: How fast we move the plates relative to each other, normalized by the distance between the plates. \emph{Also known as} \textbf{velocity gradient}, \textbf{rate of shear}. \emph{Units} \textbf{\si[text-series-to-math]{\per\second}}. \emph{Denoted by} $\bm{\dot{\gamma}}$.\footnote{"gamma dot"} \emph{Given by}
    \begin{equation*}
        \dot{\gamma} := \frac{U}{H}
    \end{equation*}
    \begin{itemize}
        \item $U$ is the difference in velocity between the top and bottom plates.
        \item $H$ is the distance between the top and bottom plates.
    \end{itemize}
    \item \textbf{Shear stress}: The pressure resulting from the viscous force applied over the area of the moving plate. \emph{Units} \textbf{\si[text-series-to-math]{\pascal}}. \emph{Denoted by} $\bm{\tau}$, $\bm{\tau_{yx}}$. \emph{Given by}
    \begin{equation*}
        \tau := \frac{\text{viscous force}}{\text{area}} = \eta\dot{\gamma}
    \end{equation*}
    \begin{itemize}
        \item The $\tau_{yx}$ notation comes from the fact that stress is a tensor, and the $yx$ element of this tensor is the shear stress.
        \item However, we are not concerned with tensors in this class, so we may think of shear stress as nothing more than a scalar with units of pressure.
    \end{itemize}
    \item \textbf{Newtonian} (fluid): A fluid that satisfies the following condition. \emph{Constraint}
    \begin{equation*}
        \eta \neq \eta(\gamma_\text{tot},\dot{\gamma})
    \end{equation*}
    \begin{itemize}
        \item The above constraint states that for a Newtonian fluid, the viscosity $\eta$ is a constant. In particular, viscosity is \emph{not} a function of the shear strain $\gamma$ or the shear rate $\dot{\gamma}$ (as is the case for some non-Newtonian fluids).
    \end{itemize}
    \item \textbf{Non-Newtonian} (fluid): Any fluid that is not Newtonian.
    \begin{itemize}
        \item This encompases a \emph{broad} range of fluid properties.
        \item Polymers are very non-Newtonian.
    \end{itemize}
    \item This concludes our introduction to the terminology of fluid mechanics.
    \item We now address today's focus, the behavior of dilute ($\phi_2\ll 1$) polymer-solvent solutions at small shear rates ($\dot{\gamma}\ll 1$).
    \begin{figure}[h!]
        \centering
        \begin{subfigure}[b]{0.4\linewidth}
            \centering
            \includegraphics[width=0.95\linewidth]{smallSheara.png}
            \caption{Data on dilute PS solutions.}
            \label{fig:smallSheara}
        \end{subfigure}
        \begin{subfigure}[b]{0.4\linewidth}
            \centering
            \includegraphics[width=0.8\linewidth]{smallShearb.png}
            \caption{Shear thinning, graphically.}
            \label{fig:smallShearb}
        \end{subfigure}
        \caption{Polymer viscosity at varying shear rates.}
        \label{fig:smallShear}
    \end{figure}
    \begin{itemize}
        \item Figure \ref{fig:smallSheara} compiles a bunch of data from the literature on viscosity vs. shear rates.
        \begin{itemize}
            \item The $y$-axis is the viscosity $\eta_p$ of the polymer-solvent solution, normalized by said solution's asymptotic viscosity $\eta_{p0}$ as you go to very small shear rates.
            \item The $x$-axis is the shear rate, again normalized by some constant $\lambda_Z$. Here, we have picked a scaling $\lambda_Z$ that makes our data dimensionless (i.e., 1 on the $x$-axis) and reveals the two important trends that\dots
            \begin{itemize}
                \item Faster shearing ($\lambda_Z\dot{\gamma}>1$) leads to a decrease in viscosity known as \textbf{shear thinning};
                \item Lower shearing ($\Lambda_Z\dot{\gamma}<1$) leads to the asymptotic viscosity.
            \end{itemize}
            \item Below a certain shear rate, we observe Newtonian behavior and "zero shear viscosity."
        \end{itemize}
        \item Figure \ref{fig:smallShearb} sketches the idea of shear thinning.
        \begin{itemize}
            \item \textbf{Zero shear viscosity} occurs at left.
            \begin{itemize}
                \item Note that this regime would more aptly be termed the "low shear viscosity" regime, since we're only asymptotically approaching zero shear.
                \item However, "zero shear viscosity" is what's in the lexicon.
            \end{itemize}
            \item Shear thinning occurs in the middle.
            \item At right, we enter the asymptotic regime wherein most viscosity comes from the solvent, not the polymer.
        \end{itemize}
        \item Reference: \textcite{bib:smallShear}.
    \end{itemize}
    \item \textbf{Shear thinning}: A property of certain fluids wherein viscosity decreases the faster the fluid is sheared (i.e., as the shear rate $\dot{\gamma}$ increases).
    \item \textbf{Zero shear viscosity}: The viscosity under a shear rate sufficiently low that the fluid behaves as if its Newtonian. \emph{Denoted by} $\bm{\eta_0}$. \emph{Given by}
    \begin{equation*}
        \eta_0 := \lim_{\dot{\gamma}\to 0}\eta
    \end{equation*}
    \item \textbf{Weissenberg number}: A dimensionless shear rate. \emph{Denoted by} $\bm{Wi}$. \emph{Given by}
    \begin{equation*}
        Wi := \dot{\gamma}\tau_\text{polymer}
    \end{equation*}
    \item Before we dive deeper, we should state an important assumption about shear rate underpinning our analysis.
    \begin{figure}[h!]
        \centering
        \begin{subfigure}[b]{0.3\linewidth}
            \centering
            \includegraphics[width=0.95\linewidth]{stretchTumblea.jpeg}
            \caption{Terminology definitions.}
            \label{fig:stretchTumblea}
        \end{subfigure}
        \begin{subfigure}[b]{0.3\linewidth}
            \centering
            \includegraphics[width=0.4\linewidth]{stretchTumbleb.png}
            \caption{DNA being sheared.}
            \label{fig:stretchTumbleb}
        \end{subfigure}
        \caption{Stretching and tumbling upon shearing.}
        \label{fig:stretchTumble}
    \end{figure}
    \begin{itemize}
        \item In this class, we will assume that polymer coils are \emph{not} deformed by the flow (i.e., $Wi<1$).
        \item Typically, polymers \textbf{stretch} and \textbf{tumble} when $Wi>0$ (Figure \ref{fig:stretchTumblea}).
        \begin{itemize}
            \item For example, DNA has been micrographed stretching and tumbling at higher shear rates and Weissenberg numbers (Figure \ref{fig:stretchTumbleb}).
        \end{itemize}
        \item Reference: \textcite{bib:stretchTumble}.
    \end{itemize}
    \item \textbf{Stretch}: A behavior of a sheared polymer in which a usually flexible coil is elongated by shear forces.
    \item \textbf{Tumble}: A behavior of a sheared polymer in which a stretched coil is compressed with rotation by shear forces.
    \item We now describe how the Navier-Stokes equations treat viscosity.
    \begin{figure}[h!]
        \centering
        \includegraphics[width=0.1\linewidth]{stokesViscosity.png}
        \caption{Drag coefficient on a hard sphere in a viscous fluid.}
        \label{fig:stokesViscosity}
    \end{figure}
    \begin{itemize}
        \item Consider a hard sphere of radius $r_s$ falling through a fluid of viscosity $\eta_0$ with velocity $v$.
        \item By Newton's laws, the fluid will exert a viscous force $F_\text{viscous}$ on the sphere. Moreover, this force will be given by the following.
        \begin{equation*}
            F_\text{viscous} = fv
        \end{equation*}
        \begin{itemize}
            \item $v$ is the velocity with which the sphere is moving.
            \item $f$ is the drag coefficient or friction factor. It can be thought of as the proportionality factor between the velocity and viscous force.
        \end{itemize}
        \item Solving the Navier-Stokes equations for this scenario, we obtain \textbf{Stokes' law}.
        \item How, then, can we treat a polymer molecule moving through a viscous fluid? There are two limiting cases.
        \begin{enumerate}
            \item The polymer coil acts as an impenetrable sphere with dimensions given by the pervaded volume.
            \item The polymer coil is entirely penetrable, with only individual monomers interacting with the viscous fluid.
        \end{enumerate}
    \end{itemize}
    \item \textbf{Stokes' law}: The Navier-Stokes equations' relation between the drag coefficient $f$ on a sphere of radius $r_s$ falling through a fluid, and the fluid's viscosity $\eta_0$. \emph{Given by}
    \begin{equation*}
        f = 6\pi\eta_0r_s
    \end{equation*}
    \begin{itemize}
        \item The $6\pi$ falls out of the math when solving the Navier-Stokes equations.
        \item It makes intuitive sense that $f\propto\eta_0$ and $f\propto r_s$, so it is not hard to rationalize or visualize that such proportionality manifests itself as a bilinear relation.
    \end{itemize}
    \item Let's now extend our investigation into the possible behaviors of a polymer molecule in shear flow.
    \begin{figure}[h!]
        \centering
        \begin{subfigure}[b]{0.35\linewidth}
            \centering
            \includegraphics[width=0.85\linewidth]{polFluida.png}
            \caption{Penetrable coil.}
            \label{fig:polFluida}
        \end{subfigure}
        \begin{subfigure}[b]{0.35\linewidth}
            \centering
            \includegraphics[width=0.7\linewidth]{polFluidb.JPG}
            \caption{Penetrable coil (nanoscale).}
            \label{fig:polFluidb}
        \end{subfigure}\\[2em]
        \begin{subfigure}[b]{0.35\linewidth}
            \centering
            \includegraphics[width=0.8\linewidth]{polFluidc.png}
            \caption{Impenetrable coil.}
            \label{fig:polFluidc}
        \end{subfigure}
        \begin{subfigure}[b]{0.35\linewidth}
            \centering
            \includegraphics[width=0.95\linewidth]{polFluidd.png}
            \caption{Rod-like polymer.}
            \label{fig:polFluidd}
        \end{subfigure}
        \caption{Modeling a polymer in a viscous fluid.}
        \label{fig:polFluid}
    \end{figure}
    \begin{enumerate}
        \item Penetrable coil, i.e., Rouse's free draining model.
        \begin{equation*}
            f \propto N_2\xi^*
        \end{equation*}
        \begin{itemize}
            \item Notation.
            \begin{itemize}
                \item We still let $N_2$ be the number of segments per chain (e.g., number of Kuhn steps).
                \item $\xi^*=\zeta$ is the \textbf{monomeric friction factor}.
            \end{itemize}
            \item Globally, the fluid passes through the coil freely (Figure \ref{fig:polFluida}).
            \item However, very locally (Figure \ref{fig:polFluidb}), there is no slip. In other words, each monomer contributes a small amount $\xi^*$ to the total friction factor. It follows that in this case, the friction factor scales linearly with the number of units $N_2$ in the polymer!
        \end{itemize}
        \item Impenetrable sphere, i.e., Zimm's non-free draining model (Figure \ref{fig:polFluidc}).
        \begin{align*}
            f &= 6\pi\eta_0R_H&
            R_H &\propto \prb{R_g^2}^{1/2}&
            f &\propto N_2^\nu
        \end{align*}
        \begin{itemize}
            \item $R_H$ denotes the sphere's \textbf{hydrodynamic radius}.
            \item The rightmost relation follows from transitivity and the scaling of $R_g$.
            \begin{itemize}
                \item $\nu$ is the same measure of solvent quality (e.g., $1/2$, $3/5$, or $1/3$) discussed previously.
            \end{itemize}
            \item There is a strong \textbf{hydrodynamic interaction} (HI), or coupling, of the motion of monomers in dilute solution.
            \begin{itemize}
                \item Essentially, when one monomer moves somewhere along a polymer chain, it yanks others in the chain along with it.
                \item This interaction decreases the farther out along the chain you go from the monomer that is moved. Symbolically, $\text{HI}\propto 1/r$, where $r$ is the relative separation of monomers.
                \item Additional equation for $\mathbf{v}(\mathbf{r})$??
            \end{itemize}
            \item This model is appropriate for a (1) high molecular weight and (2) flexible polymer.
        \end{itemize}
        \item A rod-like molecule (Figure \ref{fig:polFluidd}).
        \begin{align*}
            f &= 6\pi\eta_0R_H&
            R_H &\propto R_\text{max} = L&
            f &\propto N_2
        \end{align*}
        \begin{itemize}
            \item This is one more commonly encountered possiblity.
            \item The above relations should be fairly self-explanatory.
        \end{itemize}
    \end{enumerate}
    \item When choosing which of the above three models to use for a given situation, remember that the one that is most appropriate depends on the size and concentration of the polymers.
    \item How might we measure the friction factor $f$?
    \begin{figure}[h!]
        \centering
        \includegraphics[width=0.13\linewidth]{diffusion.JPG}
        \caption{A diffusing molecule.}
        \label{fig:diffusion}
    \end{figure}
    \begin{itemize}
        \item Consider the Stokes-Einstein diffusivity / diffusion coefficient
        \begin{equation*}
            D_t = \frac{\kB T}{\xi}
        \end{equation*}
        \begin{itemize}
            \item $\xi=f$ is the drag coefficient / friction factor.
            \item We are taking this to be of a tracer molecule, i.e., some molecule in solution far from everything else.
            \item $\kB T$ is an energy (i.e., a force times a length), and the friction factor has been defined as the quotient of (the viscous) force divided by a velocity. Thus, the units cancel out to length squared per time for $D_t$.
        \end{itemize}
        \item We have a flux $J$ satisfying \textbf{Fick's law}.
        \begin{itemize}
            \item This is a continuum view of diffusion.
        \end{itemize}
        \item The average squared displacement of a molecule as it moves around over a period of time $\tau$ is
        \begin{equation*}
            \prb{\Delta x^2+\Delta y^2+\Delta z^2}_\tau = 6D_t\tau
        \end{equation*}
        \begin{itemize}
            \item This is a molecular view of diffusion, from time $t=0$ to time $t=\tau$ (Figure \ref{fig:diffusion}).
            \item $D_t$ has units of length squared over time here, as well!
        \end{itemize}
        \item Thus, to actually get the diffusion coefficient, we could track molecules, e.g., by taking movies of them.
        \begin{itemize}
            \item Take movies of everywhere it goes, break up the path into segments like a random walk (recall that Brownian motion is a continuous random walk), and then take the ensemble or time average over a period of time of length $\tau$.
            \begin{equation*}
                \prb{\Delta r^2} = 6D_t\tau
            \end{equation*}
        \end{itemize}
        \item It follows by combining several of the above equations that
        \begin{equation*}
            R_H = \frac{\kB T}{6\pi\eta_sD_t}
        \end{equation*}
        \item Additionally, it is important that
        \begin{equation*}
            R_H \propto R_g \propto R
        \end{equation*}
    \end{itemize}
    \item \textbf{Fick's law}: A relationship between flux, diffusivity, concentration, and distance. \emph{Given by}
    \begin{equation*}
        J = -D_m\pdv{c}{x}
    \end{equation*}
    \begin{itemize}
        \item $J$ is the flux.
        \item $c$ is the concentration.
        \item $x$ is the distance.
    \end{itemize}
    \item How long does a polymer have to be to obey Zimm scaling?
    \begin{itemize}
        \item For most systems in which we are interested, we will use Zimm's non-draining sphere.
        \item However, there is still significant debate in the literature about when this non-draining sphere assumption is ok to use.
        \item \textcite{bib:ZimmLength1} posits that once $N>10$, we can use this model.
        \item \textcite{bib:ZimmLength2} --- published just two years ago --- posits that we need to wait until $N>30$.
        \item Many polymers we have are well into the hundreds or thousands, though. Above $N=100$ is a generally solid cutoff for this behavior, so we can think of these as very good and very non-draining.
    \end{itemize}
    \item Scaling behavior of the friction factor.
    \begin{itemize}
        \item $\nu$ values reviewed again.
        \item For high MW flexible chains, the correct model is non-draining impenetrable units (Zimm-like).
        \begin{itemize}
            \item As discussed above, this model obeys $\xi=f=6\pi\eta_sR_H$, where $R_H\propto R_g$.
        \end{itemize}
    \end{itemize}
    \item $R_H$ vs. $R_g$ vs. $R$.
    \begin{table}[h!]
        \centering
        \includegraphics[width=0.6\linewidth]{RgRH.png}
        \caption{Ratio of radius of gyration and hydrodynamic radius for different polymer architectures.}
        \label{tab:RgRH}
    \end{table}
    \begin{itemize}
        \item The prefactor $\gamma$ in
        \begin{equation*}
            R_g = \gamma R_H
        \end{equation*}
        is on the order of 1.
        \begin{itemize}
            \item The above equation is really important! We will use it later today.
        \end{itemize}
        \item Recall that $R_H$ is related to a hydrodynamic interaction tensor??
        \item Note that $\gamma=R_g/R_H$ converges to 1 (middle-right column in Table \ref{tab:RgRH}) as the number of arms on a polymer increases because, as you get more arms, you do just start to look like a sphere.
        \item \textcite{bib:RubinsteinColby} has some good explanations on this; Table \ref{tab:RgRH} is also its "Table 8.4."
    \end{itemize}
    \item For the rest of class, we focus on viscosity. We'll get as far into this topic as we can toay, and then we'll carry on next time.
    \item Einstein model for viscosity.
    \begin{itemize}
        \item Simulating viscosity in full is a computationally intensive process.
        \item On pen and paper, we will focus on what we can do for idealized hard spheres in a Newtonian fluid. This is the setup Einstein considered.
        \item Let the hard spheres be of volume fraction $\phi_{hs}$ in the solution.
        \item Einstein did his calculation in the dilute limit, obtaining
        \begin{equation*}
            \eta(\phi_{hs}) = \eta_s(1+2.5\phi_{hs}+\cdots)
        \end{equation*}
        \begin{itemize}
            \item Einstein's detailed calculation got $2.5$ as the linear correction.
            \item The power series continues, but in the dilute limit ($\phi_2\ll 1$ and $c_2\ll c_2^*$, where $c_2^*$ is the overlap concentration), we only need the linear correction.
        \end{itemize}
        \item This solution will have a viscosity $\eta$, composed of a linear contribution from the solvent $\eta_s$, and then an added contribution from the polymer in the form of a \textbf{specific viscosity} $\eta_{sp}$. This yields
        \begin{equation*}
            \eta = \eta_s+\eta_s\eta_{sp}
        \end{equation*}
        \item Thus, from the above by transitivity,
        \begin{equation*}
            \frac{\eta}{\eta_s}-1 = \eta_{sp} = \underbrace{2.5\phi_{hs}}_\text{for hard spheres}+\cdots
        \end{equation*}
        \item From the specific viscosity, we can define the \textbf{intrinsic viscosity} $[\eta]$.
        \item It follows that in general, our solution has a viscosity
        \begin{equation*}
            \eta = \eta_s\left( 1+c\cdot[\eta]+K_\text{H}c^2[\eta]^2+\cdots \right)
        \end{equation*}
        \begin{itemize}
            \item This equation represents viscosity as a perturbation in concentration $c$.
            \item The first prefactor is the intrinsic viscocity.
            \item $K_\text{H}$ denotes the \textbf{Huggins coefficient}. As in Flory-Huggins theory, the 2nd term reflects pairwise interactions of solute molecules.
        \end{itemize}
        \item We now want to relate the intrinsic viscosity to our polymer models.
        \begin{itemize}
            \item Let's go back to hard spheres.
            \item The volume fraction of hard spheres is the number of spheres times the sphere volume $V_{hs}$ divided by the total volume.
            \item To get it in terms of concentration so as to bridge this model over to our new concentration-based model, we obtain
            \begin{equation*}
                \phi_{hs} = \frac{(\#\text{ spheres})(\text{vol sphere})}{\text{total volume}}
                = \frac{cN_\text{A}V_{hs}}{M}
            \end{equation*}
            \begin{itemize}
                \item We measure concentration in \si[per-mode=symbol]{\gram\per\liter}.
                \item We measure molar mass in \si[per-mode=symbol]{\gram\per\mole}.
                \item The units of Avogadro's number $N_\text{A}$ is \si{\per\mole}.
                \item Thus, in the above expression, the units do work out correctly to number times volume divided by volume!
            \end{itemize}
        \end{itemize}
        \item Thus, we can now put everything together.
        \begin{itemize}
            \item The Einstein model's correction is
            \begin{equation*}
                2.5\phi_{hs} = \frac{2.5cN_\text{A}V_{hs}}{M}
            \end{equation*}
            \item We can relate the hard sphere volume back to polymer radius with
            \begin{equation*}
                V_{hs} = \frac{4}{3}\pi R_{hs}^3
            \end{equation*}
            \item The specific viscosity is then equal to $2.5\phi_{hs}$ by definition, so
            \begin{equation*}
                \eta_{sp} = \frac{2.5\cdot 4\pi R_H^3cN_\text{A}}{3M}
            \end{equation*}
            where we have replaced the hard sphere radius with the polymer's hydrodynamic radius.
            \item Now $R_H=\gamma R_g$, as mentioned earlier today. Additionally, we can divide through by concentration to turn the specific viscosity into an intrinsic viscosity. Therefore,
            \begin{equation*}
                [\eta] = \frac{10\pi N_\text{A}\gamma^3\prb{R_g^2}^{3/2}}{3M}
            \end{equation*}
            \item The above result is our final theory --- a very commonly used theory.
        \end{itemize}
        \item It follows that for our polymer-solvent solution, we have the following scaling.
        \begin{equation*}
            [\eta] \propto \frac{R_g^3}{M} \propto M^{3\nu-1} \propto N^{3\nu-1}
        \end{equation*}
        \begin{itemize}
            \item We still have $\nu$ dependent on the solvent type exactly as discussed repeatedly above.
            \item Thus, for a $\theta$ solvent, scaling of the intrinsic viscosity is $3(1/2)-1=1/2$. For a good solvent, $3(3/5)-1=4/5$.
            \item Note that we get the scaling because $R_g\propto R\propto N^\nu$, and $N\propto M$.
        \end{itemize}
    \end{itemize}
    \item \textbf{Specific viscosity}: A dimensionless quantity describing how much more viscous a solution becomes as polymers are introduced. \emph{Denoted by} $\bm{\eta_{sp}}$. \emph{Given by}
    \begin{equation*}
        \eta_{sp} := \frac{\eta}{\eta_s}-1
    \end{equation*}
    \begin{itemize}
        \item From the definition, we can see that for a pure solution (only solvent), the specific viscosity is zero. As the viscosity of the solution increases with increasing solute, $\eta_{sp}$ will obtain a value greater than zero (and likewise for decreases in viscosity).
    \end{itemize}
    \item \textbf{Intrinsic viscosity}: A measure of a solute's contribution to the viscosity of a solution. \emph{Units} $\textbf{conc}^{\bm{-1}}$, \textbf{\si[text-series-to-math,per-mode=symbol]{\milli\liter\per\gram}}. \emph{Denoted by} $\bm{[\eta]}$. \emph{Given by}
    \begin{equation*}
        [\eta] := \lim_{c\to 0}\left( \frac{\eta_{sp}}{c} \right)
        = \lim_{c\to 0}\left( \frac{\eta-\eta_s}{c\eta_s} \right)
    \end{equation*}
    \item \textbf{Huggins coefficient}: The viscometric equivalent of a Virial coefficient. \emph{Denoted by} $\bm{K_H}$.
    \item Implicit assumptions used today.
    \begin{itemize}
        \item We are in a dilute solution.
        \item If we use this viscosity in a real flow, our Weissenberg number must be much less than one ($Wi\ll 1$, where $Wi:=\tau_p\dot{\gamma}$).
        \item The Zimm/non-draining argument is essential.
    \end{itemize}
    \item Do we need to know anything else about the radius of gyration?
    \begin{itemize}
        \item We will not be doing any detailed calcualations, so no.
        \item Just know that all of the different radii are applicable in different circumstances, and all related by order 1 coefficients. Thus, we can convert between all of them as needed.
    \end{itemize}
    \item \textbf{Mark-Houwink-Sakurada model}: A scaling of the intrinsic viscosity derived above. \emph{Also known as} \textbf{Mark-Houwink model}. \emph{Given by}
    \begin{equation*}
        [\eta] = KM^a
    \end{equation*}
    \begin{itemize}
        \item $K,a$ --- the \textbf{Mark-Houwink parameters} --- depend on the polymer and solvent quality, respectively.
        \item Exponent scaling.
        \begin{itemize}
            \item $a<1/2$ implies a poor solvent.
            \item $a=1/2$ implies a $\theta$ solvent.
            \item $4/5>a>1/2$ implies a good solvent.
        \end{itemize}
        \item There's lots of data for various polymer types.
    \end{itemize}
\end{itemize}



\section{Intrinsic Viscosity - 2}
\begin{itemize}
    \item \marginnote{10/9:}Review of last lecture (various terms introduced last time).
    \item To begin, we'll wrap up some concepts from last time.
    \item More types of viscosity.
    \begin{itemize}
        \item \textbf{Relative}, \textbf{reduced}, and \textbf{inherent} viscosity.
        \item These exist, but we don't need to know much about them.
    \end{itemize}
    \item \textbf{Relative viscosity}: The total viscosity normalized by that of the solvent. \emph{Denoted by} $\bm{\eta_r}$. \emph{Given by}
    \begin{equation*}
        \eta_r := \frac{\eta}{\eta_s}
    \end{equation*}
    \item \textbf{Reduced viscosity}: The increase in viscosity, normalized by the concentration of the added solute. \emph{Denoted by} $\bm{\eta_\textbf{red}}$. \emph{Given by}
    \begin{equation*}
        \eta_\text{red} := \frac{1}{c}\left( \frac{\eta}{\eta_s}-1 \right)
    \end{equation*}
    \item \textbf{Inherent viscosity}: The measure of viscosity defined as follows. \emph{Denoted by} $\bm{\eta_\textbf{inh}}$. \emph{Given by}
    \begin{equation*}
        \eta_\text{inh} := \frac{1}{c}\ln(\frac{\eta}{\eta_s})
    \end{equation*}
    \item On the scaling of the intrinsic viscosity with the polymer molecular weight.
    \begin{itemize}
        \item Doyle provides the justification I thought up for the $3\nu-1$ proportionality discussed last time!
        \item The Mark-Houwink-Sakurada model's benefit is that it allows us to fit our data to a more general law that doesn't depend on "good" or "bad" solvents so explicitly, but rather allows you to interpolate between them.
    \end{itemize}
    \item Doyle skips the slide on Mark-Houwink parameters, but know that these are tabulated.
    \item Last class, we built a lot of theory around viscosity. But is that theory consistent with our theory from Topics 1-2?
    \item Let's show that it is in one example: Intrinsic viscosity and $c^*$.
    \begin{itemize}
        \item Recall Figure \ref{fig:concentrationRegimes}.
        \item Recall the geometric argument that
        \begin{equation*}
            c^* = \left( \frac{M}{N_\text{A}} \right)\frac{1}{V_p} = \frac{3M}{4\pi N_\text{A}R_g^3}
        \end{equation*}
        \begin{itemize}
            \item $M$ is the mass of 1 mole of polymers.
            \item Thus, $M/N_\text{A}$ is the mass of a single polymer and $M/N_\text{A}V_p$ is the concentration of polymer within the excluded volume.
            \item But this is equal to the concentration of $n$ polymers within $n$ of their excluded volumes, or of a whole solution at the critical concentration!
            \item $V_p$ is related to the radius of gyration by the equation for the volume of a sphere.
        \end{itemize}
        \item Alternatively, the critical concentration occurs when the effective volume fraction $\phi_\text{eff}=\phi_{hs}=1$. Thus, substituting in the values at the critical moment,
        \begin{align*}
            [\eta] &= \frac{2.5\phi_{hs}}{c_2}\\
            [\eta]_c &= \frac{2.5\cdot 1}{c^*}\\
            c^* &= \frac{2.5}{[\eta]} = \frac{3M}{4\pi N_\text{A}R_g^3\gamma^3}
        \end{align*}
        \begin{itemize}
            \item The second line above is a \emph{very} famous equation that shows up in many research articles.
        \end{itemize}
        \item The two definitions of $c^*$ are very similar!
        \begin{itemize}
            \item This provides a good "sanity check" for our theory.
            \item We can rest easy about the $\gamma^3$ difference because $\gamma$ is an order 1 term, so it doesn't change the value of the expression much.
        \end{itemize}
    \end{itemize}
    \item Determination of the intrinsic viscosity.
    \begin{figure}[h!]
        \centering
        \begin{subfigure}[b]{0.3\linewidth}
            \centering
            \includegraphics[width=0.8\linewidth]{intViscosityDetermina.png}
            \caption{Series of concentrations.}
            \label{fig:intViscosityDetermina}
        \end{subfigure}
        \begin{subfigure}[b]{0.3\linewidth}
            \centering
            \includegraphics[width=0.8\linewidth]{intViscosityDeterminb.png}
            \caption{Series of MWs.}
            \label{fig:intViscosityDeterminb}
        \end{subfigure}
        \caption{Determination of intrinsic viscosity.}
        \label{fig:intViscosityDetermin}
    \end{figure}
    \begin{itemize}
        \item Let's start with a derivation.
        \begin{itemize}
            \item Recall that for $c<c^*$,
            \begin{align*}
                \eta &\approx \eta_s\left( 1+c\cdot[\eta]+K_\text{H}c^2[\eta]^2 \right)&
                \eta_{sp} &= \frac{\eta}{\eta_s}-1
            \end{align*}
            \item Thus, we can divide the right equation through by concentration and the solvent viscosity, rearrange, and relate it to the intrinsic viscosity, as follows.
            \begin{align*}
                \frac{\eta}{\eta_sc} &= \frac{1}{c}+[\eta]+K_\text{H}[\eta]^2c\\
                \underbrace{\frac{\eta/\eta_s-1}{c}}_{\eta_{sp}/c} &= [\eta]+\underbrace{K_\text{H}[\eta]^2}_\text{constant}c
            \end{align*}
            \item The equation above tells us that given a series of concentrations, we can plot $\eta_{sp}/c$ vs. $c$ and get a linear plot, the $y$-intercept of which will be the intrinsic viscosity and the slope of which could give us the Huggins coefficient (Figure \ref{fig:intViscosityDetermina}). 
            \item Notice that in the limit of $c\to 0$, the above equation reduces to the definition of intrinsic viscosity!
        \end{itemize}
        \item Now that we have the intrinsic viscosity, we can do even more.
        \begin{itemize}
            \item The Mark-Houwink model tells us that given a series of molecular weights and their intrinsic viscosities, we can make a log-log plot to measure $K$ and $a$ (Figure \ref{fig:intViscosityDeterminb}).
        \end{itemize}
        \item The data on the next slide shows agreement with the theoretical expectation.
    \end{itemize}
    \item Doyle is skipping the Intrinsic Viscosity - Key Idea \& Viscosity-Average Molecular Weight derivations.
    \begin{itemize}
        \item Key result: Each of our species $i$ in solution obeys its own Mark-Houwink equation.
        \item The \textbf{viscosity-averaged molecular weight} tends to lie between the number- and weight-averaged MW; it's what you use for $M$ in the Mark-Houwink plot analysis.
    \end{itemize}
    \item Example of the above method of measuring intrinsic velocity, from a recent article in Nature.
    \begin{itemize}
        \item The authors were interested in the degradation of cellulose.
        \item They measured the decrease in molecular weight by Mark-Houwink analysis, but in reverse! They knew $K,a$, measured $[\eta]$, and found $M$ for each sample.
        \item Reference: \textcite{bib:MHScellulose}.
    \end{itemize}
    \item We now move into today's lecture content in earnest: Measuring intrinsic visocosity, from classic methods to the state-of-the-art ones people are currently researching.
    \item Challenges to the measurement of intrinsic viscosity.
    \begin{itemize}
        \item Low concentrations of polymer.
        \begin{itemize}
            \item This implies that the viscosity of the polymer solution is very close to that of the solvent, so we need very accurate instruments.
        \end{itemize}
        \item Need to operate in the $Wi<1$ regime.
        \begin{itemize}
            \item This is because we need the polymer to not be deformed, so we can't shear \emph{too} fast (or we'll get into stretching and tumbling regimes that throw off our calculations, as in Figure \ref{fig:stretchTumble}).
            \item Since the shear stress $\tau=\eta\dot{\gamma}$,
            \begin{equation*}
                Wi = \tau\dot{\gamma}
                = \eta\dot{\gamma}^2
            \end{equation*}
            and so we need $\eta\dot{\gamma}^2$ less than one.
        \end{itemize}
        \item We may not have much of our sample to commit to this experiment, especially in the case of precious materials.
        \begin{itemize}
            \item This is especially challenging because relatively large volumes of solutions are typically needed for viscosity measurements.
        \end{itemize}
    \end{itemize}
    \item Classic method: Capillary viscometer.
    \begin{figure}[h!]
        \centering
        \includegraphics[width=0.2\linewidth]{capillaryViscometer.png}
        \caption{Capillary viscometer.}
        \label{fig:capillaryViscometer}
    \end{figure}
    \begin{itemize}
        \item From the 1800s.
        \item We suck some liquid up into a straw.
        \item Gravity will then want to equilibrate the chambers, and we measure the time that the equilibration takes.
        \item All you need is the set up, a means to draw your liquid up, and a stopwatch.
    \end{itemize}
    \item Basic (without Navier-Stokes) physics for the why a capillary viscometer works.
    \begin{figure}[h!]
        \centering
        \includegraphics[width=0.25\linewidth]{HagenPoiseuille.JPG}
        \caption{Hagen-Poiseuille derivation.}
        \label{fig:HagenPoiseuille}
    \end{figure}
    \begin{itemize}
        \item Hagen and Poiseuille\footnote{"POY-say"} solved a nice problem for a Newtonian fluid (constant viscosity $\eta$) traveling through a straight tube of diameter $2r$ and length $L=h_\text{entrance}-h_\text{exit}$ at pressure $\Delta P$.
        \begin{itemize}
            \item The flow $Q$ is assumed to be sufficiently slow so as not to be \textbf{turbulent}. In other words, the flow is \textbf{laminar}.
            \begin{itemize}
                \item Specifically, they wanted the flow to have \textbf{Reynolds number} $\text{Re}\leq 2000$.
                \item Note that $Q$ is measured in units of volume per second (e.g., \si[per-mode=symbol]{\cubic\meter\per\second}).
            \end{itemize}
            \item The result of the derivation is that the flow is
            \begin{equation*}
                Q = \frac{\pi r^4}{8\eta L}\Delta P
            \end{equation*}
            \begin{itemize}
                \item $\Delta P=P+\rho gh$ is a dynamic pressure, which is equal to the regular pressure $P$ plus $\rho gh$. It is the pressure difference between the two ends of the pipe.
            \end{itemize}
        \end{itemize}
        \item It follows that --- in the context of a capillary viscometer --- the time to drain a volume $V$ is going to be
        \begin{equation*}
            \frac{V}{Q} \propto \underbrace{\frac{VL}{r^4g\Delta h}}_\text{constant}\cdot\frac{\eta}{\rho}
        \end{equation*}
        \begin{itemize}
            \item Thus, we have a relation between our observable ($V/Q$), the desired quantity ($\eta$), and parameters baked into our system.
            \item Note that the parameters baked into the system can be broken down into two classes. The first is the contents of the constant, which are all specific to a given capillary viscometer. The second is the density of the fluid, which we can experimentally measure independently of our capillary viscometer (e.g., with a micropipette and a balance) for any fluid whose viscosity we seek to determine.
        \end{itemize}
        \item A convenient rewrite of the above equation (in the specific case of water, a common solvent) is
        \begin{equation*}
            t_{\ce{H2O}} = \text{constant}\cdot\frac{\eta_{\ce{H2O}}}{\rho_{\ce{H2O}}}
        \end{equation*}
        \begin{itemize}
            \item Water is very extensively studied, so we can look up the right terms above in a handbook.
        \end{itemize}
        \item Doing the same for our polymer solution and combining the two equations, we get the following result. This result will give us our material properties.
        \begin{equation*}
            \frac{t_{\ce{H2O}}}{t_\text{polymer}} = \frac{\eta_{\ce{H2O}}}{\rho_{\ce{H2O}}}\cdot\frac{\rho_\text{polymer}}{\eta_\text{polymer}}
        \end{equation*}
        \begin{itemize}
            \item We measure $t_{\ce{H2O}}$ in our specific viscometer, either measure or look up $\eta_{\ce{H2O}}$ and $\rho_{\ce{H2O}}$, determine the density of our polymer-water solution, and then can obtain the viscosity of our polymer!
            \begin{itemize}
                \item This equation eliminates the need to measure the constant for our specific viscometer.
                \item Basically, it replaces measuring a constant with calibrating the viscometer against a known solvent.
            \end{itemize}
            \item Note that every term subscripted "polymer" refers to a property of the polymer-solvent solution. This is because we want to measure quantities at the conditions under which we're running the experiment.
            \item We can also use the approximation for $c\ll c^*$ that $\rho_\text{polymer}\approx\rho_\text{solvent}$.
        \end{itemize}
        \item We need about \SIrange{5}{10}{\milli\liter} of solution to run this experiment.
        \item These instruments are \$200-\$400 from Sigma today!
    \end{itemize}
    \item \textbf{Reynolds number}: A dimensionless quantity that helps predict fluid flow patterns (e.g., laminar vs. turbulent). \emph{Denoted by} $\textbf{Re}$. \emph{Given by}
    \begin{equation*}
        \text{Re} := \frac{2rU\rho}{\eta}
    \end{equation*}
    \item \textbf{Kinematic viscosity}: A measure of the viscosity of a solution that is directly proportional to the time the solution takes to flow under certain conditions. \emph{Given by}
    \begin{equation*}
        \frac{\eta}{\rho}
    \end{equation*}
    \item We now begin discussing \textbf{rheology}.
    \item \textbf{Rheology}: The study of the flow of matter. \emph{Etymology} from Greek "rhei" (to flow or stream) and "logos" (the study of).
    \item History of rheology.
    \begin{itemize}
        \item Plato (400 BC): "All things move and nothing stands still."
        \item Simplicius (500 AD): "Panta rhei."
        \begin{itemize}
            \item Translation: Everything flows.
            \item This is true! Some things just appear not to because we're looking at them at too short of a timescale.
        \end{itemize}
        \item Prof. Bingham (1920s): Coins the term "rheology."
        \item Influence of the idea that everything flows: The \textbf{Deborah number}.
        \begin{itemize}
            \item Named after Deborah in the Bible's Book of Judges, who famously said, "the mountains flowed before the Lord."
            \item The idea is that on the scale of observation of God (which is infinite), even mountains will flow.
        \end{itemize}
    \end{itemize}
    \item \textbf{Deborah number}: A dimensionless number used to characterize the fluidity of materials over a given timescale. \emph{Denoted by} $\textbf{De}$. \emph{Given by}
    \begin{equation*}
        \text{De} := \frac{\tau}{\tau_\text{flow}}
    \end{equation*}
    \begin{itemize}
        \item $\tau$ is the time it takes for a material to react to a stimulus.
        \item $\tau_\text{flow}$ is the time over which the change is observed.
        \item Thus, low Deborah numbers correspond to fast-flowing substances, and high Deborah numbers correspond to more immutable substances (e.g., the proverbial mountains).
    \end{itemize}
    \item What is a rheometer?
    \begin{itemize}
        \item A rheometer applies a controlled flow or stress via a specific geometry.
        \item Many geometries to do this.
        \item A spinaret is common. It's a moving boundary, you put your sample inside, you get your force.
    \end{itemize}
    \item Intrinsic viscosity is measured via a co-centric cylinder rheometer.
    \begin{figure}[h!]
        \centering
        \includegraphics[width=0.15\linewidth]{rheometerCyl.png}
        \caption{Co-centric cylinder rheometer.}
        \label{fig:rheometerCyl}
    \end{figure}
    \begin{itemize}
        \item The gap is very small relative to the size of the central bob. Thus, the surface looks flat locally.
        \item A bit of geometry gets you your shear rate and shear stress, from which we know viscosity $\eta=\tau/\dot{\gamma}$.
        \begin{align*}
            \dot{\gamma} &= \frac{2\omega R_c^2}{R_c^2-R_b^2}&
            \tau &= \frac{M}{2\pi R_b^2L}
        \end{align*}
        \item Typical sample volume: \SIrange{1}{50}{\milli\liter}.
        \item People like these because you put your sample in, press a button, and get your result.
    \end{itemize}
    \item Aside: Just how sensitive are rheometers?
    \begin{itemize}
        \item Gareth McKinley (in MechE) will tell you that rheometers can typically measure 6 orders of magnitude dynamic range.
        \item Modern rheometers are about an order of magnitude more sensitive than the capillary rheometer.
    \end{itemize}
    \item Microfluidics approaches to rheometry.
    \begin{itemize}
        \item Still pretty similar to Hagen-Poiseuille.
        \item Advantages.
        \begin{itemize}
            \item Help when you only have very little sample (as is often the case in pharma).
            \item Subject fluids to variety of flows in a single device.
            \item Etc.
        \end{itemize}
        \item Challenges.
        \begin{itemize}
            \item No moving surface, so no direct analog to Figure \ref{fig:rheometerCyl}.
            \begin{itemize}
                \item Thus more similar to the capillary viscometer, fundamentally speaking.
            \end{itemize}
            \item Key challenge: We have to extrapolate to zero concentration to take the limit in the definition of intrinsic viscosity.
        \end{itemize}
    \end{itemize}
    \item Microrheometry overview.
    \begin{itemize}
        \item Many types.
        \item Rheometry on a chip (a device you can buy now).
        \item Microrheology, also a thing we'll discuss.
    \end{itemize}
    \item Measuring intrinsic viscosity on a chip.
    \begin{itemize}
        \item More benefits and challenges.
    \end{itemize}
    \item Theory for an intrinsic viscometer.
    \begin{figure}[h!]
        \centering
        \includegraphics[width=0.5\linewidth]{viscometerInt.png}
        \caption{Intrinsic viscometer schematic.}
        \label{fig:viscometerInt}
    \end{figure}
    \begin{itemize}
        \item Black lines are very small; width on the order of our hair (\SI{100}{\micro\meter}).
        \item The rightmost circle is just an outlet; that's trash.
        \item One reservoir has our sample, of which we'd like to take a measurement.
        \item The other reservoir has just the solvent in which the sample is dissolved, and a dye!
        \item Flow rates are given by volumetric resistances. Then the flows combine with conservation of mass.
        \item You create a dilution series with optically detectable concentration.
    \end{itemize}
    \item How an intrinsic viscometer works.
    \begin{itemize}
        \item Recall that $\Delta P=QR_H$, where $R_H=8\eta L/\pi r^4$ is the \textbf{hydrodynamic resistance}.
        \begin{itemize}
            \item Some of $R_H$ is geometric, and some is the viscosity.
            \item The geometric terms we denote $\tilde{R}_H=8L/\pi r^4$.
            \item Hydrodynamic resistance is analogous to electrical resistance and Ohm's law $V=IR$.
        \end{itemize}
        \item Four equations.
        \begin{gather*}
            \Delta P = P_A-P^*=\tilde{R}_1\eta_pQ_1\tag{1}\\
            P_B-P^* = \tilde{R}_2\eta_sQ_2\tag{2}\\
            P^*-P^{**} = \tilde{R}_3\eta(Q_1+Q_2)\tag{3}\\
            P^{**}-P^* = \tilde{R}_4\eta(Q_1+Q_2)\tag{4}
        \end{gather*}
        \begin{itemize}
            \item $P^*$ is the pressure right where the two flows meet.
            \item $P^{**}$ is the pressure once the two flows have equilibrated.
            \begin{itemize}
                \item We don't know what either of these are, hence why we use placeholder variables for them.
            \end{itemize}
            \item $\eta_p$ is the viscosity of the polymer solution.
        \end{itemize}
        \item Suppose $P_A=P_B$.
        \begin{itemize}
            \item Then the lefthand sides of equations 1 and 2 are the same, so
            \begin{align*}
                \tilde{R}_1\eta_pQ_1 &= \tilde{R}_2\eta_sQ_2\\
                \eta_p &= \underbrace{\frac{\tilde{R}_2\eta_s}{\tilde{R}_1}}_\text{known}\cdot\frac{Q_2}{Q_1}
            \end{align*}
            \begin{itemize}
                \item Pressure and flow are related in a dependent/independent relation; we can control one of them or the other, but not both. Thus, the one that we're not controlling will have to be measured.
                \item In this case, we're applying pressures, so we don't know the flows $Q_1,Q_2$.
            \end{itemize}
            \item The detector signal divided by the max signal gives
            \begin{equation*}
                Y = \frac{Q_2}{Q_1+Q_2}
            \end{equation*}
        \end{itemize}
        \item The remaining math is hard and not worth our time. Using all four allows us to get rid of the unknowns $P^*,P^{**}$ and fully analyze the system.
    \end{itemize}
    \item Intrinsic viscometer: Device performance.
    \begin{itemize}
        \item \textcite{bib:viscometerInt} showed that it worked pretty well, even with a tiny sample volume of \SI{5}{\micro\liter}.
        \begin{itemize}
            \item Recall that a traditional viscometer requires closer to \SI{5}{\milli\liter}.
        \end{itemize}
        \item The authors did both kinds of measurements in Figure \ref{fig:intViscosityDetermin} successfully.
        \item Design calculations: The authors wanted this to be a device that everyone could use, so they defined an operating range.
        \begin{itemize}
            \item The Deborah or Weissenberg number must be less than 1.
            \item Assuming a good solvent, $\lambda$ should scale as molecular weight to the $1.8$ power.
            \item We need very fast diffusion laterally across the channel so that the two fluids merge into one. This is the Pe calculation. The ratio cancelling out with fast diffusion time implies that \textbf{Peclet number} is greater than 1.
        \end{itemize}
    \end{itemize}
    \item Commercial realization: Formulaction.
    \begin{itemize}
        \item Dye might interact with polymer; ininial device was too complicated for commercialization.
        \item Now we let the fluids \emph{not} mix quickly and look at the index of refraction for them flowing next to each other.
        \item Hydrodynamic resistance is related to the width.
    \end{itemize}
    \item Microrheology.
    \begin{itemize}
        \item Spherical bead probes act like constant stress rheometers.
        \item Look at variations in Brownian motion!
        \item We can ensemble average the trajectories (as in Figure \ref{fig:diffusion}), plot against $\tau$, and get $2D$ as the parameter.
        \item By Einstein, $D=\kB T/6\pi a\eta$, so we can back out viscosity!
        \item Can do this with a few microliters, can do this inside living objects, etc.
        \item Developed by Doyle's lab!
        \item Here, $\tau$ is time, not the shear stress.
        \item $\kB T$ divided by a length is a force (the way the units work out). Thus, $\kB T$ divided by a volume ($L^3$) is a force over length squared, which is a pressure. This pressure is what we call the \textbf{thermal stress}.
    \end{itemize}
    \item High throughput microrheology.
    \begin{itemize}
        \item By one of Doyle's old UROPs who's now doing a PhD w/ Eric Furst.
        \item Each drop is a microliter experiment.
    \end{itemize}
    \item More on Figure \ref{fig:viscosityScaling}.
    \begin{itemize}
        \item Doyle recapitulates what he said when he originally showed the figure.
    \end{itemize}
    \item Expectations for PSet 3 and the Quiz: We're not gonna have to solve a pipe flow problem or anything, but the diffusivity relation to mean squared displacement is well within our wheelhouse.
    \item Class next Tuesday confirmed!
    \item Exam.
    \begin{itemize}
        \item Average 80, standard deviation of 10.
    \end{itemize}
\end{itemize}



\section{Office Hours (Doyle)}
\begin{itemize}
    \item \marginnote{10/14:}Zimm's non-free draining model: The following equation from the slides?
    \begin{equation*}
        \mathbf{v}(\mathbf{r}) = \frac{1}{8\pi\eta}\left( \frac{\delta}{\mathbf{r}}+\frac{\mathbf{r}\mathbf{r}}{\mathbf{r}^3} \right)\cdot\mathbf{F}
    \end{equation*}
    \begin{itemize}
        \item This is an equation from an advanced fluid dynamics textbook, and we do not need to know it.
        \item Basically, this equation gives the vector field $\mathbf{v}$ induced by displacing one particle along the polymer by $\mathbf{r}$.
        \item The only relevance it has to us is that it shows that there is a generic scaling of $1/r$ in both terms to the hydrodynamic interaction $\text{HI}=\mathbf{v}$.
    \end{itemize}
    \item PSet 3: General approach to the questions?
    \begin{itemize}
        \item It is basically a plug and chug PSet.
        \item I have the correct scalings for $N$ in PSet 3, Q1a-b! All of the rest of the factors are less important.
        \item PSet 3, Q2 is just a cheap way of teaching us something (sedimentation) that we don't have time to cover in class, so that we recognize it if we ever come across it. It is as simple as it seems.
        \item PSet 3, Q3 is purely plug and chug. Moreover, for PSet 3, Q3d, I did have it right that all we need is a qualitative recognition of what the scaling between intrinsic velocity and the mass tells us about solvent quality; we do not have to do anything quantitative with the numbers here.
    \end{itemize}
    \item What do we need to know about Fick's law?
    \begin{itemize}
        \item For anyone who has taken Transport Phenomena, Fick's law is a touchstone. Doyle just wanted to show all of the students in class that the tracer diffusion coefficient we discussed in class is the "same" as the mutual diffusion coefficient in Fick's law.
        \item Now, as the textbook would imply, there are subtle differences between $D_t$ and $D_m$, but we don't need to worry about them at our level.
    \end{itemize}
    \item What is the Weissenberg number?
    \begin{itemize}
        \item The Weissenberg number (sometimes used interchangeably with the Deborah number) is part of something called \textbf{dimensionless groups}.
        \begin{itemize}
            \item The core idea here is relative scaling. Dimensionless groups address questions like "how fast is fast" by setting a reference point and talking about how fast something is \emph{relatively} (\emph{dimensionlessly}) compared to something else.
        \end{itemize}
        \item With the Weissenberg number, we are concerned with the timescale of polymer relaxation (e.g., from being stretched) relative to the timescale of fluid flow.
        \item Since the shear rate is measured in units of reciprocal seconds, an approximate relaxation time scale $\tau_\text{flow}$ for the flow is $\tau_\text{flow}=1/\dot{\gamma}$. It follows that the Weissenberg number $Wi$, as a dimensionless group, should be equal to the ratio
        \begin{equation*}
            \frac{\tau_\text{polymer}}{\tau_\text{flow}} = \frac{\tau_\text{polymer}}{1/\dot{\gamma}}
            = \dot{\gamma}\tau_\text{polymer}
        \end{equation*}
        as it is.
        \begin{itemize}
            \item From the leftmost definition above, we can see that for $Wi<1$, the polymer relaxes faster than the flow. This means that stretching and tumbling can be neglected, which is the regime we'd like to analyze.
            \item This is why we set $Wi<1$ in this class.
        \end{itemize}
    \end{itemize}
    \item Do you have any intuition for the shear rate $\dot{\gamma}$?
    \begin{itemize}
        \item In the context of Figure \ref{fig:viscosity}, $\dot{\gamma}$ is a constant equal to the slope of the velocity profile.
        \item This is because $\dot{\gamma}=\dv*{U}{H}$, and Figure \ref{fig:viscosity}'s velocity profile plots the speed $U$ as a function of height $H$ (or $y$).
        \item Note that the calculus definition is important because in some scenarios, the velocity profile may not be constant. For instance, in the Hagen-Poiseuille capillary, the velocity profile is parabolic.
    \end{itemize}
\end{itemize}



\section{Osmometry and GPC}
\begin{itemize}
    \item Announcements.
    \begin{itemize}
        \item Exam solutions posted soon.
        \item Thursday's lecture will almost certainly be an asychronous video.
    \end{itemize}
    \item Lecture outline.
    \begin{itemize}
        \item Gel permeation chromatography.
        \item Osmometry.
        \begin{itemize}
            \item Derivations will be abbreviated, but overviewed.
        \end{itemize}
    \end{itemize}
    \item "GPC" and "SEC" will be used interchangeably in this class; the difference has to do with the material packed inside the column.
    \item A broad overview of how GPC works.
    \begin{figure}[h!]
        \centering
        \includegraphics[width=0.55\linewidth]{GPCoverview.png}
        \caption{Gel permeation chromatography overview.}
        \label{fig:GPCoverview}
    \end{figure}
    \begin{itemize}
        \item Porous resin beads are packed within the column.
        \begin{itemize}
            \item Pores are similar in size to the radius of gyration of the molecules you want to chromatograph.
        \end{itemize}
        \item Put your polymer mixture in at the top, and elute it with pressure.
        \item Measure the chromatogram with some kind of detector (e.g., light scattering or refractive index).
        \item Bigger polymers move through the interstitial space first because its entropically unfavorable for them to fit in pores.
        \begin{itemize}
            \item Little polymers sample both the interstitial and bead volume, and thus take more time to elute.
        \end{itemize}
    \end{itemize}
    \item GPC-selling companies (e.g., Waters or Agilent) sell various types of beads.
    \begin{itemize}
        \item Sometimes, you go through multiple columns in series to separate different molecular weights better and get better resolution overall. Doyle has seen up to 6 columns wired in series.
    \end{itemize}
    \item GPC theory.
    \begin{itemize}
        \item There are two important volumes.
        \begin{itemize}
            \item $V_i$ is the total volume \underline{i}nside the porous beads' pores.
            \item $V_o$ is the volume \underline{o}utside the beads.
        \end{itemize}
        \item The elution time of a given species depends on the volume explored. This volume can be $V_i$, $V_o$, or a combination of the two. Symbolically,
        \begin{equation*}
            t_\text{elute} = \frac{V_\text{explored}}{Q_\text{injection}}
        \end{equation*}
        \begin{itemize}
            \item $Q_\text{injection}$ is the \textbf{volumetric flow rate}, i.e., the cubic meters per second your pump is flowing through the column.
        \end{itemize}
        \item We commonly think of elution time in terms of an alternative measure, the \textbf{retention volume}.
    \end{itemize}
    \item \textbf{Retention volume}: The volume explored by a molecule as it passes through a GPC column. \emph{Denoted by} $\bm{V_r}$. \emph{Given by}
    \begin{equation*}
        V_r := t_\text{elute}\cdot Q_\text{injection}
    \end{equation*}
    \item Let's look at a typical response curve.
    \begin{figure}[h!]
        \centering
        \includegraphics[width=0.3\linewidth]{GPCresponse.JPG}
        \caption{GPC response curve.}
        \label{fig:GPCresponse}
    \end{figure}
    \begin{itemize}
        \item We plot the log of the molar mass of whatever's coming out vs. $V_r$.
        \item For big enough molecules (less volume is explored than $V_o$), everything will come out.
        \item Then there's a region where we get separation.
        \begin{itemize}
            \item The width of this interesting region is equal to the volume inside the pores, $V_i$. 
            \item A \textbf{partition coefficient} $K\in[0,1]$ interpolates between how much volume a given polymer can sample. Specifically,
            \begin{equation*}
                V_r = V_o+V_iK
            \end{equation*}
            \begin{itemize}
                \item In other words, $K$ determines the concentration inside and outside the pores for a given polymer.
            \end{itemize}
            \item Let's consider the extrema of $K$.
            \begin{itemize}
                \item When $K=0$, $M$ is large and the corresponding particles do not enter the pores at all.
                \item When $K=1$, $M$ is small and the correspondign particles sample all the space inside the pores \emph{and} outside in solution.
            \end{itemize}
            \item $K$ will be a function of $R_g$ divided by the diameter of our pores.
        \end{itemize}
        \item Then the molecules are so small ($V_r>V_0+V_i$) that they practically don't come out at all / all come out at once.
    \end{itemize}
    \item \textbf{Partition coefficient}: A real number between 0 and 1 describing the extent to which polymers enter the pores. \emph{Also known as} \textbf{size exclusion equilibrium constant}. \emph{Denoted by} $\bm{K}$. \emph{Given by}
    \begin{equation*}
        K = \frac{c_{p,\text{inside bead}}}{c_\text{p,outside bead}}
    \end{equation*}
    \begin{itemize}
        \item In words, $K$ is equal to the concentration of the polymer inside vs. outside a bead.
        \item Since being in a pore is equivalent to a polymer being in a higher energy state, $K$ is governed by the Boltzmann distribution
        \begin{equation*}
            K = \exp(-\frac{\Delta\hat{G}_{pp}}{RT})
        \end{equation*}
        \begin{itemize}
            \item $\Delta\hat{G}_{pp}$ is the change in Gibbs free energy upon 1 mole of polymer entering pores.
            \item Note that this Boltzmann distribution --- like all of them --- describes a \emph{dynamic} equilibrium under which polymers are constantly going in and out of various pores.
        \end{itemize}
        \item Rearranging the above Boltzmann distribution and invoking the definition of the Gibbs free energy affords
        \begin{align*}
            \Delta\hat{G}_{pp} &= -RT\ln K\\
            &= \Delta\hat{H}_{pp}-T\Delta\hat{S}_{pp}
        \end{align*}
        \begin{itemize}
            \item We typically choose systems (i.e., column material and solvent) so that there is no enthalpic change upon entering or leaving a pore.
            \item This implies that $\Delta\hat{H}_{pp}=0$ in systems of interest to us.
        \end{itemize}
        \item Thus,
        \begin{equation*}
            K = \exp(\frac{\Delta\hat{S}_{pp}}{R})
        \end{equation*}
        \begin{itemize}
            \item Sanity check: If there's no change in entropy (because our polymer is so small it doesn't even notice it's in the pore), $K\to 1$.
            \item Sanity check: If entropy is decreasing substantially (because the polymer is so large that it must significantly restrict its conformational freedom to enter a pore), $K\to 0$.
        \end{itemize}
    \end{itemize}
    \item The polymer has some entropy loss when it enters a pore.
    \begin{figure}[h!]
        \centering
        \includegraphics[width=0.25\linewidth]{GPCpore.JPG}
        \caption{Theoretical porous resin microstructure.}
        \label{fig:GPCpore}
    \end{figure}
    \begin{itemize}
        \item SEM and TEM images reveal that the porous beads used in GPC are highly irregular, but we will assume that these irregularities cancel to something more uniform so that our math is easier.
        \item Specifically, we will assume that our beads have a bunch of identical, cylindrical pores of diameter $D$.
    \end{itemize}
    \item Nobel Prize (1991) to de Gennes for the "blob concept."
    \begin{figure}[h!]
        \centering
        \includegraphics[width=0.25\linewidth]{GPCblobs.JPG}
        \caption{Segmenting a polymer into blobs.}
        \label{fig:GPCblobs}
    \end{figure}
    \begin{itemize}
        \item Very interesting guy who used to smoke while giving lectures (despite No Smoking signs), and sit in the front of other lectures and smoke.
        \item Consider a polymer with $N=10$ Kuhn steps.
        \item Now double the MW to $N=20$ Kuhn steps, still of the same length. This corresponds to the addition of the 10 orange Kuhn steps in Figure \ref{fig:GPCblobs}.
        \item We could say that the orange part has its own mean squared end-to-end distance instead of looking at the whole polymer! In other words, we can split a single polymer into self-similar subsections called \textbf{blobs}.
    \end{itemize}
    \item Stuffing our polymer into a pore.
    \begin{figure}[h!]
        \centering
        \includegraphics[width=0.3\linewidth]{GPCblobPore.jpg}
        \caption{Polymer blobs in a pore.}
        \label{fig:GPCblobPore}
    \end{figure}
    \begin{itemize}
        \item If we stuff a small enough part of the total polymer into the pore, that blob doesn't know it's in a pore! This is because there's no conformational restriction to this blob on average.
        \item Essentially, the polymer chain doesn't know about the pore on length scales $\leq D$.
        \item Assume a blob has $g$ Kuhn steps. Let the blob size be equal to $D$. Then $D=g^{1/2}l_k$. It follows that
        \begin{equation*}
            g = \left( \frac{D}{l_k} \right)^2
        \end{equation*}
        \item Two important consequences of this.
        \begin{itemize}
            \item It follows that the number of blobs is $N/g$: The total number of Kuhn steps divided by the number of Kuhn steps per blob.
            \item The energy pentalty per blob is $\kB T$, so the entropy penalty per blob is $\kB T/T=\kB$. This is the amount of freedom we're taking away by fixing the orientation between each blob.
        \end{itemize}
        \item Thus, the total entropic penalty is $(N/g)\cdot\kB$.
        \item It follows by using some of the above substitutions that
        \begin{equation*}
            \Delta S_{pp} = -\frac{\kB Nl_k^2}{D^2}
        \end{equation*}
        \item But in the unconfined bulk, $\prb{R^2}=Nl_k^2$, so the change in entropy \emph{per polymer} is
        \begin{equation*}
            \Delta S_{pp} = -\frac{\kB\prb{R^2}}{D^2}
        \end{equation*}
        \item It follows that the change in entropy \emph{per mole} of polymer is
        \begin{equation*}
            \Delta\hat{S}_{pp} = -\frac{R\prb{R^2}}{D^2}
        \end{equation*}
        \begin{itemize}
            \item Be careful here! $R$ denotes both the ideal gas constant and the end-to-end distance; which is which depends on context.
        \end{itemize}
        \item Thus,
        \begin{equation*}
            K = \exp(-\frac{\prb{R^2}}{D^2})
        \end{equation*}
    \end{itemize}
    \item Most of the stuff in the slides reflects what we did above; Doyle does not go through it in any sort of depth.
    \item Now how do we measure the molecular weight of the peaks? One common way is with standards.
    \begin{figure}[h!]
        \centering
        \includegraphics[width=0.3\linewidth]{GPCstandards.png}
        \caption{Calibrant response curves.}
        \label{fig:GPCstandards}
    \end{figure}
    \begin{itemize}
        \item Use essentially monodisperse GPC calibration standards that generate response curves between which you can interpolate.
        \item Example: Malvern's standards.
    \end{itemize}
    \item An alternative method of measuring molecular weights: Universal calibration.
    \begin{figure}[h!]
        \centering
        \includegraphics[width=0.3\linewidth]{GPCunivCalib.png}
        \caption{Universal calibration curve.}
        \label{fig:GPCunivCalib}
    \end{figure}
    \begin{itemize}
        \item Recall that
        \begin{equation*}
            [\eta] \propto \frac{R_g^3}{M}
            \propto \frac{V_H}{M}
            \propto \frac{V_r}{M}
        \end{equation*}
        \begin{itemize}
            \item Hence, the retention volume $V_r$ is proportional to $M[\eta]$.
            \item It follows --- assuming that the proportionality between $M[\eta]$ and $V_r$ is independent of molecular structure --- that if two species have the same retention volume, they have the same value of $M[\eta]$. This is how we are able to generate Figure \ref{fig:GPCunivCalib}, which maps every polymer to the same $\log(M[\eta])$ value based only on its $V_r$.
            \item We will now use this universal calibration curve to derive an unknown polymer's molar mass in terms of a known polymer's molar mass.
        \end{itemize}
        \item Recall from the Mark-Houwink model that $[\eta]=kM^a$.
        \item Let some standard $s$ (with retention volume $V_r$, as measured by GPC) have Mark-Houwink parameters $k_s,a_s$. Then for the standard,
        \begin{equation*}
            ([\eta]M)_s = k_sM_s^{a_s}\cdot M_s = k_sM_s^{1+a_s}
        \end{equation*}
        \begin{itemize}
            \item Similarly, for an unknown $u$ (with equal retention volume $V_r$, as measured by GPC), we can calculate
            \begin{equation*}
                ([\eta]M)_u = k_uM_u^{1+a_u}
            \end{equation*}
        \end{itemize}
        \item But if the standard and unknown have the same retention volume, then
        \begin{equation*}
            k_sM_s^{1+a_s} = ([\eta]M)_s
            = \text{const}\cdot V_r
            = ([\eta]M)_u
            = k_uM_u^{1+a_u}
        \end{equation*}
        and hence
        \begin{equation*}
            M_u = \left[ \frac{k_sM_s^{1+a_s}}{k_u} \right]^{\frac{1}{1+a_u}}
        \end{equation*}
        \begin{itemize}
            \item We can look up $k_s,k_u,a_s,a_u$, and then we can convert (from one measurement of $M_s$) to $M_u$ without having to create a purpose-built calibration curve with multiple samples of $u$.
        \end{itemize}
    \end{itemize}
    \item We now switch gears to osmometry, specifically membrane osmometry.
    \item Membrane osmometry overview, as in a capillary membrane osmometer.
    \begin{figure}[h!]
        \centering
        \includegraphics[width=0.33\linewidth]{membraneOsmometry.png}
        \caption{Membrane osmometer.}
        \label{fig:membraneOsmometry}
    \end{figure}
    \begin{itemize}
        \item Pure solvent on one side of a membrane, polymer solution on the other side. The membrane is semipermeable, letting solvent through but not polymer.
        \item Initially, you have solutions (open to atmosphere, and hence under constant pressure) of equal volume on both side of the membrane.
        \item Entropy will cause the solvent to want to mix with the polymer, with solvent flowing until the chemical potentials are equal.
    \end{itemize}
    \item \textbf{Osmotic pressure}: A colligative property which depends only on the number of solute molecules in the solution. \emph{Denoted by} $\bm{\Pi}$.
    \begin{itemize}
        \item Your equilibrium is established when the chemical potential $\mu_1^\circ$ of the pure solvent is equal to that of the other one ($\mu_1$) plus a mechanical force.
        \begin{equation*}
            \mu_1^\circ = \mu_1+\underbrace{\int_{P_0}^{P_0+\Pi}\left( \pdv{\mu_1}{P} \right)\dd{P}}_\text{work term}
        \end{equation*}
        \item We now invoke a Maxwell relation.
        \begin{align*}
            \left( \pdv{\mu}{P} \right) = \pdv{P}(\pdv{G}{n_1})
                = \pdv{n_1}(\pdv{G}{P})
                = \pdv{V}{n_1}
                = \bar{V}_1
        \end{align*}
        \begin{itemize}
            \item This is based on $\dd{G}=V\dd{P}-T\dd{S}+\sum_i\mu_i\dd{n_i}$.
            \item $\bar{V}_1$ is the partial molar volume of the solvent. It is a material property parameter.
        \end{itemize}
        \item It follows by evaluating the above integral that 
        \begin{align*}
            \mu_1^\circ &= \mu_1+\Pi\bar{V}_1\\
            \mu_1-\mu_1^\circ &= -\Pi\bar{V}_1
        \end{align*}
        \begin{itemize}
            \item The term on the right above is the \textbf{osmotic pressure}.
        \end{itemize}
    \end{itemize}
    \item The full derivation is on the slides, but Doyle will not go through it.
    \item We will now skip ahead to getting the Flory $\chi$ parameter.
    \begin{itemize}
        \item Just above, we derived an expression for $\mu_1-\mu_1^\circ$.
        \item But recall from Lecture 2.2 the additional expression
        \begin{equation*}
            \mu_1-\mu_1^\circ = RT\left[ \ln\phi_1+\left( 1-\frac{1}{N_2} \right)\phi_2+\chi\phi_2^2 \right]
        \end{equation*}
        \item Additionally, recall that with a dilute polymer solution ($\phi_2<\phi_1$), we may obtain from the Taylor series expansion of $\ln(1-\phi_2)$ that
        \begin{equation*}
            \mu_1-\mu_1^\circ = RT\left[ -\frac{\phi_2}{N_2}+\left( \chi-\frac{1}{2} \right)\phi_2^2 \right]
        \end{equation*}
        \item Setting equal the two expressions, we obtain
        \begin{align*}
            RT\left[ -\frac{\phi_2}{N_2}+\left( \chi-\frac{1}{2} \right)\phi_2^2 \right] &= -\Pi\bar{V}_1
        \end{align*}
    \end{itemize}
    \item Comments on the osmometry derivation.
    \begin{itemize}
        \item Some rearranging, and then putting it in terms of concentration.
        \item Then the slope in a $\Pi/c$ vs. $c$ plot gives us $\chi$, via the following equation.
        \begin{equation*}
            \frac{\Pi}{c} = RT\left[ \frac{1}{M_n}+\left( \frac{1}{2}-\chi \right)\frac{V_1N_2^2}{M_n^2}\cdot c \right]
        \end{equation*}
        \begin{itemize}
            \item This is the most important result, and main takeaway from osmometry theory.
        \end{itemize}
        \item We also get something analogous to the ideal gas law ($PV=nRT$ becomes $\Pi V=n_2RT$).
        \item The virial expansion has an analogous osmotic pressure form. $A_1$ is related to the molecular weight; $A_2$ is related to $\chi$, the density, and some other things.
        \item Takeaway: The full osmometry derivation is not testable material, but he would expect us to be able to bring in Flory-Huggins and work with limits of small/big $\phi$ to do the first couple of steps.
    \end{itemize}
    \item Last note on our osmometry theory: This theory works best when either (1) $\phi_2\ll\phi_1$ or (2) Flory-Huggins theory takes over in the melt case, i.e., $c>c^*$.
    \begin{itemize}
        \item So we want to be concentrated, but not too concentrated. "Thermodynamically concentrated but mathematically dilute."
    \end{itemize}
\end{itemize}



\section{Chapter 9: Dynamics of Dilute Polymer Solutions}
\emph{From \textcite{bib:HiemenzLodge}.}
\begin{itemize}
    \item \marginnote{10/9:}What relates "the viscosity of polymer solutions, the diffusion of polymer molecules, the technique of dynamic light scattering, the phenomenon of hydrodynamic interaction, and the separation and analysis of polymers by size exclusion chromatography" \parencite[377]{bib:HiemenzLodge}? They all help determine polymer molecular weight, and they all depend on the spatial extent of polymer coils.
    \item We have studied \emph{static} properties of polymer up to this point. Now, we will discuss \emph{time-dependent} dynamics.
    \begin{itemize}
        \item We begin our study of dynamics with dilute solutions, so as to highlight the properties of individual polymer molecules.
    \end{itemize}
    \item \textbf{Molecular friction factor}: The proportionality factor between the force $\mathbf{F}$ directionally applied to a polymer in a solvent, and the polymer's equilibrium velocity $\mathbf{v}$ after accelerating to the point where drag counterbalances $\mathbf{F}$. \emph{Units} \textbf{\si[text-series-to-math,per-mode=symbol]{\kilo\gram\per\second}}. \emph{Denoted by} $\bm{f}$. \emph{Given by}
    \begin{equation*}
        f = \frac{|\mathbf{F}|}{|\mathbf{v}|}
    \end{equation*}
    \begin{itemize}
        \item Figure \ref{fig:stokesViscosity} may be helpful in visualizing this definition.
        \item The textbook uses the units of \si[per-mode=symbol]{\gram\per\second}.
        \item Typical values of $f$ for a polymer in water fall between \SIrange[per-mode=symbol]{e-7}{e-6}{\gram\per\second}.
    \end{itemize}
    \item Two questions we will focus on answering.
    \begin{itemize}
        \item How can we expermentally measure $f$?
        \item What can $f$ tell us about polymers?
    \end{itemize}
    \item First model we will use: The impenetrable sphere.
    \begin{itemize}
        \item Surprisingly accurate for certain polymers (e.g., floppy random coils, as discussed in class).
        \item Allows us to use the computationally simple ideal solution model to underpin our analysis!
    \end{itemize}
    \item Defining the viscosity of a fluid, and related terms (see Figure \ref{fig:viscosity}).
    \begin{itemize}
        \item A \emph{fluid} is preliminarily defined to be "a set of infinitesimally thin layers moving parallel to each other, each with a characteristic velocity" \parencite[377]{bib:HiemenzLodge}.
        \item We postulate that our fluid has no \textbf{slip} at the interface between the stationary and flowing phases.
        \begin{itemize}
            \item This is a good approximation for the systems in which we are interested.
        \end{itemize}
        \item This leads our fluid to behave as a \textbf{steady flow}.
    \end{itemize}
    \item \textbf{Slip}: Any difference in velocity between those fluid layers that are adjacent to nonflowing surfaces and the nonflowing surfaces themselves.
    \item \textbf{Steady flow}: The time-independent veocity profile developed when the upper plate (Figure \ref{fig:viscosity}) is moving at a constant velocity.
    \item We now build up to justifying the notation "$\dot{\gamma}$" for the shear rate.
    \item \textbf{Shear displacement}: The distance $\Delta x$ that the top layer in the velocity profile moves relative to the bottom layer during a short time interval $\Delta t$.
    \item \textbf{Shear strain}: The shear displacement per unit distance $H$ between the two plates. \emph{Denoted by} $\gamma$. \emph{Given by}
    \begin{equation*}
        \gamma := \frac{\Delta x}{H}
    \end{equation*}
    \item Thus, the shear rate (i.e., the rate at which the shear strain develops) is given by the first time derivative of the shear strain $\dv*{\gamma}{t}$. But in Newton's notation, first derivatives are represented by putting a single dot over the variable being differentiated. Thus, we represent the shear rate as $\dot{\gamma}$.
    \item \textbf{Shear force}: The force applied to the top plate (Figure \ref{fig:viscosity}) to develop a velocity profile. \emph{Also known as} \textbf{viscous force}. \emph{Denoted by} $\mathbf{F}$.
    \item \textbf{Newton's law of viscosity}: The statement that the shear stress $\tau$ depends bilinearly on the viscosity $\eta$ and shear rate $\dot{\gamma}$ with no proportionality constant. \emph{Given by}
    \begin{equation*}
        \tau = \eta\dot{\gamma}
    \end{equation*}
    \item \textbf{Newtonian} (fluid): A fluid that satisfies Newton's law of viscosity, i.e., for which the viscosity $\eta$ is independent of the shear rate $\dot{\gamma}$.
    \begin{itemize}
        \item Liquids of low molecular weight compounds are generally Newtonian.
        \item Newtonian fluids are characterized by a single viscosity.
    \end{itemize}
    \item Any liquid that undergoes shear thinning is non-Newtonian, as a Newtonian liquid would have a straight line on the graphs in Figure \ref{fig:smallShear} ($\eta$ independent of $\dot{\gamma}$).
    \begin{itemize}
        \item Shear thinning is often observed for polymer solutions or melts.
    \end{itemize}
    \item Most viscometers can determine of a fluid is Newtonian by varying the shear rate $\dot{\gamma}$ and measuring whether or not the viscosity stays constant.
    \item \textcite[379-80]{bib:HiemenzLodge} covers \textbf{viscous heating}, which was not mentioned in class.
    \item \marginnote{10/14:}\textbf{Equation of motion} (of a fluid): The differential equation --- obtained by considering all $x$-, $y$-, and $z$-forces of gravitational or mechanical origin acting on a volume element of liquid --- whose solution would give the velocity $v$ of the flowing liquid as a function of time and position within the sample.
    \item Comments on Einstein's model for the viscosity of a suspension of hard spheres \parencite[384]{bib:HiemenzLodge}.
    \begin{itemize}
        \item The viscosity does not depend on the radius of the spheres, only their total volume fraction.
        \item By describing the concentration dependence of viscosity as a power series, Einstein's theory plays a comparable role for viscosity as our "virial expansion" in Flory-Huggins theory.
        \item The volume fraction emerges from the Einstein derivation as the natural concentration unit to describe viscosity. This parallels the way volume fraction arises as a natural thermodynamic concentration unit in the Flory-Huggins theory.
    \end{itemize}
    \item Common features and assumptions between Stokes' law and Einstein's model.
    \begin{itemize}
        \item The liquid medium is a continuum.
        \begin{itemize}
            \item It follows that the results may be suspect for spheres so small that the molecular nature of the solvent cannot be ignored.
            \item However, even in this case, the results often hold up experimentally surprisingly well.
        \end{itemize}
        \item Both relationships hold up well to experimental verification, across many systems and spheres of many diameters.
        \item For particles that are shaped differently than spheres (e.g., elongated ellipsoids of revolution), there are related derivations.
        \item The disturbance of the flow streamlines is assumed to be produced by a single particle, hence the limitation to dilute solutions. In other words, the net effect of an array of spheres is treated as the sum of the individual nonoverlapping disturbances. Moreover, contributions from the walls of the container are assumed to be absent.
    \end{itemize}
    \item On the hydrodynamic volume.
    \begin{itemize}
        \item By assuming that hydrodynamic volume is related to the volume of a sphere based on the radus of gyration, we assume that "the volume that matters in the viscosity experiment is not the volume actually occupied by the polymer segments (which would be the degree of polymerization times the volume of the monomer), but the volume pervaded by the entire molecule" \parencite[386]{bib:HiemenzLodge}.
    \end{itemize}
    \item \textbf{Poiseuille equation}: An expression for the flow rate through a vertical capillary. \emph{Given by}
    \begin{equation*}
        Q = \frac{\pi r^4}{8\eta L}(\rho g L+\Delta P)
    \end{equation*}
    \begin{itemize}
        \item \textcite[393-95]{bib:HiemenzLodge} fully derives this. The not-too-difficult calculus may be worth going through!
    \end{itemize}
    \item On co-centric cylinder rheometers \parencite[397-98]{bib:HiemenzLodge}.
    \begin{itemize}
        \item The range of applicability is very wide, extending at least from $\eta\approx\SIrange{0.01}{e10}{\pascal\second}$.
        \item The design permits different velocity gradients to be considered, so that non-Newtonian behavior (e.g., shear thinning) can be investigated.
        \item Some of the mathematics is derived.
    \end{itemize}
    \item \textbf{Diffusion coefficient}: A measure of how fast molecules move. \emph{Denoted by} $\bm{D}$.
    \item \textbf{Tracer} (diffusion coefficient): A diffusion coefficient describing the motion of a single Brownian particle. \emph{Units} \textbf{\si[text-series-to-math,per-mode=symbol]{\meter\squared\per\second}}. \emph{Denoted by} $\bm{D_t}$.
    \begin{itemize}
        \item The factor of 6 in $\prb{\Delta r^2}=6D\tau$ is a historical convention.
    \end{itemize}
    \item \textbf{van Hove space-time self-correlation function}: The probability of finding a particle a distance $r$ away from where it was at time $t=0$ at time $t=\tau$. \emph{Given by}
    \begin{equation*}
        P(r,\tau) = 4\pi r^2\left( \frac{1}{4\pi D_t\tau} \right)^{3/2}\exp(-\frac{r^2}{4D_t\tau})
    \end{equation*}
    \begin{itemize}
        \item Notice that the form is mathematically equivalent to the three-dimensional Gaussian distribution for a polymer's end-to-end distance derived in Lecture 1.3, using the substitution $6D_t\tau=\prb{\Delta r^2}=Na^2$.
    \end{itemize}
    \item \textbf{Stokes-Einstein equation}: An expression for the diffusion of a hard-sphere tracer particle of radius $r_s$ in a continuous solvent. \emph{Given by}
    \begin{equation*}
        D_t = \frac{\kB T}{6\pi\eta_sr_s}
    \end{equation*}
    \item \textbf{Hydrodynamic radius}: For any polymer or particle, the radius of the hard sphere that would have the same friction factor or diffusivity. \emph{Denoted by} $\bm{R_H}$. \emph{Given by}
    \begin{equation*}
        R_H := \frac{\kB T}{6\pi\eta_sD_t}
    \end{equation*}
    \begin{itemize}
        \item Since flexible molecules under hydrodynamic conditions behave like hard spheres with radius proportional to the radius of gyration, they also diffuse this way. This is where $R_H\propto R_g$ comes from!
        \item Another important set of proportionalities is $D_t\propto R_H^{-1}\propto M^{-\nu}$, for the typical values of $\nu$ depending on solvent quality.
    \end{itemize}
    \item \textbf{Shape factor}: The dimensionless proportionality constant between the hydrodynamic radius and radius of gyration. \emph{Denoted by} $\bm{\gamma}$. \emph{Given by}
    \begin{equation*}
        \gamma := \frac{R_g}{R_H}
    \end{equation*}
    \begin{itemize}
        \item As the name suggests, $\gamma$ provides insight into how spherical particles are: $\gamma\to 1$ as particles become more spherical.
    \end{itemize}
    \item \textbf{Mutual} (diffusion coefficient): A diffusion coefficient describing how a collection of Brownian particles will distribute themselves in space. \emph{Denoted by} $\bm{D_m}$.
    \begin{itemize}
        \item "Mutual diffusion acts to eliminate any gradients in concentration" \parencite[401]{bib:HiemenzLodge}.
    \end{itemize}
    \item On Fick's law.
    \begin{itemize}
        \item This equation quantifies the idea that mass diffusion is analogous to thermal diffusion (in the sense of eliminating concentration gradients), and is mathematically an adaptation of Fourier's law of heat conduction to the transport of material.
        \item $J$ has units of mass per area per time.
        \item The statement of Fick's law given corresponds to one-dimensional diffusion in the $x$-direction.
        \item \textcite[401-05]{bib:HiemenzLodge} covers Fick's other laws, and relates diffusion to chemical potential!
    \end{itemize}
    \item \textcite[406-09]{bib:HiemenzLodge} covers dynamic light scattering.
\end{itemize}




\end{document}