\documentclass[../notes.tex]{subfiles}

\pagestyle{main}
\renewcommand{\chaptermark}[1]{\markboth{\chaptername\ \thechapter\ (#1)}{}}
\setcounter{chapter}{2}

\begin{document}




\chapter{Dilute Solutions}
\section{Intrinsic Viscosity - 1}
\begin{itemize}
    \item \marginnote{10/7:}Announcements.
    \begin{itemize}
        \item Grade our HWs on Canvas.
        \item The profs are grading the first quiz.
        \item PSet 3 will be posted tomorrow morning.
        \begin{itemize}
            \item After today's lecture, we'll have a lot of fodder to get going on it!
        \end{itemize}
    \end{itemize}
    \item Overview of Topic 3.
    \begin{itemize}
        \item We've built up a lot of theory, but now we want to discuss how we measure the parameters we've introduced.
        \begin{itemize}
            \item Example: Measuring polymer size and conformation.
        \end{itemize}
        \item We'll also touch on why such parameters are important for various material properties.
        \begin{itemize}
            \item Examples of where polymer size and conformation are important: Non-entangled rubber elasticity, shear thiceners, elastic modulus of crosslinked networks, and electrical conductivity.
        \end{itemize}
    \end{itemize}
    \item Upcoming lectures by day.
    \begin{itemize}
        \item Today: Viscometry. What is the viscosity, intrisic viscosity, etc. of a polymer? How rheology tells us stuff about a polymer sample.
        \begin{itemize}
            \item Specifically, we'll relate the viscosity of a dilute polymer-solvent solution to the polymer molecular weight.
        \end{itemize}
        \item Thursday: Standard through state-of-the-art ways to measure the viscosity of a polymer solution.
        \item Next Tuesday: Fractionation (e.g., via GPC), and how it works (based on our theory). Also osmotic pressure.
        \begin{itemize}
            \item Fractionation techniques can provide us the full molecular weight distribution of a polymer sample.
            \item Osmotic pressure can provide information on molecular size and polymer-polymer interactions.
        \end{itemize}
        \item Next Thursday: A high-level discussion of light scattering. Different power sources (e.g., NMR), and how scattering tells us something about the polymer size.
    \end{itemize}
    \item Outline of today's lecture.
    \begin{itemize}
        \item Drag coefficient of a polymer.
        \begin{itemize}
            \item Also referred to (e.g., by the textbook) as the "friction factor."
            \item This number is related to a polymer's conformation.
        \end{itemize}
        \item Draining and free draining models.
        \item The diffusion coefficient, and how molecules randomly move in a solution due to thermal energy.
        \item Intrinsic viscisity.
        \begin{itemize}
            \item This is key, and the first three topics all build up to it.
            \item The intrinsic viscosity has a scaling with the diffusion coefficient?? Did we cover this today?
        \end{itemize}
        \item Mark-Houwink-Sakurada model.
    \end{itemize}
    \item Before we get to the viscosity of dilute solutions of polymers, let's introduce some basic concepts around just plain viscosity. This will basically serve as a "fluid dynamics 101."
    \item To help visualize the following definitions, consider this thought experiment.
    \begin{figure}[h!]
        \centering
        \includegraphics[width=0.3\linewidth]{viscosity.png}
        \caption{Viscosity visualization.}
        \label{fig:viscosity}
    \end{figure}
    \begin{itemize}
        \item Put a liquid between two plates, and move the top one.
        \item You can show (via the fluid equation or \textbf{Navier-Stokes equations}) that you will quickly develop a velocity profile. Molecular friction exerts a force, and the force is related to the \textbf{viscosity}.
        \item Consideration of the viscosity leads us to the other definitions below.
    \end{itemize}
    \item \textbf{Navier-Stokes equations}: The governing equations of fluids.
    \item \textbf{Viscosity}: A relationship which tells us something about when we exert stresses on surfaces as a result of flow, and the relation to how fast we're moving a fluid. \emph{Also known as} \textbf{coefficient of viscosity}. \emph{Units} \textbf{\si[text-series-to-math]{\pascal\second}}. \emph{Denoted by} $\bm{\eta}$. \emph{Given by}
    \begin{equation*}
        \eta := \frac{\text{shear stress}}{\text{rate of shear}}
    \end{equation*}
    \item \textbf{Shear stress}: The pressure resulting from the viscous force applied over the area of the moving plate. \emph{Units} \textbf{\si[text-series-to-math]{\pascal}}. \emph{Denoted by} $\bm{\tau}$. \emph{Given by}
    \begin{equation*}
        \tau := \frac{\text{viscous force}}{\text{area}}
    \end{equation*}
    \item \textbf{Shear rate}: How fast we move the plates relative to each other, normalized by the distance between the plates. \emph{Also known as} \textbf{velocity gradient}, \textbf{rate of shear}. \emph{Units} \textbf{\si[text-series-to-math]{\per\second}}. \emph{Denoted by} $\bm{\dot{\gamma}}$.\footnote{"gamma dot"} \emph{Given by}
    \begin{equation*}
        \dot{\gamma} := \frac{U}{H}
    \end{equation*}
    \begin{itemize}
        \item $U$ is the difference in velocity between the top and bottom plates.
        \item $H$ is the distance between the top and bottom plates.
    \end{itemize}
    \item \textbf{Viscous stress}. \emph{Also known as} \textbf{shear stress}. \emph{Denoted by} $\bm{\tau_{yx}}$. \emph{Given by}
    \begin{equation*}
        \tau_{yx} := \eta\dot{\gamma}
    \end{equation*}
    \begin{itemize}
        \item The faster you shear something, it's linearly proportional to the tensor $\tau_{yx}$?? What is this, and what is its relation to the definition of $\tau$ above?
    \end{itemize}
    \item \textbf{Newtonian} (fluid): A fluid that satisfies the following condition. \emph{Constraint}
    \begin{equation*}
        \eta \neq \eta(\gamma_\text{tot},\dot{\gamma})
    \end{equation*}
    \begin{itemize}
        \item What does this constraint mean?? Shouldn't a Newtonian fluid be one for which the viscosity is independent of the magnitude of the shear rate $\dot{\gamma}$?
    \end{itemize}
    \item \textbf{Non-Newtonian} (fluid): Any fluid that is not Newtonian.
    \begin{itemize}
        \item This encompases a \emph{broad} range of fluid properties.
        \item Polymers are very non-Newtonian.
    \end{itemize}
    \item This concludes our introduction to the terminology of fluid mechanics.
    \item We now address today's focus, the behavior of dilute ($\phi_2\ll 1$) polymer-solvent solutions at small shear rates ($\dot{\gamma}\ll 1$).
    \begin{figure}[h!]
        \centering
        \begin{subfigure}[b]{0.4\linewidth}
            \centering
            \includegraphics[width=0.95\linewidth]{smallSheara.png}
            \caption{Data on dilute PS solutions.}
            \label{fig:smallSheara}
        \end{subfigure}
        \begin{subfigure}[b]{0.4\linewidth}
            \centering
            \includegraphics[width=0.8\linewidth]{smallShearb.png}
            \caption{Shear thinning, graphically.}
            \label{fig:smallShearb}
        \end{subfigure}
        \caption{Polymer viscosity at varying shear rates.}
        \label{fig:smallShear}
    \end{figure}
    \begin{itemize}
        \item Figure \ref{fig:smallSheara} compiles a bunch of data from the literature on viscosity vs. shear rates.
        \begin{itemize}
            \item The $y$-axis is the viscosity $\eta_p$ of the polymer-solvent solution, normalized by said solution's asymptotic viscosity $\eta_{p0}$ as you go to very small shear rates.
            \item The $x$-axis is the shear rate, again normalized by some constant $\lambda_Z$. Here, we have picked a scaling $\lambda_Z$ that makes our data dimensionless (i.e., 1 on the $x$-axis) and reveals the two important trends that\dots
            \begin{itemize}
                \item Faster shearing ($\lambda_Z\dot{\gamma}>1$) leads to a decrease in viscosity known as \textbf{shear thinning};
                \item Lower shearing ($\Lambda_Z\dot{\gamma}<1$) leads to the asymptotic viscosity.
            \end{itemize}
            \item Below a certain shear rate, we observe Newtonian behavior and "zero shear viscosity."
        \end{itemize}
        \item Figure \ref{fig:smallShearb} sketches the idea of shear thinning.
        \begin{itemize}
            \item \textbf{Zero shear viscosity} occurs at left.
            \begin{itemize}
                \item Note that this regime would more aptly be termed the "low shear viscosity" regime, since we're only asymptotically approaching zero shear.
                \item However, "zero shear viscosity" is what's in the lexicon.
            \end{itemize}
            \item Shear thinning occurs in the middle.
            \item At right, we enter the asymptotic regime wherein most viscosity comes from the solvent, not the polymer.
        \end{itemize}
        \item Reference: \textcite{bib:smallShear}.
    \end{itemize}
    \item \textbf{Shear thinning}: A property of certain fluids wherein viscosity decreases the faster the fluid is sheared (i.e., as the shear rate $\dot{\gamma}$ increases).
    \item \textbf{Zero shear viscosity}: The viscosity under a shear rate sufficiently low that the fluid behaves as if its Newtonian. \emph{Denoted by} $\bm{\eta_0}$. \emph{Given by}
    \begin{equation*}
        \eta_0 := \lim_{\dot{\gamma}\to 0}\eta
    \end{equation*}
    \item \textbf{Weissenberg number}: A dimensionless shear rate. \emph{Denoted by} $\bm{Wi}$. \emph{Given by}
    \begin{equation*}
        Wi := \dot{\gamma}\tau_\text{polymer}
    \end{equation*}
    \item Before we dive deeper, we should state an important assumption about shear rate underpinning our analysis.
    \begin{figure}[h!]
        \centering
        \begin{subfigure}[b]{0.3\linewidth}
            \centering
            \includegraphics[width=0.95\linewidth]{stretchTumblea.jpeg}
            \caption{Terminology definitions.}
            \label{fig:stretchTumblea}
        \end{subfigure}
        \begin{subfigure}[b]{0.3\linewidth}
            \centering
            \includegraphics[width=0.4\linewidth]{stretchTumbleb.png}
            \caption{DNA being sheared.}
            \label{fig:stretchTumbleb}
        \end{subfigure}
        \caption{Stretching and tumbling upon shearing.}
        \label{fig:stretchTumble}
    \end{figure}
    \begin{itemize}
        \item In this class, we will assume that polymer coils are \emph{not} deformed by the flow (i.e., $Wi<1$).
        \item Typically, polymers \textbf{stretch} and \textbf{tumble} when $Wi>0$ (Figure \ref{fig:stretchTumblea}).
        \begin{itemize}
            \item For example, DNA has been micrographed stretching and tumbling at higher shear rates and Weissenberg numbers (Figure \ref{fig:stretchTumbleb}).
        \end{itemize}
        \item Reference: \textcite{bib:stretchTumble}.
    \end{itemize}
    \item \textbf{Stretch}: A behavior of a sheared polymer in which a usually flexible coil is elongated by shear forces.
    \item \textbf{Tumble}: A behavior of a sheared polymer in which a stretched coil is compressed with rotation by shear forces.
    \item We now describe how the Navier-Stokes equations treat viscosity.
    \begin{figure}[h!]
        \centering
        \includegraphics[width=0.1\linewidth]{stokesViscosity.png}
        \caption{Drag coefficient on a hard sphere in a viscous fluid.}
        \label{fig:stokesViscosity}
    \end{figure}
    \begin{itemize}
        \item Consider a hard sphere of size $r_s$ falling through a fluid of viscosity $\eta_0$ with velocity $v$.
        \item By Newton's laws, the fluid will exert a viscous force $F_\text{viscous}$ on the sphere. Moreover, this force will be given by the following.
        \begin{equation*}
            F_\text{viscous} = fv
        \end{equation*}
        \begin{itemize}
            \item $v$ is the velocity with which the sphere is moving.
            \item $f$ is the drag coefficient or friction factor. It can be thought of as the proportionality factor between the velocity and viscous force.
        \end{itemize}
        \item Solving the Navier-Stokes equations for this scenario, we obtain \textbf{Stokes' law}.
        \item How, then, can we treat a polymer molecule moving through a viscous fluid? There are two limiting cases.
        \begin{enumerate}
            \item The polymer coil acts as an impenetrable sphere with dimensions given by the pervaded volume.
            \item The polymer coil is entirely penetrable, with only individual monomers interacting with the viscous fluid.
        \end{enumerate}
    \end{itemize}
    \item \textbf{Stokes' law}: The Navier-Stokes equations' relation between the drag coefficient $f$ on a sphere of size $r_s$ falling through a fluid, and the fluid's viscosity $\eta_0$. \emph{Given by}
    \begin{equation*}
        f = 6\pi\eta_0r_s
    \end{equation*}
    \begin{itemize}
        \item The $6\pi$ falls out of the math when solving the Navier-Stokes equations.
        \item It makes intuitive sense that $f\propto\eta_0$ and $f\propto r_s$, so it is not hard to rationalize or visualize that such proportionality manifests itself as a bilinear relation.
    \end{itemize}
    \item Let's now extend our investigation into the possible behaviors of a polymer molecule in shear flow.
    \begin{figure}[h!]
        \centering
        \begin{subfigure}[b]{0.35\linewidth}
            \centering
            \includegraphics[width=0.85\linewidth]{polFluida.png}
            \caption{Penetrable coil.}
            \label{fig:polFluida}
        \end{subfigure}
        \begin{subfigure}[b]{0.35\linewidth}
            \centering
            \includegraphics[width=0.7\linewidth]{polFluidb.JPG}
            \caption{Penetrable coil (nanoscale).}
            \label{fig:polFluidb}
        \end{subfigure}\\[2em]
        \begin{subfigure}[b]{0.35\linewidth}
            \centering
            \includegraphics[width=0.8\linewidth]{polFluidc.png}
            \caption{Impenetrable coil.}
            \label{fig:polFluidc}
        \end{subfigure}
        \begin{subfigure}[b]{0.35\linewidth}
            \centering
            \includegraphics[width=0.95\linewidth]{polFluidd.png}
            \caption{Rod-like polymer.}
            \label{fig:polFluidd}
        \end{subfigure}
        \caption{Modeling a polymer in a viscous fluid.}
        \label{fig:polFluid}
    \end{figure}
    \begin{enumerate}
        \item Penetrable coil, i.e., Rouse's free draining model.
        \begin{equation*}
            f \propto N_2\xi^*
        \end{equation*}
        \begin{itemize}
            \item Notation.
            \begin{itemize}
                \item We still let $N_2$ be the number of segments per chain (e.g., number of Kuhn steps).
                \item $\xi^*$ is the \textbf{monomeric friction factor}.
            \end{itemize}
            \item Globally, the fluid passes through the coil freely (Figure \ref{fig:polFluida}).
            \item However, very locally (Figure \ref{fig:polFluidb}), there is no slip. In other words, each monomer contributes a small amount $\xi^*$ to the total friction factor. It follows that in this case, the friction factor scales linearly with the number of units $N_2$ in the polymer!
        \end{itemize}
        \item Impenetrable sphere, i.e., Zimm's non-free draining model (Figure \ref{fig:polFluidc}).
        \begin{align*}
            f &= 6\pi\eta_0R_H&
            R_H &\propto \prb{R_g^2}^{1/2}&
            f &\propto N_2^\nu
        \end{align*}
        \begin{itemize}
            \item $R_H$ denotes the sphere's \textbf{hydrodynamic radius}.
            \item The rightmost relation follows from transitivity and the scaling of $R_g$.
            \begin{itemize}
                \item $\nu$ is the same measure of solvent quality (e.g., $1/2$, $3/5$, or $1/3$) discussed previously.
            \end{itemize}
            \item There is a strong hydrodynamic interaction (HI), or coupling, of the motion of monomers in dilute solution.
            \begin{itemize}
                \item Essentially, when one monomer moves somewhere along a polymer chain, it yanks others in the chain along with it.
                \item $\text{HI}\propto 1/r$, where $r$ is the relative separation of monomers (like the bond length??).
                \item Additional equation for $\mathbf{v}(\mathbf{r})$??
            \end{itemize}
            \item This model is appropriate for a (1) high molecular weight and (2) flexible polymer.
            \item Is $2R_H$ the radius or diameter of the hydrodynamic sphere?? See the slides.
        \end{itemize}
        \item A rod-like molecule (Figure \ref{fig:polFluidd}).
        \begin{align*}
            f &= 6\pi\eta_0R_H&
            R_H &\propto R_\text{max} = L&
            f &\propto N_2
        \end{align*}
        \begin{itemize}
            \item This is one more commonly encountered possiblity.
            \item The above relations should be fairly self-explanatory.
        \end{itemize}
    \end{enumerate}
    \item When choosing which of the above three models to use for a given situation, remember that the one that is most appropriate depends on the size and concentration of the polymers.
    \item How might we measure the friction factor $f$?
    \begin{figure}[h!]
        \centering
        \includegraphics[width=0.13\linewidth]{diffusion.JPG}
        \caption{A diffusing molecule.}
        \label{fig:diffusion}
    \end{figure}
    \begin{itemize}
        \item Consider the Stokes-Einstein diffusivity / diffusion coefficient
        \begin{equation*}
            D_t = \frac{\kB T}{\xi}
        \end{equation*}
        \begin{itemize}
            \item $\xi=f$ is the drag coefficient / friction factor.
            \item We are taking this to be of a tracer molecule, i.e., some molecule in solution far from everything else.
            \item $\kB T$ is an energy (i.e., a force times a length), and the friction factor has been defined as the quotient of (the viscous) force divided by a velocity. Thus, the units cancel out to length squared per time for $D$.
        \end{itemize}
        \item We have a flux $J$ satisfying \textbf{Fick's law}.
        \begin{itemize}
            \item This is a continuum view of diffusion.
        \end{itemize}
        \item The average squared displacement of a molecule as it moves around over a period of time $\tau$ is
        \begin{equation*}
            \prb{\Delta x^2+\Delta y^2+\Delta z^2}_\tau = 6D\tau
        \end{equation*}
        \begin{itemize}
            \item This is a molecular view of diffusion, from time $t=0$ to time $t=\tau$ (Figure \ref{fig:diffusion}).
            \item $D$ has units of length squared over time here, as well!
        \end{itemize}
        \item Thus, to actually get the diffusion coefficient, we could track molecules, e.g., by taking movies of them.
        \begin{itemize}
            \item Take movies of everywhere it goes, break up the path into segments $\tau$, and then take the ensemble or time average
            \begin{equation*}
                \prb{\Delta r^2} = 6D\tau
            \end{equation*}
        \end{itemize}
        \item It follows by combining several of the above equations that
        \begin{equation*}
            R_H = \frac{\kB T}{6\pi\eta D}
        \end{equation*}
        \item Additionally, it is important that
        \begin{equation*}
            R_H \propto R_g \propto R
        \end{equation*}
    \end{itemize}
    \item \textbf{Fick's law}: A relationship between flux, diffusivity, concentration, and distance. \emph{Given by}
    \begin{equation*}
        J = -D\pdv{c}{x}
    \end{equation*}
    \begin{itemize}
        \item $J$ is the flux.
        \item $c$ is the concentration.
        \item $x$ is the distance.
    \end{itemize}
    \item How long does a polymer have to be to obey Zimm scaling?
    \begin{itemize}
        \item For most systems in which we are interested, we will use Zimm's non-draining sphere.
        \item However, there is still significant debate in the literature about when this non-draining sphere assumption is ok to use.
        \item \textcite{bib:ZimmLength1} posits that once $N>10$, we can use this model.
        \item \textcite{bib:ZimmLength2} --- published just two years ago --- posits that we need to wait until $N>30$.
        \item Many polymers we have are well into the hundreds or thousands, though. Above $N=100$ is a generally solid cutoff for this behavior, so we can think of these as very good and very non-draining.
    \end{itemize}
    \item Scaling behavior of the friction factor.
    \begin{itemize}
        \item $\nu$ values reviewed again.
        \item For high MW flexible chains, the correct model is non-draining impenetrable units (Zimm-like).
        \begin{itemize}
            \item As discussed above, this model obeys $\xi=f=6\pi\eta_sR_H$, where $R_H\propto R_g$.
        \end{itemize}
    \end{itemize}
    \item $R_H$ vs. $R_g$ vs. $R$.
    \begin{table}[h!]
        \centering
        \includegraphics[width=0.6\linewidth]{RgRH.png}
        \caption{Ratio of radius of gyration and hydrodynamic radius for different polymer architectures.}
        \label{tab:RgRH}
    \end{table}
    \begin{itemize}
        \item The prefactor $\gamma$ in
        \begin{equation*}
            R_H = \gamma R_g
        \end{equation*}
        is on the order of 1.
        \begin{itemize}
            \item The above equation is really important! We will use it later today.
        \end{itemize}
        \item Recall that $R_H$ is related to a hydrodynamic interaction tensor??
        \item Note that $\gamma=R_g/R_H$ converges to 1 (middle-right column in Table \ref{tab:RgRH}) as the number of arms on a polymer increases because, as you get more arms, you do just start to look like a sphere.
        \item \textcite{bib:RubinsteinColby} has some good explanations on this; Table \ref{tab:RgRH} is also lifted from "Table 8.4" in this textbook.
    \end{itemize}
    \item For the rest of class, we focus on viscosity. We'll get as far into this topic as we can toay, and then we'll carry on next time.
    \item Einstein model for viscosity.
    \begin{itemize}
        \item Simulating viscosity in full is a computationally intensive process.
        \item On pen and paper, we will focus on what we can do for idealized hard spheres in a Newtonian fluid. This is the setup Einstein considered.
        \item Let the hard spheres be of volume fraction $\phi_{hs}$ in the solution.
        \item Einstein did his calculation in the dilute limit, obtaining
        \begin{equation*}
            \eta(\phi_{hs}) = \eta_s(1+2.5\phi_{hs}+\cdots)
        \end{equation*}
        \begin{itemize}
            \item Einstein's detailed calculation got $2.5$ as the linear correction.
            \item The power series continues, but in the dilute limit ($\phi_2\ll 1$ and $c_2\ll c_2^*$, where $c_2^*$ is the overlap concentration), we only need the linear correction.
        \end{itemize}
        \item This solution will have a viscosity $\eta$, composed of a linear contribution from the solvent $\eta_s$, and then an added contribution from the polymer in the form of a \textbf{specific viscosity} $\eta_{sp}$. This gives us the following.
        \begin{equation*}
            \eta = \eta_s+\eta_s\eta_{sp}
        \end{equation*}
        \item Thus, from the above by transitivity,
        \begin{equation*}
            \frac{\eta}{\eta_s}-1 = \eta_{sp} = \underbrace{2.5\phi_{hs}}_\text{for hard spheres}+\cdots
        \end{equation*}
        \item From the specific viscosity, we can define the \textbf{intrinsic viscosity} $[\eta]$.
        \item It follows that in general, our solution has a viscosity
        \begin{equation*}
            \eta = \eta_s\left( 1+c\cdot[\eta]+K_\text{H}c^2[\eta]^2+\cdots \right)
        \end{equation*}
        \begin{itemize}
            \item This equation represents viscosity as a perturbation in concentration $c$.
            \item The first prefactor is the intrinsic viscocity.
            \item $K_\text{H}$ denotes the \textbf{Huggins coefficient}.
        \end{itemize}
        \item We now want to relate the intrinsic viscosity to our polymer models.
        \begin{itemize}
            \item Let's go back to hard spheres.
            \item The volume fraction of hard spheres is the number of spheres times the sphere volume $V_{hs}$ divided by the total volume.
            \item To get it in terms of concentration so as to bridge this model over to our new concentration-based model, we obtain
            \begin{equation*}
                \phi_{hs} = \frac{(\#\text{ spheres})(\text{vol sphere})}{\text{total volume}}
                = \frac{cN_\text{A}V_{hs}}{M}
            \end{equation*}
            \begin{itemize}
                \item We measure concentration in \si[per-mode=symbol]{\gram\per\liter}.
                \item We measure molar mass in \si[per-mode=symbol]{\gram\per\mole}.
                \item The units of Avogadro's number $N_\text{A}$ is \si{\per\mole}.
                \item Thus, in the above expression, the units do work out correctly to number times volume divided by volume!
            \end{itemize}
        \end{itemize}
        \item Thus, we can now put everything together.
        \begin{itemize}
            \item The Einstein model's correction is
            \begin{equation*}
                2.5\phi_{hs} = \frac{2.5cN_\text{A}V_{hs}}{M}
            \end{equation*}
            \item We can relate the hard sphere volume back to polymer radius with
            \begin{equation*}
                V_{hs} = \frac{4}{3}\pi R_{hs}^3
            \end{equation*}
            \item The specific viscosity is then equal to $2.5\phi_{hs}$ by definition, so
            \begin{equation*}
                \eta_{sp} = \frac{2.5\cdot 4\pi R_H^3cN_\text{A}}{3M}
            \end{equation*}
            where we have replaced the hard sphere radius with the polymer's hydrodynamic radius.
            \item Now $R_H=\gamma R_g$, as mentioned earlier today. Additionally, we can divide through by concentration to turn the specific viscosity into an intrinsic viscosity. Therefore,
            \begin{equation*}
                [\eta] = \frac{10N_\text{A}\pi\gamma^3\prb{R_g^2}^{3/2}}{3M}
            \end{equation*}
            \item The above result is our final theory --- a very commonly used theory.
        \end{itemize}
        \item It follows that for our polymer-solvent solution, we have the following scaling.
        \begin{equation*}
            [\eta] \propto \frac{R_g^3}{M} \propto M^{3\nu-1} \propto N^{3\nu-1}
        \end{equation*}
        \begin{itemize}
            \item We still have $\nu$ dependent on the solvent type exactly as discussed repeatedly above.
            \item Thus, for a $\theta$ solvent, scaling of the intrinsic viscosity is $3(1/2)-1=1/2$. For a good solvent, $3(3/5)-1=4/5$.
            \item Note that we get the scaling because $R_g\propto R\propto N^\nu$, and $N\propto M$.
        \end{itemize}
    \end{itemize}
    \item \textbf{Specific viscosity}: A dimensionless quantity describing how much more viscous a solution becomes as polymers are introduced. \emph{Denoted by} $\bm{\eta_{sp}}$. \emph{Given by}
    \begin{equation*}
        \eta_{sp} := \frac{\eta}{\eta_s}-1
    \end{equation*}
    \item \textbf{Intrinsic viscosity}: A measure of a solute's contribution to the viscosity of a solution. \emph{Units} $\textbf{conc}^{\bm{-1}}$. \emph{Denoted by} $\bm{[\eta]}$. \emph{Given by}
    \begin{equation*}
        [\eta] := \lim_{c\to 0}\left( \frac{\eta_{sp}}{c} \right)
        = \lim_{c\to 0}\left( \frac{\eta-\eta_s}{c\eta_s} \right)
    \end{equation*}
    \item \textbf{Huggins coefficient}: The viscometric equivalent of a Virial coefficient. \emph{Denoted by} $\bm{K_H}$.
    \item Implicit assumptions used today.
    \begin{itemize}
        \item We are in a dilute solution.
        \item If we use this viscosity in a real flow, our Weissenberg number must be much less than one ($Wi\ll 1$, where $Wi:=\tau_p\dot{\gamma}$).
        \item The Zimm/non-draining argument is essential.
    \end{itemize}
    \item Do we need to know anything else about the radius of gyration?
    \begin{itemize}
        \item We will not be doing any detailed calcualations, so no.
        \item Just know that all of the different radii are applicable in different circumstances, and all related by order 1 coefficients. Thus, we can convert between all of them as needed.
    \end{itemize}
    \item \textbf{Mark-Houwink-Sakurada model}: A scaling of the intrinsic viscosity derived above. \emph{Given by}
    \begin{equation*}
        [\eta] = KM^a
    \end{equation*}
    \begin{itemize}
        \item $K,a$ depend on solvent quality and polymer.
        \item Exponent scaling.
        \begin{itemize}
            \item $a<1/2$ implies a poor solvent.
            \item $a=1/2$ implies a $\theta$ solvent.
            \item $4/5>a>1/2$ implies a good solvent.
        \end{itemize}
        \item There's lots of data for various polymer types.
    \end{itemize}
\end{itemize}



\section{Chapter 9: Dynamics of Dilute Polymer Solutions}
\emph{From \textcite{bib:HiemenzLodge}.}
\begin{itemize}
    \item \marginnote{10/9:}What relates "the viscosity of polymer solutions, the diffusion of polymer molecules, the technique of dynamic light scattering, the phenomenon of hydrodynamic interaction, and the separation and analysis of polymers by size exclusion chromatography" \parencite[377]{bib:HiemenzLodge}? They all help determine polymer molecular weight, and they all depend on the spatial extent of polymer coils.
    \item We have studied \emph{static} properties of polymer up to this point. Now, we will discuss \emph{time-dependent} dynamics.
    \begin{itemize}
        \item We begin our study of dynamics with dilute solutions, so as to highlight the properties of individual polymer molecules.
    \end{itemize}
    \item \textbf{Molecular friction factor}: The proportionality factor between the force $\mathbf{F}$ directionally applied to a polymer in a solvent, and the polymer's equilibrium velocity $\mathbf{v}$ after accelerating to the point where drag counterbalances $\mathbf{F}$. \emph{Units} \textbf{\si[text-family-to-math,per-mode=symbol]{\kilo\gram\per\second}}. \emph{Denoted by} $\bm{f}$. \emph{Given by}
    \begin{equation*}
        f = \frac{|\mathbf{F}|}{|\mathbf{v}|}
    \end{equation*}
    \begin{itemize}
        \item Figure \ref{fig:stokesViscosity} may be helpful in visualizing this definition.
        \item The textbook uses the units of \si[per-mode=symbol]{\gram\per\second}.
        \item Typical values of $f$ for a polymer in water fall between \SIrange[per-mode=symbol]{e-7}{e-6}{\gram\per\second}.
    \end{itemize}
    \item Two questions we will focus on answering.
    \begin{itemize}
        \item How can we expermentally measure $f$?
        \item What can $f$ tell us about polymers?
    \end{itemize}
    \item First model we will use: The impenetrable sphere.
    \begin{itemize}
        \item Surprisingly accurate for certain polymers (e.g., floppy random coils, as discussed in class).
        \item Allows us to use the computationally simple ideal solution model to underpin our analysis!
    \end{itemize}
    \item Defining the viscosity of a fluid, and related terms (see Figure \ref{fig:viscosity}).
    \begin{itemize}
        \item A \emph{fluid} is preliminarily defined to be "a set of infinitesimally thin layers moving parallel to each other, each with a characteristic velocity" \parencite[377]{bib:HiemenzLodge}.
        \item We postulate that our fluid has no \textbf{slip} at the interface between the stationary and flowing phases.
        \begin{itemize}
            \item This is a good approximation for the systems in which we are interested.
        \end{itemize}
        \item This leads our fluid to behave as a \textbf{steady flow}.
    \end{itemize}
    \item \textbf{Slip}: Any difference in velocity between those fluid layers that are adjacent to nonflowing surfaces and the nonflowing surfaces themselves.
    \item \textbf{Steady flow}: The time-independent veocity profile developed when the upper plate (Figure \ref{fig:viscosity}) is moving at a constant velocity.
    \item We now build up to justifying the notation "$\dot{\gamma}$" for the shear rate.
    \item \textbf{Shear displacement}: The distance $\Delta x$ that the top layer in the velocity profile moves relative to the bottom layer during a short time interval $\Delta t$.
    \item \textbf{Shear strain}: The shear displacement per unit distance $H$ between the two plates. \emph{Denoted by} $\gamma$. \emph{Given by}
    \begin{equation*}
        \gamma := \frac{\Delta x}{H}
    \end{equation*}
    \item Thus, the shear rate (i.e., the rate at which the shear strain develops) is given by the first time derivative of the shear strain $\dv*{\gamma}{t}$. But in Newton's notation, first derivatives are represented by putting a single dot over the variable being differentiated. Thus, we represent the shear rate as $\dot{\gamma}$.
    \item \textbf{Shear force}: The force applied to the top plate (Figure \ref{fig:viscosity}) to develop a velocity profile. \emph{Also known as} \textbf{viscous force}. \emph{Denoted by} $\mathbf{F}$.
    \item \textbf{Newton's law of viscosity}: The statement that the shear stress $\tau$ depends bilinearly on the viscosity $\eta$ and shear rate $\dot{\gamma}$ with no proportionality constant. \emph{Given by}
    \begin{equation*}
        \tau = \eta\dot{\gamma}
    \end{equation*}
    \item \textbf{Newtonian} (fluid): A fluid that satisfies Newton's law of viscosity, i.e., for which the viscosity $\eta$ is independent of the shear rate $\dot{\gamma}$.
    \begin{itemize}
        \item Liquids of low molecular weight compounds are generally Newtonian.
        \item Newtonian fluids are characterized by a single viscosity.
    \end{itemize}
    \item Any liquid that undergoes shear thinning is non-Newtonian, as a Newtonian liquid would have a straight line on the graphs in Figure \ref{fig:smallShear} ($\eta$ independent of $\dot{\gamma}$).
    \begin{itemize}
        \item Shear thinning is often observed for polymer solutions or melts.
    \end{itemize}
    \item Most viscometers can determine of a fluid is Newtonian by varying the shear rate $\dot{\gamma}$ and measuring whether or not the viscosity stays constant.
    \item \textcite[379-80]{bib:HiemenzLodge} covers \textbf{viscous heating}, which was not mentioned in class.
\end{itemize}




\end{document}