\documentclass[../notes.tex]{subfiles}

\pagestyle{main}
\renewcommand{\chaptermark}[1]{\markboth{\chaptername\ \thechapter\ (#1)}{}}
\stepcounter{chapter}

\begin{document}




\chapter{Solution Thermodynamics}
\section{Flory-Huggins Theory}
\begin{itemize}
    \item \marginnote{9/18:}Outline of the next three lectures.
    \begin{itemize}
        \item Thermodynamics of polymer solutions and blends.
        \begin{itemize}
            \item Entropy of mixing.
            \item Enthalpy of mixing.
            \item Flory interaction parameter (definition and measurements).
            \item Solutions and melts (the theta temperature).
            \item LCST vs. UCST.
        \end{itemize}
        \item Copolymers.
        \begin{itemize}
            \item Microphase separation.
            \item Interfacial free energy.
            \item Chain stretching and configurational free energy.
        \end{itemize}
    \end{itemize}
    \item Huggins\footnote{"HOY-gins"}.
    \item We're going to start by approximating polymers as straight chains on a lattice.
    \begin{figure}[h!]
        \centering
        \includegraphics[width=0.6\linewidth]{polymPhase.png}
        \caption{Lattice theory for polymer phase behavior.}
        \label{fig:polymPhase}
    \end{figure}
    \begin{itemize}
        \item When we heat them up, will they stay phase separated or mix (go into State 2)?
        \item This will depend on how \textbf{compatible} they are.
    \end{itemize}
    \item \textbf{Compatible} (polymers): Two different types of polymers that like to mix with each other to form a single phase.
    \item The original model for phase behavior was postulated by \textcite{bib:alloyPhase} for small molecules and alloys.
    \begin{itemize}
        \item \textcite{bib:polymPhasea} and \textcite{bib:polymPhaseb} generalized this model to longer things (polymers).
        \item Flory was a Stanford prof., but started in the petroleum business (needed to separate chains and understand how they behave).
    \end{itemize}
    \item The thermodynamics of polymer solutions and blends are important for many applications, such as\dots
    \begin{itemize}
        \item Phase diagrams;
        \item Fractionation by molecular weight and/or by composition;
        \item $T_m$ depression in a semicrystalline polymer by a 2nd component;
        \item Swelling behavior of networks/gels;
        \begin{itemize}
            \item Covered much later in the course.
        \end{itemize}
    \end{itemize}
    \item High-impact polystyrene (HIPS).
    \begin{figure}[h!]
        \centering
        \includegraphics[width=0.4\linewidth]{salamiPhase.png}
        \caption{Salami phase micrograph in styrene-butadiene mixtures.}
        \label{fig:salamiPhase}
    \end{figure}
    \begin{itemize}
        \item Example: circular plastic dishes in lab. Hard, but very brittle.
        \item Idea to make better: Mix a stiff but brittle polymer (PS) with a soft elastic polymer (polybutadiene, PB) to get better mechanical properties.
        \item Cracks cannot propagate because they will hit rubbery phases of PB that have phase separated on the nanoscale, and be absorbed.
        \begin{itemize}
            \item This is called the \textbf{salami phase} because of how micrographs look.
        \end{itemize}
        \item Effect: Stress-strain curve elongates significantly (\textbf{toughness} increases because that is the area under the curve).
    \end{itemize}
    \item Aside: Making things look clear requires a lot of polymer engineering, because you have to make things very amorphous and not have nanoscale crystals.
    \item Thermodynamics of polymer blends.
    \begin{itemize}
        \item Legos are made of statistical copolymers of acrylonitrile-butadiene-styrene (ABS).
        \item Acrylonitrile gives resistance to repeated clicking and unclicking, butadiene makes it rubbery, styrene makes it shiny.
    \end{itemize}
    \item Today: Derive a free energy functional.
    \begin{itemize}
        \item Last lecture, we derived a free energy functional for single chains.
    \end{itemize}
    \item Today, we're looking at $G=H-TS$.
    \begin{itemize}
        \item What we're really interested in is the free energy of mixing,
        \begin{equation*}
            \Delta G_M = G_{1,2}-(G_1+G_2)
        \end{equation*}
        \item In multicomponent systems --- besides the typical parameters of excluded volume, etc. --- we need to know\dots
        \begin{itemize}
            \item How many chains we have of each type;
            \item What their degree of polymerization is;
            \item What total volume do they occupy.
        \end{itemize}
        \item Thus, in State 1, we have
        \begin{align*}
            V_1 &= n_1N_1v_1&
            V_2 &= n_2N_2v_2
        \end{align*}
        which describes two separate phases\dots
        \begin{itemize}
            \item Containing $n_i$ moles of species $i$;
            \item With degree of polymerization $N_i$;
            \item Each occupying a total volume $V_i$;
            \item Where the volume of each monomer/solvent molecule is given by $v_i$.
        \end{itemize}
        \item In State 2, we have a mixed phase with total volume
        \begin{equation*}
            V = \underbrace{n_1N_1v_1}_{V_1}+\underbrace{n_2N_2v_2}_{V_2}
        \end{equation*}
        \item Note that we're assuming that there is no change in volume $\Delta V$ during mixing.
        \item Nomenclature: If a system is comprised of a solvent and polymer, name the solvent "1" and the polymer "2".
    \end{itemize}
    \item To understand the thermodynamics of mixing, we'll start with the \emph{entropy} of mixing.
    \begin{itemize}
        \item Comments.
        \begin{itemize}
            \item In a melt, most chains do not feel themselves because other chains screen the interaction of the original chain with itself.
            \item This is great for us, because everything behaves like a truly random walk with scaling relation $N^{1/2}$.
            \begin{itemize}
                \item This realization is what won Flory his Nobel Prize!
            \end{itemize}
        \end{itemize}
        \item There is only 1 possible way to arrange a pure component in its volume.
        \begin{itemize}
            \item This follows from the binomial expression $\binom{n_iN_i}{n_iN_i}$.
            \item Thus, when phases are separated, each phase has entropy $S=0$.
        \end{itemize}
    \end{itemize}
    \item Mean field lattice theory.
    \begin{figure}[H]
        \centering
        \includegraphics[width=0.3\linewidth]{latticeConfig.JPG}
        \caption{Configurations in mean field lattice theory.}
        \label{fig:latticeConfig}
    \end{figure}
    \begin{itemize}
        \item We get another binomial because we're adjacent to a random walk where we have 1 or 2 in each adjacent cell as we go along.
        \item As we fill up the grid, we first have access to all $X_0$ of the objects. Then we have 1 less, then we have 2 less, etc.
        \begin{itemize}
            \item But since all the objects in group 1 or group 2 are the same, we need to divide out by the number of objects $X_1$ in category 1. We need to do the same because all objects in $X_2$ are the same.
            \item Thus,
            \begin{equation*}
                \Omega_{1,2} = \frac{X_0!}{X_1!X_2!}
            \end{equation*}
        \end{itemize}
        \item It follows that the change entropy $\Delta S$ upon mixing is
        \begin{equation*}
            \Delta S_M = \kB\ln\Omega_{1,2}-0
        \end{equation*}
        \begin{itemize}
            \item Remember that the initial entropy is zero!
        \end{itemize}
        \item Invoking Stirling's approximation and remembering that $X_1+X_2=X_0$, we can then get
        \begin{align*}
            \frac{\Delta S_M}{\kB} &= X_0\ln X_0-X_0-\left[ X_1\ln X_1-X_1+X_2\ln X_2-X_2 \right]\\
            &= X_0\ln X_0-X_1\ln X_1-X_2\ln X_2
        \end{align*}
        \item It follows that the entropy of mixing per site $\Delta S_M/\kB X_0$ is
        \begin{align*}
            \frac{\Delta S_M}{\kB X_0} &= \frac{1}{X_0}\left[ (X_1+X_2)\ln X_0-X_1\ln X_1-X_2\ln X_2 \right]\\
            &= -\frac{X_1}{X_0}(\ln X_1-\ln X_0)-\frac{X_2}{X_0}(\ln X_2-\ln X_0)\\
            &= -\frac{X_1}{X_0}\ln\frac{X_1}{X_0}-\frac{X_2}{X_0}\ln\frac{X_2}{X_0}\\
            &= -\phi_1\ln\phi_1-\phi_2\ln\phi_2
        \end{align*}
        \begin{itemize}
            \item The new variables $\phi_i=X_i/X_0$ are the volume fractions for spaces $i$.
            \item Consequence: $\phi_1+\phi_2=1$.
        \end{itemize}
    \end{itemize}
    \item Note: An assumption underlying the use of the Boltzmann equation is that all microstates have equal energy. This isn't strictly true, but it's a good enough approximation.
    \begin{figure}[h!]
        \centering
        \begin{subfigure}[b]{0.25\linewidth}
            \centering
            \begin{tikzpicture}
                \draw [xshift=-0.5cm,yshift=-0.5cm] (0,0) grid (2,2);
                \draw [gray,dashed] (0,0) rectangle (1,1);

                \filldraw [draw=gray,fill=rex] (0,0) circle (2.5mm);
                \filldraw [draw=gray,fill=blx] (1,0) circle (2.5mm);
                \filldraw [draw=gray,fill=blx] (0,1) circle (2.5mm);
                \filldraw [draw=gray,fill=rex] (1,1) circle (2.5mm);
            \end{tikzpicture}
            \caption{Energy $E_1$ system.}
            \label{fig:nearNeighbora}
        \end{subfigure}
        \begin{subfigure}[b]{0.25\linewidth}
            \centering
            \begin{tikzpicture}
                \draw [xshift=-0.5cm,yshift=-0.5cm] (0,0) grid (2,2);
                \draw [gray,dashed]
                    (0,0) -- (1,0)
                    (0,1) -- (1,1)
                ;

                \filldraw [draw=gray,fill=blx] (0,0) circle (2.5mm);
                \filldraw [draw=gray,fill=rex] (1,0) circle (2.5mm);
                \filldraw [draw=gray,fill=blx] (0,1) circle (2.5mm);
                \filldraw [draw=gray,fill=rex] (1,1) circle (2.5mm);
            \end{tikzpicture}
            \caption{Energy $E_2$ system.}
            \label{fig:nearNeighborb}
        \end{subfigure}
        \caption{Nearest neighbor interactions.}
        \label{fig:nearNeighbor}
    \end{figure}
    \begin{itemize}
        \item Example: Assume nearest neighbor interactions matter.
        \item Let opposing objects in neighboring cells contribute energy $\varepsilon_1$ to the total energy of the system. This means that in Figure \ref{fig:nearNeighbor}, $E_1=4\varepsilon_1$ and $E_2=2\varepsilon_2$.
        \item In big systems, the energy won't fluctuate much, though, so Boltzmann equation is more of an approximation, but it's \emph{good enough}.
    \end{itemize}
    \item Mean field mixing enthalpy.
    \begin{itemize}
        \item Assume the lattice is such that each point has $z$ nearest neighbor cells.
        \begin{itemize}
            \item For example, each cell in a square lattice (Figure \ref{fig:latticeConfig}) has 4 nearest neighbors: One above, below, right, and left.
        \end{itemize}
        \item To calculate enthalpic interactions, we consider the number of pairwise interactions.
        \item But in a mean field approximation, we wash out some detail by mixing red and blue to make purple. We say that \emph{on average}, your neighbor one away from you is proportional to the composition (because it might be red, then blue, then red again). Quick exchange of neighbors.
        \begin{itemize}
            \item You can build on this with weights, but this is the purest sense of a mean field approximation.
        \end{itemize}
        \item The mean field approximation breaks down when mixing breaks down, i.e., when you start to get some clusters of pure one thing and pure another thing.
    \end{itemize}
    \item Enthalpy of mixing.
    \begin{figure}[h!]
        \centering
        \includegraphics[width=0.2\linewidth]{Uwell.JPG}
        \caption{Potential well.}
        \label{fig:Uwell}
    \end{figure}
    \begin{itemize}
        \item We need to count each interaction. It follows from the above that the number of $ij$ interactions is
        \begin{align*}
            \xi_{11} &= \frac{X_1z\phi_1}{2}&
            \xi_{22} &= \frac{X_2z\phi_2}{2}&
            \xi_{12} &= X_1z\phi_2
        \end{align*}
        \begin{itemize}
            \item Let's rationalize the formula for $\xi_{11}$. $X_1$ sites each have $z$ nearest neighbors, so there are $X_1z$ nearest neighbor interactions where one of the species involved is species 1. The probability that a nearest neighbor is also species 1 is $\phi_1$. Thus, there are $X_1z\phi_1$ interactions where the other partner is also species 1. However, since there are two neighbors involved, we have currently accounted for each interaction twice: Once from the perspective of each neighbor. Thus, we need to divide by 2.
            \item Same rationalization for $\xi_{22}$.
            \item For $\xi_{12}$, a similar rationalization applies but we are not overcounting because we only have the perspective of one of the two interaction partners taken into account, so no dividing by 2 is necessary. Alternatively, from the perspective of both species, we have
            \begin{equation*}
                \xi_{12} = \frac{X_1z\phi_2}{2}+\frac{X_2z\phi_1}{2} = X_1z\phi_2
            \end{equation*}
        \end{itemize}
        \item In a typical attraction well, we have a most probable distance $a$, at which the energy depth is $\varepsilon$ (see Figure \ref{fig:Uwell}).
        \item Let $\varepsilon_{ij}$ refer to how deep the well is between species $i$ and $j$, where $i,j\in\{1,2\}$.
        \item It follows that in a mixed enthalpic state, the interaction energy is the following.
        \begin{align*}
            H_{1,2} &= \xi_{12}\varepsilon_{12}+\xi_{11}\varepsilon_{11}+\xi_{22}\varepsilon_{22}\\
            &= zX_1\phi_2\varepsilon_{12}+\frac{zX_1\phi_1\varepsilon_{11}}{2}+\frac{zX_2\phi_2\varepsilon_{22}}{2}
        \end{align*}
        \begin{itemize}
            \item In pure enthalpic states, the interaction energies are the following.
            \begin{align*}
                H_1 &= \frac{zX_1\varepsilon_{11}}{2}&
                H_2 &= \frac{zX_2\varepsilon_{22}}{2}
            \end{align*}
        \end{itemize}
        \item Assuming that volume is constant, energy and enthalpy are the same (we're actually calculating energy but operating under this assumption).
        \item It follows that
        \begin{align*}
            \Delta H_M &= H_{1,2}-(H_1+H_2)\\
            &= z\left[ X_1\phi_2\varepsilon_{12}+\frac{X_1\varepsilon_{11}}{2}(\phi_1-1)+\frac{X_2\varepsilon_{22}}{2}(\phi_2-1) \right]
        \end{align*}
        and hence
        \begin{align*}
            \frac{\Delta H_M}{X_0} &= z\left[ \phi_1\phi_2\varepsilon_{12}+\frac{\phi_1\varepsilon_{11}}{2}(-\phi_2)+\frac{\phi_2\varepsilon_{22}}{2}(-\phi_1) \right]\\
            \frac{\Delta H_M}{X_0\kB T} &= \frac{z}{\kB T}\left\{ \phi_1\phi_2\left[ \varepsilon_{12}-\frac{1}{2}(\varepsilon_{11}+\varepsilon_{22}) \right] \right\}\\
            &= \chi\phi_1\phi_2
        \end{align*}
        where
        \begin{equation*}
            \chi := \frac{z}{\kB T}\left[ \varepsilon_{12}-\frac{1}{2}(\varepsilon_{11}+\varepsilon_{22}) \right]
        \end{equation*}
        is the \textbf{Flory $\bm{\chi}$ parameter}.
        \begin{itemize}
            \item From its definition, we can see that the sign of $\chi$ determines whether mixing will be enthalpically favorable or not. Moreover, the sign of $\chi$ is determined by the interplay between how much the components like each other, and how much they like themselves on average.
        \end{itemize}
        \item Implication: If the two components like themselves more than they like each other (right diagram at bottom of slide 14), $\chi$ will be positive and will between the middle two lines (we don't know here if it will mix or demix)??
    \end{itemize}
    \item The $\chi$ parameter is still being debated today; Alfredo is writing a paper on it!
    \begin{itemize}
        \item The parameter as written is linear, but that's based on a mean field assumption. Should it have a quadratic term? Maybe it does at a (more accurate) higher level of theory.
    \end{itemize}
    \item At this point, we can assemble everything into the free energy of mixing for monomers.
    \begin{align*}
        \frac{\Delta G_M}{X_0} &= \frac{\Delta H_M}{X_0}-T\frac{\Delta S_M}{X_0}\\
        &= \kB T\chi\phi_1\phi_2-T\cdot\kB(-\phi_1\ln\phi_1-\phi_2\ln\phi_2)\\
        &= \kB T(\chi\phi_1\phi_2+\phi_1\ln\phi_1+\phi_2\ln\phi_2)
    \end{align*}
    \item How about for polymers?
    \begin{figure}[h!]
        \centering
        \includegraphics[width=0.25\linewidth]{latticePolym.JPG}
        \caption{Fitting polymers into a square lattice.}
        \label{fig:latticePolym}
    \end{figure}
    \begin{itemize}
        \item For enthalpy of mixing, we still use a mean field approximation, so nothing changes. We justify this by noting, as earlier, that in a melt, most chains do not feel themselves because other chains screen the interaction of the original chain with itself.
        \item For entropy of mixing, things do change a bit.
        \begin{itemize}
            \item As in Figure \ref{fig:latticeConfig}, we have to pick $N$ adjacent sites.
            \item Then the next time we pick, what's left is the number of sites minus $N$. And on and on.
            \item This essentially reduces the number of sites by $N_1$ and $N_2$.
            \item All of this is part of a complex derivation done by Flory, but the simple and intuitive result is that the entropy of mixing decreases by approximately $1/N$ due to the connectivity of the $N$ segments we cannot arrange any further apart. Mathematically, we obtain
            \begin{equation*}
                \frac{\Delta S_M}{\kB X_0} = -\frac{\phi_1}{N_1}\ln\phi_1-\frac{\phi_2}{N_2}\ln\phi_2
            \end{equation*}
            \item This equation tells us that when $N$ is big, entropy doesn't play a huge role in driving mixing (because $N_1,N_2$ are in the denominator). Thus, enthalpy matters more for the mixing of polymers.
        \end{itemize}
    \end{itemize}
    \item Using the modified mixing entropy, we can now finally state the "famous Flory-Huggins theory for the free energy of mixing."
    \begin{equation*}
        \frac{\Delta G_M}{X_0} = \kB T\left( \chi\phi_1\phi_2+\frac{\phi_1}{N_1}\ln\phi_1+\frac{\phi_2}{N_2}\ln\phi_2 \right)
    \end{equation*}
    \begin{itemize}
        \item This equation is applicable to solvent-solvent ($1=N_1=N_2$), polymer-solvent ($1=N_1\neq N_2$), and polymer-polymer ($1\neq N_1,N_2$) mixing.
        \item This equation also gives us a tool to investigate phase behavior and how it varies with $T,\chi,\phi_i,N_i$. Namely, values of these variables that lead to a negative $\Delta G_M$ will correspond to mixing, and values of these variables that lead to a positive $\Delta G_M$ will correspond to demixing.
    \end{itemize}
    \item How do we measure $\chi$ experimentally?
    \begin{itemize}
        \item We will explore this soon.
    \end{itemize}
    \item Influence of $\chi$ on phase behavior.
    \begin{figure}[h!]
        \centering
        \begin{subfigure}[b]{0.33\linewidth}
            \centering
            \includegraphics[width=0.95\linewidth]{mixingChia.png}
            \caption{Mixing enthalpy.}
            \label{fig:mixingChia}
        \end{subfigure}
        \begin{subfigure}[b]{0.32\linewidth}
            \centering
            \includegraphics[width=0.95\linewidth]{mixingChib.png}
            \caption{Mixing entropy.}
            \label{fig:mixingChib}
        \end{subfigure}
        \begin{subfigure}[b]{0.33\linewidth}
            \centering
            \includegraphics[width=0.95\linewidth]{mixingChic.png}
            \caption{Mixing energy.}
            \label{fig:mixingChic}
        \end{subfigure}
        \caption{Mixing or demixing based on the Flory $\chi$ parameter.}
        \label{fig:mixingChi}
    \end{figure}
    \begin{itemize}
        \item Notice how adding the curves in Figures \ref{fig:mixingChia}-\ref{fig:mixingChib} gives the curves in Figure \ref{fig:mixingChic}.
        \item From Figure \ref{fig:mixingChic}, we can see that when $\chi$ reaches the range of 2-3, we get demixing.
        \begin{itemize}
            \item This is because (at $\chi=3$, for example) it will be more energetically favorable to form (i) a phase that is approximately 90\% component 1 and 10\% component 2 and (ii) a phase that is approximately 10\% component 1 and 90\% component 2 than it will be to completely mix.
        \end{itemize}
        \item When $\chi=2$, free energy is largely flat. Thus, very different compositions have similar free energy, which means that the system will "undergo a 2nd order transition where all length scales are viable??"
    \end{itemize}
    \item A parameter that has gained more importance in recent years: The coordination number $z$.
    \begin{itemize}
        \item It turns out that the more neighbors you have, the better mean-field behavior you get.
        \item This is essentially because you're averaging over more values.
        \item Recent finding: $\chi$ is pretty good if $z$ is big; if $z$ is small, quadratic and other higher order corrections may be necessary for $\chi$.
    \end{itemize}
    \item Phase behavior of blends.
    \begin{itemize}
        \item The critical value $\chi_c$ of the Flory $\chi$ parameter is the value of $\chi$ at which you see phase separation.
        \item $\chi_c$ decreases exponentially with increasing chain length $N_1=N_2=N$.
        \begin{itemize}
            \item This relates to the phenomenon discussed earlier in which the mixing entropy shrinks as chain length grows. If mixing entropy is shrinking, the system can tolerate less enthalpic repulsion before demixing.
        \end{itemize}
    \end{itemize}
    \item Preliminaries to next class.
    \begin{itemize}
        \item Construction of phase diagrams.
        \begin{itemize}
            \item You have a critical point graph that gets flipped??
            \item Inside one is binodal line; outside one is spinodal line.
        \end{itemize}
        \item Between the two inflection points, the system is unstable.
        \item Concave curvature puts sum of two free energies below the points.
        \item You will evolve toward the two lowest energy points, phase separating as needed.
        \item Demixing occurs by nucleation and growth.
    \end{itemize}
\end{itemize}



\section{The Theta State}
\begin{itemize}
    \item \marginnote{9/23:}Announcements.
    \begin{itemize}
        \item PSet 2 posted; due midnight next Tuesday.
    \end{itemize}
    \item Last time: Flory-Huggins polymers.
    \item Today, let's begin by thinking about the equilibrium between the two different phases of a system.
    \begin{itemize}
        \item We will quantify this with binodal and spinodal stuff.
    \end{itemize}
    \item \textbf{Chemical potential}: The Gibbs free energy of a substance at a given concentration. \emph{Denoted by} $\bm{\mu_i}$. \emph{Given by}
    \begin{equation*}
        \mu_i := \left( \pdv{G}{n_i} \right)_{T,P,N,n_j}\tag{$j\neq i$}
    \end{equation*}
    \item \textbf{Coexistence curve}: The set of points where the chemical potentials are equal. \emph{Also known as} \textbf{binodal curve}.
    \begin{itemize}
        \item The coexistence curve encloses all compositions wherein the mixture demixes into two distinct, coexisting phases.
        \item Symbolically, letting the two phases in the mixture be called prime and double prime, we have for each component 1 and 2 that
        \begin{align*}
            \mu_1' &= \mu_1''&
            \mu_2' &= \mu_2''
        \end{align*}
        \item In PSet 2, we'll derive one of the expressions on this slide.
    \end{itemize}
    \item Spinodal inflection points are where the 2nd derivative is zero.
    \begin{itemize}
        \item Critical points has 2nd and 3rd derivatives equal to zero.
        \item \emph{equations in slides}
    \end{itemize}
    \item Polymer-solvent solutions.
    \begin{figure}[h!]
        \centering
        \includegraphics[width=0.25\linewidth]{demix.JPG}
        \caption{Demixing of a polymer-solvent solution.}
        \label{fig:demix}
    \end{figure}
    \begin{itemize}
        \item An interesting way of thinking about this: Osmometry.
        \item Imagine a uniform, mixed polymer solution.
        \begin{itemize}
            \item We can demix to a state where the polymer is all clumped together.
            \item The clumps still have some solvent in them though.
        \end{itemize}
        \item The phase separation stops when a solvent molecule \emph{inside} and \emph{outside} the polymer clump has the same chemical potential.
        \item Important relevant expressions.
        \begin{align*}
            \mu_1-\mu_1^\circ &= RT\left[ \ln\phi_1+\left( 1-\frac{1}{N_2} \right)\phi_2+\chi\phi_2^2 \right]\\
            \mu_2-\mu_2^\circ &= RT\left[ \ln\phi_2+(N_2-1)\phi_2+N_2\chi\phi_2^2 \right]
        \end{align*}
        \begin{itemize}
            \item Remember that by convention, phase 1 is the solvent and phase 2 is the polymer.
            \item Notice the multiplying vs. dividing of $N_2$.
            \item For a polydisperse system of polymer chains, let $N_2=\prb{N_2}=\Mn$.
        \end{itemize}
        \item We'll now massage the above expressions to get some more mechanistic understanding out of them.
    \end{itemize}
    \item Chemical potential for a dilute solution.
    \begin{itemize}
        \item Since we are positing a dilute solution, we may use the approximation that the volume fraction of component 2 is small. This will allow us to expand the expressions.
        \item For small $x$, the following approximation holds.
        \begin{equation*}
            \ln(1-x) = -x-\frac{x^2}{2}-\cdots-\frac{x^n}{n}-\cdots
        \end{equation*}
        \begin{itemize}
            \item We can use fewer terms for quite small $x$.
        \end{itemize}
        \item It also follows from the assumptions that (1) the solution is dilute, i.e., $n_1\gg n_2N_2$ and (2) that the volume of the monomers is approximately equal, i.e., $v_1\approx v_2$ that
        \begin{align*}
            \phi_1 &= \frac{n_1v_1}{n_1v_1+n_2N_2v_2}&
            \phi_2 &= \frac{n_2N_2v_2}{n_1v_1+n_2N_2v_2} \approx \frac{n_2N_2}{n_1}
        \end{align*}
        \item We thus expand
        \begin{align*}
            \ln(1-\phi_2) &\approx -\phi_2-\frac{1}{2}\phi_2^2
        \end{align*}
        \item It follows that
        \begin{align*}
            \frac{\mu_1-\mu_1^\circ}{RT} &= \ln(1-\phi_2)+1\cdot\phi_2-\frac{1}{N_2}\cdot\phi_2+\chi\phi_2^2\\
            &= -\phi_2-\frac{1}{2}\phi_2^2+\phi_2-\frac{\phi_2}{N_2}+\chi\phi_2^2\\
            &= -\frac{\phi_2}{N_2}+\left( \chi-\frac{1}{2} \right)\phi_2^2
        \end{align*}
        \begin{itemize}
            \item Since it's negative, this tells us that the chemical potential $\mu_1$ is always less than $\mu_1^\circ$, which means that the term is always negative, so everything wants to mix.
            \item In a subsequent course, we describe more results based off of the above equation!
        \end{itemize}
    \end{itemize}
    \item Phase diagram for a dilute polymer solution.
    \begin{itemize}
        \item The condition where ?? is the $\theta$ condition.
        \item Positive $\chi$ means that polymers don't like the solvent as much as they like themselves.
        \item As $N_2$ increases, we push to lower fractions.
    \end{itemize}
    \item It is very difficult to mix high MW polymers because you will need $\chi<2/N$.
    \item Solubility parameter and $\chi$.
    \begin{figure}[h!]
        \centering
        \includegraphics[width=0.3\linewidth]{Hildebrand.JPG}
        \caption{Hildebrand's experiment.}
        \label{fig:Hildebrand}
    \end{figure}
    \begin{itemize}
        \item How do we estimate $\chi$?
        \begin{itemize}
            \item Hildebrand's interesting idea was to use the enthalpy of vaporization $\Delta H_v$.
        \end{itemize}
        \item Experimental setup.
        \begin{itemize}
            \item Take your liquid, heat it up, measure how much heat goes into the system, turn it into a gas, and see how the heat has turned into kinetic energy.
            \item By the time you have heated an object in a lattice, your are neighborless; you have no potential energy.
            \item We now have
            \begin{align*}
                \delta &= \left( \frac{\Delta E}{V} \right)^{1/2}
            \end{align*}
            \item We'll now calculate $\delta_1$ and $\delta_2$ for species 1 and 2, the \textbf{solubility parameters} of the components.
            \begin{align*}
                \delta_1 &= \sqrt{\frac{Z}{2}\frac{\varepsilon_{11}}{v}}&
                \delta_2 &= \sqrt{\frac{Z}{2}\frac{\varepsilon_{22}}{v}}
            \end{align*}
            \item We then use \textbf{Berthelot's mixing rule} (which uses the geometric mean) to get $\varepsilon_{12}$:
            \begin{equation*}
                \varepsilon_{12} = \sqrt{\varepsilon_{11}\varepsilon_{22}}
            \end{equation*}
            \item Now imagine two points 1 and 2 separated by a distance $r$, as in Figure \ref{fig:monomerGasc}.
            \begin{equation*}
                U_\text{attractive} = -\frac{\alpha_1\alpha_2}{r^6}
            \end{equation*}
            \begin{itemize}
                \item Scales as $1/r^6$, and also has the \textbf{polarizability} / \textbf{polarizability volumes}.
                \item This is related to dipole-induced dipole attractions; when you average over all possible combinations, this relation falls out. And that's what Lennard and Jones based their use of $1/r^6$ as the attractive term on!
                \item In PSet 2, we will prove that this attraction rule is "like likes like."
            \end{itemize}
            \item Then
            \begin{equation*}
                (\delta_1-\delta_2)^2 = \frac{1}{v}\left( \frac{z\varepsilon_{11}}{2}+\frac{z\varepsilon_{22}}{2}-z\varepsilon_{12} \right)
            \end{equation*}
            \item Now just multiply by $v/\kB T$ to get the $\chi$ parameter.
            \begin{itemize}
                \item $v$ is a volume.
            \end{itemize}
            \item This gets us to the \textbf{Hildebrand equation}
            \begin{equation*}
                \Delta H_M = V_m\phi_1\phi_2(\delta_1-\delta_2)^2 \geq 0
            \end{equation*}
            \begin{itemize}
                \item This works better for nonpolar than polar species.
                \item $V_m$ is the average molar volume of solvent / monomers.
            \end{itemize}
            \item See \textcite{bib:RubinsteinColby} for an in-depth discussion of this setup.
        \end{itemize}
    \end{itemize}
    \item Let's now compare the curve with experiments.
    \begin{itemize}
        \item PS in cyclohexane.
        \item Dashed lines are the Flory-Huggins theory, which clearly doesn't look like the experiment at all.
        \begin{itemize}
            \item This is because of Flory's mean field assumption, which doesn't hold here. Indeed, as you heat up, you will be more likely to have the same neighbor.
            \item Actual curve is wrong, but scaling is correct (this happens in several of Flory's theories).
        \end{itemize}
        \item Reference: \textcite{bib:ShultzFlory}.
    \end{itemize}
    \item Phase diagrams of polymer-polymer blends.
    \begin{itemize}
        \item $\phi$ for the fraction that a chain occupies is
        \begin{equation*}
            \phi = \frac{Nv}{(N^{1/2}\ell)^3} \approx \frac{v}{\ell^3}N^{-1/2}
        \end{equation*}
    \end{itemize}
    \item Two principal types of phase diagrams.
    \begin{itemize}
        \item Demixing at higher temperatures, vs. mixing at higher temperatures.
        \item Poly(methyl methacrylate) / styrene-\emph{co}-acrylonitrile demixes at increased temperature (because molecules are polar).
        \item Polystyrene / polyisoprene mixes at higher temperatures.
        \item PEG and PMMA have a negative $\chi$ at room temperature. PEG and \ce{H2O} is similar (you heat it up, and the polymer comes out of solution).
        \item pNIPAM undergoes a transition around \SIrange{32}{34}{\celsius}.
        \item Attraction gives rise to a low or negative $\chi$.
        \item A number of references on polymer blends are included in the slides!
    \end{itemize}
    \item You can arrest a spinodal decomposition by heating and then cooling very quickly.
    \item Applications of FH theory.
    \begin{itemize}
        \item Biocondensates and membrane-free organelles (like the nucleolus and centrioles). Identified another bunch of these after they expanded their defintion of organelles! These things come together because of FH theory.
    \end{itemize}
    \item Next time.
    \begin{itemize}
        \item Self-assembly.
        \item The PSet 2 might be a bit long, so start early! We should currently be able to do every problem up to 3, and after Thursday, we should be able to do every problem.
    \end{itemize}
\end{itemize}



\section{Phase Behavior, Melting Point Depression, Osmometry, and Microphase Separation}
\begin{itemize}
    \item \marginnote{9/25:}Last time.
    \begin{itemize}
        \item Entities that are not covalently bonded.
    \end{itemize}
    \item Today.
    \begin{itemize}
        \item Entities that \emph{are} connected together.
        \begin{itemize}
            \item You cannot get rigid phase separation here.
            \item Self-assembly is a thing.
        \end{itemize}
    \end{itemize}
    \item Lecture outline.
    \begin{itemize}
        \item Copolymers.
        \begin{itemize}
            \item Microphase separation.
            \item Interfacial free energy.
            \item Chain stretching and configuraitonal free energy.
            \begin{itemize}
                \item This will bring back concepts from Professor Doyle's class.
            \end{itemize}
        \end{itemize}
    \end{itemize}
    \item A bit more on biocondensates (not testable material).
    \begin{figure}[h!]
        \centering
        \includegraphics[width=0.2\linewidth]{bioNetwork.JPG}
        \caption{Simplified graph of a biochemical network.}
        \label{fig:bioNetwork}
    \end{figure}
    \begin{itemize}
        \item Correction: The protein discussed last time goes into solution if you \emph{add} salt.
        \item Principle in biology: At some point, it gets better to have things that do multiple tasks poorly than one task really well.
        \begin{itemize}
            \item This is because it takes energy to produce proteins.
            \item Example: In computer science, engineers used to spend a lot of time to make 1 really nice transistor. But now, they go for a lot of transistors that are almost all the same and you connect them in different ways. Now you can do basically any task, but not all of them are great. In D. E. Shaw, they have a computer that \emph{only} runs molecular dynamics (1000 times faster than Nvidia GPUs), but that's the only thing it does.
            \item So since we need a lot of functions in a cell and we don't want to produce a lot of very specialized proteins, it's better to be a bit more general.
        \end{itemize}
        \item Suppose you have a (biochemical) network, and we control each transition between nodes locally (Figure \ref{fig:bioNetwork}).
        \begin{itemize}
            \item If we want to actually do complex computation with the system, having junctions that act on a number of different nodes is helpful.
        \end{itemize}
        \item Takeaway: Random things and disorder in a cell gives you capabilities beyond perfectly folded structures, like proteins and enzymes.
    \end{itemize}
    \item This concludes content from last time; we now move onto today's content.
    \item Self-assembly of simplified systems (relative to cells).
    \begin{figure}[H]
        \centering
        \includegraphics[width=0.4\linewidth]{selfAssembly.png}
        \caption{Self-assembly of an ABC triblock copolymer.}
        \label{fig:selfAssembly}
    \end{figure}
    \begin{itemize}
        \item What aspects of an ABC triblock copolymer affect its organization?
        \item There will be \textbf{intra interactions} $\chi_{AB}$, $\chi_{AC}$, and $\chi_{BC}$ between the various components of the chain.
        \item Let the chain have total length $N$, and let each of the three segments have length $N_i$.
        \begin{itemize}
            \item Then we have $N_A$, $N_B$, and $N_C=N-N_A-N_B$.
        \end{itemize}
        \item If we have something like (a) of Figure \ref{fig:selfAssembly}, then we probably have $N_A=N_B=N_C$ (becasue everything is nice and equally ordered) and $\chi_{AC}<\chi_{AB}\approx\chi_{BC}$ (because we have AB and BC interfaces, but not AC interfaces).
        \item Note: These images are not made up; all of them have been seen.
    \end{itemize}
    \item If we can do everything in Figure \ref{fig:selfAssembly} with 3 things, imagine how much we can do with the 20 amino acids!
    \begin{itemize}
        \item Note on the "hydrophobic" amino acids: They have branching (see valine, leucine, isoleucine)! Nature doesn't just use \emph{n}-alkyl chains of different length because the methyl groups sticking off have partial charges of 0.4 (40\% the charge of an electron), which makes them still pretty polar.
        \item Tyrosine can use its phenolic substiuent to \emph{enhance} its $\pi$-cation non-covalent interactions relative to phenylalanine.
    \end{itemize}
    \item Key question: How can a homogeneous state go to a semiordered state, to an even more ordered state?
    \begin{itemize}
        \item Example: Unfolded protein, to good prions, to rogue prions.
        \item Aside: In rogue prions, there is an exposed $\beta$-pleated sheet, which will stack vertically with the $\beta$-pleated sheets of other prions. This stacking is what causes the brain to shut down in Mad Cow Disease.
    \end{itemize}
    \item Goals for self-assembly.
    \begin{itemize}
        \item Understand the key concepts behind the process of self-assembly, in particular for the case of block copolymers.
        \item Construct a simple formalism to determine which variables contribute more relative to other ones.
        \item Recent stuff on how to control self-assembly using external methods.
    \end{itemize}
    \item \textbf{Min-max principle}: Phases are most stable when we (1) minimize interfacial energy and (2) maximize the conformational entropy of the chains.
    \item The min-max principle governs self-assembly.
    \begin{figure}[H]
        \centering
        \includegraphics[width=0.4\linewidth]{ordering.png}
        \caption{Ordering of a diblock copolymer.}
        \label{fig:ordering}
    \end{figure}
    \begin{itemize}
        \item As we transition from a homogeneous, disordered state to an ordered state, we develop an interface.
        \begin{itemize}
            \item This interface is technically termed the \textbf{IMDS}, or inter-material dividing surface.
        \end{itemize}
        \item In the disordered phase, entropy is maximized\dots but we're paying an enthalpic price because of the contact between groups that don't like each other.
        \item The subscript $c$ in Figure \ref{fig:ordering} means "critical."
        \begin{itemize}
            \item Remember that $\chi N$ controls whether or not we develop microdomains (more on this below). We will investigate this in PSet 2, too.
        \end{itemize}
    \end{itemize}
    \item Principles of self-assembly: Microphase separation in diblock copolymers.
    \begin{figure}[h!]
        \centering
        \begin{subfigure}[b]{0.2\linewidth}
            \centering
            \includegraphics[width=0.7\linewidth]{microphaseSepa.png}
            \caption{$T>T_\text{ODT}$.}
            \label{fig:microphaseSepa}
        \end{subfigure}
        \begin{subfigure}[b]{0.2\linewidth}
            \centering
            \includegraphics[width=0.7\linewidth]{microphaseSepb.png}
            \caption{$T\approx T_\text{ODT}$.}
            \label{fig:microphaseSepb}
        \end{subfigure}
        \begin{subfigure}[b]{0.2\linewidth}
            \centering
            \includegraphics[width=0.7\linewidth]{microphaseSepc.png}
            \caption{$T<T_\text{ODT}$.}
            \label{fig:microphaseSepc}
        \end{subfigure}
        \caption{Microphase separation in diblock copolymers.}
        \label{fig:microphaseSep}
    \end{figure}
    \begin{itemize}
        \item Some domains start to form and you get lamellae in time.
        \item Misconception: Things are not perfectly mixed at one extreme; you start seeing domains earlier. As you go from Figure \ref{fig:microphaseSepa}-\ref{fig:microphaseSepc}, you get into a lamellar state.
    \end{itemize}
    \item Microdomain morphologies: Diblock copolymers.
    \begin{figure}[H]
        \centering
        \includegraphics[width=0.65\linewidth]{microphaseMorph.png}
        \caption{Microphase morphologies in diblock copolymers.}
        \label{fig:microphaseMorph}
    \end{figure}
    \begin{itemize}
        \item In contrast to Figure \ref{fig:microphaseMorph}, 40-60\% gets you a lamellar state. This accounts for the fact that going on either side of 50\% is equivalent.
        \item On the outskirts of this, you get \textbf{bicontinuous phases}.
        \begin{itemize}
            \item The Double Diamond is heavily sought after in optics.
            \item Double Gyroid is more common.
            \item Difference is tri- vs. tetracoordination at the nodes.
        \end{itemize}
        \item Then cylinders.
        \item And an even smaller amount of green gets you spheres.
    \end{itemize}
    \item \textbf{Bicontinuous} (phases): Two demixed phases such that for any two points in a single phase, there exists a path between them that never crosses a phase boundary.
    \item Where are the above morphologies used?
    \begin{itemize}
        \item Example: Krayton's / green rubbers.
        \begin{itemize}
            \item This is a PS-\emph{block}-PB-\emph{block}-PS polymer, with a big PB domain.
        \end{itemize}
        \item The PS ends either land in another domain, or come back to the same domain.
        \item Good for high-performance applications, like the rubber in an F1 track.
    \end{itemize}
    \item We now investigate microdomain spacing for diblock copolymers.
    \item Variables to be aware of.
    \begin{itemize}
        \item $G$ is the free energy per chain;
        \item $N=N_A+N_B$ is the number of segments per chain.
        \item $a$ is the step size.
        \item $\lambda$ is the domain periodicity. \emph{look up definitions!!}
        \item $\Sigma$ is the interfacial area where the chains actually interact.
        \item $\gamma_\text{AB}$ is the interfacial energy per unit area. It will be computed using Helfand's equation.
        \begin{equation*}
            \gamma_\text{AB} = \frac{\kB T}{a^2}\sqrt{\frac{\chi_\text{AB}}{6}}
        \end{equation*}
        \begin{itemize}
            \item $a$ is the Kuhn length or monomer length, varying depending on the context.
        \end{itemize}
        \item $\chi_\text{AB}$ is the same Flory-Huggins interaction parameter we've been looking at in previous lectures.
    \end{itemize}
    \item Free energies of these diblock copolymers.
    \begin{figure}[H]
        \centering
        \includegraphics[width=0.35\linewidth]{diblockStretch.JPG}
        \caption{Entropy and enthalpy changes as diblock copolymers are stretched.}
        \label{fig:diblockStretch}
    \end{figure}
    \begin{itemize}
        \item We have
        \begin{align*}
            \Delta G &= (\underbrace{H_2-H_1}_{\Delta H})-T(\underbrace{S_2-S_1}_{\Delta S})\\
            &= \gamma_\text{AB}\underbrace{\sum-N\chi_\text{AB}\phi_\text{A}\phi_\text{B}\kB T}_\text{Enthalpic Term}+\underbrace{\frac{3}{2}\kB T\left[ \frac{(\lambda/2)^2}{Na^2}-1 \right]}_\text{Entropic Term}
        \end{align*}
        \begin{itemize}
            \item The entropic term relates to the springiness of the polymer.
            \item Note that $Na^3=\Sigma\cdot\lambda/2$. This essentially equates (1) the total volume occupied by $N$ monomers each of volume $a^3$ and (2) the volume of the cylinder bounding said monomers, a cylinder having height $\lambda/2$ and base area $\Sigma$.
        \end{itemize}
        \item Important assumption: Chains want to stretch away from the interface.
        \begin{itemize}
            \item Enthalpy goes way down when you have linear strands (everything is near each other). Entropy goes way up here, though, because we're stretching.
            \item In the other regime, though, area is way bigger. This means we get more enthalpy.
            \item $\Delta S\approx\lambda^2/a^2N$.
            \item System will try to find the optimal balance between the two; we want to optimize the length $\lambda$.
        \end{itemize}
        \item To find the optimal $\lambda$, we'll want to find the minimum Gibbs free energy as a function of $\lambda$.
        \begin{equation*}
            \Delta G(\lambda) = \underbrace{\frac{\kB T}{a^2}\sqrt{\frac{\chi_\text{AB}}{6}}}_{\gamma_\text{AB}}\underbrace{\frac{Na^3}{\lambda/2}}_\Sigma-N\chi_\text{AB}\phi_\text{A}\phi_\text{B}\kB T+\frac{3}{2}\kB T\left[ \frac{(\lambda/2)^2}{Na^2}-1 \right]
        \end{equation*}
        \begin{itemize}
            \item By above definitions and equalities, the first term is the interfacial energy per unit area times the area of the interface.
        \end{itemize}
        \item Compressing every non-$\lambda$ variable in the above expression into a constant (termed $\alpha$, $\beta$, $\text{const}_1$, or $\text{const}_2$) reveals that the above equation is of the following general form.
        \begin{equation*}
            \Delta G(\lambda) = \frac{\alpha}{\lambda}-\text{const}_1+\beta\lambda^2-\text{const}_2
        \end{equation*}
        \item Thus, the optimum period of the lamellae repeat unit is
        \begin{align*}
            0 &= \pdv{\Delta G}{\lambda}\\
            &= -\frac{\alpha}{\lambda_\text{opt}^2}+2\beta\lambda_\text{opt}\\
            \lambda_\text{opt} &= \sqrt[3]{\frac{\alpha}{2\beta}} = aN^{2/3}\chi_\text{AB}^{1/6}
        \end{align*}
        \begin{itemize}
            \item The result that $\lambda_\text{opt}$ scales as $N^{2/3}$ is important! It implies that chains in microdomains are stretched compared to the homogeneous melt state (in which scaling is the smaller $N^{1/2}$).
        \end{itemize}
    \end{itemize}
    \item Let's now investigate the order-disorder transition temperature.
    \begin{itemize}
        \item By substituting $\lambda_\text{opt}$ into our expression for $\Delta G(\lambda)$, we obtain the estimate that
        \begin{align*}
            \Delta G(\lambda_\text{opt}) &= \frac{2}{\sqrt{6}}\kB TN\chi_\text{AB}^{1/2}a\lambda_\text{opt}^{-1}-N\chi_\text{AB}\phi_\text{A}\phi_\text{B}\kB T+\frac{3}{8}\kB T\frac{\lambda_\text{opt}^2}{Na^2}-\frac{3}{2}\kB T\\
            &= \left( \frac{2}{\sqrt{6}}+\frac{3}{8} \right)\kB TN^{1/3}\chi_\text{AB}^{1/3}-N\chi_\text{AB}\phi_\text{A}\phi_\text{B}\kB T-\frac{3}{2}\kB T\\
            &\approx 1.2\kB TN^{1/3}\chi_\text{AB}^{1/3}-N\chi_\text{AB}\phi_\text{A}\phi_\text{B}\kB T-\frac{3}{2}\kB T\\
            &\approx 1.2\kB TN^{1/3}\chi_\text{AB}^{1/3}-N\chi_\text{AB}\phi_\text{A}\phi_\text{B}\kB T
        \end{align*}
        \item Since the first two terms are both much greater than the thrid term, we neglect it.
        \begin{itemize}
            \item Thus, the sign of $\Delta G$ will depend on which of the two remaining terms is bigger.
        \end{itemize}
        \item Let's analyze the case of a 50/50 volume fraction of components A and B. Specifically, we want to know what the critical $N\chi$ value is above which $\Delta G=-$ and we form lamellar microdimains, and below which $\Delta G=+$ and we stay in a homogenous melt.
        \begin{itemize}
            \item In a 50/50 split, $\phi_\text{A}=\phi_\text{B}=1/2$. Thus,
            \begin{equation*}
                \phi_\text{A}\phi_\text{B} = \frac{1}{4}
            \end{equation*}
            \item It follows that the critical $N\chi$ value $(N\chi)_c$ is
            \begin{align*}
                \frac{(N\chi)_c}{4} &= 1.2(N\chi)_c^{1/3}\\
                (N\chi)_c^{2/3} &= 4.8\\
                (N\chi)_c &\approx 10.5
            \end{align*}
            \item Therefore, if $N\chi<10.5$, we'll get a homogeneus, mixed melt; and if $N\chi>10.5$, we'll get demixing into lamellar microdomains.
        \end{itemize}
    \end{itemize}
    \item Other interfaces: Polymer brushes.
    \begin{itemize}
        \item Consider a series of polymer strands grown off of a 2D surface.
        \begin{itemize}
            \item Let each polymer strand be a distance $D$ away from the next nearest strand. In this sense, each polymer strand can be thought to inhabit a volume of diamater $D$ and height $H$ away from the surface.
        \end{itemize}
        \item The polymer strands stretch out more ($H$ increases) when they don't want to interact with the 2D interface.
        \item What is the energy or enthalpy?
        \begin{equation*}
            \Delta H \approx vc^2\cdot HD^2
        \end{equation*}
        \begin{itemize}
            \item $H,D$ are defined as above.
            \item $v$ is how many two body interactions tere are (counted by mole).
            \item $c^2$ describes how dense the system is.
        \end{itemize}
        \item Here, $\Delta S\approx H^2/a^2N$ (as opposed to $\lambda^2/a^2N$ from earlier).
        \item We want to minimize $H^2$ and $1/H$ on an $H$ vs. Gibbs free energy graph.
    \end{itemize}
\end{itemize}



\section{Chapter 7: Thermodynamics of Polymer Mixtures}
\emph{From \textcite{bib:HiemenzLodge}.}
\begin{itemize}
    \item \marginnote{9/14:}Goals for this chapter.
    \begin{itemize}
        \item Thermodynamically analyze a solution of a polymer in a low molecular weight solvent.
        \item Determine the phase equilibria relevant to this situation.
    \end{itemize}
    \item \textbf{Polymer blend}: A mixture of two polymers.
    \item \textbf{Pure} (thermodynamics): The purely phenomenological study of observable thermodynamic quantities and the relationships among them.
    \item \textbf{Statistical} (thermodynamics): The atomistic model justifying purely thermodynamic observations.
    \begin{itemize}
        \item "\emph{Doing} thermodynamics does not even require knowledge that molecules exist\dots whereas \emph{understanding} thermodynamics benefits considerably from the molecular point of view" \parencite[271]{bib:HiemenzLodge}.
    \end{itemize}
    \item In this chapter, we are concerned with the state of a two-component system at equilibrium. The Gibbs free energy relates to this equilibrium, and in this case, it is given by
    \begin{align*}
        \dd{G} &= \left( \pdv{G}{P} \right)_{T,n_1,n_2}\dd{P}+\left( \pdv{G}{T} \right)_{P,n_1,n_2}\dd{T}+\left( \pdv{G}{n_1} \right)_{P,T,n_2}\dd{n_1}+\left( \pdv{G}{n_2} \right)_{P,T,n_1}\dd{n_2}\\
        &= V\dd{P}-S\dd{T}+\sum_{i=1}^2\mu_i\dd{n_i}
    \end{align*}
    \item \textbf{Partial molar} (quantity $Y$ of component $i$): The amount of $Y$ contributed to the whole by each mole of component $i$ in a mixture. \emph{Denoted by} $\bm{\bar{Y}_i}$. \emph{Units} $\textbf{mol}^{\bm{-1}}$. \emph{Given by}
    \begin{equation*}
        \bar{Y}_i := \left( \pdv{Y}{n_i} \right)_{P,T,n_{j\neq i}}
    \end{equation*}
    \begin{itemize}
        \item Example: The chemical potential of component $i$ is the amount of Gibbs free energy contributed to the total Gibbs free energy $G$ by each mole of $i$.
        \item There exist a partial molar volume, enthalpy, and entropy.
        \item The value of partial molar quantities depends on the overall composition of the mixture.
        \begin{itemize}
            \item Example: $\bar{V}_{\ce{H2O}}$ is not the same for a water-alcohol mixture that is 10\% water as for one that is 90\% water.
        \end{itemize}
        \item For a pure substance, partial molar quantities are equal to \textbf{molar values}.
        \begin{itemize}
            \item Example: $\mu_i=\hat{G}_i$.
        \end{itemize}
        \item Properties of a mixture are linear combinations of mole-weighted contributions of the partial molar properties of the components.
        \begin{equation*}
            Y_\text{m} = \sum_in_i\bar{Y}_i
        \end{equation*}
        \item The value of $Y_\text{m}$ on a per mole basis is given by \textbf{mole fractions} as follows.
        \begin{equation*}
            \frac{Y_\text{m}}{\sum_in_i} = \sum_ix_i\bar{Y}_i
        \end{equation*}
        \item Partial molar quantities exhibit the same relations as ordinary thermodynamic variables.
        \begin{itemize}
            \item Examples:
            \begin{align*}
                \mu_i &= \bar{H}_i-T\bar{S}_i&
                \bar{V}_i &= \left( \pdv{\mu_i}{P} \right)_{T,n_{j\neq i}}
            \end{align*}
        \end{itemize}
    \end{itemize}
    \item \textbf{Molar} (quantity $Y$ of substance $i$): The amount of $Y$ contributed by each mole of a substance $i$ when pure. \emph{Dentoed by} $\bm{\hat{Y}_i}$. \emph{Units} $\textbf{mol}^{\bm{-1}}$.
    \item \textbf{Mole fraction} (of component $i$): The fraction of moles of component $i$ relative to the total number of moles in the mixture. \emph{Denoted by} $\bm{x_i}$. \emph{Given by}
    \begin{equation*}
        x_i := \frac{n_i}{\sum_in_i}
    \end{equation*}
    \item \textbf{Standard state} (value of $Y_i$): The value of $Y_i$ when the substance $i$ is pure. \emph{Denoted by} $\bm{Y_i^\circ}$.
    \item \textbf{Activity}: A thermodynamic concentration and measure of the nonideality of solutions. \emph{Denoted by} $\bm{a_i}$. \emph{Given by}
    \begin{equation*}
        \mu_i = \mu_i^\circ+RT\ln a_i
    \end{equation*}
    \item Notation.
    \begin{itemize}
        \item We've established $n_i$ as the number of \emph{moles} of component $i$.
        \item Let $m_i$ denote the number of \emph{molecules} of component $i$. Thus,
        \begin{equation*}
            m_i = N_\text{A}n_i
        \end{equation*}
        where $N_\text{A}=\num{6.022e23}$ is \textbf{Avogadro's number}.
        \begin{itemize}
            \item This is also equal to $X_i$ from class!
        \end{itemize}
    \end{itemize}
    \item \textbf{Coordination number}: The number of nearest neighbors that surround a central lattice point. \emph{Denoted by} $\bm{z}$.
    \begin{itemize}
        \item Example: A cell in a 2D square lattice has $z=4$.
    \end{itemize}
    \item Regular solution theory: A simple statistical model that provides a useful expression for the free energy of mixing for a binary solution of two components.
    \begin{itemize}
        \item Assume that the two molecules in the mixture have equal volumes.
        \item Assume that the two components have equal (and concentration-independent) partial molar volumes, i.e., $\bar{V}_1=\bar{V}_2$.
        \item Imagine each molecule occupying a cell in a lattice with volume equal to the molecular volume.
        \item Let the lattice have coordination number $z$.
    \end{itemize}
\end{itemize}




\end{document}