\documentclass[../notes.tex]{subfiles}

\pagestyle{main}
\renewcommand{\chaptermark}[1]{\markboth{\chaptername\ \thechapter\ (#1)}{}}
\stepcounter{chapter}

\begin{document}




\chapter{Solution Thermodynamics}
\section{Flory-Huggins Theory}
\begin{itemize}
    \item \marginnote{9/18:}Outline of the next three lectures.
    \begin{itemize}
        \item Thermodynamics of polymer solutions and blends.
        \begin{itemize}
            \item Entropy of mixing.
            \item Enthalpy of mixing.
            \item Flory interaction parameter (definition and measurements).
            \item Solutions and melts (the theta temperature).
            \item LCST vs. UCST.
        \end{itemize}
        \item Copolymers.
        \begin{itemize}
            \item Microphase separation.
            \item Interfacial free energy.
            \item Chain stretching and configurational free energy.
        \end{itemize}
    \end{itemize}
    \item Huggins\footnote{"HOY-gins"}.
    \item We're going to start by approximating polymers as straight chains on a lattice.
    \begin{figure}[h!]
        \centering
        \includegraphics[width=0.6\linewidth]{polymPhase.png}
        \caption{Lattice theory for polymer phase behavior.}
        \label{fig:polymPhase}
    \end{figure}
    \begin{itemize}
        \item When we heat them up, will they stay phase separated or mix (go into State 2)?
        \item This will depend on how \textbf{compatible} they are.
    \end{itemize}
    \item \textbf{Compatible} (polymers): Two different types of polymers that like to mix with each other to form a single phase.
    \item The original model for phase behavior was postulated by \textcite{bib:alloyPhase} for small molecules and alloys.
    \begin{itemize}
        \item \textcite{bib:polymPhasea} and \textcite{bib:polymPhaseb} generalized this model to longer things (polymers).
        \item Flory was a Stanford prof., but started in the petroleum business (needed to separate chains and understand how they behave).
    \end{itemize}
    \item The thermodynamics of polymer solutions and blends are important for many applications, such as\dots
    \begin{itemize}
        \item Phase diagrams;
        \item Fractionation by molecular weight and/or by composition;
        \item $T_m$ depression in a semicrystalline polymer by a 2nd component;
        \item Swelling behavior of networks/gels;
        \begin{itemize}
            \item Covered much later in the course.
        \end{itemize}
    \end{itemize}
    \item High-impact polystyrene (HIPS).
    \begin{figure}[h!]
        \centering
        \includegraphics[width=0.4\linewidth]{salamiPhase.png}
        \caption{Salami phase micrograph in styrene-butadiene mixtures.}
        \label{fig:salamiPhase}
    \end{figure}
    \begin{itemize}
        \item Example: circular plastic dishes in lab. Hard, but very brittle.
        \item Idea to make better: Mix a stiff but brittle polymer (PS) with a soft elastic polymer (polybutadiene, PB) to get better mechanical properties.
        \item Cracks cannot propagate because they will hit rubbery phases of PB that have phase separated on the nanoscale, and be absorbed.
        \begin{itemize}
            \item This is called the \textbf{salami phase} because of how micrographs look.
        \end{itemize}
        \item Effect: Stress-strain curve elongates significantly (\textbf{toughness} increases because that is the area under the curve).
    \end{itemize}
    \item Aside: Making things look clear requires a lot of polymer engineering, because you have to make things very amorphous and not have nanoscale crystals.
    \item Thermodynamics of polymer blends.
    \begin{itemize}
        \item Legos are made of statistical copolymers of acrylonitrile-butadiene-styrene (ABS).
        \item Acrylonitrile gives resistance to repeated clicking and unclicking, butadiene makes it rubbery, styrene makes it shiny.
    \end{itemize}
    \item Today: Derive a free energy functional.
    \begin{itemize}
        \item Last lecture, we derived a free energy functional for single chains.
    \end{itemize}
    \item Today, we're looking at $G=H-TS$.
    \begin{itemize}
        \item What we're really interested in is the free energy of mixing,
        \begin{equation*}
            \Delta G_M = G_{1,2}-(G_1+G_2)
        \end{equation*}
        \item In multicomponent systems --- besides the typical parameters of excluded volume, etc. --- we need to know\dots
        \begin{itemize}
            \item How many chains we have of each type;
            \item What their degree of polymerization is;
            \item What total volume do they occupy.
        \end{itemize}
        \item Thus, in State 1, we have
        \begin{align*}
            V_1 &= n_1N_1v_1&
            V_2 &= n_2N_2v_2
        \end{align*}
        which describes two separate phases\dots
        \begin{itemize}
            \item Containing $n_i$ moles of species $i$;
            \item With degree of polymerization $N_i$;
            \item Each occupying a total volume $V_i$;
            \item Where the volume of each monomer/solvent molecule is given by $v_i$.
        \end{itemize}
        \item In State 2, we have a mixed phase with total volume
        \begin{equation*}
            V = \underbrace{n_1N_1v_1}_{V_1}+\underbrace{n_2N_2v_2}_{V_2}
        \end{equation*}
        \item Note that we're assuming that there is no change in volume $\Delta V$ during mixing.
        \item Nomenclature: If a system is comprised of a solvent and polymer, name the solvent "1" and the polymer "2".
    \end{itemize}
    \item To understand the thermodynamics of mixing, we'll start with the \emph{entropy} of mixing.
    \begin{itemize}
        \item Comments.
        \begin{itemize}
            \item In a melt, most chains do not feel themselves because other chains screen the interaction of the original chain with itself.
            \item This is great for us, because everything behaves like a truly random walk with scaling relation $N^{1/2}$.
            \begin{itemize}
                \item This realization is what won Flory his Nobel Prize!
            \end{itemize}
        \end{itemize}
        \item There is only 1 possible way to arrange a pure component in its volume.
        \begin{itemize}
            \item This follows from the binomial expression $\binom{n_iN_i}{n_iN_i}$.
            \item Thus, when phases are separated, each phase has entropy $S=\kB\ln(1)=0$.
        \end{itemize}
    \end{itemize}
    \item Mean field lattice theory.
    \begin{figure}[h!]
        \centering
        \includegraphics[width=0.3\linewidth]{latticeConfig.JPG}
        \caption{Configurations in mean field lattice theory.}
        \label{fig:latticeConfig}
    \end{figure}
    \begin{itemize}
        \item We get another binomial because we're adjacent to a random walk where we have 1 or 2 in each adjacent cell as we go along.
        \item As we fill up the grid, we first have access to all $X_0$ of the objects. Then we have 1 less, then we have 2 less, etc.
        \begin{itemize}
            \item But since all the objects in group 1 or group 2 are the same, we need to divide out by the number of objects $X_1$ in category 1. We need to do the same because all objects in $X_2$ are the same.
            \item Thus,
            \begin{equation*}
                \Omega_{1,2} = \frac{X_0!}{X_1!X_2!}
            \end{equation*}
        \end{itemize}
        \item It follows that the change entropy $\Delta S$ upon mixing is
        \begin{equation*}
            \Delta S_M = \kB\ln\Omega_{1,2}-0
        \end{equation*}
        \begin{itemize}
            \item Remember that the initial entropy is zero!
        \end{itemize}
        \item Invoking Stirling's approximation and remembering that $X_1+X_2=X_0$, we can then get
        \begin{align*}
            \frac{\Delta S_M}{\kB} &= X_0\ln X_0-X_0-\left[ X_1\ln X_1-X_1+X_2\ln X_2-X_2 \right]\\
            &= X_0\ln X_0-X_1\ln X_1-X_2\ln X_2
        \end{align*}
        \item It follows that the entropy of mixing per site $\Delta S_M/\kB X_0$ is
        \begin{align*}
            \frac{\Delta S_M}{\kB X_0} &= \frac{1}{X_0}\left[ (X_1+X_2)\ln X_0-X_1\ln X_1-X_2\ln X_2 \right]\\
            &= -\frac{X_1}{X_0}(\ln X_1-\ln X_0)-\frac{X_2}{X_0}(\ln X_2-\ln X_0)\\
            &= -\frac{X_1}{X_0}\ln\frac{X_1}{X_0}-\frac{X_2}{X_0}\ln\frac{X_2}{X_0}\\
            &= -\phi_1\ln\phi_1-\phi_2\ln\phi_2
        \end{align*}
        where $\phi_i$ denotes the \textbf{volume fraction} of spaces occupied by species $i$.
    \end{itemize}
    \item \textbf{Volume fraction} (of $i$): The fraction of lattice sites occupied by species $i$. \emph{Denoted by} $\bm{\phi_i}$. \emph{Given by}
    \begin{equation*}
        \phi_i := \frac{X_i}{X_0}
    \end{equation*}
    \begin{itemize}
        \item It follows from this definition that $\phi_1+\phi_2=1$.
    \end{itemize}
    \item Note: An assumption underlying the use of the Boltzmann equation is that all microstates have equal energy. This isn't strictly true, but it's a good enough approximation.
    \begin{figure}[h!]
        \centering
        \begin{subfigure}[b]{0.25\linewidth}
            \centering
            \begin{tikzpicture}
                \draw [xshift=-0.5cm,yshift=-0.5cm] (0,0) grid (2,2);
                \draw [gray,dashed] (0,0) rectangle (1,1);

                \filldraw [draw=gray,fill=rex] (0,0) circle (2.5mm);
                \filldraw [draw=gray,fill=blx] (1,0) circle (2.5mm);
                \filldraw [draw=gray,fill=blx] (0,1) circle (2.5mm);
                \filldraw [draw=gray,fill=rex] (1,1) circle (2.5mm);
            \end{tikzpicture}
            \caption{Energy $E_1$ system.}
            \label{fig:nearNeighbora}
        \end{subfigure}
        \begin{subfigure}[b]{0.25\linewidth}
            \centering
            \begin{tikzpicture}
                \draw [xshift=-0.5cm,yshift=-0.5cm] (0,0) grid (2,2);
                \draw [gray,dashed]
                    (0,0) -- (1,0)
                    (0,1) -- (1,1)
                ;

                \filldraw [draw=gray,fill=blx] (0,0) circle (2.5mm);
                \filldraw [draw=gray,fill=rex] (1,0) circle (2.5mm);
                \filldraw [draw=gray,fill=blx] (0,1) circle (2.5mm);
                \filldraw [draw=gray,fill=rex] (1,1) circle (2.5mm);
            \end{tikzpicture}
            \caption{Energy $E_2$ system.}
            \label{fig:nearNeighborb}
        \end{subfigure}
        \caption{Nearest neighbor interactions.}
        \label{fig:nearNeighbor}
    \end{figure}
    \begin{itemize}
        \item Example: Assume nearest neighbor interactions matter.
        \item Let opposing objects in neighboring cells contribute energy $\varepsilon_1$ to the total energy of the system. This means that in Figure \ref{fig:nearNeighbor}, $E_1=4\varepsilon_1$ and $E_2=2\varepsilon_2$.
        \item In big systems, the energy won't fluctuate much, though, so Boltzmann equation is more of an approximation, but it's \emph{good enough}.
    \end{itemize}
    \item Mean field mixing enthalpy.
    \begin{itemize}
        \item Assume the lattice is such that each point has $z$ nearest neighbor cells.
        \begin{itemize}
            \item For example, each cell in a square lattice (Figure \ref{fig:latticeConfig}) has 4 nearest neighbors: One above, below, right, and left.
        \end{itemize}
        \item To calculate enthalpic interactions, we consider the number of pairwise interactions.
        \item But in a mean field approximation, we wash out some detail by mixing red and blue to make purple. We say that \emph{on average}, your neighbor one away from you is proportional to the composition (because it might be red, then blue, then red again). Quick exchange of neighbors.
        \begin{itemize}
            \item You can build on this with weights, but this is the purest sense of a mean field approximation.
        \end{itemize}
        \item The mean field approximation breaks down when mixing breaks down, i.e., when you start to get some clusters of pure one thing and pure another thing.
    \end{itemize}
    \item Enthalpy of mixing.
    \begin{figure}[h!]
        \centering
        \includegraphics[width=0.2\linewidth]{Uwell.JPG}
        \caption{Potential well.}
        \label{fig:Uwell}
    \end{figure}
    \begin{itemize}
        \item We need to count each interaction. It follows from the above that the number of $ij$ interactions is
        \begin{align*}
            \xi_{11} &= \frac{X_1z\phi_1}{2}&
            \xi_{22} &= \frac{X_2z\phi_2}{2}&
            \xi_{12} &= X_1z\phi_2
        \end{align*}
        \begin{itemize}
            \item Let's rationalize the formula for $\xi_{11}$. $X_1$ sites each have $z$ nearest neighbors, so there are $X_1z$ nearest neighbor interactions where one of the species involved is species 1. The probability that a nearest neighbor is also species 1 is $\phi_1$. Thus, there are $X_1z\phi_1$ interactions where the other partner is also species 1. However, since there are two neighbors involved, we have currently accounted for each interaction twice: Once from the perspective of each neighbor. Thus, we need to divide by 2.
            \item Same rationalization for $\xi_{22}$.
            \item For $\xi_{12}$, a similar rationalization applies but we are not overcounting because we only have the perspective of one of the two interaction partners taken into account, so no dividing by 2 is necessary. Alternatively, from the perspective of both species, we have
            \begin{equation*}
                \xi_{12} = \frac{X_1z\phi_2}{2}+\frac{X_2z\phi_1}{2} = X_1z\phi_2
            \end{equation*}
        \end{itemize}
        \item In a typical attraction well, we have a most probable distance $a$, at which the energy depth is $\varepsilon$ (see Figure \ref{fig:Uwell}).
        \item Let $\varepsilon_{ij}$ refer to how deep the well is between species $i$ and $j$, where $i,j\in\{1,2\}$.
        \item It follows that in a mixed enthalpic state, the interaction energy is the following.
        \begin{align*}
            H_{1,2} &= \xi_{12}\varepsilon_{12}+\xi_{11}\varepsilon_{11}+\xi_{22}\varepsilon_{22}\\
            &= zX_1\phi_2\varepsilon_{12}+\frac{zX_1\phi_1\varepsilon_{11}}{2}+\frac{zX_2\phi_2\varepsilon_{22}}{2}
        \end{align*}
        \begin{itemize}
            \item In pure enthalpic states, the interaction energies are the following.
            \begin{align*}
                H_1 &= \frac{zX_1\varepsilon_{11}}{2}&
                H_2 &= \frac{zX_2\varepsilon_{22}}{2}
            \end{align*}
        \end{itemize}
        \item Note that we are actually computing the energy $U$ above, not the enthalpy $H=U+PV$. However, we equate $H=U$ by assuming that there is no volume change upon mixing.
        \begin{itemize}
            \item Such an assumption is consistent with the lattice approach (i.e., the assumption that all species fit into a lattice site of equal volume both before and after mixing).
        \end{itemize}
        \item It follows that
        \begin{align*}
            \Delta H_M &= H_{1,2}-(H_1+H_2)\\
            &= z\left[ X_1\phi_2\varepsilon_{12}+\frac{X_1\varepsilon_{11}}{2}(\phi_1-1)+\frac{X_2\varepsilon_{22}}{2}(\phi_2-1) \right]
        \end{align*}
        and hence
        \begin{align*}
            \frac{\Delta H_M}{X_0} &= z\left[ \phi_1\phi_2\varepsilon_{12}+\frac{\phi_1\varepsilon_{11}}{2}(-\phi_2)+\frac{\phi_2\varepsilon_{22}}{2}(-\phi_1) \right]\\
            \frac{\Delta H_M}{X_0\kB T} &= \frac{z}{\kB T}\left\{ \phi_1\phi_2\left[ \varepsilon_{12}-\frac{1}{2}(\varepsilon_{11}+\varepsilon_{22}) \right] \right\}\\
            &= \chi\phi_1\phi_2
        \end{align*}
        where $\chi$ denotes the \textbf{Flory $\bm{\chi}$ parameter}.
    \end{itemize}
    \item \textbf{Flory $\bm{\chi}$ parameter}: A measure of the degree to which the two species like each other vs. themselves. \emph{Also known as} \textbf{interaction parameter}. \emph{Denoted by} $\bm{\chi}$. \emph{Given by}
    \begin{equation*}
        \chi := \frac{z}{\kB T}\left[ \varepsilon_{12}-\frac{1}{2}(\varepsilon_{11}+\varepsilon_{22}) \right]
    \end{equation*}
    \begin{itemize}
        \item From its definition, we can see that the sign of $\chi$ determines whether mixing will be enthalpically favorable or not. Moreover, the sign of $\chi$ is determined by the interplay between how much the components like each other, and how much they like themselves on average.
        \item Implication: If the two components like themselves more than they like each other (right diagram at bottom of slide 14), $\chi$ will be positive and will between the middle two lines (we don't know here if it will mix or demix)??
    \end{itemize}
    \item The $\chi$ parameter is still being debated today; Alfredo is writing a paper on it!
    \begin{itemize}
        \item The parameter as written is linear, but that's based on a mean field assumption. Should it have a quadratic term? Maybe it does at a (more accurate) higher level of theory.
    \end{itemize}
    \item At this point, we can assemble everything into the free energy of mixing for monomers.
    \begin{align*}
        \frac{\Delta G_M}{X_0} &= \frac{\Delta H_M}{X_0}-T\frac{\Delta S_M}{X_0}\\
        &= \kB T\chi\phi_1\phi_2-T\cdot\kB(-\phi_1\ln\phi_1-\phi_2\ln\phi_2)\\
        &= \kB T(\chi\phi_1\phi_2+\phi_1\ln\phi_1+\phi_2\ln\phi_2)
    \end{align*}
    \item How about for polymers?
    \begin{figure}[h!]
        \centering
        \includegraphics[width=0.25\linewidth]{latticePolym.JPG}
        \caption{Fitting polymers into a square lattice.}
        \label{fig:latticePolym}
    \end{figure}
    \begin{itemize}
        \item For enthalpy of mixing, we still use a mean field approximation, so nothing changes. We justify this by noting, as earlier, that in a melt, most chains do not feel themselves because other chains screen the interaction of the original chain with itself.
        \item For entropy of mixing, things do change a bit.
        \begin{itemize}
            \item As in Figure \ref{fig:latticeConfig}, we have to pick $N$ adjacent sites.
            \item Then the next time we pick, what's left is the number of sites minus $N$. And on and on.
            \item This essentially reduces the number of sites by $N_1$ and $N_2$.
            \item All of this is part of a complex derivation done by Flory, but the simple and intuitive result is that the entropy of mixing decreases by approximately $1/N$ due to the connectivity of the $N$ segments we cannot arrange any further apart. Mathematically, we obtain
            \begin{equation*}
                \frac{\Delta S_M}{\kB X_0} = -\frac{\phi_1}{N_1}\ln\phi_1-\frac{\phi_2}{N_2}\ln\phi_2
            \end{equation*}
            \item This equation tells us that when $N$ is big, entropy doesn't play a huge role in driving mixing (because $N_1,N_2$ are in the denominator). Thus, enthalpy matters more for the mixing of polymers.
        \end{itemize}
    \end{itemize}
    \item Using the modified mixing entropy, we can now finally state the "famous Flory-Huggins theory for the free energy of mixing."
    \begin{equation*}
        \frac{\Delta G_M}{X_0} = \kB T\left( \chi\phi_1\phi_2+\frac{\phi_1}{N_1}\ln\phi_1+\frac{\phi_2}{N_2}\ln\phi_2 \right)
    \end{equation*}
    \begin{itemize}
        \item This equation is applicable to solvent-solvent ($1=N_1=N_2$), polymer-solvent ($1=N_1\neq N_2$), and polymer-polymer ($1\neq N_1,N_2$) mixing.
        \item This equation also gives us a tool to investigate phase behavior and how it varies with $T,\chi,\phi_i,N_i$. Namely, values of these variables that lead to a negative $\Delta G_M$ will correspond to mixing, and values of these variables that lead to a positive $\Delta G_M$ will correspond to demixing.
    \end{itemize}
    \item How do we measure $\chi$ experimentally?
    \begin{itemize}
        \item We will explore this soon.
    \end{itemize}
    \item Influence of $\chi$ on phase behavior.
    \begin{figure}[h!]
        \centering
        \begin{subfigure}[b]{0.33\linewidth}
            \centering
            \includegraphics[width=0.95\linewidth]{mixingChia.png}
            \caption{Mixing enthalpy.}
            \label{fig:mixingChia}
        \end{subfigure}
        \begin{subfigure}[b]{0.32\linewidth}
            \centering
            \includegraphics[width=0.95\linewidth]{mixingChib.png}
            \caption{Mixing entropy.}
            \label{fig:mixingChib}
        \end{subfigure}
        \begin{subfigure}[b]{0.33\linewidth}
            \centering
            \includegraphics[width=0.95\linewidth]{mixingChic.png}
            \caption{Mixing energy.}
            \label{fig:mixingChic}
        \end{subfigure}
        \caption{Mixing or demixing based on the Flory $\chi$ parameter.}
        \label{fig:mixingChi}
    \end{figure}
    \begin{itemize}
        \item Notice how adding the curves in Figures \ref{fig:mixingChia}-\ref{fig:mixingChib} gives the curves in Figure \ref{fig:mixingChic}.
        \item From Figure \ref{fig:mixingChic}, we can see that when $\chi$ reaches the range of 2-3, we get demixing.
        \begin{itemize}
            \item This is because (at $\chi=3$ and initially equal volumes of both components, for example) it will be more energetically favorable to form (i) a phase that is approximately 90\% component 1 and 10\% component 2 and (ii) a phase that is approximately 10\% component 1 and 90\% component 2 than it will be to completely mix.
        \end{itemize}
        \item When $\chi=2$, free energy is largely flat. Thus, very different compositions have similar free energy, which means that the system will "undergo a 2nd order transition where all length scales are viable??"
    \end{itemize}
    \item A parameter that has gained more importance in recent years: The coordination number $z$.
    \begin{itemize}
        \item It turns out that the more neighbors you have, the better mean-field behavior you get.
        \item This is essentially because you're averaging over more values.
        \item Recent finding: $\chi$ is pretty good if $z$ is big; if $z$ is small, quadratic and other higher order corrections may be necessary for $\chi$.
    \end{itemize}
    \item Phase behavior of blends.
    \begin{itemize}
        \item The critical value $\chi_c$ of the Flory $\chi$ parameter is the value of $\chi$ at which you see phase separation.
        \item $\chi_c$ decreases exponentially with increasing chain length $N_1=N_2=N$.
        \begin{itemize}
            \item This relates to the phenomenon discussed earlier in which the mixing entropy shrinks as chain length grows. If mixing entropy is shrinking, the system can tolerate less enthalpic repulsion before demixing.
        \end{itemize}
    \end{itemize}
    \item Preliminaries to next class.
    \begin{figure}[h!]
        \centering
        \begin{subfigure}[b]{0.4\linewidth}
            \centering
            \includegraphics[width=0.7\linewidth]{phaseDiaa.png}
            \caption{Phase diagram.}
            \label{fig:phaseDiaa}
        \end{subfigure}
        \begin{subfigure}[b]{0.4\linewidth}
            \centering
            \includegraphics[width=0.9\linewidth]{phaseDiab.png}
            \caption{Composite phase diagram and energy plot.}
            \label{fig:phaseDiab}
        \end{subfigure}
        \caption{Example phase diagram and explanation.}
        \label{fig:phaseDia}
    \end{figure}
    \begin{itemize}
        \item Construction of phase diagrams.
        \begin{itemize}
            \item You have a critical point graph that gets flipped??
            \item The inside, dashed line is the \textbf{spinodal curve}.
            \item The outside, solid line is the \textbf{binodal curve}.
            \item We are essentially thinking of $\phi_2$ and $\chi N$ as two independent variables that together define a system. A dependent variable of interest in this system is the mixing energy $\Delta G_M$. Therefore, a representation of the set of systems as a 2D graph in 3D space is warranted. Figure \ref{fig:mixingChic} plots one set of 2D cross sections of this surface (with color as a pseudo-third dimension), and Figure \ref{fig:phaseDia} plots other important contours of this surface.
        \end{itemize}
        \item The spinodal curve plots the inflection points in Figure \ref{fig:mixingChic}.
        \begin{itemize}
            \item For example, the pink curve has two inflection points around 0.2 and 0.8, and the green curve has one (special) inflection point around 0.5.
            \item Between the two inflection points, the system is unstable.
            \item Concave curvature puts the sum of two free energies below the points.
            \item You will evolve toward the two lowest energy points, phase separating as needed.
        \end{itemize}
        \item Once we have reached a \textbf{metastable} state at the spinodal, demixing to the binodal occurs by nucleation and growth.
    \end{itemize}
\end{itemize}



\section{The Theta State}
\begin{itemize}
    \item \marginnote{9/23:}Announcements.
    \begin{itemize}
        \item PSet 2 posted; due midnight next Tuesday.
    \end{itemize}
    \item Last time: Flory-Huggins polymers.
    \item Today, let's begin by thinking about the equilibrium between the two different phases of a system.
    \begin{itemize}
        \item We will quantify this with binodal and spinodal stuff.
    \end{itemize}
    \item \textbf{Chemical potential}: The Gibbs free energy of a substance at a given concentration. \emph{Denoted by} $\bm{\mu_i}$. \emph{Given by}
    \begin{equation*}
        \mu_i := \left( \pdv{G}{n_i} \right)_{T,P,N,n_j}\tag{$j\neq i$}
    \end{equation*}
    \item \textbf{Coexistence curve}: The set of points where the chemical potentials of each component in each phase are equal. \emph{Also known as} \textbf{binodal curve}. \emph{Constraints}
    \begin{align*}
        \mu_1' &= \mu_1''&
            \mu_2' &= \mu_2''\\
        \mu_1'-\mu_1^\circ &= \mu_1''-\mu_1^\circ&
            \mu_2'-\mu_2^\circ &= \mu_2''-\mu_2^\circ
    \end{align*}
    \begin{itemize}
        \item Note that one of the phases (e.g., the clumped phase in Figure \ref{fig:demix}) is denoted by a single prime, and the other phase (e.g., the mostly solvent/unclumped phase in Figure \ref{fig:demix}) is denoted by a double prime.
        \begin{itemize}
            \item The lower line of constraints indicates that equal chemical potentials are also equal with respect to the reference chemical potential.
            \item We include it because $\mu_i-\mu_i^\circ$ is easier to calculate than just $\mu_i$. This is because we have expressions for $\Delta G_M$, not $G_M$, so we would prefer to use the left derivative below than the right.
            \begin{align*}
                \mu_i-\mu_i^\circ &= \pdv{\Delta G}{n_i}&
                \mu_i &= \pdv{G}{n_i}
            \end{align*}
        \end{itemize}
        \item The coexistence curve encloses all compositions wherein the mixture demixes into two distinct, coexisting phases.
        \item Alternatively, the coexistence curve is given by finding, for each $\Delta G_M(\phi)$ curve, the intersections of said curve with a line tangent to the two wells.\footnote{See Office Hours for more on this tangent line.}
    \end{itemize}
    \item \textbf{Spinodal curve}: The set of inflection points along the $\Delta G_M(\phi)$ curves, generated as $\chi$ is varied. \emph{Also known as} \textbf{stability limit}. \emph{Constraint}
    \begin{equation*}
        \pdv[2]{\Delta G_M}{\phi_1} = 0
    \end{equation*}
    \begin{itemize}
        \item Recall that at each inflection point, the 2nd derivative is zero. This is why we use the criterion above.
    \end{itemize}
    \item \textbf{Critical point}: The point on a phase diagram where a solution first begins to demix, or first begins to mix. \emph{Denoted by} $\bm{(\phi_{1,c},\chi_c)}$. \emph{Constraints}
    \begin{align*}
        \eval{\pdv[2]{\Delta G_M}{\phi_1}}_{(\phi_{1,c},\chi_c)} &= 0&
        \eval{\pdv[3]{\Delta G_M}{\phi_1}}_{\phi_{1,c}} &= 0
    \end{align*}
    \begin{itemize}
        \item The spinodal and coexistence curves intersect at this point.
        \item Because the critical point lies on the spinodal curve, it will satisfy the left constraint above (i.e., 2nd derivative is zero).
        \item To explain the right constraint, we need to think a bit more. Usually, an inflection point identifies where the concavity of a curve changes from up to down, or vice versa. At such a point, the concavity is zero. However, on the $\Delta G_M(\phi)$ curve at the \emph{boundary} between mixing and demixing (e.g., the green line in Figure \ref{fig:mixingChic}), there is an inflection point where the concavity goes from up, to zero, to up again. At such an inflection point, the second derivative is still equal to zero \emph{but the third derivative is, too}.
        \item Thus, altogether, the right constraint pinpoints the concentration $\phi_{1,c}$ at which demixing first occurs as we raise $\chi$, and the left constraint relates this concentration to the interaction parameter $\chi_c$ necessary to produce a mixing curve with an inflection point at $\phi_{1,c}$.
    \end{itemize}
    \item Per the above, a few derivatives suffice to compute the spinodal curve. But if we want to understand the binodal curve, we should find a way to compute the chemical potentials of our solutions.
    \item The chemical potential of a polymer-solvent solution.
    \begin{figure}[h!]
        \centering
        \includegraphics[width=0.25\linewidth]{demix.JPG}
        \caption{Demixing of a polymer-solvent solution.}
        \label{fig:demix}
    \end{figure}
    \begin{itemize}
        \item An interesting way of thinking about this: Osmometry.
        \begin{itemize}
            \item Imagine a uniform, mixed polymer solution.
            \item We can demix to a state where the polymer is all clumped together, but the clumps still have some solvent in them though.
            \item The phase separation stops when a solvent molecule \emph{inside} and \emph{outside} the polymer clump has the same chemical potential.
        \end{itemize}
        \item We now state --- without proof --- that the change in chemical potentials upon mixing are as follows.
        \begin{align*}
            \mu_1-\mu_1^\circ &= RT\left[ \ln(1-\phi_2)+\left( 1-\frac{1}{N_2} \right)\phi_2+\chi\phi_2^2 \right]\\
            \mu_2-\mu_2^\circ &= RT\left[ \ln\phi_2+(N_2-1)\phi_2+N_2\chi\phi_1^2 \right]
        \end{align*}
        \begin{itemize}
            \item Remember that by convention, phase 1 is the solvent and phase 2 is the polymer.
            \item Notice the multiplying vs. dividing of $N_2$.
            \item For a polydisperse system of polymer chains, let $N_2=\prb{N_2}=\Mn$.
            \item In PSet 2, Q1a, we'll derive the top one of these expressions.
        \end{itemize}
    \end{itemize}
    \item We'll now massage the above expressions to get some more mechanistic understanding out of them, specifically for dilute solutions.
    \begin{itemize}
        \item Since we are postulating a dilute solution, we may use the approximation that the volume fraction of component 2 is small. This will allow us to expand the logarithms.
        \item Recall from math class that for small $x$, the following approximation holds.
        \begin{equation*}
            \ln(1-x) = -x-\frac{x^2}{2}-\cdots-\frac{x^n}{n}-\cdots
        \end{equation*}
        \begin{itemize}
            \item Moreover, we can choose to use only the first couple of terms for sufficiently small $x$.
        \end{itemize}
        \item It also follows from the assumptions that (1) the solution is dilute (i.e., $n_1\gg n_2N_2$) and (2) that the volume of the monomers is approximately equal (i.e., $v_1\approx v_2$) that
        \begin{align*}
            \phi_1 &= \frac{n_1v_1}{n_1v_1+n_2N_2v_2}&
            \phi_2 &= \frac{n_2N_2v_2}{n_1v_1+n_2N_2v_2} \approx \frac{n_2N_2}{n_1}
        \end{align*}
        \item We thus expand
        \begin{align*}
            \ln(1-\phi_2) &\approx -\phi_2-\frac{1}{2}\phi_2^2
        \end{align*}
        \item It follows that
        \begin{align*}
            \frac{\mu_1-\mu_1^\circ}{RT} &= \ln(1-\phi_2)+1\cdot\phi_2-\frac{1}{N_2}\cdot\phi_2+\chi\phi_2^2\\
            &\approx -\phi_2-\frac{1}{2}\phi_2^2+\phi_2-\frac{\phi_2}{N_2}+\chi\phi_2^2\\
            &= -\frac{\phi_2}{N_2}+\left( \chi-\frac{1}{2} \right)\phi_2^2
        \end{align*}
        \begin{itemize}
            \item This expression implies that when $\chi<1/2$, the whole thing will be negative.
            \begin{itemize}
                \item It follows that if $\chi<1/2$, the chemical potential $\mu_1$ will be less than that of the standard, unmixed state $\mu_1^\circ$. Therefore, to lower the energy of the system, everything will want to mix.
            \end{itemize}
            \item In a subsequent course, we describe more results based off of the above equation!
        \end{itemize}
    \end{itemize}
    \item This chemical potential expression allows us to get a more precise definition of the $\theta$ condition.
    \begin{itemize}
        \item First, recall from thermodynamics that for an \textbf{ideal solution}, the chemical potential of the solvent is related to its mole fraction $x_1$ via
        \begin{equation*}
            \mu_1 = \mu_1^\circ+RT\ln x_1
        \end{equation*}
        \begin{itemize}
            \item Substituting $x_1=1-x_2$ and using the same small logarithm approximation as above, we can learn that for an ideal solution,
            \begin{equation*}
                \frac{\mu_1-\mu_1^\circ}{RT} = \ln(1-x_2) \approx -x_2 = -\frac{\phi_2}{N_2}
            \end{equation*}
        \end{itemize}
        \item Let's now compare this expression to the one we just derived for the chemical potential of a polymer-solvent solution.
        \begin{itemize}
            \item Doing so, we'll see that they both have a $-\phi_2/N_2$ term, but the polymer-solvent soution \emph{also} has the following term.
            \begin{equation*}
                RT\left( \chi-\frac{1}{2} \right)\phi_2^2
            \end{equation*}
            \item This term is called the \textbf{excess chemical potential}.
        \end{itemize}
        \item The excess chemical potential describes\dots
        \begin{itemize}
            \item Contact interactions, which relate to solvent quality as follows;
            \begin{equation*}
                \chi\phi_2^2RT
            \end{equation*}
            \item Excluded volume as follows.
            \begin{equation*}
                -\frac{1}{2}\phi_2^2RT
            \end{equation*}
        \end{itemize}
        \item When contact interactions and excluded volume balance, this is the $\theta$ condition. It follows that the $\theta$ condition occurs when the excess chemical potentential equals zero. But mathematically, this happens when $\chi=1/2$!
        \begin{itemize}
            \item It also follows (as might be intuitive) that the when we are in the $\theta$ condition, the solution behaves as an ideal solution (i.e., with its chemical potential expression equal to that of an ideal solution).
            \item This is our definition of the $\theta$ condition.
            \item Moreover, $\chi>1/2$ implies an elevated chemical potential for the solvent. The solvent would thus rather be closer to its pure state, and demixing will ensue. In other words, $\chi>1/2$ is indicative of a "bad solvent," one that polymers don't like as much as they like themselves.
            \item On the other hand, $\chi<1/2$ implies a decreased chemical potential for the solvent, one that will be augmented by further mixing. Therefore, $\chi<1/2$ is indicative of a "good solvent."
        \end{itemize}
        \item Lastly, there is the matter of $N_2$. As $N_2$ increases, the chemical potential of the solvent also increases. This means that as chains get longer, solvents tend to get "worse."
    \end{itemize}
    \item Implications of Flory-Huggins theory for the behavior polymer-solvent solutions, as seen on their phase diagrams.
    \begin{figure}[h!]
        \centering
        \includegraphics[width=0.25\linewidth]{polSolPhase.png}
        \caption{Binodal curves for polymer-solvent solutions as $N$ increases.}
        \label{fig:polSolPhase}
    \end{figure}
    \begin{itemize}
        \item Recall that a polymer-solvent solution occurs when component 1 is a solvent ($N_1=1$), component 2 is a polymer ($N_2\gg 1$), and the solution is dilute ($n_1\gg n_2N_2$).
        \item It is harder to mix small amounts of high molecular weight polymer into a solvent than it is to mix large amounts of said polymer into a solvent.
        \begin{itemize}
            \item This makes sense becasue beneath the critical concentration $c^*$, polymers will want to coil up into small localized pockets that essentially function as a second phase!
            \item The temperature required to ensure "even mixing" is the temperature needed to stretch the coils out more.
        \end{itemize}
        \item Additionally, $T_c$ increases as $N$ increases.
        \begin{itemize}
            \item This illustrates the point that higher MW polymers require more thermal energy to mix well.
        \end{itemize}
    \end{itemize}
    \item General expressions for the critical composition and critical interaction parameters.
    \begin{table}[h!]
        \centering
        \small
        \renewcommand{\arraystretch}{1.2}
        \begin{tabular}{ll|c|c}
            \toprule
            \multicolumn{2}{c|}{\textbf{Binary System}} & $\bm{\phi_{1,c}}$ & $\bm{\chi_c}$\\
            \midrule
            Low molar mass liquids & $N_1=N_2=1$ & $0.5$ & $2$\\
            Polymer-solvent blend & $N_1=1$, $N_2$ & $\dfrac{\sqrt{N_2}}{1+\sqrt{N_2}}$ & $\dfrac{1}{2}\left( 1+\dfrac{1}{\sqrt{N_2}} \right)^2$\\[1.2em]
            Symmetric polymer blend & $N_1=N=N_2$ & $0.5$ & $\dfrac{2}{N}$\\[1em]
            General & $N_1,N_2$ & $\dfrac{\sqrt{N_2}}{\sqrt{N_1}+\sqrt{N_2}}$ & $\dfrac{1}{2}\left( \dfrac{1}{\sqrt{N_1}}+\dfrac{1}{\sqrt{N_2}} \right)^2$\\
            \bottomrule
        \end{tabular}
        \caption{Critical composition and interaction parameters for binary blends.}
        \label{tab:criticalBlend}
    \end{table}
    \begin{itemize}
        \item There are four regimes for which it is important to know these parameters: A regular solution of two liquids, a polymer-solvent blend, a \textbf{symmetric} polymer-polymer blend, and a general polymer-polymer blend.
        \item You will derive these expressions in PSet 2, Q1e.
    \end{itemize}
    \item \textbf{Symmetric} (polymer-polymer blend): A polymer-polymer blend for which the degrees of polymerization for each component are equal. \emph{Constraint}
    \begin{equation*}
        N_1 = N = N_2
    \end{equation*}
    \item The limit of solvents getting "worse" occurs when you mix high molecular weight polymers.
    \begin{itemize}
        \item Here, you don't just need $\chi<1/2$ for a good solvent, but it turns out you will need $\chi<2/N$.
        \begin{itemize}
            \item You will prove the $\chi<2/N$ rule in PSet 2, Q1e.
        \end{itemize}
        \item This is why it is very difficult to mix high MW polymers.
    \end{itemize}
    \item Solubility parameter and $\chi$.
    \begin{figure}[h!]
        \centering
        \includegraphics[width=0.3\linewidth]{Hildebrand.JPG}
        \caption{Hildebrand's experiment.}
        \label{fig:Hildebrand}
    \end{figure}
    \begin{itemize}
        \item How do we estimate $\chi$?
        \begin{itemize}
            \item Hildebrand's interesting idea was to use the enthalpy of vaporization $\Delta H_v$.
        \end{itemize}
        \item Experimental setup.
        \begin{itemize}
            \item Take your liquid, heat it up, measure how much heat goes into the system, turn it into a gas, and see how the heat has turned into kinetic energy.
            \item By the time you have heated an object in a lattice, your are neighborless; you have no potential energy.
            \item We now have
            \begin{align*}
                \delta &= \left( \frac{\Delta E}{V} \right)^{1/2}
            \end{align*}
            \item We'll now calculate $\delta_1$ and $\delta_2$ for species 1 and 2, the \textbf{solubility parameters} of the components.
            \begin{align*}
                \delta_1 &= \sqrt{\frac{Z}{2}\frac{\varepsilon_{11}}{v}}&
                \delta_2 &= \sqrt{\frac{Z}{2}\frac{\varepsilon_{22}}{v}}
            \end{align*}
            \item We then use \textbf{Berthelot's mixing rule} (which uses the geometric mean) to get $\varepsilon_{12}$:
            \begin{equation*}
                \varepsilon_{12} = \sqrt{\varepsilon_{11}\varepsilon_{22}}
            \end{equation*}
            \item Now imagine two points 1 and 2 separated by a distance $r$, as in Figure \ref{fig:monomerGasc}.
            \begin{equation*}
                U_\text{attractive} = -\frac{\alpha_1\alpha_2}{r^6}
            \end{equation*}
            \begin{itemize}
                \item Scales as $1/r^6$, and also has the \textbf{polarizability} / \textbf{polarizability volumes}.
                \item This is related to dipole-induced dipole attractions; when you average over all possible combinations, this relation falls out. And that's what Lennard and Jones based their use of $1/r^6$ as the attractive term on!
                \item In PSet 2, we will prove that this attraction rule is "like likes like."
            \end{itemize}
            \item Then
            \begin{equation*}
                (\delta_1-\delta_2)^2 = \frac{1}{v}\left( \frac{z\varepsilon_{11}}{2}+\frac{z\varepsilon_{22}}{2}-z\varepsilon_{12} \right)
            \end{equation*}
            \item Now just multiply by $v/\kB T$ to get the $\chi$ parameter.
            \begin{itemize}
                \item $v$ is a volume.
            \end{itemize}
            \item This gets us to the \textbf{Hildebrand equation}
            \begin{equation*}
                \Delta H_M = V_m\phi_1\phi_2(\delta_1-\delta_2)^2 \geq 0
            \end{equation*}
            \begin{itemize}
                \item This works better for nonpolar than polar species.
                \item $V_m$ is the average molar volume of solvent / monomers.
            \end{itemize}
            \item See \textcite{bib:RubinsteinColby} for an in-depth discussion of this setup.
        \end{itemize}
    \end{itemize}
    \item At this point, we've written a lot of theoretical equations. Now let's see how Flory-Huggins theory compares with experimental measurements of polymer-solvent blends.
    \begin{figure}[h!]
        \centering
        \includegraphics[width=0.35\linewidth]{FHexpt.png}
        \caption{Flory-Huggins theory vs. experiment.}
        \label{fig:FHexpt}
    \end{figure}
    \begin{itemize}
        \item To do so, we'll look at the classic data of \textcite{bib:ShultzFlory} on the binodal curves of PS in cyclohexane.
        \begin{itemize}
            \item Note that Figure \ref{fig:FHexpt} is lifted from \textcite[300]{bib:HiemenzLodge} rather than the original paper as in the slides, since the textbook's figure is more consistent with the notation used thus far.
        \end{itemize}
        \item The dashed lines are the Flory-Huggins theory, which clearly differ significantly from the experimental results.
        \begin{itemize}
            \item This difference is because of Flory's mean field assumption, which doesn't hold here. Indeed, as you heat up, you will be more likely to have the same neighbor.
            \item However, while the predicted curve is wrong, the vertical scaling is correct! This phenomenon happens in several of Flory's theories.
        \end{itemize}
    \end{itemize}
    \item Phase diagrams of polymer-polymer blends.
    \begin{figure}[h!]
        \centering
        \includegraphics[width=0.3\linewidth]{polPolPhase.png}
        \caption{Binodal curves for PS-PB blends at almost symmetric compositions.}
        \label{fig:polPolPhase}
    \end{figure}
    \begin{itemize}
        \item Recall that the volume fraction that a chain occupies is
        \begin{equation*}
            \phi = \frac{Nv}{(N^{1/2}\ell)^3} \propto N^{-1/2}
        \end{equation*}
        \begin{itemize}
            \item Specifically, this is the ratio of a polymer's hard volume to its pervaded volume.
            \item The scaling in this equation implies that a polymer with $100$ repeat units occupies about 10\% of its pervaded volume, a polymer with \num{e4} repeat units occupies about 1\% of its pervaded volume, etc.
        \end{itemize}
        \item Therefore, molecules in a polymer chain are far more likely to interact with neighbors on a different polymer chain then they are to interact with themselves.
        \item This leads to the melt condition in which excluded volume effects are largely washed out, and hence phase diagrams (as in Figure \ref{fig:polPolPhase}) look quite similar to those for a solvent-solvent mixture.
        \begin{itemize}
            \item This average of the interactions of one polymer with many other chains also justifies the mean field assumption in this case.
        \end{itemize}
    \end{itemize}
    \item At this point, we know what phase diagrams look like and what kinds of curves Flory-Huggins theory predicts they should have.
    \item However, real systems can exhibit additional phenomena beyond the capacity of Flory-Huggins theory to describe. As such, let's now look at the two principal types of real phase diagrams.
    \begin{figure}[h!]
        \centering
        \includegraphics[width=0.3\linewidth]{realPhaseD.png}
        \caption{The upper and lower critical solution temperatures.}
        \label{fig:realPhaseD}
    \end{figure}
    \begin{itemize}
        \item Up until now, we have only considered solutions that undergo mixing at higher temperatures. Systems like this have a \textbf{UCST}.
        \item However, there also exist systems that \emph{demix} at higher temperatures! Such systems have a \textbf{LCST}.
        \item Some systems, such as the idealized one sketched in Figure \ref{fig:realPhaseD}, exhibit both (or more/other!) types of behavior.
    \end{itemize}
    \item \textbf{Upper critical solution temperature}: The critical point on a phase boundary that separates a two-phase region at low temperature from a one-phase region at high temperature. \emph{Also known as} \textbf{UCST}.
    \item \textbf{Lower critical solution temperature}: The critical point on a phase boundary that separates a two-phase region at high temperature from a one-phase region at low temperature. \emph{Also known as} \textbf{LCST}.
    \item Having covered the fundamentals of real phase diagrams, let's now look at some examples of real systems.
    \begin{itemize}
        \item Poly(methyl methacrylate) / styrene-\emph{co}-acrylonitrile demixes at increased temperature.
        \begin{itemize}
            \item This particular system exhibits such behavior because the molecules involved are polar, and thus they preferentially bond to each other provided an excess of thermal energy.
        \end{itemize}
        \item Polystyrene / polyisoprene mixes at higher temperatures.
        \begin{itemize}
            \item This mimicks the systems described by Flory-Huggins theory.
        \end{itemize}
        \item PEG and PMMA have a negative $\chi$ at room temperature. PEG and \ce{H2O} is similar (you heat it up, and the polymer comes out of solution).
        \item A polymer-solvent solution of pNIPAM in \ce{H2O} undergoes a transition around \SIrange{32}{34}{\celsius}.
        \begin{itemize}
            \item Specifically, this is the temperature at which water molecules solubilizing pNIPAM will cease their enthalpically stable hydrogen bonding to the \ce{C=O} and \ce{N-H} moieties on the polymer.
            \item Once the water molecules begin moving around more randomly, they interact more with the hydrophobic \ce{C-C} backbone, which is not enthalpically favorable. In fact, this new interaction is so unfavorable that phase separation occurs!
        \end{itemize}
    \end{itemize}
    \item Key takeaway: Strong attraction gives rise to a low or negative $\chi$, and this commonly leads to deviations from Flory-Huggins theory.
    \begin{figure}[h!]
        \centering
        \begin{subfigure}[b]{0.2\linewidth}
            \centering
            \includegraphics[width=0.9\linewidth]{negChia.JPG}
            \caption{The components.}
            \label{fig:negChia}
        \end{subfigure}
        \begin{subfigure}[b]{0.2\linewidth}
            \centering
            \includegraphics[width=0.7\linewidth]{negChib.JPG}
            \caption{Low $T$ regime.}
            \label{fig:negChib}
        \end{subfigure}
        \begin{subfigure}[b]{0.2\linewidth}
            \centering
            \includegraphics[width=0.75\linewidth]{negChic.JPG}
            \caption{High $T$ regime.}
            \label{fig:negChic}
        \end{subfigure}
        \caption{Systems with negative Flory $\chi$ parameters.}
        \label{fig:negChi}
    \end{figure}
    \begin{itemize}
        \item Example system (Figure \ref{fig:negChia}).
        \begin{itemize}
            \item Consider a model two-component system, where one component is circular and the other component is rectangular. We may consider the components to be either free (like small molecules) or covalently bound together (like polymers); this distinction will not affect the following arugment.
            \item Suppose these two components have the ability to form productive interactions (e.g., a hydrogen bond) at the shaded regions.
        \end{itemize}
        \item At low temperatures (Figure \ref{fig:negChib}), these productive interactions will take hold and drive mixing.
        \begin{itemize}
            \item Productive enthalpic interactions corresponds to a negative $\chi$!
        \end{itemize}
        \item At high temperatures (Figure \ref{fig:negChic}), however, we will rupture these attractive interactions and "like dissolves like" will take hold.
        \begin{itemize}
            \item Notice how the "molecules" are pulling apart.
        \end{itemize}
    \end{itemize}
    \item Let's now discuss the LCST a bit more, especially as it pertains to polymer blends.
    \begin{itemize}
        \item You can arrest a spinodal decomposition by \textbf{up-quenching} (heating the system into the unstable regime) and then --- after a short hold, the time of which you vary on successive experiments --- cooling the system very quickly.
        \begin{itemize}
            \item This process allows you to "trap" the structure of the polymer blend partway through its spinodal decomposition, at a timepoint into the decomposition determined by the length of the hold!
        \end{itemize}
        \item You can then observe the structure of the material using microscopy (e.g., TEM).
        \item Results from one such study.
        \begin{itemize}
            \item \textcite{bib:LCSTblend} found that the spinodal decomposition of a SAN/PMMA blend leads to \textbf{bicontinuous} structures. We will discuss bicontinuous phases more next class!
            \item Allowing the spinodal decomposition to proceed too far will result in coarsening and droplet formation.
            \item Note that this behavior mimicks how metal alloys behave under such temperature changes.
        \end{itemize}
    \end{itemize}
    \item A number of references on polymer blends are included in the slides!
    \item Applications of FH theory.
    \begin{itemize}
        \item A Nature paper published just a few days ago used Flory-Huggins theory to explain how free amino acids (solvent) stabilize proteins (polymers) within cells.
        \item Biocondensates.
        \begin{itemize}
            \item These are also known as membrane-free organelles; many of these have been discovered since scientists expanded their definition of "organelle" beyond the requirement of a region cordoned off by a plasma membrane.
            \item Examples: The nucleolus and centrioles.
            \item These things come together because of FH theory!
        \end{itemize}
    \end{itemize}
    \item Next time.
    \begin{itemize}
        \item Self-assembly.
        \item The PSet 2 might be a bit long, so start early! We should currently be able to do every problem up to 3, and after Thursday, we should be able to do every problem.
    \end{itemize}
\end{itemize}



\section{Phase Behavior, Melting Point Depression, Osmometry, and Microphase Separation}
\begin{itemize}
    \item \marginnote{9/25:}Last time.
    \begin{itemize}
        \item Entities that are not covalently bonded.
    \end{itemize}
    \item Today.
    \begin{itemize}
        \item Entities that \emph{are} connected together.
        \begin{itemize}
            \item You cannot get rigid phase separation here.
            \item Self-assembly is a thing.
        \end{itemize}
    \end{itemize}
    \item Lecture outline.
    \begin{itemize}
        \item Copolymers.
        \begin{itemize}
            \item Microphase separation.
            \item Interfacial free energy.
            \item Chain stretching and configuraitonal free energy.
            \begin{itemize}
                \item This will bring back concepts from Professor Doyle's class.
            \end{itemize}
        \end{itemize}
    \end{itemize}
    \item A bit more on biocondensates (not testable material).
    \begin{figure}[h!]
        \centering
        \includegraphics[width=0.2\linewidth]{bioNetwork.JPG}
        \caption{Simplified graph of a biochemical network.}
        \label{fig:bioNetwork}
    \end{figure}
    \begin{itemize}
        \item Correction: The protein discussed last time goes into solution if you \emph{add} salt.
        \item Principle in biology: At some point, it gets better to have things that do multiple tasks poorly than one task really well.
        \begin{itemize}
            \item This is because it takes energy to produce proteins.
            \item Example: In computer science, engineers used to spend a lot of time to make 1 really nice transistor. But now, they go for a lot of transistors that are almost all the same and you connect them in different ways. Now you can do basically any task, but not all of them are great. In D. E. Shaw, they have a computer that \emph{only} runs molecular dynamics (1000 times faster than Nvidia GPUs), but that's the only thing it does.
            \item So since we need a lot of functions in a cell and we don't want to produce a lot of very specialized proteins, it's better to be a bit more general.
        \end{itemize}
        \item Suppose you have a (biochemical) network, and we control each transition between nodes locally (Figure \ref{fig:bioNetwork}).
        \begin{itemize}
            \item If we want to actually do complex computation with the system, having junctions that act on a number of different nodes is helpful.
        \end{itemize}
        \item Takeaway: Random things and disorder in a cell gives you capabilities beyond perfectly folded structures, like proteins and enzymes.
    \end{itemize}
    \item This concludes content from last time; we now move onto today's content.
    \item Self-assembly of simplified systems (relative to cells).
    \begin{figure}[h!]
        \centering
        \includegraphics[width=0.4\linewidth]{selfAssembly.png}
        \caption{Self-assembly of an ABC triblock copolymer.}
        \label{fig:selfAssembly}
    \end{figure}
    \begin{itemize}
        \item What aspects of an ABC triblock copolymer affect its organization?
        \item There will be \textbf{intra interactions} $\chi_{AB}$, $\chi_{AC}$, and $\chi_{BC}$ between the various components of the chain.
        \item Let the chain have total length $N$, and let each of the three segments have length $N_i$.
        \begin{itemize}
            \item Then we have $N_A$, $N_B$, and $N_C=N-N_A-N_B$.
        \end{itemize}
        \item If we have something like (a) of Figure \ref{fig:selfAssembly}, then we probably have $N_A=N_B=N_C$ (becasue everything is nice and equally ordered) and $\chi_{AC}<\chi_{AB}\approx\chi_{BC}$ (because we have AB and BC interfaces, but not AC interfaces).
        \item Note: These images are not made up; all of them have been seen.
    \end{itemize}
    \item If we can do everything in Figure \ref{fig:selfAssembly} with 3 things, imagine how much we can do with the 20 amino acids!
    \begin{itemize}
        \item Note on the "hydrophobic" amino acids: They have branching (see valine, leucine, isoleucine)! Nature doesn't just use \emph{n}-alkyl chains of different length because the methyl groups sticking off have partial charges of 0.4 (40\% the charge of an electron), which makes them still pretty polar.
        \item Tyrosine can use its phenolic substiuent to \emph{enhance} its $\pi$-cation non-covalent interactions relative to phenylalanine.
    \end{itemize}
    \item Key question: How can a homogeneous state go to a semiordered state, to an even more ordered state?
    \begin{itemize}
        \item Example: Unfolded protein, to good prions, to rogue prions.
        \item Aside: In rogue prions, there is an exposed $\beta$-pleated sheet, which will stack vertically with the $\beta$-pleated sheets of other prions. This stacking is what causes the brain to shut down in Mad Cow Disease.
    \end{itemize}
    \item Goals for self-assembly.
    \begin{itemize}
        \item Understand the key concepts behind the process of self-assembly, in particular for the case of block copolymers.
        \item Construct a simple formalism to determine which variables contribute more relative to other ones.
        \item Recent stuff on how to control self-assembly using external methods.
    \end{itemize}
    \item \textbf{Min-max principle}: Phases are most stable when we (1) minimize interfacial energy and (2) maximize the conformational entropy of the chains.
    \item The min-max principle governs self-assembly.
    \begin{figure}[h!]
        \centering
        \includegraphics[width=0.4\linewidth]{ordering.png}
        \caption{Ordering of a diblock copolymer.}
        \label{fig:ordering}
    \end{figure}
    \begin{itemize}
        \item As we transition from a homogeneous, disordered state to an ordered state, we develop an interface.
        \begin{itemize}
            \item This interface is technically termed the \textbf{IMDS}, or inter-material dividing surface.
        \end{itemize}
        \item In the disordered phase, entropy is maximized\dots but we're paying an enthalpic price because of the contact between groups that don't like each other.
        \item The subscript $c$ in Figure \ref{fig:ordering} means "critical."
        \begin{itemize}
            \item Remember that $\chi N$ controls whether or not we develop microdomains (more on this below). We will investigate this in PSet 2, too.
        \end{itemize}
    \end{itemize}
    \item Principles of self-assembly: Microphase separation in diblock copolymers.
    \begin{figure}[h!]
        \centering
        \begin{subfigure}[b]{0.2\linewidth}
            \centering
            \includegraphics[width=0.7\linewidth]{microphaseSepa.png}
            \caption{$T>T_\text{ODT}$.}
            \label{fig:microphaseSepa}
        \end{subfigure}
        \begin{subfigure}[b]{0.2\linewidth}
            \centering
            \includegraphics[width=0.7\linewidth]{microphaseSepb.png}
            \caption{$T\approx T_\text{ODT}$.}
            \label{fig:microphaseSepb}
        \end{subfigure}
        \begin{subfigure}[b]{0.2\linewidth}
            \centering
            \includegraphics[width=0.7\linewidth]{microphaseSepc.png}
            \caption{$T<T_\text{ODT}$.}
            \label{fig:microphaseSepc}
        \end{subfigure}
        \caption{Microphase separation in diblock copolymers.}
        \label{fig:microphaseSep}
    \end{figure}
    \begin{itemize}
        \item Some domains start to form and you get lamellae in time.
        \item Misconception: Things are not perfectly mixed at one extreme; you start seeing domains earlier. As you go from Figure \ref{fig:microphaseSepa}-\ref{fig:microphaseSepc}, you get into a lamellar state.
    \end{itemize}
    \item Microdomain morphologies: Diblock copolymers.
    \begin{figure}[h!]
        \centering
        \includegraphics[width=0.65\linewidth]{microphaseMorph.png}
        \caption{Microphase morphologies in diblock copolymers.}
        \label{fig:microphaseMorph}
    \end{figure}
    \begin{itemize}
        \item In contrast to Figure \ref{fig:microphaseMorph}, 40-60\% gets you a lamellar state. This accounts for the fact that going on either side of 50\% is equivalent.
        \item On the outskirts of this, you get \textbf{bicontinuous phases}.
        \begin{itemize}
            \item The Double Diamond is heavily sought after in optics.
            \item Double Gyroid is more common.
            \item Difference is tri- vs. tetracoordination at the nodes.
        \end{itemize}
        \item Then cylinders.
        \item And an even smaller amount of green gets you spheres.
    \end{itemize}
    \item \textbf{Bicontinuous} (phases): Two demixed phases such that for any two points in a single phase, there exists a path between them that never crosses a phase boundary.
    \item Where are the above morphologies used?
    \begin{itemize}
        \item Example: Krayton's / green rubbers.
        \begin{itemize}
            \item This is a PS-\emph{block}-PB-\emph{block}-PS polymer, with a big PB domain.
        \end{itemize}
        \item The PS ends either land in another domain, or come back to the same domain.
        \item Good for high-performance applications, like the rubber in an F1 track.
    \end{itemize}
    \item We now investigate microdomain spacing for diblock copolymers.
    \item Variables to be aware of.
    \begin{itemize}
        \item $G$ is the free energy per chain;
        \item $N=N_A+N_B$ is the number of segments per chain.
        \item $a$ is the step size.
        \item $\lambda$ is the domain periodicity. \emph{look up definitions!!}
        \item $\Sigma$ is the interfacial area where the chains actually interact.
        \item $\gamma_\text{AB}$ is the interfacial energy per unit area. It will be computed using Helfand's equation.
        \begin{equation*}
            \gamma_\text{AB} = \frac{\kB T}{a^2}\sqrt{\frac{\chi_\text{AB}}{6}}
        \end{equation*}
        \begin{itemize}
            \item $a$ is the Kuhn length or monomer length, varying depending on the context.
        \end{itemize}
        \item $\chi_\text{AB}$ is the same Flory-Huggins interaction parameter we've been looking at in previous lectures.
    \end{itemize}
    \item Free energies of these diblock copolymers.
    \begin{figure}[h!]
        \centering
        \includegraphics[width=0.35\linewidth]{diblockStretch.JPG}
        \caption{Entropy and enthalpy changes as diblock copolymers are stretched.}
        \label{fig:diblockStretch}
    \end{figure}
    \begin{itemize}
        \item We have
        \begin{align*}
            \Delta G &= (\underbrace{H_2-H_1}_{\Delta H})-T(\underbrace{S_2-S_1}_{\Delta S})\\
            &= \gamma_\text{AB}\underbrace{\sum-N\chi_\text{AB}\phi_\text{A}\phi_\text{B}\kB T}_\text{Enthalpic Term}+\underbrace{\frac{3}{2}\kB T\left[ \frac{(\lambda/2)^2}{Na^2}-1 \right]}_\text{Entropic Term}
        \end{align*}
        \begin{itemize}
            \item The entropic term relates to the springiness of the polymer.
            \item Note that $Na^3=\Sigma\cdot\lambda/2$. This essentially equates (1) the total volume occupied by $N$ monomers each of volume $a^3$ and (2) the volume of the cylinder bounding said monomers, a cylinder having height $\lambda/2$ and base area $\Sigma$.
        \end{itemize}
        \item Important assumption: Chains want to stretch away from the interface.
        \begin{itemize}
            \item Enthalpy goes way down when you have linear strands (everything is near each other). Entropy goes way up here, though, because we're stretching.
            \item In the other regime, though, area is way bigger. This means we get more enthalpy.
            \item $\Delta S\approx\lambda^2/a^2N$.
            \item System will try to find the optimal balance between the two; we want to optimize the length $\lambda$.
        \end{itemize}
        \item To find the optimal $\lambda$, we'll want to find the minimum Gibbs free energy as a function of $\lambda$.
        \begin{equation*}
            \Delta G(\lambda) = \underbrace{\frac{\kB T}{a^2}\sqrt{\frac{\chi_\text{AB}}{6}}}_{\gamma_\text{AB}}\underbrace{\frac{Na^3}{\lambda/2}}_\Sigma-N\chi_\text{AB}\phi_\text{A}\phi_\text{B}\kB T+\frac{3}{2}\kB T\left[ \frac{(\lambda/2)^2}{Na^2}-1 \right]
        \end{equation*}
        \begin{itemize}
            \item By above definitions and equalities, the first term is the interfacial energy per unit area times the area of the interface.
        \end{itemize}
        \item Compressing every non-$\lambda$ variable in the above expression into a constant (termed $\alpha$, $\beta$, $\text{const}_1$, or $\text{const}_2$) reveals that the above equation is of the following general form.
        \begin{equation*}
            \Delta G(\lambda) = \frac{\alpha}{\lambda}-\text{const}_1+\beta\lambda^2-\text{const}_2
        \end{equation*}
        \item Thus, the optimum period of the lamellae repeat unit is
        \begin{align*}
            0 &= \pdv{\Delta G}{\lambda}\\
            &= -\frac{\alpha}{\lambda_\text{opt}^2}+2\beta\lambda_\text{opt}\\
            \lambda_\text{opt} &= \sqrt[3]{\frac{\alpha}{2\beta}} = aN^{2/3}\chi_\text{AB}^{1/6}
        \end{align*}
        \begin{itemize}
            \item The result that $\lambda_\text{opt}$ scales as $N^{2/3}$ is important! It implies that chains in microdomains are stretched compared to the homogeneous melt state (in which scaling is the smaller $N^{1/2}$).
        \end{itemize}
    \end{itemize}
    \item Let's now investigate the order-disorder transition temperature.
    \begin{itemize}
        \item By substituting $\lambda_\text{opt}$ into our expression for $\Delta G(\lambda)$, we obtain the estimate that
        \begin{align*}
            \Delta G(\lambda_\text{opt}) &= \frac{2}{\sqrt{6}}\kB TN\chi_\text{AB}^{1/2}a\lambda_\text{opt}^{-1}-N\chi_\text{AB}\phi_\text{A}\phi_\text{B}\kB T+\frac{3}{8}\kB T\frac{\lambda_\text{opt}^2}{Na^2}-\frac{3}{2}\kB T\\
            &= \left( \frac{2}{\sqrt{6}}+\frac{3}{8} \right)\kB TN^{1/3}\chi_\text{AB}^{1/3}-N\chi_\text{AB}\phi_\text{A}\phi_\text{B}\kB T-\frac{3}{2}\kB T\\
            &\approx 1.2\kB TN^{1/3}\chi_\text{AB}^{1/3}-N\chi_\text{AB}\phi_\text{A}\phi_\text{B}\kB T-\frac{3}{2}\kB T\\
            &\approx 1.2\kB TN^{1/3}\chi_\text{AB}^{1/3}-N\chi_\text{AB}\phi_\text{A}\phi_\text{B}\kB T
        \end{align*}
        \item Since the first two terms are both much greater than the thrid term, we neglect it.
        \begin{itemize}
            \item Thus, the sign of $\Delta G$ will depend on which of the two remaining terms is bigger.
        \end{itemize}
        \item Let's analyze the case of a 50/50 volume fraction of components A and B. Specifically, we want to know what the critical $N\chi$ value is above which $\Delta G=-$ and we form lamellar microdimains, and below which $\Delta G=+$ and we stay in a homogenous melt.
        \begin{itemize}
            \item In a 50/50 split, $\phi_\text{A}=\phi_\text{B}=1/2$. Thus,
            \begin{equation*}
                \phi_\text{A}\phi_\text{B} = \frac{1}{4}
            \end{equation*}
            \item It follows that the critical $N\chi$ value $(N\chi)_c$ is
            \begin{align*}
                \frac{(N\chi)_c}{4} &= 1.2(N\chi)_c^{1/3}\\
                (N\chi)_c^{2/3} &= 4.8\\
                (N\chi)_c &\approx 10.5
            \end{align*}
            \item Therefore, if $N\chi<10.5$, we'll get a homogeneus, mixed melt; and if $N\chi>10.5$, we'll get demixing into lamellar microdomains.
        \end{itemize}
    \end{itemize}
    \item Phase transitions between different microstructures can also be represented on a $\chi N$ vs. composition phase diagram.
    \begin{figure}[h!]
        \centering
        \includegraphics[width=0.3\linewidth]{microphaseDia.png}
        \caption{Microphase transition diagram.}
        \label{fig:microphaseDia}
    \end{figure}
    \begin{itemize}
        \item Notice as we go in from the outside (homogeneous, disordered mixed melt), we progressively go through spheres (S), cylinders (C), gyroids (G), and lamellae (L).
    \end{itemize}
    \item Other interfaces: Polymer brushes.
    \begin{itemize}
        \item Consider a series of polymer strands grown off of a 2D surface.
        \begin{itemize}
            \item Let each polymer strand be a distance $D$ away from the next nearest strand. In this sense, each polymer strand can be thought to inhabit a volume of diamater $D$ and height $H$ away from the surface.
        \end{itemize}
        \item The polymer strands stretch out more ($H$ increases) when they don't want to interact with the 2D interface.
        \item What is the energy or enthalpy?
        \begin{equation*}
            \Delta H \approx vc^2\cdot HD^2
        \end{equation*}
        \begin{itemize}
            \item $H,D$ are defined as above.
            \item $v$ is how many two body interactions tere are (counted by mole).
            \item $c^2$ describes how dense the system is.
        \end{itemize}
        \item Here, $\Delta S\approx H^2/a^2N$ (as opposed to $\lambda^2/a^2N$ from earlier).
        \item We want to minimize $H^2$ and $1/H$ on an $H$ vs. Gibbs free energy graph.
    \end{itemize}
\end{itemize}



\section{Office Hours (Alexander-Katz)}
\begin{itemize}
    \item \marginnote{9/30:}PSet 2, Q5.
    \begin{figure}[h!]
        \centering
        \includegraphics[width=0.5\linewidth]{PSet2Q5setup.JPG}
        \caption{PSet 2, Q5 setup.}
        \label{fig:PSet2Q5setup}
    \end{figure}
    \begin{itemize}
        \item Let state 1 be the free chain floating in solution. State 2 is then the brushes stuck/adsorbed to the surface.
        \item If they're stuck at a particular grafting density $D$, you first have to calculate the free energy of a chain being that close to another.
        \item Bringing in Prof. Doyle's lectures,
        \begin{equation*}
            \Delta G = \frac{H^2}{Na^2}+V_\text{ex}\rho^2\cdot D^2H-\varepsilon
        \end{equation*}
        \begin{itemize}
            \item $D^2H$ is a volume.
            \item $\rho=N/D^2H$ is the number density of monomers.
            \item So then we ned the optimal $H$, which we will term $H^*$. $H^*$ minimizes $\Delta G(H^*)$, i.e., $\ev{\pdv*{\Delta G}{H}}_{H=H^*}=0$.
            \item Then we back feed $H^*$ into the $\Delta G$ equation above.
        \end{itemize}
        \item To find $D$, we have to evaluate the stability relative to state 1.
        \begin{itemize}
            \item $\mu_\text{bulk}$ or $\mu_1$ ($\mu_1$ seems to be the notation Alexander-Katz used most consistently) can be the chemical potential / energy of state 1. We will have to define this variable.
            \item We consider this because we need the energy to take a chain out of state 1 (and into state 2), but changes in state relative to concentration is chemical potential.
        \end{itemize}
        \item Then the last thing is
        \begin{equation*}
            \Delta G_2-\Delta G_1 = \frac{(H^*)^2}{Na^2}+\frac{vN^2}{D^2H^*}-\varepsilon-\mu_\text{bulk}
        \end{equation*}
        \begin{itemize}
            \item And this $\Delta\Delta G$ must be negative if grafting is going to happen.
        \end{itemize}
        \item $a$ can be a Kuhn length or segment length or whatever; we're forgetting about all of the chemistry once we get into thermodynamic models.
        \item We need to figure out the per strand energy, hence why we're concerned with $\varepsilon$ and not $N\varepsilon$ (which would be for the whole system).
        \item You need to account for the entropy of stretching, the enthalpy of sticking, and excluded volume.
        \begin{itemize}
            \item $vN^2/HD^2$.
            \item Stretching term is $H^2/Na^2$. Excluded volume is $v\rho^2\cdot D^2H$.
            \item Balance the two terms $H^2/Na^2$ and $vN^2/D^2H$ to find $H^*$.
            \item \emph{Then} factor in $\varepsilon$.
        \end{itemize}
    \end{itemize}
    \item Phase diagrams.
    \begin{itemize}
        \item $\chi N$ or $\chi$ on the $y$-axis.
        \begin{itemize}
            \item This is approximately $1/T$, so going up is low $T$.
        \end{itemize}
        \item Inner dashed line is \textbf{spinodal} or \textbf{stability} curve.
        \begin{itemize}
            \item The inflection points on the pink curve in Figure \ref{fig:mixingChic} defines the spinodal curve.
            \item Flat bottomed curve defines critical point!
        \end{itemize}
        \item Outer solid line is \textbf{binodal} or \textbf{equilibrium} or \textbf{coexistence} curve.
        \item Lowest point where the two curves touch is the \textbf{critical} point.
        \item Real phase separation is at the binodal line.
        \begin{itemize}
            \item If we wait a long time.
        \end{itemize}
        \item For a symmetric system, we get flat bottomed tangents.
        \item Why does the spinodal matter? Won't we just get to the lowest energies?
        \begin{itemize}
            \item Free energy is the sum of the two states. Downhill from the middle to the spinodal points.
            \item Splitting our system in two past the spinodal is not downhill. The way we go beyond this is \textbf{nucleation}. You will end up at a higher energy until the nucleating thing is big enough.
        \end{itemize}
        \item Nucleation between the curves, spinodal decomposition (to the spinodals) in the middle.
        \item Example of utility: QDot synthesis.
        \begin{itemize}
            \item They first jump into the spinodal zone that will phase separate, creating a bunch of little dots.
            \item Then we quickly move the solution (once we have similarly sized nuclei) into the intercurve region. Here we no longer have spontaneous nucleation, but we have growth. And everything will grow at the same rate.
            \item People who do nanoparticles play this game all the time. Heterogeneous nucleation: Dumping some trash in to nucleate other stuff, e.g., gold nanoparticles to nucleate silicon-oil phase separation.
        \end{itemize}
    \end{itemize}
    \item PSet 2, Q3b: When we need to determine $A$? Finding $\chi_c$ based on other problems, but then how do we get $A$?
    \begin{itemize}
        \item Critical temperature in the thousands of kelvins? Should be $A=45$ and $\chi_c\approx\SI{8000}{\kelvin}$.
        \item We're not accounting for the entropic term in $\chi$, so we get an unreasonable value.
    \end{itemize}
    \item $\Sigma\gamma_\text{AB}$ is the interfacial energy, which is kind of like the surface tension.
    \item Helfand parameter.
    \begin{itemize}
        \item For a particular $\chi$, what will be the expected loop crossing the interface into the region of the other polymer.
    \end{itemize}
    \item PSet 2, Q4 for the star polymer. Is the setup a B cylinder surrounded by AC mixed region?
    \begin{itemize}
        \item Yes, something like that.
    \end{itemize}
    \item Aside: It is really hard to get away from the phases in Figure \ref{fig:selfAssembly} and get into new ones.
    \begin{itemize}
        \item Stadler's goldmine.
    \end{itemize}
    \item nVT vs. nPT constant regimes.
    \begin{itemize}
        \item Lowest common energy is the \emph{tangent} line, not the line connecting the two minima.
        \item For the calculation for a real phase diagram, the math gets dicey because we have to deal with tangent lines.
        \item Implication: The combination of the two unequal wells will give you the total lowest energy, rather than the lowest and the lowest. Moreover, we need to not be at the bottom of the wells, but slightly to the side so we get the tangent line rather than the line connecting the minima.
    \end{itemize}
\end{itemize}



\section{Review for Quiz 1}
\begin{itemize}
    \item Quiz 1 details.
    \begin{itemize}
        \item Starts at 3:05 PM sharp.
        \item 80 minutes to take it.
        \begin{itemize}
            \item They're not here to fail us; they will put things appropriate for us to do in 80 minutes.
        \end{itemize}
        \item Open book (bring Kwangwook's textbook), notes (print out notes!!), cheat sheets ("you'll probably want one of these"), but no electronic except for the occasional use of a calculator (check TI-84 batteries!! Phone calculator allowed, though).
        \item Undergrads need to answer 3/4 questions; grad students must answer all of them.
        \item Quiz is based on the HW, so make sure you know how to do it!
    \end{itemize}
    \item Announcements.
    \begin{itemize}
        \item 2nd HW solutions will be posted tomorrow.
        \item A pre-test has been posted.
    \end{itemize}
    \item Review starts now.
    \item Topic 1 (ideal chains) topics to know.
    \begin{itemize}
        \item General characteristics of polymers: Degree of polymerization $N$, tacticity, bond length, etc.
        \item Properties of polymer systems: $\Mw$, $\Mn$, architecture.
        \item Different models for polymers: Freely jointed chain (FJC), freely rotating chain (FRC), symmetric hindered rotations (and rotational isomeric states).
        \begin{itemize}
            \item Things that come out of these models: End-to-end distances $\mathbf{R}$ (with $\prb{R^2}=C_\infty Nl^2$), characteristic ratio $C_\infty$.
        \end{itemize}
        \item An important model for polymers: Kuhn's equivalent chain (with its effective bond length $l_k$ and degree of polymerization $N_k$).
        \begin{itemize}
            \item This is the only model we've used since we introduced it, as we've increasingly neglected the \emph{chemistry} involved to focus on the thermodynamics.
        \end{itemize}
        \item The end-to-end distribution is a Gaussian.
        \begin{itemize}
            \item Since the probability $p\propto\e[-\Delta G/\kB T]$ and $\Delta G=F=3\kB TR^2/2Nb^2$, we get $\Delta S/\kB T\propto R^2/Nb^2$.
            \item Spring constant in our Hookean spring is $3\kB T/Nb^2$.
        \end{itemize}
        \item Crossover concentrations (Figure \ref{fig:concentrationRegimes}).
        \begin{itemize}
            \item The chain inhabits a sphere of volume $R^3$, and the volume per chain is $V_c\propto R_e^3$.
            \item The number density in the bead is $c=N/R_e^3$. When $c$ is greater than the critical concentration $N/R_e^3$, we start to get overlap.
        \end{itemize}
    \end{itemize}
    \item Topic 2 (real chains) things to know.
    \begin{itemize}
        \item Excluded volume $v$ comes from a hard core and the attractive part of the potential.
        \item The Mayer-$f$ function is
        \begin{equation*}
            f = \e[-U/\kB T]-1
        \end{equation*}
        \begin{itemize}
            \item The excluded volume is the integral of $f$ over all space.
            \item Intuition for why the integral is excluded volume?
            \begin{itemize}
                \item The Mayer-$f$ function gives the probability of finding two things close to each other.
                \item It essentially tells us how this probability relates to that of an ideal gas where there is no interaction.
                \item The attractive component raises the probability that things are close together.
                \item The entropic component compensates for this, and doing the integral takes both into account. It measures the effective volume a particle feels like it has when in a sea of other particles.
                \item If the interaction is only repulsive, a given particle will feel the volume of the box minus the volume of other monomers.
                \item If an attractive component exists, you can eventually balance out repulsion. It's not that the volume has been reduced; it's just that negative excluded volumes mean you don't care about repulsion, you just want to be closer together. So negative excluded volumes effectively reduce the volume the polymer wants to occupy.
            \end{itemize}
        \end{itemize}
        \item The $\theta$ condition for a chain is where excluded volume is zero.
        \item Solvent quality.
        \begin{itemize}
            \item A polymer in a vacuum attracts to itself because of van der Waals forces; it doesn't just float around.
        \end{itemize}
        \item Interaction energies.
        \begin{equation*}
            \frac{F_\text{int}}{V\kB T} = \frac{1}{2}vc^2+\frac{1}{3!}\omega c^3
        \end{equation*}
        \begin{itemize}
            \item We often called these enthalpic energies.
            \item $F_\text{int}/V$ is an energy density, and dividing by $\kB T$ takes away the units from the righthand side.
            \item When $v<0$, we need the next term to the right.
            \item Volume of the chain, Flory's free energy for a single chain:
            \begin{equation*}
                F = \underbrace{\frac{R^2}{Nb^2}}_\text{expansion term}+\underbrace{\frac{Nb^2}{R^2}}_\text{compression term}+\frac{1}{2}v\frac{N^2}{R^3}+\frac{1}{3!}\omega\frac{N^3}{R^6}
            \end{equation*}
            \begin{itemize}
                \item There is a handwavey and a formal way to come up with the compression term.
                \item First two terms are the "entropic terms," in contrast to the latter two "enthalpic terms."
            \end{itemize}
        \end{itemize}
        \item Scaling of polymer chains.
        \begin{itemize}
            \item The size $R$ will be proportional to $b$ (the segment length, commonly Kuhn length) times $N^\nu$.
            \item In a $\theta$ solvent, $\nu=1/2$. For a good solvent, $\nu=3/5$. For a bad solvent, $\nu=1/3$.
            \item We don't use all terms when we're minimizing things; we use the ones that matter. If the chain is going to expand, we get rid of the compression term. If one of them is negative, we use the last term. Know when to use which terms!!
        \end{itemize}
        \item Dimensionality matters.
        \begin{itemize}
            \item $\nu_\text{good}=1,2,3,4$ for $\nu=1,3/4,3/5,1/2$.
        \end{itemize}
    \end{itemize}
    \item Topic 3 (Flory-Huggins) things to know.
    \begin{itemize}
        \item Blends and solutions of polymers and (sometimes) solvents.
        \item We're interested in the free energy of mixing $\Delta G_M=\Delta H_M-T\Delta S_M$.
        \begin{itemize}
            \item An entropic term and enthalpic term come together to make
            \begin{equation*}
                \frac{\Delta G_M}{\kB TX_0} = \frac{\phi_1}{N_1}\ln\phi_1+\frac{\phi_2}{N_2}\ln\phi_2+\chi\phi_1\phi_2
            \end{equation*}
            \item The Flory $\chi$ parameter quantifies interactions.
            \begin{equation*}
                \chi = \frac{z}{\kB T}\left[ \varepsilon_{12}-\frac{1}{2}(\varepsilon_{11}+\varepsilon_{22}) \right]
            \end{equation*}
            \begin{itemize}
                \item Take the coordination number, divide by your units of energy, and then compare the interphase attractive energy to the average of the two intraphase attraction energies.
                \item $\chi$ scales as $A/T$, for some number $A$. To make the real world fit the system, we often add another term $B$ as the "entropic contribution." Symbolically,
                \begin{equation*}
                    \chi = \frac{A}{T}+B
                \end{equation*}
                \item The above equation allows us to interconvert between $\chi$ and $T$, which we often do in phase diagrams, etc.
            \end{itemize}
        \end{itemize}
        \item Phase diagrams.
        \begin{itemize}
            \item 3 systems of interest.
            \begin{itemize}
                \item Polymer-polymer systems.
                \item Monomer-monomer systems.
                \item Polymer-monomer systems.
            \end{itemize}
            \item For polymer-polymer and polymer-monomer systems, $\chi N$ really controls things.
            \item Outside the coexistence / equilibrium / binodal curve, you mix into 1 phase.
            \begin{itemize}
                \item Inside both, you demix to the spinodal. Then with nucleation over a long period of time, you demix to the binodal.
            \end{itemize}
            \item Solubility parameters: Interconverting from $\chi$ to $\delta$, as in PSet 2.
            \item UCST and LCST transitions. Heating it up goes into an LCST regime, and cooling it down goes into a UCST regime.
        \end{itemize}
    \end{itemize}
    \item Topic 4 (self-assembly).
    \begin{itemize}
        \item Be familiar with the min-max principle of (1) minimizing energy and (2) maximizing entropy.
        \item Phases of diblock copolymers.
        \begin{itemize}
            \item Spheres (BCC), cylinders (hexagonal; $20\%-35\%$), bicontinuous ($35\%\pm 2\%$; gyroid, double diamond, and perforated lamellae), lamellae (40-60\%).
        \end{itemize}
        \item Aside: There is no excluded volume in self-assembly.
        \item Be familiar with diagrams like the binodal/spinodal one but with many nested curves, each one for a different phase transition (Figure \ref{fig:microphaseDia}).
        \begin{itemize}
            \item Not binodal and spinodal but actual different phases where you have spheres, cylinders, gyroids, lamellae. Happens theoretically around $(\chi N)_c=10.5$; experimentally around 20.
            \item $f$ on the $x$-axis; the fraction of one thing relative to the other, which accounts for polymer structure instead of just $\phi$ for monomers.
        \end{itemize}
    \end{itemize}
    \item Should we know the polymer names and monomer repeat units?
    \begin{itemize}
        \item Print this!!
    \end{itemize}
    \item For the scaling of polymer chains, is this the same as the radius of gyration?
    \begin{itemize}
        \item Slight difference in prefactor, but for linear chains, it's very close.
        \item There's a $\sqrt{6}$ term in the radius of gyration.
    \end{itemize}
\end{itemize}



\section{Office Hours (Alexander-Katz)}
\begin{itemize}
    \item \marginnote{10/7:}With the pNIPAM example, wouldn't increasing entropy of mixing at higher temperatures cause everything to mix?
    \begin{itemize}
        \item Phase separation will occur in systems with negative $\chi$ as temperature increases \emph{regardless} of the fact that the entropic mixing energy becomes more extreme.
        \item That being said, at sufficiently high temperatures, any system will remix due to entropic considerations (but the system may decompose before we are able to reach such a high temperature).
        \item This remixing means that binodal lines are \emph{always} "eye-shaped" if we extend the vertical axis to high enough temperatures.
    \end{itemize}
    \item What are some good resources to read about phase behavior in block copolymers, attachment to surfaces, interfacial energy, and the other topics from Lecture 2.3 that are not covered in \textcite{bib:HiemenzLodge}?
    \begin{itemize}
        \item \textcite{bib:RubinsteinColby} has some good information on brushes.
        \item \textcite{bib:Strobl} has some good information on block copolymers.
        \item \textcite{bib:PhysicsToday} --- which was also referenced in the slides, though not cited in full --- has some good overview of the concepts as well.
    \end{itemize}
\end{itemize}



\section{Chapter 7: Thermodynamics of Polymer Mixtures}
\emph{From \textcite{bib:HiemenzLodge}.}
\begin{itemize}
    \item \marginnote{9/30:}Goals for this chapter.
    \begin{itemize}
        \item Thermodynamically analyze a solution of a polymer in a low molecular weight solvent.
        \item Determine the phase equilibria relevant to this situation.
    \end{itemize}
    \item \textbf{Polymer blend}: A mixture of two polymers.
    \item \textbf{Pure} (thermodynamics): The purely phenomenological study of observable thermodynamic quantities and the relationships among them.
    \item \textbf{Statistical} (thermodynamics): The atomistic model justifying purely thermodynamic observations.
    \begin{itemize}
        \item "\emph{Doing} thermodynamics does not even require knowledge that molecules exist\dots whereas \emph{understanding} thermodynamics benefits considerably from the molecular point of view" \parencite[271]{bib:HiemenzLodge}.
    \end{itemize}
    \item In this chapter, we are concerned with the state of a two-component system at equilibrium. The Gibbs free energy relates to this equilibrium, and in this case, it is given by
    \begin{align*}
        \dd{G} &= \left( \pdv{G}{P} \right)_{T,n_1,n_2}\dd{P}+\left( \pdv{G}{T} \right)_{P,n_1,n_2}\dd{T}+\left( \pdv{G}{n_1} \right)_{P,T,n_2}\dd{n_1}+\left( \pdv{G}{n_2} \right)_{P,T,n_1}\dd{n_2}\\
        &= V\dd{P}-S\dd{T}+\sum_{i=1}^2\mu_i\dd{n_i}
    \end{align*}
    \item \textbf{Partial molar} (quantity $Y$ of component $i$): The amount of $Y$ contributed to the whole by each mole of component $i$ in a mixture. \emph{Denoted by} $\bm{\bar{Y}_i}$. \emph{Units} $\textbf{mol}^{\bm{-1}}$. \emph{Given by}
    \begin{equation*}
        \bar{Y}_i := \left( \pdv{Y}{n_i} \right)_{P,T,n_{j\neq i}}
    \end{equation*}
    \begin{itemize}
        \item Example: The chemical potential of component $i$ is the amount of Gibbs free energy contributed to the total Gibbs free energy $G$ by each mole of $i$.
        \item There exist a partial molar volume, enthalpy, and entropy.
        \item The value of partial molar quantities depends on the overall composition of the mixture.
        \begin{itemize}
            \item Example: $\bar{V}_{\ce{H2O}}$ is not the same for a water-alcohol mixture that is 10\% water as for one that is 90\% water.
        \end{itemize}
        \item For a pure substance, partial molar quantities are equal to \textbf{molar values}.
        \begin{itemize}
            \item Example: $\mu_i=\hat{G}_i$.
        \end{itemize}
        \item Properties of a mixture are linear combinations of mole-weighted contributions of the partial molar properties of the components.
        \begin{equation*}
            Y_\text{m} = \sum_in_i\bar{Y}_i
        \end{equation*}
        \item The value of $Y_\text{m}$ on a per mole basis is given by \textbf{mole fractions} as follows.
        \begin{equation*}
            \frac{Y_\text{m}}{\sum_in_i} = \sum_ix_i\bar{Y}_i
        \end{equation*}
        \item Partial molar quantities exhibit the same relations as ordinary thermodynamic variables.
        \begin{itemize}
            \item Examples:
            \begin{align*}
                \mu_i &= \bar{H}_i-T\bar{S}_i&
                \bar{V}_i &= \left( \pdv{\mu_i}{P} \right)_{T,n_{j\neq i}}
            \end{align*}
        \end{itemize}
    \end{itemize}
    \item \textbf{Molar} (quantity $Y$ of substance $i$): The amount of $Y$ contributed by each mole of a substance $i$ when pure. \emph{Dentoed by} $\bm{\hat{Y}_i}$. \emph{Units} $\textbf{mol}^{\bm{-1}}$.
    \item \textbf{Mole fraction} (of component $i$): The fraction of moles of component $i$ relative to the total number of moles in the mixture. \emph{Denoted by} $\bm{x_i}$. \emph{Given by}
    \begin{equation*}
        x_i := \frac{n_i}{\sum_in_i}
    \end{equation*}
    \item \textbf{Standard state} (value of $Y_i$): The value of $Y_i$ when the substance $i$ is pure. \emph{Denoted by} $\bm{Y_i^\circ}$.
    \item \textbf{Activity}: A thermodynamic concentration and measure of the nonideality of solutions. \emph{Denoted by} $\bm{a_i}$. \emph{Given by}
    \begin{equation*}
        \mu_i = \mu_i^\circ+RT\ln a_i
    \end{equation*}
    \item Notation.
    \begin{itemize}
        \item We've established $n_i$ as the number of \emph{moles} of component $i$.
        \item Let $m_i$ denote the number of \emph{molecules} of component $i$. Thus,
        \begin{equation*}
            m_i = N_\text{A}n_i
        \end{equation*}
        where $N_\text{A}=\num{6.022e23}$ is \textbf{Avogadro's number}.
        \begin{itemize}
            \item This is also equal to $X_i$ from class!
        \end{itemize}
    \end{itemize}
    \item \textbf{Coordination number}: The number of nearest neighbors that surround a central lattice point. \emph{Denoted by} $\bm{z}$.
    \begin{itemize}
        \item Example: A cell in a 2D square lattice has $z=4$.
    \end{itemize}
    \item Regular solution theory: A simple statistical model that provides a useful expression for the free energy of mixing for a binary solution of two components.
    \begin{itemize}
        \item Assume that the two molecules in the mixture have equal volumes.
        \item Assume that the two components have equal (and concentration-independent) partial molar volumes, i.e., $\bar{V}_1=\bar{V}_2$.
        \item Imagine each molecule occupying a cell in a lattice with volume equal to the molecular volume.
        \item Let the lattice have coordination number $z$.
    \end{itemize}
    \item \marginnote{10/2:}\textcite[275]{bib:HiemenzLodge} presents the entropy of mixing in the four useful forms: Original, with $R$ and number of moles, per site, and per mole of sites.
    \begin{itemize}
        \item This really helps clarify some of the rearrangements in PSet 2, Q1a.
    \end{itemize}
    \item Comments on the entropy of mixing.
    \begin{itemize}
        \item $\phi_i<1$ always, so $\ln\phi_i<0$ always. This implies that $\Delta S_M$ is always positive, so configurational entropy always favors spontaneous mixing!
        \item The expression derived is symmetric with respect to exchange of 1 and 2. Such symmetry comes from the assumption that both mixing molecules are the same size, and is difficult to satisfy in real situations.
        \item We are assuming that all configurations are equally probable (i.e., that 1 is equally likely to be next to 1 as it is to 2 and vice versa). But if there was an energetic preference, each configuration would need to be weighted by the appropriate Boltzmann factor.
    \end{itemize}
    \item \textcite[276]{bib:HiemenzLodge} take the perspective that each $\varepsilon_{ij}$ is negative, rather than being the absolute value of the depth of the well in Figure \ref{fig:Uwell}.
    \begin{itemize}
        \item It is unclear which perspective we take in class. However, it does make sense for these terms to be negative, as every type of molecule attracts at least a bit to every other type due to London dispersion forces at minimum.
    \end{itemize}
    \item \textbf{Exchange energy}: The difference between the attractive cross-interaction of 1 and 2 and the average self-interaction of 1 with 1 and 2 with 2. \emph{Denoted by} $\bm{\Delta\varepsilon}$. \emph{Given by}
    \begin{equation*}
        \Delta\varepsilon := \varepsilon_{12}-\frac{1}{2}(\varepsilon_{11}+\varepsilon_{22})
    \end{equation*}
    \begin{itemize}
        \item When only London dispersion forces are present, we preferentially have $\Delta\varepsilon\geq 0$ because "like dissolves like."
    \end{itemize}
    \item On the Flory $\chi$ parameter.
    \begin{itemize}
        \item This parameter is equal to the exchange energy per molecule, normalized by the thermal energy $\kB T$.
        \item The Flory $\chi$ parameter is the energy penalty you must pay (as a fraction of $\kB T$) in order to lift one molecule of type 1 out of a beaker of pure 1, one molecule of type 2 out of a beaker of pure 2, and exchange them. See \textcite[277]{bib:HiemenzLodge}. This is because when we do this in a typical square lattice, four things happen.
        \begin{enumerate}
            \item First, removing a moleule of 1 from a lattice of pure 1 disrupts its interaction with each of its $z=4$ nearest neighbors. But as established earlier, each nearest neighbor interaction only contributes $\varepsilon_{11}/2$ to the total energy/enthalpy of the pure state because we `see' the interaction from the perspective of each molecule involved. Thus, the energy penalty of this removal is $4\cdot\varepsilon_{11}/2=2\varepsilon_{11}$.
            \item Removing a molecule of 2 is analogous, and costs $2\varepsilon_{22}$.
            \item Adding a molecule of 2 to the now-empty space in a beaker of pure 1 \emph{rewards} you $2\varepsilon_{12}$.
            \item Analogously, adding a molecule of 1 to the now-empty space in a beaker of pure 2 rewards you $2\varepsilon_{12}$.
        \end{enumerate}
        \item Adding up all of these contributions gives $\chi\kB T$.
        \item Then finally dividing by $\kB T$ makes $\chi$ unitless, by expressing the energy summed above as a numerical fraction of a known reference energy (namely, $\kB T$).
    \end{itemize}
    \item \textbf{Mean field} (theory): A theory that assumes that the local interactions are determined solely by the bulk average composition.
    \item On volume fractions.
    \begin{itemize}
        \item For solvent-solvent systems, volume fractions are equal to mole fractions.
        \item For polymer-solvent and polymer-polymer systems, volume fractions are easier to use because calculating them only requires measuring the mass of each component and comparing it against known densities. Measuring mole fractions, on the other hand, requires precise knowledge of the full molecular weight distribution. Therefore, although mole fractions and volume fractions are mathematically equivalent, expressing our results in terms of volume fractions provides an empirically easier to use equation.
    \end{itemize}
    \item \textcite[279-82]{bib:HiemenzLodge} derives the $1/N$ prefactors in the $\Delta S_M$ equations more rigorously, as alluded to in class.
    \begin{itemize}
        \item $\Delta H_M$ would also change a bit in the polymer cases, but this complication is ignored.
    \end{itemize}
    \item Note that because monomers are not equally dispersed in dilute solutions (see Figure \ref{fig:concentrationRegimes}, left), the mean field approximation is not as good here.
    \begin{itemize}
        \item It follows that the mean-field approximation (and hence Flory-Huggins theory) should get better as $c\to c^*$. This turns out to be the case!
    \end{itemize}
    \item A summary of the assumptions used in Flory-Huggins theory \parencite[283]{bib:HiemenzLodge}.
    \begin{enumerate}
        \item There is no volume change on mixing, and $\bar{V}_i=\hat{V}_i$ is independent of concentration.
        \item $\Delta S_M$ arises entirely from the ideal combinatorial entropy of mixing.
        \item $\Delta H_M$ arises entirely from the internal energy of mixing.
        \item Both $\Delta S_M$ and $\Delta H_M$ are computed assuming random mixing.
        \item The interactions are short-ranged (nearest neighbors only), isotropic, and pairwise additive.
        \item The local concentration is always given by the bulk average composition (the mean-field assumption).
    \end{enumerate}
    \item \textcite[283-89]{bib:HiemenzLodge} covers some good osmotic pressure content that may be useful someday. Reviewing activity, activity coefficient, osmotic pressure, and Virial expansion info from Thermo will be necessary before delving into this.
    \item \textbf{Phase diagram}: A mapping of the temperature-composition plane of a solution at fixed pressure, divided into regions wherein we find different phases.
    \item \textcite[291]{bib:HiemenzLodge} view a phase diagram in terms of temperature, not $\chi$.
    \item The three features of a phase diagram (in terms of $\chi$).
    \begin{enumerate}
        \item A \textbf{critical point} below which a one-phase solution is formed in all compositions.
        \item A \textbf{binodal} describing the composition of the two phases that coexist at equilibrium, after liquid-liquid separation at some fixed $\chi>\chi_c$. Any solution prepared such that $(T,\phi_1)$ lies above the binodal will be out of equilibrium until it has undergone phase separation into two phases with compositions along the binodal.
        \item A \textbf{spinodal} dividing the two-phase region into a \textbf{metastable} window between the binodal and spinodal, and an \textbf{unstable} region above the spinodal.
    \end{enumerate}
    \item An additional consequence of Figure \ref{fig:mixingChic} is that even on the pink curve, \emph{some} mixing is still favorable. Thus, $\Delta G_M<0$ here, too!
    \item \marginnote{10/6:}Postulate: "Phase separation will occur whenever the system can lower its total free energy by dividing into two phases" \parencite[293]{bib:HiemenzLodge}.
    \item Let's discuss when exactly phase separation will or will not occur.
    \begin{figure}[h!]
        \centering
        \begin{tikzpicture}[xscale=5,yscale=20]
            \small
            \draw [-latex] (-0.1,0) -- (1.1,0) node[right]{$x_1$};
            \draw [-latex] (0,-0.15) -- (0,0.05) node[above]{$\dfrac{\Delta G_M}{X_0\kB T}$};

            \footnotesize
            \draw (0.35,-0.005) -- ++(0,0.01) node[above]{$x_1^\alpha$};
            \draw (0.48,-0.005) -- ++(0,0.01) node[above]{$\prb{x_0}$};
            \draw (0.75,-0.005) -- ++(0,0.01) node[above]{$x_1^\beta$};

            \draw [rex,thick,/pgf/fpu,/pgf/fpu/output format=fixed] plot [domain=0.0001:0.99999,samples=100,smooth] (\x,{2*\x*(1-\x)+\x*ln(\x)+(1-\x)/2*ln(1-\x)});

            \node [circle,fill=rex,inner sep=0.15em,label={left:$\Delta G_M(\prb{x_1})$}] at (0.48,-0.0231) {};
            \node (a) [circle,fill=rex,inner sep=0.15em,label={left:$\Delta G_M(x_1^\alpha)$}] at (0.35,-0.0524) {};
            \node [circle,fill=rex,inner sep=0.15em,label={[fill=white,inner sep=1pt,below=3pt]below:$\Delta G_M(x_1^\beta)$}] at (0.75,-0.014) {}
                edge [semithick,rex,densely dashed] (a)
            ;
            \node [circle,draw=rex,inner sep=0.15em,label={[xshift=3mm]below:$\Delta G_M^{\alpha\beta}(\prb{x_1})$}] at (0.48,-0.0399) {};
        \end{tikzpicture}
        \caption{Tie lines predict the favorability of demixing.}
        \label{fig:tieLine}
    \end{figure}
    \begin{itemize}
        \item Suppose we have a two-component solution with overall composition
        \begin{equation*}
            \prb{x_1} = \frac{n_1}{n_1+n_2}
        \end{equation*}
        where $n_i$ is the total number of moles of component $i$ present in the solution.
        \begin{itemize}
            \item We will justify the notation "$\prb{x_1}$" shortly.
        \end{itemize}
        \item Let $x_1^\alpha$ and $x_1^\beta$ two compositions such that $x_1^\alpha\leq\prb{x_1}\leq x_1^\beta$. We want to determine if it will be energetically favorable for our solution to divide into two phases $\alpha$ and $\beta$ with respective compositions $x_1^\alpha$ and $x_1^\beta$.
        \item First, we would like to determine --- based on our choice of $x_1^\alpha$ and $x_1^\beta$ --- how many moles will be in each phase.
        \begin{itemize}
            \item Let $n_i^j$ be the number of moles of component $i$ in phase $j$. Then by definition, we may write the following expressions.
            \begin{align*}
                x_1^\alpha &= \frac{n_1^\alpha}{n_1^\alpha+n_2^\alpha}&
                x_1^\beta &= \frac{n_1^\beta}{n_1^\beta+n_2^\beta}&
                n_1^\alpha+n_1^\beta &= n_1
            \end{align*}
            \item Now is also a good time to mention that --- as the notation suggests --- we have 
            \begin{equation*}
                \prb{x_1} = x^\alpha x_1^\alpha+x^\beta x_1^\beta
            \end{equation*}
            where $x^j$ is the mole fraction of the total solution in phase $j$.
            \begin{itemize}
                \item Equality holds between this "intuitive" defintion of $\prb{x_1}$ and the original definition since
                \begin{equation*}
                    \underbrace{\frac{n_1^\alpha+n_2^\alpha}{n_1+n_2}}_{x^\alpha}\cdot\underbrace{\frac{n_1^\alpha}{n_1^\alpha+n_2^\alpha}}_{x_1^\alpha}+\underbrace{\frac{n_1^\beta+n_2^\beta}{n_1+n_2}}_{x^\beta}\cdot\underbrace{\frac{n_1^\beta}{n_1^\beta+n_2^\beta}}_{x_1^\alpha}
                    = \frac{n_1^\alpha+n_1^\beta}{n_1+n_2}
                    = \frac{n_1}{n_1+n_2}
                \end{equation*}
            \end{itemize}
            \item With these definitions, we may derive the following \textbf{lever rule}. See my Thermo notes for more on lever rules.
            \begin{align*}
                n_1^\alpha+n_1^\beta &= n_1\\
                x_1^\alpha(n_1^\alpha+n_2^\alpha)+x_1^\beta(n_1^\beta+n_2^\beta) &= \prb{x_1}\cdot(n_1+n_2)\\
                &= \prb{x_1}\cdot(n_1^\alpha+n_2^\alpha)+\prb{x_1}\cdot(n_1^\beta+n_2^\beta)\\
                \frac{n_1^\alpha+n_2^\alpha}{n_1^\beta+n_2^\beta}\cdot x_1^\alpha+x_1^\beta &= \frac{n_1^\alpha+n_2^\alpha}{n_1^\beta+n_2^\beta}\cdot\prb{x_1}+\prb{x_1}\\
                \underbrace{\frac{n_1^\alpha+n_2^\alpha}{n_1^\beta+n_2^\beta}}_{n^\alpha/n^\beta} &= \frac{\prb{x_1}-x_1^\beta}{x_1^\alpha-\prb{x_1}}
            \end{align*}
            \item With the lever rule and the fact that $n^\alpha+n^\beta=n$, we can derive the following expressions for $n^\alpha,n^\beta$.
            \begin{align*}
                n^\alpha &= \frac{n(\prb{x_1}-x_1^\beta)}{x_1^\alpha-x_1^\beta}&
                n^\beta &= \frac{n(x_1^\alpha-\prb{x_1})}{x_1^\alpha-x_1^\beta}
            \end{align*}
            \begin{itemize}
                \item Notice that we have two variables and two equations, so we can solve the system of equations!
                \item If we care to, we can also use this result to derive expressions for any of the variables $n_1^\alpha,n_2^\alpha,n_1^\beta,n_2^\beta$ in terms of the givens. For example,
                % \begin{align*}
                %     n_1^\alpha &= \frac{\prb{x_1}-x_1^\beta}{x_1^\alpha-\prb{x_1}}\cdot(n_1^\beta+n_2^\beta)-n_2^\alpha\tag{Lever rule}\\
                %     &= \frac{\prb{x_1}-x_1^\beta}{x_1^\alpha-\prb{x_1}}\cdot[n-(n_1^\alpha+n_2^\alpha)]-n_2^\alpha\tag{$n=n_1^\alpha+n_2^\alpha+n_1^\beta+n_2^\beta$}\\
                %     &= \frac{\prb{x_1}-x_1^\beta}{x_1^\alpha-\prb{x_1}}\cdot\left( n-\frac{n_1^\alpha}{x_1^\alpha} \right)-\left( \frac{n_1^\alpha}{x_1^\alpha}-n_1^\alpha \right)\tag{Def. of $x_1^\alpha$}\\
                %     n_1^\alpha &= \frac{nx_1^\alpha(\prb{x_1}-x_1^\beta)}{x_1^\alpha-x_1^\beta}\tag{\emph{algebra}}
                % \end{align*}
                \begin{equation*}
                    n_1^\alpha = n^\alpha x_1^\alpha
                    = \frac{nx_1^\alpha(\prb{x_1}-x_1^\beta)}{x_1^\alpha-x_1^\beta}
                \end{equation*}
            \end{itemize}
        \end{itemize}
        \item We can now compare the mixing energy of the initial uniform phase with composition $\prb{x_1}$ to the total mixing energy of the two new phases with compositions $x_1^\alpha$ and $x_1^\beta$.
        \begin{itemize}
            \item The mixing energy of the initial uniform phase is just $\Delta G_M(\prb{x_1})$, but the total mixing energy of the two new phases is
            \begin{align*}
                \Delta G_M^{\alpha\beta}(\prb{x_1}) &= x^\alpha\cdot\Delta G_M(x_1^\alpha)+x^\beta\cdot\Delta G_M(x_1^\beta)\\
                &= \frac{1}{x_1^\alpha-x_1^\beta}\left[ (\prb{x_1}-x_1^\beta)\cdot\Delta G_M(x_1^\alpha)+(x_1^\alpha-\prb{x_1})\cdot\Delta G_M(x_1^\beta) \right]
            \end{align*}
            \item Let's interpret these two expressions graphically. $\Delta G_M(\prb{x_1})$ will just be a point on the $\Delta G_M$ curve. If we plot $\Delta G_M^{\alpha\beta}$ as a function of $x_1$, it will be the tie line between $\Delta G_M(x_1^\alpha)$ and $\Delta G_M(x_1^\beta)$!
            \item Therefore, if the tie line lies above the curve, demixing will raise the energy of the system and will not occur. If the tie line lies below the curve, demixing will lower the energy of the system and will occur. As such, with the choices of $\prb{x_1}$, $x_1^\alpha$, and $x_1^\beta$ in Figure \ref{fig:tieLine}, mixing will occur.
        \end{itemize}
        \item An important consequence of the tie-line finding is that --- since tie lines are an alternate way of defining concavity --- concave up is the criterion for \textbf{stability} of a phase.
    \end{itemize}
    \item \textbf{Stable} (phase): A phase such that any spontaneous, small local fluctuation in concentration will increase the free energy, and hence these out-of-equilibrium fluctuations will relax back to the expected compostion.
    \item Some personal thoughts and thought experiments on \emph{how} a solution demixes into two phases.
    \begin{itemize}
        \item A spinodal decomposition is an iterative process.
        \begin{itemize}
            \item From the initial concentration $\prb{x_1}$, we will demix into two phases with compositions only slightly higher and slightly lower than $\prb{x_1}$. These new phases then act as their own starting points, and must determine if demixing is higher or lower in energy.
            \item Demixing will continue as long as each little phase can decrease its energy by demixing, and this process will stop when the two phases lie along the spinodal.
            \begin{itemize}
                \item Indeed, regardless of the exact starting composition $\prb{x_1}$, if our starting solution is in the unstable region, it will undergo a spontaneous spinodal decomposition to two phases (of possibly different volumes) with compositions equal to the spinodal ones.
            \end{itemize}
            \item Once a phase lies at the spinodal, any choice of new phases that includes one phase outside the unstable region will correspond to a tie line above the curve (unless the other phase is \emph{vastly} different in composition).
            \item This is where nucleation must come along to `remove' some particles from the system and into a more stable nucleus.
        \end{itemize}
        \item Phase separation \emph{raises} entropy, always, as we are creating order. But it may decrease enthalpy. (Think of the enthalpy and entropy graphs separately, as in Figure \ref{fig:mixingChia}-\ref{fig:mixingChib}, and their respective tie lines.) Phase separation thus occurs when the gain in enthalpy outweights the loss in entropy.
        \begin{itemize}
            \item Temperature decreasing causes $\chi$ to increase. But by the definition of the Flory $\chi$ parameter, a decrease in temperature is mathematically equivalent to temperature staying the same and things starting to like themselves more than they like each other ($\varepsilon_{12}$ getting more negative).
        \end{itemize}
        \item An important characteristic of a metastable system at the spinodal points is that the two phases are in a \emph{dynamic} equilibrium.
        \begin{itemize}
            \item A certain amount of component $i$ can move out of one phase and into the second, just so long as the flow is balanced by an equal amount of component $i$ moving from the second phase into the first.
            \item Demixing can also be thought of as a dynamic transfer of matter that is \emph{not} in equilibrium; that is, the exodus of component 1 from a phase may be greater than the flow of component 1 into that phase, resulting in that phase decreasing in its $\phi_1$.
            \item This flow of matter is driven by the change in free energy of the system with the number of moles in it. But this is exactly the chemical potential! That's why the chemical potential is the perfect tool with which to visualize the flow of matter in demixing.
            \begin{itemize}
                \item Spinodal decomposition: The flow of matter from one phase to the other at a rate governed by the chemical potential until we have a dynamic equilibrium of matter flowing from one phase to the other that balances out.
                \item Matter will "roll downhill" until it gets to the bottom of the chemical potential, which occurs at the spinodal points.
            \end{itemize}
        \end{itemize}
        \item Nucleation of `stable' nuclei.
        \begin{itemize}
            \item Suppose ideal stability occurs for $\phi_1=0.7$. Then take 7 molecules of 1 and 3 molecules of 2 as a nucleus. It will be energically favorable for another 7 molecules of 1 and 3 molecules of 2 to come out of the phase they are in and join the growing nucleus; the remaining phases can then reequilibrate their distribution of molecules to reach stability with a smaller amount of matter.
        \end{itemize}
        \item Lattice theory can be an instructive way to visualize phases and separations.
        \begin{itemize}
            \item For example, model a beaker containing a mixture as a $4\times 4$ lattice with 8 black circles and 8 white circle randomly arranged. This could represent a homogeneous solution at high temperature.
            \item As temperature is lowered (or $\chi$ increases), there will come a point where demixing occurs into a denser, bottom phase and a lighter, top phase. Perhaps first, the top phase will gain an extra white circle on average and the bottom phase will gain an extra black circle on average.
            \item As the decomposition continues, eventually we will reach a stable point (perhaps 6 black circles and 2 white circles at bottom, and 2 black circles and 6 white circles at top). The particles still move around, as long as this general trend is maintained.
            \item Lattice theory also provides a good backdrop for visualizing the importance of energy \emph{per site}, as we can only obtain the total energy through multiplying the energy per site by each site in each region and dividing by the total nubmer of sites.
        \end{itemize}
        % \item Phase separation is governed by mass transport: If you are going to make one region more concentrated in one component, you do have to remove an equal amount of the other component. This is very easily visualized by lattice theory. We could, in theory, form a small local region of highly concentrated 1 or 2, but then the system would be out of equilibrium and we would essentially have an entirely new phase.
        % \begin{itemize}
        %     \item Does "spinodal decomposition" imply a symmetric movement of matter??
        %     \item So spinodal implies symmetric: We snap into two phases. At that point, directional movement of matter into one region or the other is unfavorable, though we can still do free diffusion as long as it's balanced (e.g., molecule of 1 leaves one phase to enter the other at the same time a molecule of 1 leaves the other phase to enter the original).
        %     \item However, when nucleation occurs, say ideal stability is $\phi_1=0.7$. Then take 7 molecules of 1 and 3 molecules of 2 as a nucleus. It will be energically favorable for another 7 molecules of 1 and 3 molecules of 2 to come out of the phase they are in and join the growing nucleus; the remaining phases can then reequilibrate their distribution of molecules to reach stability with a smaller amount of matter.
        %     \item Maybe it does move asymmetrically; it certainly could along the way. But the spinodal points are situated symmetrically, so even if we move asymmetrically to begin with, the ultimate metastable place to which we are going is symmetric.
        % \end{itemize}
        % \item Visualize matter flow with the chemical potential, dummy!!
        % \item If you create a single solution with concentration at the spinodal, will it demix?? I don't think so, because it appears from my calculations that this would just raise the energy of the system.
        % \item Unequal compositions decompose into phases with different volumes!
        % \begin{itemize}
        %     \item 6 mols 1 and 4 mols 2 will split into 7.5 moles of $x_1=0.7$ and 2.5 moles of $x_2=0.3$. So a key equation we need is
        %     \begin{gather*}
        %         \prb{\phi_1} = \phi_\alpha\phi_1+\phi_\beta\phi_2\\
        %         \phi_\alpha+\phi_\beta = 1
        %     \end{gather*}
        %     \item Regardless of the exact composition, if our starting solution is in the unstable region, it will undergo a spontaneous spinodal decomposition to two phases (of possibly different volumes) with compositions equal to the spinodal ones.
        %     \item Let $\prb{\phi_1}$ be the composition of a starting solution in the unstable region, $\phi_\alpha$ be the composition of spinodal phase $\alpha$, $\phi_\beta$ be the composition of spinodal phase $\beta$, $\phi_1$ be the volume fraction of 1 in phase $\alpha$ at the metastable equilibrium, and $\phi_2$ be the volume fraction of 2 in phase $\alpha$ at the metastable equilibrium.
        % \end{itemize}
        % \item Then what about polymer-solvent systems that have asymmetry?
        % \begin{itemize}
        %     \item Here, you \emph{will} decompose into two different phases with two different volumes!
        % \end{itemize}
    \end{itemize}
    \item Finding the binodal.
    \begin{itemize}
        \item There are infinitely many pairs $x_1^\alpha,x_1^\beta$ of phases to which an unstable system can demix, so how do we determine which pair corresponds to the most energetically stable system?
        \item Well, a phase equilibrium is established when $T,P$ are equal in both phases, and the chemical potentials are, too. Symbolically, our criteria are
        \begin{align*}
            \mu_1(x_1^\alpha) &= \mu_1(x_1^\beta)&
            \mu_2(x_1^\alpha) &= \mu_2(x_1^\beta)
        \end{align*}
        \begin{itemize}
            \item Warning: While the chemical potential of component 1 must be equal in both phases (and likewise for component 2), the chemical potential of component 1 \emph{does not} need to equal the chemical potential of component 2.
        \end{itemize}
        \item We will now show that finding the phases $x_1^\alpha,x_1^\beta$ which satisfy the above criteria is equivalent to finding the lowest possible tangent line graphically.
        \begin{itemize}
            \item Since the chemical potential is a partial molar free energy, we can write the total free energy of mixing as the mole-weighted sum of the chemical potentials, as follows.
            \begin{equation*}
                \Delta G_M = n_1\Delta\mu_1+n_2\Delta\mu_2
            \end{equation*}
            \item It follows that the free energy of mixing per mole of solution is
            \begin{equation*}
                \Delta G_M = x_1\Delta\mu_1+(1-x_1)\Delta\mu_2
                = \Delta\mu_2+x_1(\Delta\mu_1-\Delta\mu_2)
            \end{equation*}
            \item This equation corresponds to a line of the form $y=mx+b$ across an energy-composition plot and tangent to the $\Delta G_M$ curve at $x_1$. In fact, this line's $y$-intercept will be $\Delta\mu_2(x_1)$ and its intercept with the line $x_1=1$ will be $\Delta\mu_1(x_1)$! This is the key.
            \item In particular, if we have a $\Delta G_M$ curve with a bump, we can draw exactly one straight line that is tangent to the curve at \emph{two} points. We take these points to be $x_1^\alpha,x_1^\beta$ for the following reason: By the above argumnt, the line being tangent to $x_1^j$ (for $j=\alpha,\beta$) means that the $y$-intercept is $\Delta\mu_2(x_1^j)$ and the other intercept is $\Delta\mu_1(x_1^j)$. But since a single line can only have one of each intercept, we know that
            \begin{align*}
                \Delta\mu_1(x_1^\alpha) &= \Delta\mu_1(x_1^\beta)&
                    \Delta\mu_2(x_1^\alpha) &= \Delta\mu_2(x_1^\beta)\\
                \mu_1(x_1^\alpha)-\mu_1^\circ &= \mu_1(x_1^\beta)-\mu_1^\circ&
                    \mu_2(x_1^\alpha)-\mu_2^\circ &= \mu_2(x_1^\beta)-\mu_2^\circ\\
                \mu_1(x_1^\alpha) &= \mu_1(x_1^\beta)&
                    \mu_2(x_1^\alpha) &= \mu_2(x_1^\beta)
            \end{align*}
            as desired.
        \end{itemize}
        \item Note that we could also use the two original constraints and our known expressions for the chemical potentials to find $x_1^\alpha,x_1^\beta$ analytically, but the algebra would get a bit hairy.
    \end{itemize}
    \item \textbf{Metastable} (system): A system that is stable against small, spontaneous fluctuations, but not globally stable against phase separation.
    \item An alternate way of finding the spinodal points is by (1) differentiating $\Delta G_M$ with respect to $n_1$ to find the chemical potential $\mu_1$ of component 1 and then (2) setting the derivative of $\mu_1$ with respect to composition $x_1$ equal to zero to find the minimum chemical potential. This chemical potential minimum corresponds to a region of stability for the phase dominated by component 1, and hence exactly the spinodal point.
    \item On the critical point.
    \begin{itemize}
        \item In the context of the exchange energy: If it costs any more than $\chi_c\kB T$ to exchange a pair of molecules of different components, there will be phase separation.
    \end{itemize}
    \item The free energy of mixing when an arbitrary number of components $i$ are included is
    \begin{equation*}
        \Delta G_M = \sum_i\frac{\phi_i}{N_i}\ln\phi_i+\sum_{i<j}\chi_{ij}\phi_i\phi_j
    \end{equation*}
\end{itemize}




\end{document}