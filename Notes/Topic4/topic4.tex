\documentclass[../notes.tex]{subfiles}

\pagestyle{main}
\renewcommand{\chaptermark}[1]{\markboth{\chaptername\ \thechapter\ (#1)}{}}
\setcounter{chapter}{3}

\begin{document}




\chapter{Networks}
\section{Rubber Elasticity}\label{sse:14}
\begin{itemize}
    \item \marginnote{10/21:}Announcements.
    \begin{itemize}
        \item Quiz 1 solutions posted to Canvas.
        \item Ask Alfredo if you have any questions about the light scattering lecture.
        \item 2014 lecture notes are now visible on Canvas for most of the lectures in the course!
    \end{itemize}
    \item Overview of Topic 4 (by day).
    \begin{itemize}
        \item Today: Rubbers and networks.
        \item Next class: Gels.
        \item The following class (as time permits): More on networks or gels.
    \end{itemize}
    \item \textbf{Rubber}: A network of crosslinked chains.
    \begin{itemize}
        \item The crosslinked chains take many forms: You can have \textbf{topological crosslinks}, \textbf{loops}, \textbf{dangling ends}, etc.
        \item Example rubber: A rubber band. Recall that when you heat a rubber band, it tightens up. Today, we will explain this in terms of increasing the entropic spring force!
    \end{itemize}
    \item \textbf{Topological crosslinks}: Two chains that add integrity to the network by looping around each other mechanically, rather than being covalently bonded.
    \item \textbf{Loop}: A chain that connects back with itself without introducing any classical or topological crosslinks, and thus does not carry stress.
    \item \textbf{Dangling ends}: A chain strand in a network that does not bond (classically or mechanically) to any other chain or itself, and thus does not carry stress.
    \item Recall from Lecture \ref{sse:1}: Chich\'{e}n Ita\'{a} (ancient Mayan place) and natural rubber basquetbol.
    \begin{itemize}
        \item There were no hoops in this game; rather players tried to get the ball into very small holes in a big wall.
        \item Such a tight-fitting target would have necessitated the rubber be crosslinked.
    \end{itemize}
    \item \textbf{Deformation}: A change in the shape of a polymer, without changing the overall volume.
    \item \textbf{Swelling}: Enlarging a polymer network by mixing in a solvent and seeing the resulting increase in volume.
    \begin{itemize}
        \item Swelling creates a \textbf{gel}, the topic of next lecture!
    \end{itemize}
    \item Rubber elasticity assumptions.
    \begin{figure}[H]
        \centering
        \begin{subfigure}[b]{0.3\linewidth}
            \centering
            \includegraphics[width=0.8\linewidth]{rubberAssumea.JPG}
            \caption{Chain slip.}
            \label{fig:rubberAssumea}
        \end{subfigure}
        \begin{subfigure}[b]{0.3\linewidth}
            \centering
            \includegraphics[width=0.95\linewidth]{rubberAssumeb.png}
            \caption{Affine deformation.}
            \label{fig:rubberAssumeb}
        \end{subfigure}
        \caption{Rubber elasticity assumptions.}
        \label{fig:rubberAssume}
    \end{figure}
    \begin{enumerate}
        \item The chains are Gaussian between the permanent crosslinks, and there are no excluded volume effects.
        \begin{itemize}
            \item It follows that enthalpy/energy considerations are not important, i.e., $U=H=0$.
            \item Thus, basically, $\Delta G\propto\lambda^2/Nb^2$.
        \end{itemize}
        \item The temperature is well above $T_g$, so no freezing or crystallization occurs.
        \begin{itemize}
            \item We're living in a liquid state.
        \end{itemize}
        \item The chains are flexible, with relatively low backbone bond rotation potential barriers.
        \begin{itemize}
            \item Changing $\phi$ from the symmetric hindered rotations model is easy.
        \end{itemize}
        \item Chain deformation occurs by conformational changes, \emph{not} bond stretching.
        \begin{itemize}
            \item $l$ does not change.
        \end{itemize}
        \item No relaxation occurs by \textbf{chain slip}, i.e., this is a fixed, permanent network (at least on the time scale of the experiment).
        \begin{itemize}
            \item Crosslinks \emph{cannot} break.
        \end{itemize}
        \item Affine deformation.
        \begin{itemize}
            \item At every length scale, from microscopic subchains to the macroscopic network, the deformation experienced is equivalent.
            \item For example, a shear at the surface is the same shear at the microscale. Notice how in Figure \ref{fig:rubberAssumeb}, the stretch of the entire chunk is mirrored by the stretch in the polymer; we will numerically relate the variables in this diagram shortly.
            \item Mathematically, the only permissible deformations are ones that preserve lines and parallelism, but not necessarily Euclidean distances or angles. In other words, these deformations are functions composed of a linear transformation,such as rotation, shear, extension and compression, and a rigid body translation.
        \end{itemize}
        \item No crystallization at large strains.
        \begin{itemize}
            \item In real life, if all chains are fully elongated (trans-trans-trans-etc.), they can pack into crystalline domains. We assume this doesn't happen.
        \end{itemize}
        \item No change in volume or density upon deformation.
        \begin{itemize}
            \item Thus, if we pull it in one direction, it must shrink in the others.
        \end{itemize}
    \end{enumerate}
    \item \textbf{Chain slip}: The removal of a covalent or mechanical crosslink via bond cleavage or disentanglement, respectively.
    \item \textbf{Extension ratio} (in the $i$-direction): One of three ($i\in\{x,y,z\}$) orthogonal extension ratios which depend on the type of loading geometry. \emph{Denoted by} $\bm{\alpha_i},\bm{\lambda_i}$. \emph{Given by}
    \begin{align*}
        R_i &= \alpha_iR_{i0}&
        L_i &= \alpha_iL_{i0}
    \end{align*}
    where $\mathbf{R}_0$ is the end-to-end vector of an undeformed subchain in the rubber, $\mathbf{R}$ is the end-to-end vector of the same subchain after deformation, $\textbf{L}_0$ is the corner-to-corner vector of the rubber, $\mathbf{L}$ is the corner-to-corner vector of the rubber after deformation, and subscripts denote the $i^\text{th}$ component of these vectors.
    \begin{itemize}
        \item On notation, Alfredo uses $\alpha$ while the book uses $\lambda$. The two notations are completely interchangeable, and we should get used to seeing both.
        \item More explicit definitions with the above terminology and alternate notations.
        \begin{itemize}
            \item We assume that the relaxed polymer has root mean square end-to-end distance $\prb{R_0^2}^{1/2}$, and the extended polymer has root mean square end-to-end distance $\prb{R^2}^{1/2}$.
            \item These vectors are
            \begin{align*}
                \mathbf{R}_0 &= (R_{x0},R_{y0},R_{z0}) = (x,y,z)&
                \mathbf{R} &= (R_x,R_y,R_z) = (x',y',z')
            \end{align*}
        \end{itemize}
    \end{itemize}
    \item Formulating some of the assumptions mathematically in terms of the $\alpha_i$.
    \begin{itemize}
        \item Assumption 6, mathematically, is that the macro and micro definitions of $\alpha_i$ are equivalent.
        \item Assumption 8 is that $\alpha_x\alpha_y\alpha_z=1$.
    \end{itemize}
    \item We now investigate the elastic force with which the polymer tries to restore its original shape after deformation.
    \item Terminology for describing the rubber's elastic force, which arises entropically.
    \begin{itemize}
        \item State 1 is relaxed, and State 2 is extended.
        \item State 1.
        \begin{itemize}
            \item Force $F=0$.
            \item $\Omega_1$ possible conformations.
            \item Root mean square end-to-end distance (of the average chain) is $\prb{R_0^2}^{1/2}$.
            \item Extension coordinates $x,y,z$.
        \end{itemize}
        \item State 2.
        \begin{itemize}
            \item Force $F>0$.
            \item $\Omega_2$ possible conformations.
            \item Root mean square end-to-end distance (of the average chain) is $\prb{R^2}^{1/2}$.
            \item Extension coordinates $x',y',z'$.
        \end{itemize}
        \item Sanity check: $\Omega_1>\Omega_2$.
    \end{itemize}
    \item Using this terminology, the actual elastic force is derived by considering the change in the conformational entropy of the extended vs. relaxed states.
    \begin{equation*}
        \Delta S = S_2-S_1
        = \kB\ln\Omega_2-\kB\ln\Omega_1
        = \kB\ln\frac{\Omega_2}{\Omega_1}
    \end{equation*}
    \begin{itemize}
        \item Let's now look at the scaling of the $\Omega_i$ in terms of coordinates and extension ratios.
        \begin{align*}
            \Omega_1 &= \text{const}\cdot\exp(-\frac{3R_0^2}{2nl^2})&
                \Omega_2 &= \text{const}\cdot\exp(-\frac{3R^2}{2nl^2})\\
            &= \text{const}\cdot\exp\left[ -\frac{3(x^2+y^2+z^2)}{2nl^2} \right]&
                &= \text{const}\cdot\exp\left\{ -\frac{3\left[ (x')^2+(y')^2+(z')^2 \right]}{2nl^2} \right\}\\
            &= \text{const}\cdot\exp\left[ -\frac{3(1^2x^2+1^2y^2+1^2z^2)}{2nl^2} \right]&
                &= \text{const}\cdot\exp\left[ -\frac{3(\alpha_x^2x^2+\alpha_y^2y^2+\alpha_z^2z^2)}{2nl^2} \right]
        \end{align*}
        \item Substituting these definitions of $\Omega_i$ into the $\Delta S$ equation affords
        \begin{equation*}
            \Delta S = -\frac{3\kB}{2nl^2}\left[ (\alpha_x^2-1)x^2+(\alpha_y^2-1)y^2+(\alpha_z^2-1)z^2 \right]
        \end{equation*}
        \item Now averaging over the entire ensemble-like collection of subchains in our rubber, we can assume that the average subchain is isotropic. (This is also consistent with Assumption 1.) Thus,
        \begin{equation*}
            \prb{x^2} = \prb{y^2} = \prb{z^2} = \frac{\prb{R_0^2}}{3}
        \end{equation*}
        \item This combined with the fact that $R_0^2=nl^2$ gives us the final change in entropy per subchain upon deformation in terms of the extension ratios.
        \begin{align*}
            \Delta S &= -\frac{3\kB}{2nl^2}\left[ (\alpha_x^2-1)\prb{x^2}+(\alpha_y^2-1)\prb{y^2}+(\alpha_z^2-1)\prb{z^2} \right]\\
            &= -\frac{3\kB}{2nl^2}\left[ (\alpha_x^2-1)\frac{\prb{R_0^2}}{3}+(\alpha_y^2-1)\frac{\prb{R_0^2}}{3}+(\alpha_z^2-1)\frac{\prb{R_0^2}}{3} \right]\\
            &= -\frac{3\kB}{2nl^2}\frac{nl^2}{3}\left[ (\alpha_x^2-1)+(\alpha_y^2-1)+(\alpha_z^2-1) \right]\\
            &= -\frac{\kB}{2}(\alpha_x^2+\alpha_y^2+\alpha_z^2-3)
        \end{align*}
        \item It follows that the entropic spring force $F_i$ in the $i$-direction upon deformation is
        \begin{equation*}
            F_i = \pdv{G}{R_i}
            = \pdv{(0-T\Delta S)}{R_i}
            = -T\left( \pdv{\Delta S(\alpha_i)}{R_i} \right)_{T,P}
        \end{equation*}
    \end{itemize}
    \item Rubber demos.
    \begin{itemize}
        \item Rubber elasticity is of the few times you can experiment with thermodynamics on the fly.
        \begin{itemize}
            \item Other times: Filling air into tires, and spray-on sunscreen on your back feeling cold.
        \end{itemize}
        \item Today, we'll "feel" entropy by sensing the change in temperature of a rubber band as we stretch it.
        \begin{itemize}
            \item Feel temperature of a rubber band, stretch (or release) it very quickly, and feel the temperature again!
            \item When you stretch it, the system heats up; when you release it, the system cools down.
            \item We need to move quickly so that the change is (approximately) adiabatic, i.e., we do not want to allow the system to exchange heat with the environment to a meaninful extent.
        \end{itemize}
    \end{itemize}
    \item Let's now analyze the elastic force generated by the simplest type of deformation: Uniaxial deformation.
    \begin{figure}[h!]
        \centering
        \begin{subfigure}[b]{0.3\linewidth}
            \centering
            \includegraphics[width=0.5\linewidth]{uniaxa.png}
            \caption{State 1.}
            \label{fig:uniaxa}
        \end{subfigure}
        \begin{subfigure}[b]{0.3\linewidth}
            \centering
            \includegraphics[width=0.95\linewidth]{uniaxb.png}
            \caption{State 2.}
            \label{fig:uniaxb}
        \end{subfigure}
        \caption{Uniaxial deformation of a rubber.}
        \label{fig:uniax}
    \end{figure}
    \begin{itemize}
        \item In this case, $\alpha_x$ is our independent variable. In particular, $\alpha_y,\alpha_z$ depend on it via
        \begin{equation*}
            \alpha_y = \alpha_z = \alpha_x^{-1/2}
        \end{equation*}
        \begin{itemize}
            \item This is called a \textbf{Poisson contraction} in the lateral directions, and arises from the mathematical formulation of Assumption 8 (i.e., $\alpha_x\alpha_y\alpha_z=1$).
        \end{itemize}
        \item It follows that in this case,
        \begin{equation*}
            \Delta S(\alpha_x) = -\frac{\kB}{2}\left( \alpha_x^2+\frac{1}{\alpha_x}+\frac{1}{\alpha_x}-3 \right)
        \end{equation*}
        \item Thus, the entropic spring force / restoring force in the $x$-direction provided by a single chain is
        \begin{align*}
            F_x &= -T\pdv{\Delta S}{R_x}\\
            &= -T\pdv{\Delta S}{\alpha_x}\cdot\pdv{\alpha_x}{R_x}\\
            &= \frac{\kB T}{2}\pdv{\alpha_x}(\alpha_x^2+\frac{1}{\alpha_x}+\frac{1}{\alpha_x}-3)\cdot\pdv{R_x}(\frac{R_x}{R_{x0}})\\
            &= \frac{\kB T}{2}\left( 2\alpha_x-\frac{2}{\alpha_x^2} \right)\cdot\frac{1}{R_{x0}}\\
            &= \frac{\kB T}{R_{x0}}\left( \alpha_x-\frac{1}{\alpha_x^2} \right)
        \end{align*}
        \begin{itemize}
            \item The first term is Hookean ($F\propto\alpha_x$) and corresponds to the penalty for stretching.
            \item The second term is the correction for compressing the other two dimensions.
        \end{itemize}
        \item We now seek to scale this microscopic result up to the full macroscopic rubber.
        \begin{itemize}
            \item Let the full rubber contain $z$ subchains identical to the one we've analyzed, each of which contributes equally to the entropic spring force. Then the total change in entropy upon stretching the rubber is
            \begin{equation*}
                \Delta S_\text{tot} = -\frac{z\kB}{2}\left( \alpha_x^2+\frac{2}{\alpha_x}-3 \right)
            \end{equation*}
            \item This time, since we have a different system under study, we will need to differentiate with respect to the length of the \emph{rubber} instead of the length of the \emph{polymer}. Analogously to before, this affords
            \begin{equation*}
                F_{x,\text{tot}} = -T\pdv{\Delta S_\text{tot}}{l_x}
                = \frac{z\kB T}{l_{x0}}\left( \alpha_x-\frac{1}{\alpha_x^2} \right)
            \end{equation*}
            \item Now the full stress-strain relationship for the rubber is given by a tensor. The entry $\sigma_{xx}$, specifically, is the whole rubber's restoring force $F_{x,\text{tot}}$ divided by its cross-sectional area $A_0$ when unstretched. We now seek to couch the above equation in terms of these new variables.
            \item To do so, first let $V$ denoted the total volume of the rubber. Then $z/V$ is the number of subchains per unit volume. It follows by dimensional analysis (as in Lecture \ref{sse:10}) that
            \begin{equation*}
                \frac{z}{V} = \frac{\rho N_\text{A}}{M_x}
            \end{equation*}
            where $\rho$ is the density of the material in \si[per-mode=symbol]{\gram\per\liter} and $M_x$ is the number-average molecular weight of the subchains in \si[per-mode=symbol]{\gram\per\mole}.
            \item Therefore, the stress in terms of observables is
            \begin{align*}
                \sigma_{xx}(\alpha_x) &= \frac{z\kB T}{A_0l_{x0}}\left( \alpha_x-\frac{1}{\alpha_x^2} \right)\\
                &= \frac{z\kB T}{V}\left( \alpha_x-\frac{1}{\alpha_x^2} \right)\\
                &= \frac{\rho N_\text{A}\kB T}{M_x}\left( \alpha_x-\frac{1}{\alpha_x^2} \right)
            \end{align*}
            \item Note that we could also derive the same equation by considering $F_x/(R_{y0}R_{z0})$, as then we would have $1/\prb{R^2}^{3/2}=\rho N_\text{A}/M_x$ instead of $z/V$ equals that. This essentially says that the total number of subchains divided by the total volume equals the volume of one subchain divided by its volume.
        \end{itemize}
    \end{itemize}
    \item \textbf{Young's modulus}: The instantaneous stress per strain of a material when you just begin straining it. \emph{Denoted by} $\bm{E}$. \emph{Given by}
    \begin{equation*}
        E := \lim_{\alpha_x\to 1}\dv{\sigma_{xx}}{\alpha_x}
    \end{equation*}
    \item Let's now investigate the stress $\sigma$ vs. strain $\alpha$ for the uniaxial deformation case we've been considering.
    \begin{itemize}
        \item Evaluating the limit, we obtain
        \begin{align*}
            E &= \lim_{\alpha_x\to 1}\dv{\alpha_x}\left[ \frac{\rho RT}{M_x}\left( \alpha_x-\frac{1}{\alpha_x^2} \right) \right]\\
            &= \lim_{\alpha_x\to 1}\frac{\rho RT}{M_x}\left( 1+\frac{2}{\alpha_x^3} \right)\\
            &= \frac{\rho RT}{M_x}\left( 1+\frac{2}{1^3} \right)\\
            &= \frac{3\rho RT}{M_x}
        \end{align*}
        \begin{itemize}
            \item In reality, the prefactor is lower than 3. People like to bring this 3 into an "effective" molecular weight, which is a variant on $M_x$.
        \end{itemize}
        \item Notice: The Young's modulus is directly proportional to temperature, and inversely proportional to subchain molecular weight.
        \item It also follows that to measure the subchain molecular weight $M_x$, we need only measure the Young's modulus!
        \item Real-world example: Blowing a hairdryer on an extended rubber band will make it shrink up.
    \end{itemize}
    \item Comparing our theoretical behavior of elastomers under uniaxial deformation to their real-life behavior.
    \begin{figure}[h!]
        \centering
        \includegraphics[width=0.25\linewidth]{uniaxReal.png}
        \caption{Real uniaxial deformation behavior.}
        \label{fig:uniaxReal}
    \end{figure}
    \begin{itemize}
        \item We will not discuss the softening under the curve at low strains $\alpha$. However, the hardening above the curve at high strains $\alpha$ is something we can discuss.
        \item Essentially, when a rubber is almost all stretched, we start stretching bonds (violating Assumption 4). This leads to the empirically observed toughening.
    \end{itemize}
    \item Demonstration: Violating Assumption 7 in real life.
    \begin{itemize}
        \item Two students at the front of class stretch a rubber band as far as they can without breaking it, and then Alfredo starts making a small cut in the middle with a pair of scissors. The crack does not propagate. However, when the students let the rubber band relax, wait a little while, and then try to restretch it, the crack propagates further and it breaks then. Why does this happen?
        \item When everything's in the trans conformation (stretched rubber band), you start getting crystallization which will add more effective crosslinks. This is why the stretched rubber band won't break even as Alfredo cuts it.
        \item However, when the students release the strain and then stretch it again, crystallization cannot occur quick enough and the rubber band breaks.
    \end{itemize}
    \item Now instead of using entropy alone, let's look at a more realistic derivation for the length of a subchain as a function of the stretching force applied to it.
    \begin{figure}[h!]
        \centering
        \begin{tikzpicture}
            \footnotesize
            \draw [densely dashed] (0,0) coordinate (b) -- (1.2,0) coordinate (a);

            \draw [-stealth] (0,0) -- ++(-0.7,0) node[left]{$f$};
            \draw [-stealth] (1,0.5) -- ++(0.7,0) node[right]{$f$};
            \draw [blx,ultra thick] (0,0) -- (1,0.5) coordinate (c);
            \pic [draw,angle radius=5.7mm,angle eccentricity=1.2,pic text={$\theta$}] {angle=a--b--c};
        \end{tikzpicture}
        \caption{Stretching a single bond in a subchain.}
        \label{fig:subchainRealStretch}
    \end{figure}
    \begin{itemize}
        \item Consider a single bond defined by the vector $\mathbf{l}$, which is being stretched by a force $f$ in each direction at both ends (Figure \ref{fig:subchainRealStretch}).
        \item We still assume that $\mathbf{R}=\sum\mathbf{l}$.
        \item How much work $W$ do we do by rotating the bond an angle $\theta$ away from equilibrium?
        \begin{equation*}
            W(\theta) = f\Delta d
            = f\cdot l\sin\theta
        \end{equation*}
        \begin{itemize}
            \item We get the above from the definition of work as force times distance (for constant forces), and a bit of trigonometry to get the distance moved in terms of $\theta$.
        \end{itemize}
        \item We now bring in partition functions from thermodynamics.
        \begin{itemize}
            \item The probability that a chain link will be at angle $\theta$ from the ideal is $\e[-U(\theta)/\kB T]$, where $U(\theta)=W(\theta)$.
            \item Then we average over all possible lengths of this bond ($-l$ to $l$, indexed by $\theta$), weighted by the Boltzmann probability of the length occurring.
            \begin{equation*}
                \prb{l} = \frac{1}{\int_0^\pi\e[-U(\theta)/\kB T]\dd\theta}\int_0^\pi l\cos\theta\e[-U(\theta)/\kB T]\dd\theta
                = l\left( \coth\beta-\frac{1}{\beta} \right)
            \end{equation*}
            \item Sum over all $N$ bonds affords
            \begin{equation*}
                \prb{R} = Nl\left( \coth\beta-\frac{1}{\beta} \right)
            \end{equation*}
            where $\beta=fl/\kB T$ is the \textbf{relative force}.
        \end{itemize}
    \end{itemize}
    \item Thus, putting everything together, we can consider two approximate behavioral regimes for stretching a chain.
    \begin{itemize}
        \item Up until about 20\% stretching ($\beta\ll 1$), everything is pretty linear and Hookean. We can use our purely entropic model here.
        \begin{itemize}
            \item Note that by 20\% stretching, we mean stretching the chain to 20\% of it's maximum lenght. Symbolically, we mean stretching from $\prb{R_0^2}^{1/2}$ up to $0.2R_\text{max}$.
        \end{itemize}
        \item But as the force becomes very large ($\beta\gg 1$), we cannot extend beyond the total length of the system. Here, we need the Figure \ref{fig:subchainRealStretch} derivation.
        \begin{itemize}
            \item Force rises faster in these more stretched regimes because we don't have so much "rope to give." In other words, we have very few conformations left available to us.
        \end{itemize}
        \item Note that if $N$ is large, 20\% elongation may be a pretty substantial stretch. This is because $R_0=N^{1/2}b$ and $R_\text{max}=Nb$ collectively imply that
        \begin{equation*}
            \frac{R_0}{R_\text{max}} = N^{-1/2}
        \end{equation*}
    \end{itemize}
    \item Let's now see how well our two-regime model matches empirically obtained data.
    \begin{itemize}
        \item \textcite{bib:subchainRealStretch} used atomic force microscopy (AFM) to stretch a single chain of PMMA.
        \begin{itemize}
            \item Back in the '90s, people were using AFM for everything.
        \end{itemize}
        \item The data they found matched the entropic FJC model really well up to about 20\%, and then matched the Figure \ref{fig:subchainRealStretch} one past that pretty well!
        \item Note that the stretching force they applied to the single polymer was on the scale of nanonewtons
        \begin{itemize}
            \item A nanonewton of force is \emph{not} a small quantity of force on a molecular scale.
            \item For context, ligand-receptor bindings are on the scale of high femptonewtons to low piconewtons. van der Waals forces are on the scale of femptonewtons.
            \item Thus, a nanonewton is a really strong force; that's about when we start breaking covalent \ce{C-C} bonds! In other words, they had to put enough force on the polymer that it was about to break in order to get it past 20\% elongation!
        \end{itemize}
    \end{itemize}
    \item Next time.
    \begin{itemize}
        \item Gels (probs the whole class and a bit of the next one). These get more complicated, and there are more factors.
    \end{itemize}
\end{itemize}



\section{Neutral Gels}\label{sse:15}
\begin{itemize}
    \item \marginnote{10/23:}Most gels we think about are water-based, e.g., diapers, Jello, and contact lenses.
    \begin{figure}[h!]
        \centering
        \includegraphics[width=0.4\linewidth]{sperm.JPG}
        \caption{The flagellar motion of a sperm's tail is driven by polymer physics!}
        \label{fig:sperm}
    \end{figure}
    \begin{itemize}
        \item Diapers absorb \emph{a lot} of water.
        \begin{itemize}
            \item Over the course of lecture, Alfredo will pour about 2 liters of water into a standard diaper.
            \item Note, however, that this maximum swelling does take time --- when your kids pee all at once, the diaper will not absorb it all instantly!
        \end{itemize}
        \item Biological gels: Mucus.
        \begin{itemize}
            \item Home chemistry: Take a booger out, put it in water, and see what happens.
            \item This happens because boogers' heavily glycosylated (sugar) chains expand significantly.
        \end{itemize}
        \item Active biological gels (ABGs): Cells!
        \begin{itemize}
            \item Cells contain many (typically living) polymers.
            \item Examples (from stiffest to least stiff): Microtubules, actin, peptides or ssRNA.
            \begin{itemize}
                \item Recall these from Lecture \ref{sse:3} on persistence length.
            \end{itemize}
            \item Example cellular motion controlled by an active gel: The tail of a sperm (Figure \ref{fig:sperm}).
            \begin{itemize}
                \item These contain molecular motors (e.g., dyneins and kinesins) that walk along a microtubule filament.
                \item Motor proteins know which direction to walk along a microtubule because microtubules are helical, and the helix has an orientation!
                \item Flagellar motion of the tail occurs by the walking of kinesins both toward the same part of the membrane, which induces buckeling.
            \end{itemize}
            \item Example: Nature-inspired ABG.
            \begin{itemize}
                \item Self-propagating Belousov-Zhabotinsky reaction.
                \item Reference: \textcite{bib:BZgel}.
            \end{itemize}
        \end{itemize}
    \end{itemize}
    \item This concludes the introduction; we now begin the lecture content in earnest.
    \item \textbf{Gel}: A highly swollen network.
    \item Several ways to form a gel.
    \begin{itemize}
        \item Start with a monomer melt and link the monomers into a 3D network (e.g., S-\emph{co}-DVB), then swell it with a solvent.
        \begin{itemize}
            \item S-\emph{co}-DVB is hydrophobic (swell it with an organic solvent).
            \item PEG is hydrophilic (swell it with water).
        \end{itemize}
        \item Start with a polymer-solvent solution and induce network formation, e.g., crosslink the polymer via\dots
        \begin{itemize}
            \item Radiation (UV, electron beam);
            \begin{itemize}
                \item Induces covalent bonds.
            \end{itemize}
            \item Chemical means (e.g., S-\emph{co}-DVB);
            \begin{itemize}
                \item Induces covalent bonds.
            \end{itemize}
            \item Physical associations (e.g., noncovalent bonds induced by lowering the temperature or adding a nonsolvent).
            \begin{itemize}
                \item Example: Collagen forms triple helices when it cools.
            \end{itemize}
        \end{itemize}
    \end{itemize}
    \item Some gels are classified as responsive molecules.
    \item Example responsive molecule class: Polyelectrolytes.
    \begin{itemize}
        \item These undergo a collapse transition in response to solvent quality.
        \item Essentially, when the \emph{p}H is tuned one way or the other, a polyelectrolyte will either swell or collapse into a globule.
        \item Benefits: Tunable polarity, potential for self-assembly, cheap and easy to synthesize.
        \item Example: Poly(methacrylic acid).
        \begin{itemize}
            \item Hydrophilic solvent / high \emph{p}H leads to expansion.
            \item Hydrophobic solvent / low \emph{p}H induces collapse to globules.
        \end{itemize}
    \end{itemize}
    \item Flory-Rehner theory of gel swelling.
    \begin{figure}[H]
        \centering
        \includegraphics[width=0.3\linewidth]{swelling.JPG}
        \caption{Gel swelling schematic.}
        \label{fig:swelling}
    \end{figure}
    \begin{itemize}
        \item Key assumption: The energetic contributions of elasticity and mixing are linearly additive. That is,
        \begin{equation*}
            \Delta G_\text{tot} = \Delta G_\text{mix}+\Delta G_\text{elastic}
        \end{equation*}
        \item We assume a (basically infinite) reservoir of solvent, and we look at how much solvent goes in or out of the system. Solvent will go in until the chemical potential of the solvent is equal both outside and inside the polymer ($\mu'=\mu''$).
        \item Parameters to consider.
        \begin{itemize}
            \item A thermodynamicallly good solvent swells rubber. This involves a favorable $\chi$ interaction (i.e., $\chi<1/2$) and a favorable entropy of mixing.
            \item Entropic elasticity of the network (as discussed last lecture) exerts a retractive force to oppose swelling.
        \end{itemize}
    \end{itemize}
    \item Variables to be aware of in Flory-Rehner theory.
    \begin{itemize}
        \item $\alpha_x=x'/x$ is the linear swelling ratio.
        \begin{itemize}
            \item We assume extension ratios for swelling are isotropic (the same in every dimension). Thus, we only have one extension ratio $\alpha_s$ for swelling, defined by
            \begin{equation*}
                \alpha_s := \alpha_x=\alpha_y=\alpha_z
            \end{equation*}
        \end{itemize}
        \item $\phi_2=V_0/V$ is the volume fraction of the polymer in its swelled volume.
        \begin{itemize}
            \item $V_0$ is the volume of the polymer globule (containing no solvent, volume is all polymer).
            \item $V$ is the final, swelled volume.
        \end{itemize}
        \item $z=N=\rho N_\text{A}V_0/M_x$ is the number of subchains in the network, i.e., degree of crosslinking.
        \item $\mu_i=(\pdv*{G_\text{tot}}{n_i})_{T,P,n_j}$ is the chemical potential of component $i$.
        \item $\mu_1-\mu_1^\circ$ is the chemical potential difference between the solvent in the gel and the pure solvent.
    \end{itemize}
    \item Useful mathematical relations between said variables.
    \begin{itemize}
        \item The swelled volume $V$ scales from the polymer volume via the swelling ratio.
        \begin{equation*}
            V = V_0\alpha_s^3
        \end{equation*}
        \begin{itemize}
            \item An important consequence (we will see why shortly) is that $\alpha_s^3=1/\phi_2$.
        \end{itemize}
        \item It follows from the chemical potential equations that
        \begin{align*}
            \mu_1-\mu_1^\circ &= \left( \pdv{\Delta G_\text{tot}}{n_1} \right)_{T,P,n_2}\\
            &= \left( \pdv{\Delta G_\text{mix}}{n_1} \right)_{T,P,n_2}+\left( \pdv{\Delta G_\text{elastic}}{n_1} \right)_{T,P,n_2}\\
            \mu_1-\mu_1^\circ &= \left( \pdv{\Delta G_\text{mix}}{n_1} \right)_{T,P,n_2}+\left( \pdv{\Delta G_\text{elastic}}{\alpha_s} \right)\left( \pdv{\alpha_s}{n_1} \right)_{T,P,n_2}
        \end{align*}
        \begin{itemize}
            \item PSet 2, Q1a evaluated the left term above.
            \item Today, we will calculate the right term above.
        \end{itemize}
        \item In particular, recall from Lecture \ref{sse:6} that for a polymer-solvent solution,
        \begin{equation*}
            \left( \pdv{\Delta G_\text{mix}}{n_1} \right)_{T,P,n_2} = RT(\ln\phi_1+\phi_2+\chi_{12}\phi_2^2)
        \end{equation*}
        \begin{itemize}
            \item We've gotten rid of the $-\phi_2/N_2$ term because network crosslinking implies $N_2\to\infty$.
        \end{itemize}
    \end{itemize}
    \item Evaluating the $\Delta G_\text{elastic}$ term.
    \begin{itemize}
        \item Recall from last class that the change in entropy per subchain is
        \begin{equation*}
            \frac{\Delta S_\text{elastic}}{z} = -\frac{\kB}{2}\bigg\{ \left[ (\alpha_x^2+\alpha_y^2+\alpha_z^2)-3 \right]\underbrace{-\ln\frac{V}{V_0}}_\text{new term} \bigg\}
        \end{equation*}
        \begin{itemize}
            \item The new term corresponds to an additional entropy increase per subchain upon swelling, due to the increased volume that swelling makes available to each subchain.
            \item The new term turns about to be unimportant, but it isn't that negligible when you get into the details. More precisely, it is the major term when the polymer is dry, but it is not the major term when the polymer is already wet / partially swollen.
        \end{itemize}
        \item It follows that the whole network's change in Gibbs free energy upon swelling is
        \begin{equation*}
            \Delta G_\text{elastic} = -T\Delta S_\text{elastic}
            = \frac{z\kB T}{2}\left( 3\alpha_s^2-3-\ln\frac{V}{V_0} \right)
        \end{equation*}
        \item Now we take derivatives.
        \item Invoking the chain rule as above, we first investigate
        \begin{align*}
            \pdv{\Delta G_\text{elastic}}{\alpha_s} &= \frac{z\kB T}{2}\pdv{\alpha_s}(3\alpha_s^2-3-\ln\frac{\alpha_s^3V_0}{V_0})\\
            &= \frac{z\kB T}{2}\left( 6\alpha_s-\frac{3\alpha_s^2}{\alpha_s^3} \right)\\
            &= \frac{3}{2}z\kB T\left( 2\alpha_s-\frac{1}{\alpha_s} \right)
        \end{align*}
        \item To solve other derivative, we must first express $\alpha$ in terms of $n_1$ as follows.
        \begin{equation*}
            \alpha_s^3 = \frac{V}{V_0}
            = \frac{V_0+n_1\hat{V}_1}{V_0}
        \end{equation*}
        \begin{itemize}
            \item $n_1$ is the number of moles of solvent.
            \item $\hat{V}_1$ is the molar volume of the solvent.
        \end{itemize}
        \item Then, using implicit differentiation,
        \begin{align*}
            \pdv{n_1}(\alpha_s^3) &= \pdv{n_1}(\frac{V_0+n_1\hat{V}_1}{V_0})\\
            3\alpha_s^2\pdv{\alpha_s}{n_1} &= \frac{\hat{V}_1}{V_0}\\
            \pdv{\alpha_s}{n_1} &= \frac{\hat{V}_1}{3\alpha_s^2V_0}
        \end{align*}
        \item Putting everything together, we obtain
        \begin{align*}
            \left( \pdv{\Delta G_\text{elastic}}{n_1} \right)_{T,P,n_2} &= \left( \pdv{\Delta G_\text{elastic}}{\alpha_s} \right)\left( \pdv{\alpha_s}{n_1} \right)_{T,P,n_2}\\
            &= \frac{3}{2}z\kB T\left( 2\alpha_s-\frac{1}{\alpha_s} \right)\cdot\frac{\hat{V}_1}{3\alpha_s^2V_0}\\
            &= \frac{z\kB T\hat{V}_1}{2V_0}\left( \frac{2}{\alpha_s}-\frac{1}{\alpha_s^3} \right)
        \end{align*}
        \item We reexpress this in terms of $\phi_2$ and $M_x$ using the useful mathematical relations from earlier.
        \begin{align*}
            \left( \pdv{\Delta G_\text{elastic}}{n_1} \right)_{T,P,n_2} &= \frac{\rho N_\text{A}}{M_x}\cdot\frac{\kB T\hat{V}_1}{2}\left[ \frac{2}{(1/\phi_2)^{1/3}}-\frac{1}{1/\phi_2} \right]\\
            &= \frac{\rho RT\hat{V}_1}{M_x}\left( \phi_2^{1/3}-\frac{\phi_2}{2} \right)
        \end{align*}
    \end{itemize}
    \item It follows that the overall change in chemical potential upon swelling is
    \begin{equation*}
        \frac{\mu_1-\mu_1^\circ}{RT} = \left[ \ln(1-\phi_2)+\phi_2+\chi_{12}\phi_2^2 \right]+\frac{\rho\hat{V}_1}{M_x}\left( \phi_2^{1/3}-\frac{\phi_2}{2} \right)
    \end{equation*}
    \begin{itemize}
        \item This is the famous Flory-Rehner theory of gel swelling.
    \end{itemize}
    \item Once the optimal swelling has been achieved, $\mu_1=\mu_1^\circ$ (that is, $\mu_1-\mu_1^\circ=0$). Let's plug this into the Flory-Rehner theory and solve for $\phi_2$.
    \begin{itemize}
        \item First, let's assume $\phi_2\ll 1$ and expand the logarithm in the Flory-Rehner equation to two terms, as in Lecture \ref{sse:6}. This affords
        \begin{equation*}
            -\phi_2-\frac{\phi_2^2}{2}+\phi_2+\chi\phi_2^2 = -\frac{\rho\hat{V}_1}{M_x}\left( \phi_2^{1/3}-\frac{\phi_2}{2} \right)
        \end{equation*}
        \item If $\phi_2\ll 1$, then $\phi_2^{1/3}\gg\phi_2/2$. Thus, we may neglect the rightmost term above to obtain
        \begin{equation*}
            \phi_2^2\left( \chi-\frac{1}{2} \right) = -\frac{\rho\hat{V}_1\phi_2^{1/3}}{M_x}
        \end{equation*}
        \begin{itemize}
            \item The right term above is negative, so for the left term above to be negative, we must have $\chi<1/2$! This is how the math expresses our intuition that the polymer must be soluble in the solvent in order for swelling to occur.
        \end{itemize}
        \item Knowing $\chi<1/2$, we may write
        \begin{align*}
            \phi_2^2\left| \chi-\frac{1}{2} \right| &= \frac{\rho\hat{V}_1\phi_2^{1/3}}{M_x}\\
            \phi_2^{5/3} &= \frac{\rho\hat{V}_1}{M_x}\cdot\frac{1}{\left| \chi-\frac{1}{2} \right|}
        \end{align*}
        \begin{itemize}
            \item It follows that the variables we can tweak to control swelling are (1) the subchain molecular weight $M_x$ and (2) making $|\chi-1/2|$ large.
            \begin{itemize}
                \item Moreover, since $\chi$ can't be changed that much, it's really down to $M_x$.
            \end{itemize}
            \item If you want the material to swell by 2x, you need $M_x$ to grow by almost 4x because of the $3/5$ exponent.
            \item Remember that doubling the volume is equivalent to changing the lateral parameters by $\alpha=2^{1/3}$. Thus, to double the lateral parameters, you will need to \emph{octuple} the volume.
        \end{itemize}
        \item $\phi_2=\alpha^{-3}$, so
        \begin{align*}
            \frac{1}{\alpha^3} &= \left( \frac{\rho\hat{V}_1}{M_xf(\chi)} \right)^{3/5}\\
            \alpha &\propto M_x^{1/5}f(\chi)^{1/5}
        \end{align*}
        where $f(\chi)=1/|\chi-1/2|$ shorthands the relevant function of $\chi$.
        \begin{itemize}
            \item Implication: You have to change the molecular weight by \emph{a lot} in order to affect the swelling.
        \end{itemize}
    \end{itemize}
    \item Applications of gels: Contact lenses.
    \begin{figure}[H]
        \centering
        \includegraphics[width=0.5\linewidth]{contactLens.png}
        \caption{Contact lenses' mechanism of action.}
        \label{fig:contactLens}
    \end{figure}
    \begin{itemize}
        \item These are crosslinked hydrogels.
        \item How contact lenses work.
        \begin{itemize}
            \item Different eye shapes give different refractions that need to be neutralized.
        \end{itemize}
    \end{itemize}
\end{itemize}



\section{Ionic Gels}\label{sse:16}
\begin{itemize}
    \item \marginnote{10/28:}Lecture outline.
    \begin{itemize}
        \item Estimate swelling with Flory-Rehner theory.
        \item Compute charged interactions.
        \item Evaluate swelling of charged gels.
    \end{itemize}
    \item We begin by estimating the extent of swelling for gels we're likely to encounter.
    \begin{itemize}
        \item Recall from last time that if $\phi_2\ll 1$, then the equilibrium swelling is
        \begin{equation*}
            \phi_2 = \left( \frac{\rho\hat{V}_1}{M_x}\cdot\frac{1}{\left| \chi-\frac{1}{2} \right|} \right)^{3/5}
        \end{equation*}
        \item Approximations of the quantities.
        \begin{itemize}
            \item $1/f(\chi)\approx 2$ is a constant on the order of 1, so we'll neglect it outright.
            \item $\rho\approx\SI[per-mode=symbol]{1}{\gram\per\milli\liter}=\SI[per-mode=symbol]{e3}{\kilo\gram\per\cubic\meter}$ for most polymers (i.e., pretty close to that of water).
            \item $\hat{V}_1\approx N_\text{A}\cdot\SI{100}{\cubic\angstrom}$.
            \item $M_x\approx\text{\SIrange[per-mode=symbol]{e3}{e5}{\gram\per\mole}}$.
        \end{itemize}
        \item Let's now put all the approximations together, first assuming relatively short subchains.
        \begin{equation*}
            \phi_2 \approx \left[ \frac{(\SI{e3}{\kilo\gram\per\cubic\meter})(\SI{6e23}{\per\mole})(\SI{100e-30}{\meter\cubed})}{\SI{1}{\kilo\gram\per\mole}} \right]^{3/5}
            = (\num{6e-2})^{3/5}
            \approx 0.18
        \end{equation*}
        \item Increasing $M_x$ two orders of magnitude yields 0.01 as our approximation.
        \item Thus, we range from about 80\% of the volume being solvent to almost 99\%.
    \end{itemize}
    \item We now move onto charged gels.
    \begin{itemize}
        \item Recall that the $\pKa$ is the $\pH$ at which a group becomes charged.
    \end{itemize}
    \item What groups are useful for creating charged gels?
    \begin{itemize}
        \item Carboxylic acids.
        \begin{itemize}
            \item Deprotonate to contribute an anionic group to the gel.
            \item $\pKa\approx\text{\numrange{4}{5}}$.
        \end{itemize}
        \item Amines.
        \begin{itemize}
            \item Protonate to contribute a cationic group to the gel.
            \item $\pKa\approx\text{\numrange{10}{12}}$.
        \end{itemize}
        \item Sulfonates.
        \begin{itemize}
            \item Deprotonate.
            \item $\pKa\approx 0$.
        \end{itemize}
    \end{itemize}
    \item What is the $\pKa$ of a polyelectrolyte, e.g., polyacrylic acid?
    \begin{itemize}
        \item The $\pKa$ of the polyelectrolyte is \emph{higher} than that of a monoelectrolyte.
        \item This is because the like charges generated don't want to be in close proximity; rather, they have to pay an electrostatic (energetic) penalty to be so close.
        \item The $\pKa$ of a given polyelectrolyte will depend on the rigidity of the chain, how far apart all of the ionizable monomers are, etc.
    \end{itemize}
    \item Now consider a charged gel.
    \begin{figure}[h!]
        \centering
        \includegraphics[width=0.35\linewidth]{gelChargeNeg.JPG}
        \caption{A negatively charged gel.}
        \label{fig:gelChargeNeg}
    \end{figure}
    \begin{itemize}
        \item Since the gel is negatively charged along its backbones, it will have intercalating positive charges.
        \item Since there is no wall at the edge of the gel, the counterions are not trapped; rather, they can move out into the rest of solution! What we are interested in understanding is how many leave the gel.
        \item To address this question, let's begin by calculating the energy of a pair of charges.
        \begin{equation*}
            U_\text{elec} = \frac{q_+q_-}{4\pi\varepsilon r}
        \end{equation*}
        \begin{itemize}
            \item $\varepsilon$ is the permittivity of the solvent (as in Lecture \ref{sse:13}).
            \item This expression allows us to define the \textbf{Bjerrum length}.
        \end{itemize}
        \item Let's now return to the gel. How much energy does it take to move a charge from the border of the gel to just outside of it?
        \begin{itemize}
            \item A few charges may leave the gel with relatively little issue, but eventually, we will reach a point where it is unfavorable for 1 more charge to leave the gel.
            \item Let $Q_\text{macro}$ denote the number of charges that have left the gel (including the one leaving most recently), let $q_1$ denote the charge of the one that just left, and let $R$ denote the radius of the gel.
            \begin{itemize}
                \item Since positive charges are leaving the gel (as in Figure \ref{fig:gelChargeNeg}), the gel will have a net negative charge afterwards. Specifically, this net negative charge will be equal to $-q_1Q_\text{macro}$ (i.e., the opposite sign of the charge that is leaving, times the number of charges that have left).
                \item All of the charges that have left the gel before the one we're considering to have \emph{just} left are no longer part of the system; we do not account for them in our math at all.
            \end{itemize}
            \item Once the most recent charge leaves, what we essentially have is two charges separated by a distance equal to (just over) the radius of the gel. Symbolically,
            \begin{equation*}
                U_\text{elec} = \frac{q_1^2Q_\text{macro}}{4\pi\varepsilon R}
            \end{equation*}
            \item Dividing through by $\kB T$, we obtain
            \begin{equation*}
                \frac{U_\text{elec}}{\kB T} = \frac{q_1^2Q_\text{macro}}{4\pi\varepsilon R\kB T} = Q_\text{macro}\cdot\frac{\lB}{R}
            \end{equation*}
            \item Thus, if $Q_\text{macro}\cdot\lB/R>1$, then there will be no (net) transfer of charges out of the gel.
            \begin{itemize}
                \item We say no "net" transfer because diffusion of charges is still in a dynamic equilibrium.
                \item The implication is that once $Q_\text{macro}>R/\lB$, equilibrium will have been established.
            \end{itemize}
        \end{itemize}
        \item What proportion of the total amount of charge in the gel is leaving? Is it a lot or a little? To find out, let $Q_\text{tot}$ denote the original number of charges in the gel so we can investigate the ratio $Q_\text{macro}/Q_\text{tot}$.
        \begin{itemize}
            \item Let $v$ denote the volume of a monomer (assumed to be the same for polymer and solvent).
            \item If the gel is dry, it will contain $V_0/v$ monomers.
            \item It follows that the number of \emph{ionized} monomers in it is $fV_0/v$, where $f$ is the \textbf{degree of ionization}.
            \item Multiplying by $V/V=1$ and simplifying using $V_0/V=\phi_2$ (as defined last lecture) affords
            \begin{equation*}
                Q_\text{tot} = \frac{fV_0}{v}\cdot\frac{V}{V}
                = \frac{fV\phi_2}{v}
            \end{equation*}
            \item This combined with the fact that $Q_\text{macro}\propto R/\lB$ yields
            \begin{equation*}
                \frac{Q_\text{macro}}{Q_\text{tot}} = \frac{R/\lB}{fV\phi_2/v}
                \propto \frac{R/\lB}{fR^3/v\phi_2}
                = \frac{v}{\phi_2f\lB R^2}
            \end{equation*}
            where we have invoked $V\propto R^3$.
            \item Ultimately, this quantity will simplify our lives when we want to calculate swelling of charged gels.
        \end{itemize}
    \end{itemize}
    \item \textbf{Bjerrum length}: The distance $r$ at which $U_\text{elec}=\kB T$. \emph{Denoted by} $\bm{r_\textbf{B}},\bm{\ell_\textbf{B}}$. \emph{Given by}
    \begin{equation*}
        \lB = \frac{|q_+q_-|}{4\pi\varepsilon\kB T}
    \end{equation*}
    \begin{itemize}
        \item Example: The $\lB$ of water is about \SI{7}{\angstrom}. This means that two charges in water that are closer than \SI{7}{\angstrom} have a stronger interaction than $\kB T$. If they're farther apart, their interaction is less (so thermal noise will easily destroy that interaction).
    \end{itemize}
    \item \textbf{Degree of ionization}: The number of charged groups per polymer divided by the number of monomers per polymer. \emph{Denoted by} $\bm{f}$.
    \item We can now calculate the swelling of a charged gel.Assume the gel is electroneutral.
    \item In many ways, we will see that our gel is analogous to a piston.
    \begin{figure}[H]
        \centering
        \begin{subfigure}[b]{0.3\linewidth}
            \centering
            \includegraphics[width=0.8\linewidth]{gelPistona.JPG}
            \caption{Contracted piston.}
            \label{fig:gelPistona}
        \end{subfigure}
        \begin{subfigure}[b]{0.3\linewidth}
            \centering
            \includegraphics[width=0.8\linewidth]{gelPistonb.JPG}
            \caption{Expanded piston.}
            \label{fig:gelPistonb}
        \end{subfigure}
        \caption{Piston-like view of a charged gel.}
        \label{fig:gelPiston}
    \end{figure}
    \begin{itemize}
        \item The push out is the "gas" of ions pushing against the boundaries of the gel.
        \begin{itemize}
            \item We don't have to worry about electrostatics because we're assuming charge neutrality.
            \item Pressure on the boundary of the gel is maintained by the electrostatic force we derived above, which arises if charges flow out.
        \end{itemize}
        \item The pull in is the entropic spring force of the chains in the network.
        \item What we want to answer is, "how much will a chain stretch against the effect of the interactions of charged particles in the system?"
    \end{itemize}
    \item Let's calculate the free energy of the system.
    \begin{itemize}
        \item We have a solvent, a polymer, and the free energy of the gas that wants to expand. So
        \begin{equation*}
            \Delta G = \Delta G_\text{mix}+\Delta G_\text{elastic}+\Delta G_\text{ideal gas of counterions}
        \end{equation*}
        \item Flory-Huggins treated $\Delta G_\text{mix}$ and Flory-Rehner treated $\Delta G_\text{elastic}$, but what about the last term? In this case, recall from thermodynamics that the reversible work done by expanding an ideal gas isothermally (constant $T$) is given by
        \begin{align*}
            \Delta G &= -\int_{V_0}^VP\dd{V}\\
            &= -\int_{V_0}^V\frac{N_\text{counterions}\kB T}{V}\dd{V}\\
            \frac{\Delta G}{\kB T} &= -N_\text{counterions}\ln\frac{V}{V_0}\\
        \end{align*}
        \begin{itemize}
            \item This form is analogous to the "new term" from last time!
            \item Last time, this term was irrelevant. It's not irrelevant now, though, because it is multipled by the total number of counterions in the gel!
        \end{itemize}
        \item So if we bash out the math, we get analogously to last time that
        \begin{equation*}
            \left( \chi-\frac{1}{2} \right)\phi_2^2 = -\frac{\rho\hat{V}_1}{M_x}\left( \phi_2^{1/3}-\frac{\phi_2}{2}-fN\phi_2 \right)
        \end{equation*}
        \begin{itemize}
            \item $N$ is the average degree of polymerization of the subchains.
            \item Major physical implication of this equation: $\chi$ can be bigger than $1/2$.
            \begin{itemize}
                \item Essentially, we can dissolve gels that wouldn't otherwise want to dissolve in a given solvent \emph{because} of the force of ions pushing out.
                \item This makes sense as the ion pressure should "push outward," creating more room for solvent.
            \end{itemize}
            \item The rightmost term will, in fact, be a dominant term for a charged gel.
        \end{itemize}
    \end{itemize}
    \item Consider the piston with a semipermeable membrane that does not allow the "gas" molecules to pass through.
    \begin{itemize}
        \item Let the cylinder have base area $A$ and height $H$.
        \item The pressure from the gas is
        \begin{equation*}
            P_\text{gas} = \frac{N_\text{counterions}\kB T}{V}
            = \frac{fnN\kB T}{V}
        \end{equation*}
        \begin{itemize}
            \item $f$ is the degree of ionization, $n$ is the number of chains, and $N$ is the degree of polymerization of each chain.
        \end{itemize}
        \item On the other hand, recall from Lecture \ref{sse:4} that the elastic force $F$ generated by a single polymer chain is
        \begin{equation*}
            F \propto\frac{\kB TH}{Nb^2}
        \end{equation*}
        \begin{itemize}
            \item $b$ is a Kuhn length.
            \item Thus since $V=AH$, the pressure (force/area) for all $n$ chains is
            \begin{equation*}
                P_\text{elastic} = \frac{nF}{A}
                = \frac{n\kB TH}{Nb^2A}
                = \frac{n\kB TH^2}{Nb^2V}
            \end{equation*}
        \end{itemize}
        \item Then
        \begin{align*}
            P_\text{gas} &= P_\text{elastic}\\
            \frac{fnN\kB T}{V} &= \frac{n\kB TH^2}{Nb^2V}\\
            fN^2 &= \frac{H^2}{b^2}\\
            H &\propto Nbf^{1/2}
        \end{align*}
        \item Takeaway: This \textbf{counterion pressure} is huge.
    \end{itemize}
    \item What resources can I read about all of this?? It doesn't appear to be in any of our textbooks, and internet resources appear to often be too in depth.
    \item Announcements.
    \begin{itemize}
        \item Office hours time tomorrow and Friday.
        \item PSet 4 deadline extended to Sunday night.
    \end{itemize}
\end{itemize}



\section{Review for Quiz 2}
\begin{itemize}
    \item \marginnote{10/30:}Quiz 2 details.
    \begin{itemize}
        \item Quiz is 80 minutes, starts 3:05 PM sharp on 11/4.
        \item Open book, notes, calculator (except no AI).
        \item Undergrads do only 75\%.
        \item Quiz is based on homeworks, and includes viscosimetry, rubbers, and gels. No questions on scattering.
    \end{itemize}
    \item Review starts now.
    \item Viscosimetry basics.
    \begin{itemize}
        \item Understand the definition of viscosity as the ratio between the shear stress and shear rate (aka strain rate), i.e., $\tau=\eta\dot{\gamma}$.
        \item In a Newtonian fluid, $\eta$ is a constant (i.e., does not vary with $\dot{\gamma}$).
        \item In non-Newtonian fluids, we can have shear thinning (a negative deviation in the slope of a $\tau=\eta\dot{\gamma}$ graph) or thickening (a positive deviation in said slope).
        \begin{itemize}
            \item Up to some critical shear rate $\dot{\gamma}^*$, we have a linear dependence; after this, we deviate to non-Newtonian behavior.
            \item Shear thinning occurs when you shear faster than the relaxation time of the polymer, inducing stretching and tumbling!
            \item For our polymer solutions, we \emph{don't} want this to happen, so we stay in the regime in which $Wi=\tau_\text{polymer}\dot{\gamma}<1$.
            \item Look ahead: Later, we will discuss what regulates $\tau_\text{polymer}$.
        \end{itemize}
    \end{itemize}
    \item Stokesian friction (the Rouse free draining and Zimm non-draining models).
    \begin{itemize}
        \item When a fluid is pushed through a fluid at velocity $v$ (Figure \ref{fig:stokesViscosity}), it's motion is opposed by a viscous force $F_\text{viscous}=fv$.
        \item The proportionality constant $f=6\pi\eta_0r_s$, where $\eta_0$ is the viscosity of the solvent and $r_s$ is the radius of the sphere.
        \item In the Rouse model, $f=6\pi\eta_0Nb$, for the $N$ tiny spheres each of radius $b$.
        \item In the Zimm model, $f=6\pi\eta_0R_H$, where $R_H$ is the hydrodynamic radius of the polymer.
        \begin{itemize}
            \item Recall that $R_H=\gamma R_g$, where $\gamma$ is an order 1 proportionality constant. $\gamma\approx 0.7$ often.
        \end{itemize}
    \end{itemize}
    \item Diffusion of coils.
    \begin{itemize}
        \item The diffusion constant $d=\kB T/f$, where $f$ is the friction factor from above.
        \item For the draining model, $d=\kB T/6\pi\eta_0Nb$.
        \item For the non-draining model, $d=\kB T/6\pi\eta_0R_H$.
        \item It follows --- as in PSet 3, Q1 --- that the diffusion time $\tau_\text{diff}=\prb{R_g^2}/D$.
        \item Note: Since diffusion for the free draining model scales as $Nb$ while diffusion for the hard sphere scales $N^{1/2}b$, it is \emph{harder} to drag a free-draining material through a fluid!
    \end{itemize}
    \item Viscosity of dilute polymer solutions.
    \begin{itemize}
        \item Using Einstein's relationship, $\eta(\phi_{hs})=\eta_s(1+2.5\phi_{hs}+\cdots)$.
        \begin{itemize}
            \item For this to hold, we must be at a concentration $c\ll c^*$.
        \end{itemize}
        \item For polymers, we change $\phi_{hs}$ to the pervaded volume $\phi_\text{pol}\propto nR_H^3/V$, where $n$ is the number of chains.
        \item We then rephrase things in terms of the specific viscosity, and then the intrinsic viscosity.
        \item Putting everything together, we get to the Mark-Houwink-Sakurada relation of
        \begin{equation*}
            [\eta] = KM^a
        \end{equation*}
        \begin{itemize}
            \item $a=1/2$ for a $\theta$ solvent, $a<1/2$ for a bad solvent, and $1/2<a<4/5$ for a good solvent.
        \end{itemize}
        \item If you know the intrinsic viscosity, you can get the size of the coil.
        \begin{itemize}
            \item We use the equation
            \begin{equation*}
                c^* = \frac{M}{N_\text{A}V_p} = \frac{2.5}{[\eta]}
            \end{equation*}
        \end{itemize}
    \end{itemize}
    \item We now move onto Topic 4 content.
    \item Rubbers.
    \begin{itemize}
        \item Rubbers are networks.
        \item We assume rubbers deform via affine deformations (Figure \ref{fig:rubberAssumeb}). This allows us to calculate stresses at the microscopic level.
        \item The force on a chain is equal to
        \begin{equation*}
            F = -T\pdv{\Delta S}{l}
        \end{equation*}
        \begin{itemize}
            \item We can evaluate on a microscopic level, or scale up to a macroscopic level.
            \item Take $T\Delta S$, which is on the order of $(l^2-l_0^2)/Nb^2$.
        \end{itemize}
        \item We defined extension ratios.
        \begin{itemize}
            \item Noncompressibility ($\Delta V=0$) affords $\alpha_x\alpha_y\alpha_z=1$.
        \end{itemize}
        \item Types of noncompressible deformation: Uniaxial, biaxial, and holding one dimension constant while the other two change.
        \item Our analysis of the uniaxial extension of a rubber got us to the Young's modulus.
        \begin{itemize}
            \item Essentially, we got an equation of the form $\sigma=Ef(\alpha)$.
            \item The Young's modulus has the form
            \begin{equation*}
                E = \frac{3\rho RT}{M_x}
                = \frac{3\rho\kB T}{M_x/N_\text{A}}
                \propto \frac{3\kB T}{N}
            \end{equation*}
            where $N$ is the average degree of polymerization per subchain.
            \begin{itemize}
                \item Recall that $3\kB T/N$ is very similar to our Hookean spring constant from Lecture \ref{sse:4}!
            \end{itemize}
        \end{itemize}
    \end{itemize}
    \item Gels (swollen rubbers).
    \begin{itemize}
        \item Here, we don't have the noncompressibility restriction.
        \item Flory-Rehner theory describes these.
        \item In Flory-Rehner theory, we have a balance between $\Delta G_\text{elastic}$ and $\Delta G_\text{mix}$. Moreover, we assume $\chi<1/2$.
        \item Our derivation gets us to the equation
        \begin{equation*}
            \phi_2 = \left( \frac{\rho\hat{V}_1}{M_x}\cdot\frac{1}{\left| \chi-\frac{1}{2} \right|} \right)^{3/5}
        \end{equation*}
    \end{itemize}
    \item Ionic gels.
    \begin{itemize}
        \item Counterions expand ionic gels by applying pressure to the boundaries.
        \item In this case, we get the scaling $l\propto M_x$.
        \item The piston analogy is always a good one to fall back on here; counterion pressure and entropic spring force matter far more than osmotic pressures.
        \item Last thing (we haven't seen this previously): The more swollen a gel is, the lower its modulus.
    \end{itemize}
    \item No osmometry, no GPC, no light scattering.
    \item $f$ will be provided to us in any question.
    \item What Quiz 1 concepts will show up?
    \begin{itemize}
        \item Just the prerequisites to this content; not Mayer-$f$ functions or anything.
    \end{itemize}
    \item Resources for ionic gels.
    \begin{itemize}
        \item Typically in papers, but Peppa has a book on gels.
        \item Alfredo will try to post some resources.
    \end{itemize}
    \item There should be a Practice Quiz 2.
\end{itemize}



\section{Office Hours (Alexander-Katz)}
\begin{itemize}
    % \item PSet 4, Q1: Is the Rayleigh ratio of benzene \SI{46.5e4}{\per\meter}, so that it matches the order of magnitude of the values in the table better (i.e., is the negative sign a typo)? Or is it actually the tiny quanity of \SI{46.5e-4}{\per\meter}?
    \item How can we multiply $\lB/R$ by $Q_\text{macro}$ to get the condition for charges not leaving the gel?
    \begin{itemize}
        \item $Q_\text{macro}$ is the charge of the gel after \emph{many} charges have left it. $q_1$ is the charge of a single charge that has just left the gel. $Q_\text{macro}/q_1$ will be a natural number.
        \item Of all the ones that can actually leave, how many do? It's a tiny fraction.
    \end{itemize}
    \item Doesn't $V_0/V=\phi_2$?
    \begin{itemize}
        \item Yes.
    \end{itemize}
    \item What is $N_x$? Is it the degree of of polymerization of each subchain on average?
    \begin{itemize}
        \item Yep!
    \end{itemize}
    \item What is the $N$ in the Young's modulus equation from the review?
    \begin{itemize}
        \item \emph{Proportional} to $1/N$.
        \item $N$ is the average degree of polymerization between crosslinks.
        \item This equation shows that the Young's modulus is very similar to the spring constant of a single chain.
    \end{itemize}
    \item PSet 4, Q2a: The stress cross section being the plane perpendicular to the force, not the line? More info on the stress tensor and understanding it in general?
    \begin{itemize}
        \item Remember to state that we're using the \emph{engineering stress} and not the \emph{true} stress!
    \end{itemize}
    \item PSet 4, Q2b: Just sub in the $d,R,\sigma$ equations? That's it?
    \begin{itemize}
        \item Yep!
    \end{itemize}
    \item PSet 4, Q3d: What else can I do here besides rearranging?
    \begin{itemize}
        \item That's it!
    \end{itemize}
    \item PSet 4, Q3e: How can I derive this from Part (d)? In class, we got this result through orthogonal means?
    \begin{itemize}
        \item Use scalings and express $\phi_2$ in terms of $L$.
    \end{itemize}
    \item PSet 4, Q3f: How does voltage in this setup change the degree of ionization?
    \begin{itemize}
        \item We are working in an extremely oversimplified picture, so the bottom line is that we are \emph{postulating} that changing the voltage will change $f$ as the graph suggests.
        \item One way of starting to think about this is that if we are releasing positive charges into solution, for example
    \end{itemize}
    \item PSet 4, Q3f: Is the scaling relation any different than the result from part (e)?
    \begin{itemize}
        \item Yes --- we have to geometrically change the length-based result from part (e) into a result in terms.
    \end{itemize}
\end{itemize}




\end{document}